\lab{Эффект Холла в металлах}

\aim{измерение подвижности и концентрации носителей заряда в металлах.}

\equip{электромагнит с источником питания, источник постоянного тока, микровольтметр, амперметры, милливеберметр или цифровой магнитометр, образцы из меди, серебра и цинка.}

Элементарная теория свободных носителей заряда в~металлах и полупроводниках изложена во введении к разделу.

\experiment Электрическая схема установки для измерения ЭДС Холла представлена на рис.~\figref{Scheme}.

\begin{figure}[h!]
	\pic{0.9\textwidth}{3_5_1}
	\caption{Схема установки для исследования эффекта Холла в~металлах}
	\figmark{Scheme}
\end{figure}

В зазоре электромагнита (рис.~\figref{Scheme}а) создаётся постоянное магнитное поле, величину которого можно менять с помощью источника питания электромагнита. Разъём $K_1$ позволяет менять направление тока в обмотках электромагнита. Ток питания электромагнита измеряется амперметром $A_1$.

Градуировка магнита проводится с помощью милливеберметра (его описание и правила работы с ним приведены на с.~\pageref{MWB}) или цифрового магнитометра на основе датчика Холла.

Металлические образцы в форме тонких пластинок, смонтированные в специальных держателях, подключаются к блоку питания через разъём (рис.~\figref{Scheme}б). Ток через образец регулируется реостатом $R_2$ и измеряется амперметром $A_2$.

Для измерений ЭДС Холла используется микровольтметр $\text{Ф}116/1$, в котором высокая чувствительность по напряжению сочетается с малой величиной тока, потребляемого измерительной схемой: минимальный предел измерения напряжения составляет 1 мкВ, а потребляемый ток~--- всего $10^{-8}$ А.

В образце с током, помещённом в зазор электромагнита, между контактами 2 и 4 возникает холловская разность потенциалов, которая измеряется с помощью микровольтметра, если переключатель $K_3$ подключён к точке 2 образца. При подключении $K_3$ к точке 3 микровольтметр измеряет омическое падение напряжения $U_{34}$, вызванное основным током через образец. При нейтральном положении ключа входная цепь микровольтметра разомкнута.

Ключ $K_2$ позволяет менять полярность напряжения, поступающего на вход микровольтметра.

Иногда контакты 2 и 4 вследствие неточности подпайки не лежат на одной эквипотенциали, и тогда напряжение между ними связано не только с эффектом Холла, но и с омическим падением напряжения, вызванным протеканием основного тока через образец. Измеряемая разность потенциалов при одном направлении магнитного поля равна сумме ЭДС Холла и омического падения напряжения, а при другом~--- их разности. В этом случае ЭДС Холла $V_{xy}$ может быть определена как половина алгебраической разности показаний вольтметра, полученных для двух противоположных направлений магнитного поля в зазоре.

Можно исключить влияние омического падения напряжения иначе, если при каждом токе через образец измерять
напряжение $U_0$ между точками 2 и 4 в отсутствие магнитного поля. При фиксированном токе через образец это
дополнительное к ЭДС Холла напряжение остаётся неизменным. От него следует (с учётом знака) отсчитывать величину
ЭДС Холла:
\begin{equation}
	V_{xy}=U_{24}\pm U_0.
	\eqmark{3.5.1}
\end{equation}

При таком способе измерения нет необходимости проводить повторные измерения с противоположным направлением магнитного поля.

По знаку $V_{xy}$ можно определить характер проводимости~--- электронный или дырочный. Для этого необходимо знать направление тока в образце и направление магнитного поля.

Измерив ток $I$ в образце и напряжение $U_{34}$ между контактами 3 и 4 в отсутствие магнитного поля, можно, зная
параметры образца, рассчитать удельное сопотивление материала образца по очевидной формуле:
\begin{equation}
	\rho_0=\frac{U_{34}wh}{Il},
	\eqmark{3.5.2}
\end{equation}
где $l$~--- расстояние между контактами 3 и 4, $w$~--- ширина образца, $h$~--- его толщина.

\begin{lab:task}

В работе предлагается исследовать зависимость ЭДС~Холла от величины магнитного поля при различных токах через образец для определения константы Холла; определить знак носителей заряда и проводимость различных металлических образцов.

\begin{enumerate}

\item{Подготовьте приборы к работе.}

\item{ Проверьте работу цепи питания образца. Для этого подключите к разъёму блока управления один из образцов~--- медный или серебряный. Убедитесь, что ток через образец можно изменять в указанных в описании пределах.}
\item{ Проверьте работу цепи магнита. Установите разъём $K_1$ в положение I и определите диапазон изменения тока через электромагнит.}
\item{ Прокалибруйте электромагнит. Для этого вставьте в зазор электромагнита пробную катушку милливеберметра и исследуйте зависимость магнитного потока $\Phi$, пронизывающего пробную катушку, от тока $I_M$ через обмотки магнита ($\Phi=BSN$). Значение $SN$ (площадь сечения пробной катушки на число витков в ней) указано на держателе катушки.

Проведите измерения магнитного потока для 6 -- 8 значений тока через электромагнит.}

\item{ Проведите измерение ЭДС Холла. Для этого вставьте образец в зазор выключенного электромагнита и определите напряжение $U_0$ между холловскими контактами 2 и 4 при минимальном токе через образец. Это напряжение $U_0$ вызвано несовершенством контактов 2, 4 и при фиксированном токе через образец остаётся неизменным. Значение $U_0$ с учётом знака следует принять за нулевое.

Включите электромагнит и снимите зависимость напряжения $U_{24}$ от тока $I_M$ через обмотки магнита при фиксированном токе через образец. Измерения следует проводить при \important{медленном} увеличении магнитного поля. Резкие изменения магнитного поля наводят ЭДС индукции в подводящих проводах и вызывают большие отклонения стрелки микровольтметра.

Повторите измерения $U=f(I_{M})$ при постоянном токе через образец для 5 -- 6 его значений в интервале, указанном в описании конкретной установки. При каждом новом значении тока через образец величина $U_0$ будет иметь своё значение.

При максимальном токе через образец проведите измерения $U=f(I_{M})$ при другом направлении магнитного поля.

Для образца из цинка снимите зависимость $U=f(I_{M})$ при одном значении тока через образец.}

\item{Определите знак носителей в образце. Для этого необходимо знать направление тока через образец, направление магнитного поля и знак ЭДС Холла. Направление тока в образце показано знаками <<$+$>> и <<$-$>> на рис.~\figref{Scheme}. Направление тока в обмотках электромагнита при установке разъёма $K_1$ в положение I показано стрелкой на торце магнита.

Напомним, что знак потенциала, соответствующий точкам 2 или 4, можно определить по рис.~\figref{Scheme}.

Зарисуйте в тетради образец. Укажите на рисунке направление тока, магнитного поля (положение разъёма $K_1$) и знак
потенциала, соответствующий клемме 4 (положение ключа $K_2$ при отклонении стрелки вольтметра вправо).

Определите знак носителей заряда для каждого из двух образцов.}
\item{Определите удельную сопротивление образца. Для этого удалите держатель с образцом из зазора. Переключите микровольтметр (в случае необходимости) в режим измерения милливольтовых напряжений. Ключ $K_3$ поставьте в положение $U_{34}$. При токе через образец порядка максимального значения в предыдущих измерениях измерьте падение напряжения между контактами 3 и 4 для каждого из двух образцов.}

\item{ Запишите характеристики приборов и параметры образцов $l$ $w$, $h$ (длину, ширину и толщину), указанные на держателях.}
\end{enumerate}

\tasksection{Обработка результатов}
\begin{enumerate}
\item{Рассчитайте индукцию магнитного поля~$B$ для каждого значения тока и постройте график зависимости $B=f(I_{M})$.}

\item{Рассчитайте ЭДС Холла по формуле \eqref{3.5.1} и постройте на одном листе семейство характеристик $V_{xy}=f(B)$ при разных значениях тока $I$ через образец (для меди или серебра). Определите угловые коэффициенты $k(I)=\Delta V_{xy}/\Delta B$ полученных прямых.}

\item{ Постройте график $k=f(I)$. Рассчитайте угловой коэффициент прямой и по формуле \chaptereqref{3.19} определите величину постоянной Холла $R_X$.

Для цинка изобразите на графике зависимость $V_{xy}=f(B)$ и по наклону прямой рассчитайте постоянную Холла.

Для обоих образцов рассчитайте концентрацию $n$ носителей тока по формуле \chaptereqref{HallConstant}.

Оцените погрешности и сравните результаты с табличными.}

\item{ Рассчитайте удельное сопротивление $\rho_0$ материала образцов по формуле \eqref{3.5.2}.

Используя найденные значения концентрации $n$ и проводимости $\rho_0$, с помощью формулы \chaptereqref{UdSoprot} рассчитайте подвижность $b$ носителей тока в~общепринятых для этой величины внесистемных единицах: размерность напряжённости электрического поля $[E] = [U/L]=\text{В/см}$, размерность скорости $[v]=\text{см/с}$, поэтому размерность подвижности $[b]=$см$^2$/(\text{В}$\cdot$\text{с}).

Оцените погрешности и сравните результаты с табличными.
}
\end{enumerate}

\end{lab:task}

\begin{lab:questions}

\item{ Какие вещества называют диэлектриками, проводниками, полупроводниками? Чем объясняется различие их электрических свойств? Как зависит от температуры проводимость металлов и полупроводников?}

\item{ Дайте определение константы Холла. Как зависит константа Холла от температуры у металлов и полупроводников?}

\item{ Зависит ли результат измерения константы Холла от геометрии образца?}

\item{ Как устроен милливеберметр? Зависят ли его показания от сопротивления измерительной катушки? Каким должно быть это сопротивление по сравнению с сопротивлением рамки прибора: большим или маленьким?}

\end{lab:questions}

\begin{lab:literature}

\item{ \emph{Сивухин Д.В.} Общий курс физики. Т.~III. Электричество~--- М.: Наука, 1983. \S\S~98, 100.}

\item{ \emph{Парселл Э.} Электричество и магнетизм.~--- М.: Наука, 1983. Гл.~4, \S\S~4--6; Гл.~6, \S~9 (Берклеевский курс физики. Т.~II).}
\end{lab:literature}