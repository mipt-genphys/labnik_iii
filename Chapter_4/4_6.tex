\lab{Параметрический резонанс в контуре с нелинейной индуктивностью}

\aim{изучение параметрических колебаний в электрической цепи.}

\equip{параметрон (две тороидальных катушки, резисторы, интегрирующая цепочка,
конденсаторы), генератор звуковых частот, реостат, сглаживающий дроссель,
магазин ёмкостей, магазин сопротивлений, вольтметр, миллиамперметр,
осциллограф.}

Перед выполнением работы рекомендуется ознакомиться с
пп.~\ref{sec:ferromagnetism}--\ref{sec:forces} теоретического
введения к разделу.

Если периодически изменять ёмкость конденсатора или самоиндукцию катушки,
входящих в состав колебательного контура, то при определённых условиях в контуре
возбуждаются незатухающие электрические колебания. Такой способ возбуждения
называется \term{параметрическим,} поскольку колебания возникают не под
действием внешней ЭДС, а вследствие изменения параметров контура.

Рассмотрим колебательный контур, состоящий из последовательно соединённых
ёмкости $C$, индуктивности $L$ и сопротивления $R$. В силу неизбежных внешних
влияний и тепловых флуктуаций в контуре всегда имеют место небольшие колебания с
циклической частотой $\omega_0$, которая при малых потерях зависит только от реактивных
параметров $L$ и $C$:
\begin{equation}
	\eqmark{4.6.1}
	 \omega_0 = \frac{1}{\sqrt{LC}}.
\end{equation}

При этом средняя полная энергия $W$, запасённая в контуре, остаётся постоянной;
происходит лишь её периодическая перекачка с частотой $2\omega_0$ из
электрической $W_\text{Э} = q^2 / (2C)$ в магнитную $W_M = LI^2/2$ и обратно.
Здесь $q$~--- заряд на обкладках конденсатора, $I$~--- ток в катушке
индуктивности. Полную энергию системы можно изменить, если скачком поменять
величину $L$ (или $C$).

 Рассмотрим, как изменяется энергия контура при быстром уменьшении $L$
(например, растяжении катушки) в тот момент, когда ток в катушке максимален.
Сумма падений напряжения на элементах контура равна ЭДС самоиндукции:
\begin{equation*}
	RI + \frac{q}{C}= - \frac{d(LI)}{dt}.
\end{equation*}

Проинтегрируем это уравнение по времени за очень короткий промежуток $\Delta t
~(\Delta t \ll 2\pi/\omega_0)$ в течение которого изменяется индуктивность. Два
первых интеграла при этом будут близки к нулю, поэтому можно считать, что
магнитный поток $\Phi$ в катушке в течение этого времени не изменяется:
$\Delta \Phi=\Delta (LI) = 0$ или
\begin{equation}
	\eqmark{4.6.2}
	\Phi=LI =\const.
\end{equation}

Уменьшение индуктивности в тот момент, когда ток в контуре максимален, ведёт к
увеличению тока и магнитной энергии в катушке:
\begin{equation*}
	\Delta W_M = \Delta \left(\frac{\Phi^2}{2L}\right) = - (LI)^2\frac{\Delta
L}{2L^2} = - \frac{I^2}{2}\Delta L.
\end{equation*}

Если теперь через четверть периода вернуть индуктивность к прежнему значению~---
энергия системы не изменится, так как ток в этот момент равен нулю. Ещё через
четверть периода опять уменьшим $L$~--- снова возрастёт энергия. Процесс
увеличения энергии системы за счёт работы внешних сил называют 
\term{накачкой}.
Заметим, что индуктивность при этом меняется с частотой, вдвое превосходящей
собственную частоту контура.

Запишем условие раскачки колебаний.
Энергия, которую получает контур за период $T$, равна  
$2W_M = I^2_{\rm max} \Delta L$. Она должна превышать потери на активном 
сопротивлении, составляющие $W_R = RI^2_\text{эфф}T$. Видно, что это
выполняется при 
\begin{equation}
\eqmark{positive}
\Delta L > \frac{RT}{2}.
\end{equation}

При выполнении условия \eqref{positive} амплитуда колебаний в
контуре возрастает с каждым периодом. С увеличением амплитуды всё более
возрастает роль нелинейной зависимости $B(H)$, что ограничивает возрастание
амплитуды. Поэтому со временем в контуре устанавливаются колебания максимально
возможной постоянной амплитуды. Это явление называют \term{параметрическим
резонансом}. То, что амплитуда установившихся колебаний ограничивается именно 
\emph{нелинейностью}, а не потерями, является принципиальной особенностью 
параметрического резонанса.

Раскачка колебаний возможна при изменении $C$ или $L$ по любому периодическому
закону с частотами $\Omega_\text{Н}$, для которых справедливо соотношение
\begin{equation*}
	\frac{\omega_0}{\Omega_\text{Н}} = \frac{n}{2},
\end{equation*}
где $n$ --- целое число (1,~2,~$\dots$). Наиболее эффективная раскачка имеет
место при $n$ = 1, когда частота накачки $(\Omega_\text{Н})$ равна частоте
колебаний энергии $W_\text{Э}$ и $W_M~(2\omega_0).$

 В том случае, когда индуктивность изменяется по синусоидальному закону, условие
возбуждения колебаний имеет вид
\begin{equation}
	\eqmark{4.6.3}
	\Delta L > \frac{2RT}{\pi}.
\end{equation}
Величиной индуктивности можно управлять током подмагничивания.

\begin{figure}[h!]
    \centering
    \pic{0.5\textwidth}{Chapter_4/4_6_1}
    \caption{Полная и частная петли гистерезиса}
    \figmark{hysteresis loop}
\end{figure}

На рис.~\figref{hysteresis loop} показана зависимость~$B(H)$ в ферромагнитном
сердечнике~--- петля гистерезиса. Меняя поле~$H$, можно выбрать такую рабочую
точку на петле, вблизи которой зависимость~$B(H)$ обладает наиболее ярко
выраженной нелинейностью. На рис.~\figref{hysteresis loop} это точка $A$:
вблизи неё особенно резко изменяется дифференциальная магнитная
проницаемость $\mu_\text{диф}$ (производная
$d\mu_\text{диф}/dH$ максимальна).

Соответствующее подмагничивающее поле $H_\text{п}$ задаётся постоянным током,
проходящим через дополнительную (подмагничивающую) обмотку.

Небольшие колебания величины~$B$ вокруг рабочей точки можно создать, подав на
вторую подмагничивающую обмотку переменный ток подмагничивания. Магнитная
проницаемость $\mu_\text{диф}$, а с ней индуктивность $L$, будут меняться с той
же частотой, что и переменная составляющая подмагничивающего тока (из-за
нелинейности $B(H)$ в кривой $\mu(t)$ присутствуют также колебания с кратными
частотами, не представляющие для нас интереса). Если изменения индуктивности
достаточно велики, то в контуре возбуждаются незатухающие колебания, частота
которых вдвое меньше частоты изменения параметров контура (в нашем случае~---
частоты изменения индуктивности, то есть частоты подмагничивания). Такое
соотношение частот служит отличительным признаком параметрических колебаний.

\experiment
Для изучения параметрических колебаний используется <<параметрон>>~--- установка
с нелинейной индуктивностью, схема которой представлена на
рис.~\figref{experimental environment}. <<Параметрон>> включает в себя две
тороидальных катушки, интегрирующую ячейку $r_0C_0$ (см. описание
принципа работы интегрирующей ячейки в работе \ref{lab:4-5}),
резисторы $r_1$ и~$r_2$, ключи~$K_1$ и~$K_2$ и колебательный контур.
Колебательный контур состоит из двух последовательно соединённых
индуктивностей~$L_1$ и~$L_2$, ёмкости~$C$ и сопротивлений~$R_M$ и~$r_2$.
На рисунке контур заключён в пунктирную рамку.
\begin{figure}[h!]
\centering
	\pic{\textwidth}{Chapter_4/4_6_2}
	\caption{Схема установки}
	\figmark{experimental environment}
\end{figure}

Обе катушки $L_1$ и $L_2$ с одинаковым числом витков~$n_1$ намотаны на
одинаковые тороидальные ферромагнитные сердечники. Длина каждого сердечника~---
$l$, сечение~--- $S$, магнитная проницаемость~--- $\mu$. 
С~помощью теоремы о циркуляции можно показать, что общая индуктивность катушек
\begin{equation}
	\eqmark{4.6.5}
	L = 2\mu_0\mu\frac{n^2_1S}{l}.
\end{equation}

Постоянный ток подмагничивания от источника постоянного напряжения 36~В проходит
через две последовательно соединённые обмотки с числом витков~$n_2$. Ток
регулируется потенциометром~$\text{П}$. Для того чтобы увеличить сопротивление
цепи переменному току, поставлена индуктивность $L_0 = 3$~Гн. Переменный ток в
этой цепи практически отсутствует, постоянный измеряется амперметром~$A$.

Переменный ток подмагничивания, создаваемый генератором звуковых частот,
проходит через последовательно соединённые обмотки~$n_3$. Обмотка~$n_4$ имеется
всего на одном из сердечников. Она служит для измерения полного магнитного
потока, проходящего через сердечник. Обмотки $n_1$ соединены так, что
возникающие в них ЭДС имеют противоположные знаки, поэтому в колебательном
контуре не возникают токи, имеющие частоту звукового генератора.

Для измерения напряжений в схему включён вольтметр переменного тока. При
переключении ключа $K_2$ в верхнее положение вольтметр измеряет напряжение
$U_\text{ЗГ}$ на выходе генератора, при переключении в нижнее --- выходное
напряжение на ёмкости $C$. Осциллограф позволяет наблюдать петлю гистерезиса,
фиксировать момент возникновения и срыва параметрических колебаний и определять
их частоту с помощью фигур Лиссажу.

При верхнем положении ключа $K_1$ на вход $X$ осциллографа подаётся падение
напряжения между точками 1 и 7, практически равное напряжению $U_{\text{ЗГ}}$ на
генераторе (падением напряжения на резисторах $r_1$, $r_2$ можно пренебречь,
поскольку оно мало по сравнению с $U_{\text{ЗГ}}$). На вход $Y$ подаётся
напряжение с ёмкости $C$ колебательного контура. По фигурам Лиссажу, возникающим
на экране, можно сравнить частоту накачки (частоту генератора) с частотой
колебаний контура.

При нижнем положении ключа $K_1$ на вход $Y$ подаётся напряжение $U_Y$ с ёмкости
$C_0$. Эта ёмкость входит в состав интегрирующей цепочки $r_0C_0$, подключённой
к обмотке $n_4$. ЭДС индукции, возникающая в обмотке $n_4$, пропорциональна
$dB/dt$:
\begin{equation*}
	U_4 = n_4S\frac{dB}{dt}.
\end{equation*}

Параметры интегрирующей цепочки подобраны так, что сопротивление $r_0$ заметно
превышает сопротивление обмотки $n_4$ и сопротивление ёмкости:
\begin{equation*}
	r_0 \gg \frac{1}{\omega_0C_0}.
\end{equation*}

 При этом условии ток в цепочке пропорционален $dB/dt$:
\begin{equation*}
	I_0 = \frac{U_4}{r_0} = \frac{n_4S}{r_0}\frac{dB}{dt},
\end{equation*}
а напряжение $U_Y$ на конденсаторе $C_0$ пропорционально $B$:
\begin{equation}
	\eqmark{4.6.6}
	U_Y = \frac{1}{C_0}\int I_0dt = \frac{1}{r_0C_0}\int U_4dt =
\frac{n_4S}{r_0C_0}B.
\end{equation}

На вход $X$ осциллографа подаётся сумма падений напряжения на резисторах $r_1$ и
$r_2$. Напряжение, возникающее на $r_1$, пропорционально току, протекающему
через обмотки $n_3$ от генератора. В отсутствие параметрических колебаний через
$r_2$ ток не течёт, и на вход $X$ подаётся напряжение $U_X$, пропорциональное
переменному току подмагничивания $I$, которым определяется поле $H$ в
сердечнике:
\begin{equation*}
	H = \frac{n_3I}{l}.
\end{equation*}
Следовательно,
\begin{equation}
	\eqmark{4.6.7}
	U_X = Ir_1 = \frac{lr_1}{n_3}H.
\end{equation}

Таким образом, в отсутствие параметрических колебаний на экране появляется
кривая гистерезиса ферромагнитного сердечника. При возникновении колебаний в
контуре через $r_2$ начинает проходить ток, кривая резко искажается и для
измерений непригодна. Но искажение петли позволяет отметить момент возникновения
параметрических колебаний и даёт возможность измерить параметры петли при
подходе к моменту самовозбуждения. Сфотографировав или зарисовав с экрана на кальку петлю
гистерезиса, соответствующую границе возбуждения параметрических колебаний,
можно экспериментально проверить справедливость формулы \eqref{4.6.3}~---
условия самовозбуждения. Из \eqref{4.6.5} следует
\begin{equation}
	\eqmark{4.6.8}
		\Delta L = L_{\rm max} - L_{\rm min} =
        \frac{2\mu_0n^2_1S}{l}\left(\mu_{\rm max} - \mu_{\rm min}\right)
		= \frac{2n^2_1S}{l}
%         \left[\left(\frac{dB}{dH}\right)_{max} -
        \left.\frac{dB}{dH}\right|_{\rm min}^{\rm max}.
\end{equation}

Производные $dB/dH$ следует взять из чертежа, проведя касательные к кривой
$B(H)$ слева и справа от излома петли. Для расчёта масштабов выразим $B$ и $H$
через напряжения $U_Y$ и $U_X$. Подставляя \eqref{4.6.6} и \eqref{4.6.7} в
\eqref{4.6.8}, получим
\begin{equation}
	\eqmark{4.6.9}
	\Delta L = 2r_0C_0r_1\frac{n^2_1}{n_3n_4}\left[\left(\frac{\Delta
U_Y}{\Delta U_X}\right)_{\rm max} - \left(\frac{\Delta U_Y}{\Delta
U_X}\right)_{\rm min}\right].
\end{equation}
В нашей установке $n_3 = n_4$, так что $n_1^2/ (n_3n_4) = 1$. Параметры $r_0$,
$C_0$ приведены на установке.

\begin{lab:task}

\taskpreamble{В работе предлагается с помощью фигур Лиссажу найти критическое сопротивление и
определить частоту параметрических колебаний контура; с помощью кривых
гистерезиса определить критическое сопротивление и проверить условие
самовозбуждения; по кривой зависимости напряжения на конденсаторе от частоты
определить резонансную частоту и индуктивность колебательного контура.}

\item
Соберите схему согласно рис.~\figref{parametron}. Сравните схему, изображённую
на рис.~\figref{parametron}, со схемой на рис.~\figref{experimental
environment}. Подготовьте приборы к работе.
\begin{figure}[h!]
\centering
	\pic{\textwidth}{Chapter_4/4_6_3}
	\caption{Блок-схема установки}
	\figmark{parametron}
\end{figure}

\item
Установите ёмкость $C = 100$~мкФ, сопротивление магазина $R_M =~0.$ Поставьте на
минимум выходного напряжения движок потенциометра, регулирующего постоянный ток
подмагничивания. Включите питание $= 36$~В и установите постоянный ток $I
=80$~мА. Переменный ток подмагничивания установите с помощью генератора: частота
$\nu = 150$~Гц; выходное напряжение на вольтметре генератора $U_\text{ЗГ} =
15$~В.

\item
\begin{minipage}[t]{0.6\textwidth}
Для наблюдения параметрических колебаний поставьте ключ $K_1$ в положение
<<Фигура Лиссажу>>. Увеличивая постоянный ток подмагничивания, определите момент
возникновения параметрических колебаний (при $U_\text{ЗГ} = 15$~В) по появлению
на экране ЭО фигуры Лиссажу, имеющей одно самопересечение (рис.~\figref{figures
of lissage}a). Оцените интервал $\Delta I$, внутри которого эти колебания
существуют. Используя показания генератора, определите по виду фигуры Лиссажу
частоту параметрических колебаний.
\end{minipage}
\hfil
\begin{minipage}[t]{0.3\textwidth}
%\begin{figure}[h!]
\vspace*{1pt}
\centering
	\pic{\linewidth}{Chapter_4/4_6_4}
	\captionof{figure}{%
       Фигуры Лиссажу при отношении частот~1:2 (масштабы разные).
       \figmark{figures of lissage}}
%\end{figure}
\end{minipage}

\item
Убедитесь в том, что Вы наблюдаете именно параметрические колебания, внеся в
контур дополнительное затухание~--- увеличивая сопротивление магазина $R_M$.
Колебания, возбуждаемые внешним источником, при увеличении затухания постепенно
уменьшаются по амплитуде, в то время как параметрические колебания при
критическом сопротивлении $R_\text{кр}$ срываются.

\item
Определите $R_\text{кр}$ для токов: $I = 100$~мА и $I = 160$~мА. Увеличивая
сопротивление магазина, следите за постоянством напряжения на генераторе
($U_\text{ЗГ} = 15$~В).

\item
При фиксированных значениях: $I = 160$~мА, $U_\text{ЗГ} = 15$~В, $R_M = 0$~---
проследите, как изменяется фигура Лиссажу при уменьшении частоты от 150 до
50~Гц. Определите резонансную частоту и частоту срыва колебаний.

\item
Для наблюдения петли гистерезиса переключите ключ $K_1$ в положение «Петля».
Снова задайте параметры: $I = 160$~мА, $R_M = 0$, $\nu = 150$~Гц, $U_\text{ЗГ} =
15$~В. Подберите чувствительность осциллографа так, чтобы на экране была видна
петля гистерезиса в удобном масштабе.

При наличии параметрических колебаний петля гистерезиса имеет сложную форму.
Увеличьте сопротивление $R_M$ до критического. В этом случае параметрические
колебания срываются и на экране видна частная петля (на рис.~\figref{hysteresis
loop} она выделена пунктиром).

Чтобы увидеть форму полной петли, уменьшите сопротивление~$R_M$ и ток~$I$ до нуля.
При увеличении напряжения $U_\text{ЗГ}$ до 20--25~В полная петля становится
предельной.

\item
Увеличивая постоянный ток, проследите, как меняется форма петли в момент
возникновения и срыва параметрических колебаний, как перемещается частная петля.

\item
Для тока $I = 160$~мА, $U_\text{ЗГ} = 15$~В определите $R_\text{кр}$, выводя
параметры на самую границу колебаний.

\item
При сопротивлении чуть больше критического сфотографируйте или зарисуйте петлю.
Для этого установите
ручки плавной регулировки усиления по каналам $X$ и $Y$ в крайнее правое
положение (до щелчка), тогда цифры возле дискретных переключателей усиления
задают масштабы изображения $K_X$ и $K_Y$ в~мВ/дел.

Подберите коэффициенты усиления так, чтобы петля занимала практически весь
экран. Зарисуйте на кальку петлю, оси координат, деления шкалы и запишите на ней
рабочие параметры схемы и коэффициенты $K_X$ и $K_Y$.

\item Исследуйте зависимость выходного напряжения параметрона от частоты. Для
этого уменьшите сопротивление магазина до нуля и поставьте ключ $K_2$ в
положение <<$U_\text{вых}$>>. Снимите зависимость напряжения на ёмкости $C$
колебательного контура $U_\text{вых} = f(\nu)$, уменьшая частоту от 150~Гц до
срыва колебаний. Напряжение $U_\text{ЗГ} = 15$~В и ток $I = 160$~мА следует
поддерживать постоянными.

\tasksection{Обработка результатов}

\item
Определите по рисунку петли максимальный и минимальный наклоны касательных
$(\Delta U_Y/ \Delta U_X)$ и рассчитайте величину $\Delta L$ по формуле
\eqref{4.6.9}.

Проверьте справедливость условия \eqref{4.6.3}. Полное сопротивление контура
включает в себя сопротивление магазина и сопротивление параметрона между точками
5 и 7, указанное на установке.

\item
Постройте график $U_\text{вых} = f(\nu)$ и определите по нему резонансную
частоту контура $\nu_0$. Рассчитайте индуктивность контура
и проверьте справедливость условия \eqref{4.6.3} на
этой частоте, полагая $\Delta L \sim L$.

\end{lab:task}


\begin{lab:questions}
\item
Получите условие возбуждения колебаний \eqref{4.6.3}, когда индуктивность
меняется по гармоническому закону: $L = L_0(1-m\sin(2\omega_0t))$. Напишите закон
изменения тока, возбуждаемого в контуре.

\item Чем ограничивается рост амплитуды колебаний системы при параметрическом резонансе?

\item
Почему в нашем случае индуктивность пропорциональна дифференциальной магнитной
проницаемости?

\item
Нарисуйте качественный график зависимости $\mu_\text{диф}$ от величины
подмагничивающего тока для петли гистерезиса, которая изображена на
рис.~\figref{hysteresis loop}.

\item
На каких ещё частотах (в принципе) могут возбуждаться колебания в контуре
параметрона при больших изменениях индуктивности?


\end{lab:questions}


\begin{lab:literature}
\item
\SivuhinIII~--- Гл. III, \S 74; Гл. X, \S\S~122, 123, 127, 135.

\item
\Kalashnikov~--- \S~226.

\item
\KingLokOlh~--- Ч. III, Гл. 3, 3.1.

\item
\textit{Горелик~Г.С.} Колебания и волны. --- М.:~Физматгиз, 1959. Гл. III, \S~9.

\end{lab:literature}
