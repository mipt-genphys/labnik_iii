\lab{Диа- и парамагнетики}

\begin{lab:aim}
	измерение магнитной восприимчивости диа- и парамагнитного образцов.
\end{lab:aim}

\begin{lab:equipment}
	электромагнит, аналитические весы, милливеберметр,  регулируемый источник постоянного тока, образцы.
\end{lab:equipment}

Магнитная восприимчивость тел может быть определена методом измерения сил, которые действуют на тела в~магнитном поле.
Существуют два классических метода таких измерений: \important{метод Фарадея} и \important{метод Гюи.} В~методе Фарадея
исследуемые образцы, имеющие форму маленьких шариков, помещаются в~область сильно неоднородного магнитного поля и
измеряется сила, действующая на образец. При этом для расчёта магнитной восприимчивости необходимо знать величину
градиента магнитного поля в месте расположения образца. В~методе Гюи используется тонкий и длинный стержень, один из
концов которого помещают в~зазор электромагнита (обычно в~область однородного поля), а другой конец~--- вне зазора, где
величиной магнитного поля можно пренебречь. Закон изменения поля~--- от максимального до нулевого~--- в~этом случае
несуществен.

\begin{wrapfigure}{r}{0.4\textwidth}
	\pic{0.38\textwidth}{4_1_1}
	\caption{Расположение образца в зазоре электромагнита}
	\figmark{sample}
\end{wrapfigure}

Найдём выражение для магнитной силы, действующей на такой образец (рис.~\figref{sample}). Пусть площадь образца равна~$s$, его магнитная
проницаемость~---~$\mu$, а поле в~зазоре равно~$B$.

Воспользуемся для расчёта энергетическими соображениями. Магнитная сила может быть вычислена как производная от
магнитной энергии по перемещению. Из теории известно (см.~[1]), что эту производную следует брать со знаком минус, когда
постоянен поток магнитного поля через образец (например, образец в поле постоянного магнита), или со знаком плюс, как в нашем случае, когда поле в зазоре создаётся
электромагнитом, ток $I$ в обмотках которого поддерживается постоянным.

При смещении образца на расстояние~$\Delta l$ вниз магнитная сила, действующая на него, равна
\begin{equation}%1
	\eqmark{force}
	F=\left(\frac{\Delta W_m}{\Delta l}\right)_I,
\end{equation}
где $\Delta W_m$~--- изменение магнитной энергии системы при постоянном токе в обмотке электромагнита и, следовательно, при
постоянной величине магнитного поля в зазоре.

Магнитная энергия рассчитывается по формуле
\begin{equation}%2
	\eqmark{energy}
	W_m=\frac12\int H B\,dV=\frac{1}{2\mu_0}\int\frac{B^2}{\mu}\,dV,
\end{equation}
где интеграл распространён на всё пространство. При смещении образца магнитная энергия меняется только в области зазора
(в~объёме площади~$s$ и высоты~$\Delta l$), а около верхнего конца стержня остаётся неизменной, поскольку магнитного поля там
практически нет. Принимая поле внутри стержня равным измеренному нами полю в зазоре $B$, получим
\begin{equation*}
	\Delta W_m=\frac{1}{2\mu_0}\frac{B^2}{\mu}s\Delta l-\frac{1}{2\mu_0}B^2 s\Delta l=-\frac{\chi}{2\mu_0\mu}B^2s\Delta l.
\end{equation*}

Следовательно, на образец действует сила
\begin{equation}%3
	F=\frac{\Delta W_m}{\Delta l}=-\frac{\chi}{2\mu_0\mu}B^2s.
\end{equation}
Знак силы, действующей на образец, зависит от знака $\chi$: образцы из парамагнитных материалов ($\chi>0$) втягиваются
в~зазор электромагнита, а диамагнитные образцы ($\chi<0$) выталкиваются  из него.

Пренебрегая отличием $\mu$ от единицы, получаем окончательно расчётную формулу в виде
\begin{equation}%4
	\eqmark{final}
	F=-\frac{\chi B^2s}{2\mu_0}.
\end{equation}

В приложении к данной работе приведен вывод выражения \eqref{final}, в котором учитывалось отличия магнитного поля внутри образца от магнитного поле снаружи образца.

Измерив силу, действующую на образец в магнитном поле $B$, можно рассчитать магнитную восприимчивость образца.

\experiment
\begin{figure}[h!]
	\pic{0.9\textwidth}{4_1_2}
	\caption{Схема экспериментальной установки}
	\figmark{setup}
\end{figure}
\todo{На рис.new удалить фразу "GPR-11H30D". На рис.new фразу "Источник пит." заменить на фразу "Источник питания"}

Схема установки изображена на рис.~\figref{setup}. Магнитное поле с максимальной индукцией~${\simeq}1$~Тл создаётся в~зазоре
электромагнита, питаемого постоянным током. Диаметр полюсов существенно превосходит ширину зазора, поэтому поле
в~средней части зазора достаточно однородно. Величина тока, проходящего  через обмотки электромагнита, 
задаётся регулируемым источником постоянного напряжения с цифровым амперметром.

Градуировка электромагнита (связь между индукцией магнитного поля $B$ в~зазоре электромагнита и силой тока~$I$ в~его
обмотках) производится при помощи милливеберметра (описание милливеберметра и правила работы с ним приведены на
с.~\pageref{MWB}).

При измерениях образцы поочерёдно подвешиваются к~аналитическим весам так, что один конец образца оказывается в~зазоре
электромагнита, а другой~--- вне зазора, где индукцией магнитного поля можно пренебречь. При помощи аналитических весов
определяется перегрузка~$\Delta P=F$~--- сила, действующая на образец со стороны магнитного поля.

Как уже отмечалось, силы, действующие на диа- и парамагнитные образцы, очень малы. Небольшие примеси ферромагнетиков
(сотые доли процента железа или никеля) способны кардинально изменить результат опыта, поэтому образцы были специально
отобраны.

\begin{lab:task}

В работе предлагается исследовать зависимость силы, действующей на образец, размещённый в зазоре электромагнита, от
величины поля в~зазоре и по результатам измерений рассчитать магнитную восприимчивость меди и алюминия.

\begin{enumerate}

%\n Перед включением питания магнита убедитесь, что все реостаты полностью введены (такое положение реостатов показано на
%\p{2}).

\item Проверьте работу цепи питания электромагнита. Оцените диапазон изменения тока~$I$ через обмотки.

\item Прокалибруйте электромагнит. Для этого с помощью милливеберметра снимите зависимость магнитного потока $\Phi$,
пронизывающего пробную катушку, находящуюся в~зазоре, от тока~$I$ ($\Phi=BSN$). Значение $SN$ (произведение площади
сечения пробной катушки на число витков в ней) указано на установке.

\begin{lab:warning}
	Включать и отключать электромагнит следует~только~при~минимальном токе.
\end{lab:warning}

\item Убедитесь, что весы арретированы\footnote{Арретир~(фр. arreter~--- фиксировать)~--- приспособление для закрепления
чувствительного элемента измерительного прибора в нерабочем состоянии.}.

\begin{lab:warning}
	Весы следует арретировать перед каждым изменением тока.
\end{lab:warning}

\item \label{item:4} Измерьте силы, действующие на образец в магнитном поле. Для этого, не включая электромагнит, подвесьте к~весам
один из образцов. Установите на весах примерное значение массы образца (масса, диаметр и максимальное значение
перегрузки для каждого образца указаны на установке). Освободите весы и добейтесь точного равновесия весов.

Арретируйте весы. Установите минимальное значение тока и проведите измерение равновесного значения массы.

Повторите измерения $\Delta P = f(I)$ для 6~--~8 других значений тока.

\item Повторите измерения п.~\ref{item:4} для другого образца.

\end{enumerate}

\tasksection{Обработка}

\begin{enumerate}
	\item Рассчитайте поле $B$ и постройте градуировочную кривую для электромагнита: $B=f(I)$.
	\item Постройте на одном листе графики $|\Delta P|=f(B^2$) для меди и алюминия.
	\item По наклонам полученных прямых рассчитайте величину~$\chi$ с~помощью формулы \eqref{final}.
	\item Оцените погрешности измерений и сравните результаты с табличными значениями.
\end{enumerate}

\end{lab:task}


\begin{lab:questions}
	\item Объясните суть метода измерения магнитной восприимчивости.
	\item Напишите выражения для магнитной силы, действующей на~образец, помещённый в~неоднородное магнитное поле.
	\item Как можно убедиться в~однородности или неоднородности магнитного поля в~зазоре электромагнита?
	\item Как проверить экспериментально, влияет ли намагниченность весов на результаты измерения магнитной восприимчивости?
	\item Пусть у вас используется образец в виде тонкой и длинной полосы, находящейся между полосами магнита. В первом случае плоскость образца перпендикулярна линиям магнитной индукции, во втором – параллельна. Будет ли действующая на образец сила отличаться в этих двух случаях?
\end{lab:questions}


\begin{lab:literature}
	\item \emph{Сивухин Д.В.} Общий курс физики. Т.~III. Электричество.~--- М.: Физматлит 2003 \S\S~61, 75--77.
	\item \emph {Калашников С.Г.} Электричество.~--- М.: Физматлит. Гл.~ХI, \S\S~109, 117, 118.
	\item \emph{Кингсеп А.С., Локшин Г.Р., Ольхов О.А.} Основы Физики. Т.~1. Механика, электричество и магнетизм, колебания и волны, волновая оптика.~--- М.:~Физматлит, 2001. Ч.~II, гл.~5, \S~5.2.
\end{lab:literature}

\labsection{Приложение: Измерение магнитной восприимчивости диамагнетиков и парамагнетиков}

Магнитная восприимчивость тел может быть определена методом измерения сил, которые действуют на тела в магнитном поле. Существуют два классических метода таких измерений: метод Фарадея и метод Гюи. В методе Фарадея исследуемые образцы, имеющие форму маленьких шариков, помещаются в область сильно неоднородного магнитного поля и измеряется сила, действующая на образец. При этом для расчёта магнитной восприимчивости необходимо знать величину градиента магнитного поля в месте расположения образца. В методе Гюи используется тонкий и длинный стержень, один из концов которого помещают в зазор электромагнита (обычно в область однородного поля), а другой конец~--- вне зазора, где величиной магнитного поля можно пренебречь. Закон изменения поля~--- от максимального до нулевого~--- в этом случае несуществен.

Для геометрии нашего эксперимента детальный расчёт магнитного поля при наличии в зазоре стержня достаточно сложен. Те или иные приближения в расчёте могут привести к значительным погрешностям в определении изменения энергии системы при виртуальном перемещении стержня и соответственно в значении действующей на стержень силы.

С другой стороны, поскольку отличие $B$ от $\mu_0 H$ (определяемое величиной $\chi$) для всех изучаемых нами образцов не превышает $0,1\%$, поля........

\todo [author=Tiffani]{часть текста после ``для всех изучаемых нами образцов не превышает $0,1\%$, поля''отсутствует}

Найдём выражение для магнитной силы, действующей на тонкий цилиндрический стержень, расположенный между полюсами электромагнита (рис.~\figref{rod in electromagnet}). Пусть площадь поперечного сечения образца равна $s$, его магнитная восприимчивость~--- $\mu$, а поле в зазоре равно $H$.

\begin{wrapfigure}[13]{r}{0.4\textwidth}
	\pic{0.38\textwidth}{v4_7}
	\caption{Расположение образца в зазоре электромагнита}
	\figmark{rod in electromagnet}
\end{wrapfigure}

Воспользуемся общим выражением для силы, действующей на магнитный диполь с магнитным моментом $m$ во внешнем поле:

\begin{equation*}
	F = (m\nabla)B.
\end{equation*}

Нас интересует магнитная сила, действующая на образец вдоль оси $z$:

\begin{equation*}
	F_z = m_x \frac{dB_z}{dx} + m_y \frac{dB_z}{dy} + m_z \frac{dB_z}{dz}.
\end{equation*}

Выберем бесконечно малый объём стержня $dV = sdz$, где $dz$~--- малый элемент длины цилиндра на произвольной высоте $z$. Магнитный момент такого элемента объёма $dm_y = \chi H_y s dz$. Поскольку $dm_x = dm_z = 0$, то магнитная сила равна

\begin{equation*}
	dF_z = \chi H_y s \frac{dB_z}{dy} dz.
\end{equation*}

Так как в образце отсутствуют токи проводимости и токи смещения, то $rot H = 0$, а

\begin{equation*}
	\frac{dB_z}{dy} = \frac{dB_y}{dz}.
\end{equation*}

После замены производной в выражении для $dFz$ окончательно получим

\begin{equation}
	\eqmark{magnetic force-z}
	F_z = \int\limits_B^0 \frac{\chi\cdot s}{2\mu \mu_0}d\left( B_y^2 \right) = - \frac{\chi}{2\mu \mu_0} sB^2.
\end{equation}

Если $\chi > 0$ (парамагнетик)~--- стержень втягивается в зазор, если меньше (диамагнетик)~--- выталкивается из него.

По смыслу вывода $B$ в формуле \eqref{magnetic force-z}~--- поле в образце. Если приравнять его измеренному нами полю в зазоре, можно пользоваться \eqref{magnetic force-z} в качестве расчётной формулы.

Полагая равными в стержне и в зазоре векторы $H$, придём к соотношению

\begin{equation}
	\eqmark{magnetic force-magfield inside gap}
	F = \frac{\chi}{2\mu_0} sB^2.
\end{equation}

Напомним, что при переходе через границу раздела сред сохраняются нормальная составляющая вектора $B$ и тангенциальная составляющая вектора $H$. Поэтому точная величина силы лежит где-то между значениями, определяемыми формулами \eqref{magnetic force-z} и \eqref{magnetic force-magfield inside gap}, отличие между которыми лежит за пределами точности эксперимента.

Можно привести такие соображения: поскольку стержень длинный и коэффициент размагничивания для такой геометрии равен $1/2$,  получим значения $B$ и $H$ внутри образца

\begin{equation*}
	H_{\text{внутр}} = \frac{2B_0}{\mu_0 (1 + \mu)},\qquad	B_{\text{внутр}} = \frac{2\mu}{1 + \mu} B_0,
\end{equation*}
что также показывает справедливость \eqref{magnetic force-magfield inside gap} для $\mu \approx 1$.

Формулы \eqref{magnetic force-z} и \eqref{magnetic force-magfield inside gap} совпадают, если пренебречь отличием $\mu$ от единицы. Поэтому в качестве окончательной принимаем формулу \eqref{magnetic force-magfield inside gap}. Эта формула может быть получена также из энергетических соображений (см. работу 3.4.1).

Подчеркнём ещё раз, что все эти приближения справедливы только для случая $|\chi| \ll 1$.

\begin{lab:literature}
	\item~\emph{Сивухин~Д.В.} Курс общей физики.~Т.III. Электричество --- М.:~Наука, 1983. Гл 3, \S\S~74--79.
	\item~\emph{Калашников~С.Г.} Электричество.--- М.: Наука, 1977, Гл. 11.
	\item~\emph{Кингсеп~А.С., Локшин~Г.Р., Ольхов~О.А.} Основы физики.~Т.I.--- М.:~Физматлит, 2001. Ч 2, Гл V, \S\S~5.2, 5.3.
	\item~\emph{под ред. Овчинкина~В.А.} Сборник задач по общему курсу физики.~Ч. 2. Электричество и магнетизм. Оптика~--- М.:~Физматкнига, 2004.
\end{lab:literature}

\todo [author=Tiffani]{Как лучше оформить приложение?}
\labsection{Приложение. Решение задачи: стержень в магнитном поле.}

Пусть намагниченность $I = const$ (верно для длинного цилиндрического цилиндра, эквивалентного вытянутому эллипсоиду вращения).
\todo [author=Tiffani]{А цилиндр может быть нецилиндрическим?}
Тогда
\begin{equation*}
	F_{\text{внешн}} \cdot \Delta x = - \frac{1}{2} \frac{I}{c} \Delta \Phi,
\end{equation*}
где $\Delta \Phi$~--- изменение потока от нашего стержня через обмотку электромагнита.

По теореме взаимности
\begin{equation*}
	B\Delta x \cdot h = \frac{4\pi}{c} n \frac{I_{\text{магн}}}{c} \cdot \Delta x \cdot h = \frac{1}{c}HI_{\text{магн}}.
\end{equation*}

\begin{equation*}
	M = 4\pi n \Delta x P_z,
\end{equation*}
где $P_z = h I_{\text{стерж}}/c$~--- дипольный момент единицы длины стержня.
Размерность $\left[ \frac{I}{c}s \right] =$~Гс/см$^3$; $\left[ \frac{I}{c}h \right] =$~Гс/см$^4$.

Поле двумерного диполя
\begin{equation*}
	\vec B_z = \frac{4(\vec p_z \cdot \vec r) \vec r}{r^4} - \frac{2\vec p_z}{r^2}.
\end{equation*}

Граничные условия для однородно намагниченной области стержня

\begin{equation*}
	\left[ \vec B + \frac{4(\vec p_z \cdot \vec R) \vec R}{R^4} - \frac{2\vec p_z}{R^2}, \vec R \right] = \left[ \vec B - \frac{2\vec p_z}{R^2}, \vec R \right] = \left[ \frac{\vec B'}{\mu}, \vec R \right].
\end{equation*}

\begin{equation*}
	\left( \vec B + \frac{4(\vec p_z \cdot \vec R)}{R^4} - \frac{2\vec p_z}{R^2}, \vec R \right) = \left( \vec B + \frac{2\vec p_z}{R^2}, \vec R \right) = \left( \vec B', \vec R \right).
\end{equation*}
Отсюда
\begin{equation*}
	\begin{cases}
		\vec B + \frac{2\vec p_z}{R^2} = B'\\
		\vec B - \frac{2\vec p_z}{R^2} = \frac{\vec B'}{\mu}\\
	\end{cases}
	\Rightarrow \vec B' = \frac{2\mu}{\mu + 1} \vec B;~\vec p_z = \frac{\mu - 1}{\mu + 1} \vec B \frac{R^2}{2}.
\end{equation*}
И далее 
\begin{equation*}
	\frac{I}{c}\Delta \Phi = \frac{I}{c} 4\pi n \cdot \frac{\mu - 1}{\mu + 1} \vec B \frac{R^2}{2} \Delta x = \frac{\mu - 1}{\mu + 1} B^2 \frac{R^2}{2} \Delta x.
\end{equation*}
И конечный результат
\begin{equation*}
	F_{\text{внешняя}} = - \frac{1}{4} \frac{\mu - 1}{\mu + 1} R^2 B^2 = - \frac{1}{4\pi} \frac{\mu - 1}{\mu + 1} B^2 \pi R^2.
\end{equation*}

Для пара- и диамагнетиков
\begin{equation*}
F_{\text{внешняя}} \approx - \frac{1}{8\pi} \chi B^2 s,
\end{equation*}
т.е. та же величина, что и для приближённого расчёта.



