\lab{Диа- и парамагнетики}

\begin{lab:aim}
	измерение магнитной восприимчивости диа- и парамагнитного образцов.
\end{lab:aim}

\begin{lab:equipment}
	электромагнит, аналитические весы, милливеберметр,  регулируемый источник
постоянного тока, образцы.
\end{lab:equipment}

Перед выполнением работы рекомендуется ознакомиться с теоретическим
введением к разделу, пп.~\ref{sec:diamagnetism}, \ref{sec:paramagnetism} и
\ref{sec:forces}.

Магнитная восприимчивость тел может быть определена по измерению сил,
действующих на тела в~магнитном поле. Существуют два классических метода
таких измерений: \emph{метод Фарадея} и \emph{метод Гюи}. В~методе
Фарадея исследуемые образцы, имеющие форму маленьких шариков, помещаются в~область
сильно неоднородного магнитного поля и измеряется сила, действующая на образец.
При этом для расчёта магнитной восприимчивости необходимо знать величину
градиента магнитного поля в месте расположения образца. В~методе Гюи
используется тонкий и длинный стержень, один из концов которого помещают
в~зазор электромагнита (обычно в~область однородного поля), а другой конец~---
вне зазора, где величиной магнитного поля можно пренебречь. Закон изменения поля~---
от максимального до нулевого~--- в~этом случае несуществен.
В данной работе предлагается использовать метод Гюи.

\begin{wrapfigure}{r}{0.4\textwidth}
	\pic{0.38\textwidth}{Chapter_4/4_1_1}
	\caption{Расположение образца в зазоре электромагнита}
	\figmark{sample}
\end{wrapfigure}

Найдём выражение для силы силы, действующей со стороны магнитного поля
на цилиндрический стержень, помещённый в зазор электромагнита (рис.~\figref{sample}).
Пусть площадь сечения образца равна~$S$, его магнитная
проницаемость~---~$\mu$, поле в~зазоре равно~$B_0$ и образец помещён
в зазор на глубину $x$.

Ток в обмотке электромагнита $I$ поддерживается постоянным, поэтому
согласно согласно \chaptereqref{force-I} внешняя сила,
необходимая для удержания образца в магнитном поле,
равна производной магнитной энергии системы по координате.
Нас интересует сила, действующая
на образец \emph{со стороны магнитного поля},
поэтому изменим знак \chaptereqref{force-I} на противоположный:
\begin{equation}%1
	\eqmark{force}
	F_М=\left(\frac{\partial W_М}{\partial x}\right)_I,
\end{equation}
где $W_М(x)$~--- магнитная энергия системы при $I=\const$ (то есть при
$B_0=\const$) в зависимости от смещения образца $x$.

Магнитная энергия, с учётом выражения \chaptereqref{magnetic-energy-simple}
для её объёмной плотности, может быть рассчитана по формуле:
\begin{equation}%2
	\eqmark{energy}
	W_М=\frac{1}{2\mu_0}\int\frac{B^2}{\mu}\,dV,
\end{equation}
где интеграл распространён на всё пространство.

% Это неправильное объяснение! Поле у торца образца мы знать не можем!
%
% При смещении образца магнитная энергия меняется только в области зазора
% (в~объёме площади~$s$ и высоты~$\Delta l$), а около верхнего конца стержня
% остаётся неизменной, поскольку магнитного поля там
% практически нет. Принимая поле внутри стержня равным измеренному нами полю в
% зазоре $B$, получим
% \begin{equation*}
% 	\Delta W_m=\frac{1}{2\mu_0}\frac{B^2}{\mu}s\Delta l-\frac{1}{2\mu_0}B^2
% s\Delta l=-\frac{\chi}{2\mu_0\mu}B^2s\Delta l.
% \end{equation*}

\begin{wrapfigure}{r}{0.25\textwidth}
    \import{Images/Chapter_4/}{4-1-rod.pdf_tex}
    \caption{К вычислению распределения поля в образце}
    \figmark{samplex}
\end{wrapfigure}

Найдём распределение магнитного поля в длинном цилиндре, частично
помещённом зазор электромагнита.

\todo{Разделы во введении должны быть нумерованными, чтобы можно было давать
на них ссылки}
Сперва решим вспомогательную задачу:
рассмотрим бесконечный стержень с проницаемостью~$\mu$,
помещённый в перпендикулярное ему однородное магнитное поле $B_0=\mu_0 H_0$,
и найдём поле $B_{ст}$ в образце.
Эта задача имеет точное решение (см. Приложение), однако поскольку
магнитная восприимчивость диа- и парамагнетиков мала $|\chi|\ll1 $ ($\mu\approx 1$),
можно воспользоваться непрерывностью
касательной компоненты $H$ и считать, что в образце $H_{ст}=H_0$ и, следовательно,
$B_{ст} = \mu B_0$.



Вернёмся к задаче о цилиндре в электромагните.
Систему можно условно разбить на 3 части
(см. рис.~\figref{samplex}). В области~I вне электромагнита поле мало $B_{1}\approx 0$
и его вкладом в энергию можно пренебречь. В части стержня~II, погружённой в электромагнит,
поле приближенно равно $B_{2}\approx \mu B_0$.
В области~III вдали от стержня поле мало отличается от $B_3=B_0$.
Наконец, в пограничных областях между I и II и между II и III (отмечены пунктиром)
распределение поля простыми методами рассчитано быть не может.

При смещении стержня вглубь электромагнита на некоторое расстояние $\Delta x$
область~II увеличивается в объёме на $\Delta V_{2}=S\Delta x$, а область III уменьшается
на $\Delta V_{3}=-S\Delta x$.
При этом пограничный участок II--III смещается, но распределение поля в нём
практически не меняется. Изменение области I можно не учитывать, так как там нет поля,
а пограничный участок I--II остаётся практически неизменным.
Пользуясь \eqref{energy}, найдём изменение магнитной энергии
при заданном смещении:
\[
\Delta W_М(\Delta x) \approx \frac{B_{2}^2}{2\mu\mu_0} S\Delta x -
\frac{B_{3}^2}{2\mu_0} S\Delta x = (\mu-1)\frac{B_0^2}{2\mu_0} S\Delta x.
\]
Следовательно, искомая сила равна
\begin{equation}%3
    \eqmark{final}
F_М=\left(\frac{\partial W_М}{\partial x}\right)_{B_0}\approx (\mu-1)\frac{B_0^2}{2\mu_0} S
\end{equation}
Знак силы зависит от знака $\chi=\mu-1$: парамагнетики
($\chi>0$) \emph{втягиваются}
в~зазор электромагнита, а диамагнетики ($\chi<0$) \emph{выталкиваются} из него
(напомним, что положительным по $x$ мы считаем направление вглубь зазора).
Таким образом, измерив силу, действующую на образец в магнитном поле $B_0$,
можно рассчитать магнитную восприимчивость образца.

% Пренебрегая отличием $\mu$ от единицы, получаем окончательно расчётную формулу в
% виде
% \begin{equation}%4
% 	\eqmark{final}
% 	F=-\frac{\chi B^2s}{2\mu_0}.
% \end{equation}

% В приложении к данной работе приведен вывод выражения \eqref{final}, в котором
% учитывалось отличия магнитного поля внутри образца от магнитного поле снаружи
% образца.

\experiment
\begin{figure}[h!]
	\pic{0.9\textwidth}{Chapter_4/4_1_2}
	\caption{Схема экспериментальной установки}
	\figmark{setup}
\end{figure}
\todo{На рис.new удалить фразу "GPR-11H30D". На рис.new фразу "Источник пит."
заменить на фразу "Источник питания"}

Схема установки изображена на рис.~\figref{setup}. Магнитное поле с максимальной
индукцией~${\simeq}1$~Тл создаётся в~зазоре
электромагнита, питаемого постоянным током. Диаметр полюсов существенно
превосходит ширину зазора, поэтому поле
в~средней части зазора достаточно однородно. Величина тока, проходящего  через
обмотки электромагнита,
задаётся регулируемым источником постоянного напряжения.
% с цифровым амперметром.

Градуировка электромагнита (связь между индукцией магнитного поля $B$ в~зазоре
электромагнита и силой тока~$I$ в~его
обмотках) производится при помощи милливеберметра (описание милливеберметра и
правила работы с ним приведены на
с.~\pageref{MWB}).
Альтернативно магнитное поле электромагнита можно измерить
с помощью датчика Холла.

При измерениях образцы поочерёдно подвешиваются к~аналитическим весам так, что
один конец образца оказывается в~зазоре электромагнита, а другой~--- вне зазора,
где индукцией магнитного поля можно пренебречь. При помощи аналитических весов
определяется перегрузка~$\Delta P=F$~--- сила, действующая на образец со стороны
магнитного поля.

Как уже отмечалось, силы, действующие на диа- и парамагнитные образцы, очень
малы. Небольшие примеси ферромагнетиков (сотые доли процента железа или никеля)
способны кардинально изменить результат опыта, поэтому образцы были специально
отобраны.

\begin{lab:task}

В работе предлагается исследовать зависимость силы, действующей на образец,
размещённый в зазоре электромагнита, от
величины поля в~зазоре и по результатам измерений рассчитать магнитную
восприимчивость меди и алюминия.

\begin{enumerate}

%\n Перед включением питания магнита убедитесь, что все реостаты полностью
% введены (такое положение реостатов показано на
%\p{2}).

\item Проверьте работу цепи питания электромагнита. Оцените диапазон изменения
тока~$I$ через обмотки.

\item Прокалибруйте электромагнит. Для этого с помощью милливеберметра снимите
зависимость магнитного потока $\Phi$,
пронизывающего пробную катушку, находящуюся в~зазоре, от тока~$I$ ($\Phi=BSN$).
Значение $SN$ (произведение площади
сечения пробной катушки на число витков в ней) указано на установке.

\begin{lab:warning}
	Включать и отключать электромагнит следует~только~при~минимальном токе.
\end{lab:warning}

\item При работе с механическими весами убедитесь,
что весы арретированы\footnote{Арретир~(фр. arreter~---
фиксировать)~--- приспособление для закрепления
чувствительного элемента измерительного прибора в нерабочем состоянии.}.

\begin{lab:warning}
	Механические весы следует арретировать перед каждым изменением тока.
\end{lab:warning}

\item \label{item:4} Измерьте силы, действующие на образец в магнитном поле. Для
этого, не включая электромагнит, подвесьте к~весам
один из образцов. Установите на весах примерное значение массы образца (масса,
диаметр и максимальное значение
перегрузки для каждого образца указаны на установке). Освободите весы и
добейтесь точного равновесия весов.

Арретируйте весы. Установите минимальное значение тока и проведите измерение
равновесного значения массы.

Повторите измерения $\Delta P(I)$ для 6--8 других значений тока.

\item Повторите измерения п.~\ref{item:4} для другого образца.

\end{enumerate}

\tasksection{Обработка}

\begin{enumerate}
	\item Рассчитайте поле $B$ и постройте градуировочную кривую для
электромагнита: $B(I)$.
	\item Постройте на одном листе графики $|\Delta P|=f(B^2$) для меди и
алюминия.
	\item По наклонам полученных прямых рассчитайте магнитную восприимчивость~$\chi$
    с~помощью формулы \eqref{final}.
	\item Оцените погрешности измерений и сравните результаты с табличными
значениями.
\end{enumerate}

\end{lab:task}


\begin{lab:questions}
	\item Объясните суть метода измерения магнитной восприимчивости.
	\item Напишите выражения для магнитной силы, действующей на~образец,
помещённый в~неоднородное магнитное поле.
	\item Как можно убедиться в~однородности или неоднородности магнитного поля
в~зазоре электромагнита?
	\item Как проверить экспериментально, влияет ли намагниченность весов на
результаты измерения магнитной восприимчивости?
	\item Пусть в качестве образцов используется тонкая и длинная полоса.
    В первом случае плоскость образца перпендикулярна линиям магнитной индукции,
    во втором --- параллельна. Будет ли действующая на образец сила
    отличаться в этих двух случаях?
\end{lab:questions}


\begin{lab:literature}
	\item \emph{Сивухин Д.В.} Общий курс физики. Т.~III. Электричество.~--- М.:
Физматлит 2003 \S\S~61, 75--77.
	\item \emph {Калашников С.Г.} Электричество.~--- М.: Физматлит. Гл.~ХI,
\S\S~109, 117, 118.
	\item \emph{Кингсеп А.С., Локшин Г.Р., Ольхов О.А.} Основы Физики. Т.~1.
Механика, электричество и магнетизм, колебания и волны, волновая оптика.~---
М.:~Физматлит, 2001. Ч.~II, гл.~5, \S~5.2.
\end{lab:literature}

% Не нужно!
%
% \labsection{Приложение: Измерение магнитной восприимчивости диамагнетиков и
% парамагнетиков}
%
% Магнитная восприимчивость тел может быть определена методом измерения сил,
% которые действуют на тела в магнитном поле. Существуют два классических метода
% таких измерений: метод Фарадея и метод Гюи. В методе Фарадея исследуемые
% образцы, имеющие форму маленьких шариков, помещаются в область сильно
% неоднородного магнитного поля и измеряется сила, действующая на образец. При
% этом для расчёта магнитной восприимчивости необходимо знать величину градиента
% магнитного поля в месте расположения образца. В методе Гюи используется тонкий и
% длинный стержень, один из концов которого помещают в зазор электромагнита
% (обычно в область однородного поля), а другой конец~--- вне зазора, где
% величиной магнитного поля можно пренебречь. Закон изменения поля~--- от
% максимального до нулевого~--- в этом случае несуществен.
%
% Для геометрии нашего эксперимента детальный расчёт магнитного поля при наличии в
% зазоре стержня достаточно сложен. Те или иные приближения в расчёте могут
% привести к значительным погрешностям в определении изменения энергии системы при
% виртуальном перемещении стержня и соответственно в значении действующей на
% стержень силы.
%
% С другой стороны, поскольку отличие $B$ от $\mu_0 H$ (определяемое величиной
% $\chi$) для всех изучаемых нами образцов не превышает $0,1\%$, поля........
%
% \todo [author=Tiffani]{часть текста после ``для всех изучаемых нами образцов не
% превышает $0,1\%$, поля''отсутствует}
%
% Найдём выражение для магнитной силы, действующей на тонкий цилиндрический
% стержень, расположенный между полюсами электромагнита (рис.~\figref{rod in
% electromagnet}). Пусть площадь поперечного сечения образца равна $s$, его
% магнитная восприимчивость~--- $\mu$, а поле в зазоре равно $H$.
%
% \begin{wrapfigure}[13]{r}{0.4\textwidth}
% 	\pic{0.38\textwidth}{v4_7}
% 	\caption{Расположение образца в зазоре электромагнита}
% 	\figmark{rod in electromagnet}
% \end{wrapfigure}
%
% Воспользуемся общим выражением для силы, действующей на магнитный диполь с
% магнитным моментом $m$ во внешнем поле:
%
% \begin{equation*}
% 	F = (m\nabla)B.
% \end{equation*}
%
% Нас интересует магнитная сила, действующая на образец вдоль оси $z$:
%
% \begin{equation*}
% 	F_z = m_x \frac{dB_z}{dx} + m_y \frac{dB_z}{dy} + m_z \frac{dB_z}{dz}.
% \end{equation*}
%
% Выберем бесконечно малый объём стержня $dV = sdz$, где $dz$~--- малый элемент
% длины цилиндра на произвольной высоте $z$. Магнитный момент такого элемента
% объёма $dm_y = \chi H_y s dz$. Поскольку $dm_x = dm_z = 0$, то магнитная сила
% равна
%
% \begin{equation*}
% 	dF_z = \chi H_y s \frac{dB_z}{dy} dz.
% \end{equation*}
%
% Так как в образце отсутствуют токи проводимости и токи смещения, то $rot H = 0$,
% а
%
% \begin{equation*}
% 	\frac{dB_z}{dy} = \frac{dB_y}{dz}.
% \end{equation*}
%
% После замены производной в выражении для $dFz$ окончательно получим
%
% \begin{equation}
% 	\eqmark{magnetic force-z}
% 	F_z = \int\limits_B^0 \frac{\chi\cdot s}{2\mu \mu_0}d\left( B_y^2 \right) =
% - \frac{\chi}{2\mu \mu_0} sB^2.
% \end{equation}
%
% Если $\chi > 0$ (парамагнетик)~--- стержень втягивается в зазор, если меньше
% (диамагнетик)~--- выталкивается из него.
%
% По смыслу вывода $B$ в формуле \eqref{magnetic force-z}~--- поле в образце. Если
% приравнять его измеренному нами полю в зазоре, можно пользоваться
% \eqref{magnetic force-z} в качестве расчётной формулы.
%
% Полагая равными в стержне и в зазоре векторы $H$, придём к соотношению
%
% \begin{equation}
% 	\eqmark{magnetic force-magfield inside gap}
% 	F = \frac{\chi}{2\mu_0} sB^2.
% \end{equation}
%
% Напомним, что при переходе через границу раздела сред сохраняются нормальная
% составляющая вектора $B$ и тангенциальная составляющая вектора $H$. Поэтому
% точная величина силы лежит где-то между значениями, определяемыми формулами
% \eqref{magnetic force-z} и \eqref{magnetic force-magfield inside gap}, отличие
% между которыми лежит за пределами точности эксперимента.
%
% Можно привести такие соображения: поскольку стержень длинный и коэффициент
% размагничивания для такой геометрии равен $1/2$,  получим значения $B$ и $H$
% внутри образца
%
% \begin{equation*}
% 	H_{\text{внутр}} = \frac{2B_0}{\mu_0 (1 + \mu)},\qquad	B_{\text{внутр}} =
% \frac{2\mu}{1 + \mu} B_0,
% \end{equation*}
% что также показывает справедливость \eqref{magnetic force-magfield inside gap}
% для $\mu \approx 1$.
%
% Формулы \eqref{magnetic force-z} и \eqref{magnetic force-magfield inside gap}
% совпадают, если пренебречь отличием $\mu$ от единицы. Поэтому в качестве
% окончательной принимаем формулу \eqref{magnetic force-magfield inside gap}. Эта
% формула может быть получена также из энергетических соображений (см. работу
% 3.4.1).
%
% Подчеркнём ещё раз, что все эти приближения справедливы только для случая
% $|\chi| \ll 1$.
%
% \begin{lab:literature}
% 	\item~\emph{Сивухин~Д.В.} Курс общей физики.~Т.III. Электричество ---
% М.:~Наука, 1983. Гл 3, \S\S~74--79.
% 	\item~\emph{Калашников~С.Г.} Электричество.--- М.: Наука, 1977, Гл. 11.
% 	\item~\emph{Кингсеп~А.С., Локшин~Г.Р., Ольхов~О.А.} Основы физики.~Т.I.---
% М.:~Физматлит, 2001. Ч 2, Гл V, \S\S~5.2, 5.3.
% 	\item~\emph{под ред. Овчинкина~В.А.} Сборник задач по общему курсу
% физики.~Ч. 2. Электричество и магнетизм. Оптика~--- М.:~Физматкнига, 2004.
% \end{lab:literature}


\labsection{Приложение. Стержень в поперечном магнитном поле.}

% Не нужно

% Пусть намагниченность $I = const$ (верно для длинного цилиндрического цилиндра,
% эквивалентного вытянутому эллипсоиду вращения).
% \todo [author=Tiffani]{А цилиндр может быть нецилиндрическим?}
% Тогда
% \begin{equation*}
% 	F_{\text{внешн}} \cdot \Delta x = - \frac{1}{2} \frac{I}{c} \Delta \Phi,
% \end{equation*}
% где $\Delta \Phi$~--- изменение потока от нашего стержня через обмотку
% электромагнита.
%
% По теореме взаимности
% \begin{equation*}
% 	B\Delta x \cdot h = \frac{4\pi}{c} n \frac{I_{\text{магн}}}{c} \cdot \Delta
% x \cdot h = \frac{1}{c}HI_{\text{магн}}.
% \end{equation*}
%
% \begin{equation*}
% 	M = 4\pi n \Delta x P_z,
% \end{equation*}
% где $P_z = h I_{\text{стерж}}/c$~--- дипольный момент единицы длины стержня.
% Размерность $\left[ \frac{I}{c}s \right] =$~Гс/см$^3$; $\left[ \frac{I}{c}h
% \right] =$~Гс/см$^4$.

Найдём распределение магнитного поля внутри бесконечно длинного стержня
с сечением $S=\pi R^2$,
помещённого перпендикулярное ему однородное внешнее поле $\vec{H}_0$.
Будем искать решение, предполагая, что намагниченность стержня будет
постоянной (однородной) и сонаправленной ему, $\vec M=\const$. Пусть
$\vec M=\chi' \vec{H}_0$, где
$\chi'=\const$ --- некоторая безразмерная константа, имеющая смысл эффективной
магнитной восприимчивости с учётом формы образца.

С учётом сделанного предположения поле вне стержня представляет собой сумму
внешнего поля $\vec{B}_0=\mu_0 H_0=\const$
и поля двумерного диполя с дипольным моментом $\vec{p} = \vec{M} S$
(на единицу длины стержня).
Воспользуемся известным выражением для поля двумерного диполя:
\begin{equation*}
    \vec H_{дип} = \frac{1}{4\pi}
    \left(\frac{4(\vec p \cdot \vec r) \vec r}{r^4} -
    \frac{2\vec p}{r^2}\right).
\end{equation*}

\begin{wrapfigure}{r}{0.3\textwidth}
    \import{Images/Chapter_4/}{4-1-exact.pdf_tex}
    \caption{Намагниченность цилиндра во внешнем поле}
    \figmark{cylinder-exact}
\end{wrapfigure}

Воспользуемся граничными условиями для магнитного поля.
В каждой точке на границе стержня должна быть непрерывна нормальная
компонента поля $B$ (а также тангенциальная для $H$). Причём всюду внутри стержня
$\vec M = \chi \vec H$ и, следовательно,
\[
\vec B=\mu_0(\vec H+ \vec M)=\frac{\chi+1}{\chi} \mu_0\vec{M}.
\]
Рассмотрим, например, точку A (см. рис.~\figref{cylinder-exact}), для которой
$\vec{r}_A = \frac{\vec{p}}{p} r$. Вклад в поле от намагниченности стержня
в этой точке равен
\[
\vec B_{дип} = \frac{2\mu_0 \vec p}{4\pi r^2} = \frac12 \mu_0 \vec{M} = \frac12 \chi'\vec{B}_0.
\]
Снаружи стержня имеем
$B^{\rm (out)}_n = B_0 + B_{дип}(\vec r_A) = (1+\frac12 \chi') B_0$,
внутри стержня ${B}^{(\rm in)}_n = \frac{\mu\mu_0}{\mu-1} M =
\frac{\chi+1}{\chi} \chi' B_0$. Приравнивая, находим
\[
\chi' = \frac{2\chi}{2+\chi} = 2\frac{\mu-1}{\mu+1}.
\]
Предоставляем читателю возможность самостоятельно убедиться,
что соответствующие граничные условия будут удовлетворены во всех точках
на границе стержня. В силу теоремы единственности, найденное распределение
намагниченности и будет решением поставленной задачи.

Итак, во внешнем поле с напряжённостью $\vec H_0$ длинный стержень,
расположенный перпендикулярно полю, приобретает намагниченность
\[
\vec M = \frac{\chi}{1+\frac12 \chi} \vec{H_0}.
\]
Таким образом, мы показали, что \emph{размагничивающий фактор} в данной задаче
равен $N_{разм}=1/2$.

Если восприимчивость материала стержня мала, $|\chi|\ll 1$, то окончательно
имеем
\[
\vec M \approx \chi \vec{H}_0.
\]
Иными словами, в данной ситуации можно пренебречь отличием напряжённости
внешнего поля $\vec H_0$ и поля в образце $\vec H$. Для стержня конечных
размеров разность $H_0 -H$ может оказаться отличной от нуля ввиду краевых
эффектов, которыми однако можно пренебречь, если длина стержня велика по сравнению
с радиусом.


% Крокодилы должны быть отстрелены!

%
% \begin{equation*}
% 	\left[ \vec B + \frac{4(\vec p_z \cdot \vec R) \vec R}{R^4} - \frac{2\vec
% p_z}{R^2}, \vec R \right] = \left[ \vec B - \frac{2\vec p_z}{R^2}, \vec R
% \right] = \left[ \frac{\vec B'}{\mu}, \vec R \right].
% \end{equation*}
%
% \begin{equation*}
% 	\left( \vec B + \frac{4(\vec p_z \cdot \vec R)}{R^4} - \frac{2\vec
% p_z}{R^2}, \vec R \right) = \left( \vec B + \frac{2\vec p_z}{R^2}, \vec R
% \right) = \left( \vec B', \vec R \right).
% \end{equation*}
% Отсюда
% \begin{equation*}
% 	\begin{cases}
% 		\vec B + \frac{2\vec p_z}{R^2} = B'\\
% 		\vec B - \frac{2\vec p_z}{R^2} = \frac{\vec B'}{\mu}\\
% 	\end{cases}
% 	\Rightarrow \vec B' = \frac{2\mu}{\mu + 1} \vec B;~\vec p_z = \frac{\mu -
% 1}{\mu + 1} \vec B \frac{R^2}{2}.
% \end{equation*}
% И далее
% \begin{equation*}
% 	\frac{I}{c}\Delta \Phi = \frac{I}{c} 4\pi n \cdot \frac{\mu - 1}{\mu + 1}
% \vec B \frac{R^2}{2} \Delta x = \frac{\mu - 1}{\mu + 1} B^2 \frac{R^2}{2} \Delta
% x.
% \end{equation*}
% И конечный результат
% \begin{equation*}
% 	F_{\text{внешняя}} = - \frac{1}{4} \frac{\mu - 1}{\mu + 1} R^2 B^2 = -
% \frac{1}{4\pi} \frac{\mu - 1}{\mu + 1} B^2 \pi R^2.
% \end{equation*}
%
% Для пара- и диамагнетиков
% \begin{equation*}
% F_{\text{внешняя}} \approx - \frac{1}{8\pi} \chi B^2 s,
% \end{equation*}
% т.е. та же величина, что и для приближённого расчёта.



