\lab{3.4.1}{Диа- и парамагнетики}

\cel{измерение магнитной восприимчивости диа- и парамагнитного образцов.}

\prin{электромагнит, аналитические весы, милливеберметр, амперметр постоянного тока, реостаты, образцы.}

Магнитная восприимчивость тел может быть определена методом измерения сил, которые действуют на тела в~магнитном поле.
Существуют два классических метода таких измерений: \emph{метод Фарадея} и \emph{метод Гюи}. В~методе Фарадея
исследуемые образцы, имеющие форму маленьких шариков, помещаются в~область сильно неоднородного магнитного поля и
измеряется сила, действующая на образец. При этом для расчёта магнитной восприимчивости необходимо знать величину
градиента магнитного поля в месте расположения образца. В~методе Гюи используется тонкий и длинный стержень, один из
концов которого помещают в~зазор электромагнита (обычно в~область однородного поля), а другой конец~--- вне зазора, где
величиной магнитного поля можно пренебречь. Закон изменения поля~--- от максимального до нулевого~--- в~этом случае
несуществен.

\rpic{30mm}{4_1_1}{\cct Расположение образца в зазоре электромагнита}{1}

Найдём выражение для магнитной силы, действующей на такой образец (\p{1}). Пусть площадь образца равна~$s$, его магнитная
проницаемость~---~$\mu$, а поле в~зазоре равно~$B$.

Воспользуемся для расчёта энергетическими соображениями. Магнитная сила может быть вычислена как производная от
магнитной энергии по перемещению. Из теории известно (см.~[1]), что эту производную следует брать со знаком минус, когда
образец находится в поле постоянного магнита, или со знаком плюс, как в нашем случае, когда поле в зазоре создаётся
электромагнитом, ток $I$ в обмотках которого поддерживается постоянным.

При смещении образца на расстояние~$\D l$ вниз магнитная сила, действующая на него, равна
\be1
F=\left(\frac{\D W_m}{\D l}\right)_I,
\ee
где $\D W_m$~--- изменение магнитной энергии системы при постоянном токе в обмотке электромагнита и, следовательно, при
постоянной величине магнитного поля в зазоре.

Магнитная энергия рассчитывается по формуле
\be2
W_m=\frac12\int H B\,dV=\frac{1}{2\mu_0}\int\frac{B^2}{\mu}\,dV,
\ee
где интеграл распространён на всё пространство. При смещении образца магнитная энергия меняется только в области зазора
(в~объёме площади~$s$ и высоты~$\D l$), а около верхнего конца стержня остаётся неизменной, поскольку магнитного поля там
практически нет. Принимая поле внутри стержня равным измеренному нами полю в зазоре $B$, получим
\[
\D W_m=\frac{1}{2\mu_0}\frac{B^2}{\mu}s\D l-\frac{1}{2\mu_0}B^2 s\D l=\frac{1-\mu}{2\mu_0\mu}B^2s\D
l=-\frac{\chi}{2\mu_0\mu}B^2s\D l.
\]
Следовательно, на образец действует сила
\be3
F=%\frac{\D W_m}{\D l}=
-\frac{\chi}{2\mu_0\mu}B^2s.
\ee
Знак силы, действующей на образец, зависит от знака $\chi$: образцы из парамагнитных материалов ($\chi>0$) втягиваются
в~зазор электромагнита, а диамагнитные образцы ($\chi<0$) выталкиваются  из него.

Пренебрегая отличием $\mu$ от единицы, получаем окончательно расчётную формулу в виде
\be4
F=-\frac{\chi B^2s}{2\mu_0}.
\ee
Измерив силу, действующую на образец в магнитном поле $B$, можно рассчитать магнитную восприимчивость образца.

\cpic[0.7]{4_1_2}{Схема экспериментальной установки}{2}

\eo Схема установки изображена на \p{2}. Магнитное поле с максимальной индукцией~${\simeq}1,5$~Тл создаётся в~зазоре
электромагнита, питаемого постоянным током. Диаметр полюсов существенно превосходит ширину зазора, поэтому поле
в~средней части зазора достаточно однородно. Величина тока, проходящего через обмотки электромагнита, регулируется при
помощи трёх реостатов $R_1$, $R_2$ и $R_3$ и измеряется многопредельным амперметром~A. Тонкая проволока высокоомных
реостатов не рассчитана на большой ток, поэтому регулировку более низкоомными реостатами следует проводить только при
\aht{полностью} выведенных высокоомных реостатах.

Градуировка электромагнита (связь между индукцией магнитного поля~$B$ в~зазоре электромагнита и силой тока~$I$ в~его
обмотках) производится при помощи милливеберметра (описание милливеберметра и правила работы с ним приведены на
с.~\pageref{MWB}).

При измерениях образцы поочерёдно подвешиваются к~аналитическим весам так, что один конец образца оказывается в~зазоре
электромагнита, а другой~--- вне зазора, где индукцией магнитного поля можно пренебречь. При помощи аналитических весов
определяется перегрузка~$\D P=F$~--- сила, действующая на образец со стороны магнитного поля.

Как уже отмечалось, силы, действующие на диа- и парамагнитные образцы, очень малы. Небольшие примеси ферромагнетиков
(сотые доли процента железа или никеля) способны кардинально изменить результат опыта, поэтому образцы были специально
отобраны.

\zad

В работе предлагается исследовать зависимость силы, действующей на образец, размещённый в зазоре электромагнита, от
величины поля в~зазоре и по результатам измерений рассчитать магнитную восприимчивость меди и алюминия.

\dopop

%\n Перед включением питания магнита убедитесь, что все реостаты полностью введены (такое положение реостатов показано на
%\p{2}).

\n Проверьте работу цепи питания электромагнита. Оцените диапазон изменения тока~$I$ через обмотки.

\n Прокалибруйте электромагнит. Для этого с помощью милливеберметра снимите зависимость магнитного потока $\Phi$,
пронизывающего пробную катушку, находящуюся в~зазоре, от тока~$I$ ($\Phi=BSN$). Значение $SN$ (произведение площади
сечения пробной катушки на число витков в ней) указано на установке.

\ahtung{Включать и отключать электромагнит следует~только~при~минимальном токе.}

\n Убедитесь, что весы арретированы\footnote{Арретир~(фр. arreter --- фиксировать)~--- приспособление для закрепления
чувствительного элемента измерительного прибора в нерабочем состоянии.}.

\ahtung{Весы следует арретировать перед каждым изменением тока.}

\n[p4] Измерьте силы, действующие на образец в магнитном поле. Для этого, не включая электромагнит, подвесьте к~весам
один из образцов. Установите на весах примерное значение массы образца (масса, диаметр и максимальное значение
перегрузки для каждого образца указаны на установке). Освободите весы и добейтесь точного равновесия весов.

Арретируйте весы. Установите минимальное значение тока и проведите измерение равновесного значения массы.

Повторите измерения $m=f(T)$ для 6--8 других значений тока.

\n Повторите измерения п.~\r{p4} для другого образца.

\Obrab

\n Рассчитайте поле $B$ и постройте градуировочную кривую для электромагнита: $B=f(I)$.

\n Постройте на одном листе графики $|\D P|=f(B^2$) для меди и алюминия.

\n По наклонам полученных прямых рассчитайте величину~$\chi$ с~помощью формулы~(\r{4}).

\n Оцените погрешности измерений и сравните результаты с табличными значениями.

{\small

\kv

\n Объясните суть метода измерения магнитной восприимчивости.

\n Напишите выражения для магнитной силы, действующей на~образец, помещённый в~неоднородное магнитное поле.

\n Как можно убедиться в~однородности или неоднородности магнитного поля в~зазоре электромагнита?

\n Как проверить экспериментально, влияет ли намагниченность весов на результаты измерения магнитной восприимчивости?

\lit

\n \emph{Сивухин Д.В.} Общий курс физики. Т.~III. Электричество.~--- М.: Наука, 1983. \S\S~61, 75--77.

\n \emph {Калашников С.Г.} Электричество.~--- М.: Наука, 1977. Гл.~ХI, \S\S~109, 117, 118.

\n \emph{Кингсеп А.С., Локшин Г.Р., Ольхов О.А.} Основы Физики. Т.~1. Механика, электричество и магнетизм, колебания и
волны, волновая оптика.~--- М.:~Физматлит, 2001. Ч.~II, гл.~5, \S~5.2.

}
