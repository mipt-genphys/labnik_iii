\lab{Точка Кюри}


\aim{определение точки Кюри ферромагнетиков по температурным зависимостям
магнитной проницаемости и сопротивления.}


\equip{трансформатор, катушки, амперметры, вольтметры, реостаты, трубчатая
печь, цифровой вольтметр, термопары, образцы.}



В первой части работы исследуется изменение начальной магнитной проницаемости
ферромагнетика вблизи точки Кюри $\Theta_{К}$ (рис.~\figref{magnetic
permeability on temperature}), во второй~--- зависимость сопротивления образца
от температуры (фазовый переход II-го рода~--- рис.~\figref{resistance on
temperature}).

\begin{figure}[h!]
    \hfil
\parbox{5cm}{%
	\begin{minipage}[b]{5cm}
		\pic{\linewidth}{Chapter_4/4_3_1}
		\caption{Зависимость магнитной проницаемости от температуры образца}
		\figmark{magnetic permeability on temperature}
	\end{minipage}%
}
	\hfil
\parbox{5cm}{%
	\begin{minipage}[b]{5cm}
		\pic{\linewidth}{Chapter_4/4_3_2}
		\caption{Зависимость сопротивления от температцры образца}
		\figmark{resistance on temperature}
	\end{minipage}%
}
\end{figure}


Известно, что в ферромагнетике при определённой температуре, называемой
\term{точкой Кюри}, исчезает спонтанная намагниченность материала. Это
сопровождается изменением ряда физических свойств ферромагнетика: теплоёмкости,
теплопроводности, электропроводности, магнитной восприимчивости и
проницаемости; исчезает эффект магнитострикции и анизотропия намагниченности.
Поэтому, нагревая ферромагнитный образец и наблюдая за изменением его физических
свойств, можно определить точку Кюри ферромагнетика.

\labsection{А. Определение точки Кюри по изменению магнитной проницаемости}

\experiment
Экспериментальная установка, представленная на рис.~\figref{experiment of
magnetic permeability on temperature}, состоит из намагничивающей катушки $L$,
питаемой переменным током, и измерительной катушки $L_1$, замкнутой через диод
$D$ на микроамперметр $A_1$.

\begin{figure}[h!]
\centering
	\pic{0.8\textwidth}{Chapter_4/4_3_3}
	\caption{Схема экспериментальной установки для исследования $\mu(T)$}
	\figmark{experiment of magnetic permeability on temperature}
\end{figure}


При прохождении переменного тока через катушку $L$ в катушке $L_1$ возникает
индукционный ток $I_1$. Величину тока можно регулировать реостатом $R_1$.

Внутри обеих катушек находится небольшая трубчатая печь $\text{П}$, в которую
помещается образец $\text{О}$. Печь нагревается бифилярной обмоткой,
подключённой к источнику постоянного напряжения 36 В. Ток нагрева печи $I_0$
можно
регулировать реостатом и контролировать амперметром $A_0$.

Нагрев катушек $L$ и $L_1$ может изменить их сопротивления и сказаться на
показаниях микроамперметра. Для уменьшения
вредного нагрева в зазор между печкой и катушками вдувается воздух с помощью
вентилятора.
% Напряжение питания вентилятора~--- 36~В.

Температура внутри печи измеряется с помощью термопары $\text{Тп}$, соединенной
с милливольтметром $V$. По показаниям милливольтметра можно с помощью графика
чувствительности термопары, приведённого на установке, определить температуру
спая термопары относительно комнатной, а затем, зная комнатную температуру,
найти температуру образца.

При неизменных прочих условиях ток индукции $I_1$ зависит только от магнитной
проницаемости образца, помещённого в печь, т. е. $I_1 = f(\mu)$. Это
утверждение легко проверить, сняв зависимость тока $I_1$ от температуры печи,
когда в ней нет образца, и с ферромагнитным образцом. В первом случае ток
практически не зависит от температуры печи, а во втором --- изменение тока будет
значительным вблизи точки Кюри (рис.~\figref{magnetic permeability on
temperature}). При температуре выше точки Кюри показания микроамперметра будут
такими же, как и в отсутствие образца.

Точка Кюри соответствует середине участка с максимальным наклоном касательной к
кривой; рабочий диапазон $\Delta T$ должен быть несколько шире.

\labtask

В этом упражнении предлагается снять зависимость тока индукции от температуры и
рассчитать точку Кюри для двух образцов. В работе исследуются стержни из
ферромагнитных материалов (никель и пермендюр) и феррита.

\begin{lab:task}
    
\item
  \label{item:1}
  Перед началом работы с помощью графика чувствительности термопары рассчитайте
предельно допустимую разность потенциалов, если известно, что температура
образца не должна превышать $400\oC$.
\item
  Поместив ферромагнитный стержень в катушку, включите трансфоратор $\text{Тр}$
в сеть на 220~В. Образец следует опустить до упора, под которым расположена
термопара.
\item
  С помощью потенциометра $R_1$ установите в цепи катушки $L_1$ ток $I_1$,
вызывающий отклонение стрелки микроамперметра примерно на $3/4$ шкалы.
\item
  Установите реостат $R_0$ в среднее положение и подключите печь к ис­точнику
36~В. Тумблером, расположенным под катушками, включите вентилятор.
\item
  Подберите режим, удобный для определения точки Кюри: чтобы быстрее дойти от
комнатной температуры до начала рабочего участка $\Delta T$
(рис.~\figref{magnetic permeability on temperature}), установите максимальный
ток нагрева печи ($I_0\,\sim5\,A$) с помощью реостата $R_0$. Заметив начало
спада тока индукции, уменьшите ток нагрева вдвое. Оцените границы рабочего
диапазона термопары $\Delta U ( \sim\Delta T$) и интервал резкого изменения
тока. Максимальный ток $I_1$ должен быть близок к концу шкалы.

\warning{Не перегревайте катушку! (см. п. \ref{item:1}).}

Подберите ток нагрева $I_0$ так, чтобы время одной серии (нагрев или охлаждение
внутри рабочего диапазона $\Delta U$ составляло 2~--~3 мин. В течение одной
серии не следует менять чувствительность микроамперметра $(I_1)$ и ток нагрева
печи $(I_0)$, т. к. это влияет на величину тока индукции $I_1$.

Проведите предварительные измерения: при фиксированном токе нагрева $I_0$
регистрируйте $I_1$ и $U$ (дел). Полезно отметить время начала и окончания
записи, чтобы оценить продолжительность одной серии.

\item
  Выбрав режим работы, снимите зависимость тока индукции $I_1$ от термо-ЭДС $U$
при постоянном токе $I_0$; в
  области резкого изменения тока $I_1$ точки должны лежать почаще. Проведите
измерения при нагревании и охлаждении образца.

\item
  Повторите п. 2~--~6 для второго образца.
\item
  Охладив катушку, отключите печь и ток намагничивания.

\tasksection{Обработка результатов}

\item
  Постройте графики $I_1 = f(T)$ (дел), не пересчитывая каждую точку в $\Delta
T\,^{\circ} C$. Определите точку Кюри как температуру средней точки участка
кривой с максимальным наклоном касательной (в единицах $U$ дел).

Для выбранной точки пересчитайте $U$ (дел) сначала в милливольты (150 дел~---
45 мВ), а затем по графику чувствительности термопары~--- в  $\Delta
T\,^{\circ} C$. Зная комнатную температуру, определите температуру Кюри
$\Theta_{К}$.


\item
  Оцените погрешность и сравните результат с табличным.

\end{lab:task}

\labsection{Б. Определение точки Кюри по изменению сопротивления}

\experiment

\begin{figure}[h!]
\centering
	\pic{0.8\textwidth}{Chapter_4/4_3_4}
	\caption{Схема экспериментальной установки для исследования $R(T)$}
	\figmark{experiment of resistance on temperature}
\end{figure}
Экспериментальная установка для исследования зависимости омического
сопротивления ферромагнетика от температуры представлена на
рис.~\figref{experiment of resistance on temperature}. Никелевая спираль $C$,
намотанная на фарфоровую трубку, заключена в керамическую трубку и помещена в
трубчатую печь $\text{П}$. Нагрев печи регулируется переключателем $K$.
Температура фарфоровой трубки (спирали) контролируется термопарой $\text{Тп}$,
подключённой к милливольтметру $V$, прокалиброванному в градусах. Сопротивление
спирали измеряется цифровым мультиметром.
% $\text{В}7 - 27$.


\begin{lab:task}

\taskpreamble{В этом упражнении предлагается снять зависимость сопротивления никелевой
спирали от температуры и определить точку Кюри.}
    
\item
  Включите в сеть цифровой вольтметр. Измерьте сопротивление никелевой спирали
при комнатной температуре $R \sim 10\,\text{Ом}$.
\item
  Включите печь в сеть на 220 В и поставьте переключатель $K$ в среднее
положение.
\item
  Измеряйте сопротивление $R$ и температуру спирали $T$ через каждые 20\oC,
  не останавливая нагрева. При температуре образца $>200\oC$ мощность
нагрева следует увеличить.

Дойдя до предельной температуры ($T_\text{max} = 450\oC$), отключите
нагрев и проведите измерения при охлаждении образца до $100\oC$.

\tasksection{Обработка результатов}

\item
  Постройте графики $R = f(U_\text{дел})$. По изменению температурного
коэффициента сопротивления (пересечению касательных к прямолинейным участкам
графика $R = f(T)$ найдите точку Кюри (рис.~\figref{resistance on temperature}).
\item
  Градуировка милливольтметра соответствует термопаре железо-кон\-стантан. Если
в установке используется медь-константан, сделайте пересчет точки Кюри
$\Theta_{К}$, используя графики чувствительности \mbox{обеих тер}\-мопар (не
забудьте учесть комнатную температуру --- обычно $\approx20\oC$).
\item
  Оцените погрешности и сравните результат с табличным.

\end{lab:task}


\begin{lab:questions}

\item
  Чем отличаются атомы пара- и диамагнетиков по магнитным характеристикам в
отсутствие магнитного поля?
\item
  Как изменяются характеристики вещества при фазовых переходах первого и
второго рода?
\item
  Какие два конкурирующих взаимодействия между атомами характерны для
ферромагнитного вещества?
\item
  На одном графике качественно изобразите начальные кривые намагничивания
$B(H)$ для ферромагнетика при трёх температурах: комнатной, более высокой и
температуре выше точки Кюри. Укажите на оси $H$, где лежит область,
соответствующая условиям настоящей работы.

\end{lab:questions}


\begin{lab:literature}
\item \SivuhinIII~--- \S\S~74, 79.
\item \Kalashnikov~--- Гл. XI, \S\S~110, 111, 119.
\item \Kirichenko~--- \S~9.3.
\end{lab:literature}

