\lab{Петля гистерезиса (статический метод)}
\label{lab:4-4}

\begin{lab:aim}
	наблюдение начальной кривой намагничивания ферромагнетиков 
    и предельной петли гистерезиса.
\end{lab:aim}

\begin{lab:equipment}
	источник питания, тороид, соленоид, баллистический гальванометр с
осветителем и шкалой,
	амперметр, магазин сопротивлений, лабораторный автотрансформатор (ЛАТР),
разделительный трансформатор.
\end{lab:equipment}


Перед выполнением работы необходимо ознакомиться с
пп.~\ref{sec:ferromagnetism}, \ref{sec:histeresis} и~\ref{sec:measure-HB} 
теоретического Введения к разделу.

Магнитная индукция~${B}$ и напряжённость поля~${H}$
в~ферромагнитном материале неоднозначно связаны между
собой: индукция зависит не только от напряжённости, но и от предыстории образца.
Связь между~$B$ и~$H$ типичного ферромагнетика иллюстрирует
рис.~\figref{Ferromagnet hysteresis loop}.

\begin{figure}[h]
%     \hfil\pic{0.38\textwidth}{4_4_1}
    \centering
    \pic{62mm}{Chapter_4/4-4-hist}
    \caption{Петля гистерезиса ферромагнетика}
    \figmark{Ferromagnet hysteresis loop}
\end{figure}

Если к размагниченному образцу начинают прикладывать магнитное поле,
то его намагничивание следует кривой~OACD (начальная кривая намагничивания).
Кривая~OAC практически совпадает с соответствующей зависимостью~$M(H)$,
поскольку для неё второе слагаемое в выражении~\chaptereqref{fieldB}
существенно превосходит первое ($\chi\gg1$). В~точке~C намагниченность
достигает насыщения ($M\to \rm max$), соответствующее значение
индукции (индукцию насыщения) обозначим~$B_s$.
Дальнейшее увеличение индукции происходит только вследствие
роста~$H$ (линия~CD --- прямая). Нетрудно видеть, что экстраполяция прямой~CD 
до пересечения с осью ординат~$H=0$ позволяет определить намагниченность 
в насыщении~$M_s$.

При уменьшении $H$ до нуля зависимость~$B(H)$ имеет вид
кривой~DCE, в точке~E ($H=0$) имеет место некоторая остаточная индукция~$B_r$.
Чтобы размагнитить образец, то есть перевести
его в состояние $B=0$ (точка~F), необходимо приложить <<обратное>> магнитное поле~$-H_c$
(коэрцитивным поле или коэрцитивная сила).
Замкнутая кривая CEFC$'$E$'$F$'$C, возникающая при циклическом перемагничивании
образца с достижением насыщения
называется \important{предельной} петлёй гистерезиса. Если начать уменьшать поле
в некоторой промежуточной точке~A основной кривой, траектория
системы будет также представлять собой некоторую гистерезисную петлю~AA$'$
меньшей площади (обозначена пунктиром).

% Индукция $\vec{B}$ в образце состоит из индукции, связанной с~намагничивающим
% полем~$\vec{H}$, и~индукции, создаваемой самим
% намагниченным образцом. В~системе~СИ эта связь имеет вид
% \begin{equation}
% 	\eqmark{1}
% 	\vec{B}=\mu_0(\vec{H}+\vec{M}),
% \end{equation}
% где $\vec{M}$~--- \term{намагниченность}~--- магнитный момент единичного
% объёма образца, а~$\mu_0$~--- магнитная
% постоянная.

\begin{figure}[h!]
	\begin{minipage}[b]{0.55\textwidth}
		\pic{65mm}{Chapter_4/4_4_2}
		\caption{Схема для измерения индукционного тока (или заряда)}
		\figmark{Scheme for induction current}
	\end{minipage}%
\hfill
	\begin{minipage}[b]{0.4\textwidth}
        \centering
		\pic{29mm}{Chapter_4/4_4_3}
		\caption{Схема для калибровки гальванометра}
		\figmark{Scheme for galvanometr calibration}
	\end{minipage}
\end{figure}

В настоящей работе предлагается измерить начальную кривую намагничивания
и предельную петлю гистерезиса следующим методом.
На тороидальный сердечник (рис.~\figref{Scheme for
induction current}), изготовленный из исследуемого образца,
равномерно намотана \important{намагничивающая} обмотка 
с~числом витков~$N$, а поверх неё~--- \important{измерительная} 
обмотка с числом витков~$N'$.
При скачкообразном изменении тока в~намагничивающей обмотке
в~измерительной обмотке возникает ЭДС индукции. Ток, вызванный
этой ЭДС, регистрируется гальванометром~Г, работающим в баллистическом
(импульсном) режиме: его отклонение пропорционально полному
заряду $\Delta q$, протекшему через него (подробнее баллистический режим
описан в~работе \ref{lab:galvanometr}).

Напряжённость поля~$H$ в сердечнике пропорциональна току~$I$ в первичной
(намагничивающей) обмотке, а изменение магнитной индукции~$\Delta B$~---
заряду $\Delta q$, протекшему через вторичную (измерительную) обмотку. 
Таким образом, измеряя токи $I$ и суммируя отклонения $\Delta q$ гальванометра~Г, 
можно получить зависимость~$B(H)$ для материала сердечника.

\paragraph{Измерение полей}
Рассмотрим подробнее, как выразить~$B$ и~$H$ через параметры, измеряемые в
эксперименте. Напряжённость магнитного поля~$H$ в тороиде определяется током,
текущем в~намагничивающей обмотке (см.~\chaptereqref{H-toroid}):
\begin{equation}
	\eqmark{2}
	H\approx\frac{N}{\pi D}\,I,
\end{equation}
где $D$~--- средний диаметр тора.

Пусть в намагничивающей обмотке ток скачкообразно изменился
на величину~$\Delta I$. При этом пропорционально меняется поле~$H$ в тороиде:
$\Delta H\propto\Delta I$.
Изменение поля~$\Delta H$ приводит к изменению потока магнитной индукции~$\Phi$
в сердечнике, и в измерительной обмотке
сечения $S_{т}$ c числом витков~$N'$ возникает ЭДС индукции:
\begin{equation*}
	\mathcal{E}=-\frac{d\Phi}{dt}=-S N'\frac{dB}{dt}.
\end{equation*}

Через гальванометр Г протекает импульс тока; первый отброс <<зайчика>>
гальванометра, работающего в баллистическом режиме,
пропорционален величине прошедшего через гальванометр заряда~$\Delta q$:
\begin{equation*}
\Delta x=\frac{\Delta q}{b},
\end{equation*}
где $b$~--- константа, называемая \important{баллистической постоянной гальванометра}.
Свяжем отклонение $\Delta x$ с изменением магнитной индукции~$\Delta B$:
\begin{equation}
	\eqmark{3}
|\Delta  x|=\frac{1}{b}\int\! I\,dt= \frac{1}{bR}\int\!
\mathcal{E}\,dt=\frac{S_т N'}{bR}\left| \Delta B \right|,
\end{equation}
где $R$~--- полное сопротивление измерительной цепи тороида, $S$~--- площадь
поперечного сечения сердечника: $S=\pi d^2/4$.

Баллистическую постоянную $b$ можно определить, если провести аналогичные
измерения, взяв вместо тороида с~сердечником
пустотелый соленоид с~числом витков~$N_{с}$, на который намотана короткая
измерительная катушка с~числом
витков~$N_{с}'$ (рис.~\figref{Scheme for galvanometr calibration}). В длинном
пустом соленоиде имеем
% (практически достаточно, чтобы его длина превышала~6 диаметров: $l_C >
% 6d_C$)
\begin{equation*}
B_{с}=\mu_0 H_{с} \approx \frac{N_{с}}{l_с} I,
\end{equation*}
где $l_с$ --- его длина. Отклонение гальванометра при изменении тока
в соленоиде найдём по формуле, аналогичной \eqref{3}:
\begin{equation}%5
	\eqmark{5}
	\left| \varphi_с \right|=\frac{S_с\,N_{с}'}{bR_{с}}\,\left|\Delta B_с
    \right| = \frac{\mu_0 S_с\,N_{с}'N_{с}}{bR_{с}l_с}\left|\Delta I_с\right|
\end{equation}
Здесь $R_{с}$~--- полное сопротивление измерительной цепи соленоида, $S_{с}$~---
площадь поперечного сечения соленоида: $S_{с}=\pi d_{с}^2/4$.

Таким образом, выражения \eqref{3} и~\eqref{5} позволяют, исключив
баллистическую постоянную~$b$, установить связь между отклонением 
<<зайчика>> $\Delta x$ и изменением магнитной индукции $\Delta B$ 
в сердечнике тороида:
\begin{equation}%6
\eqmark{6}
\Delta B=\mu_0 N_{с} \frac{N_{с}'}{N'}  \frac{R}{R_{с}}
\frac{d_с^2}{d^2} \frac{\Delta I_с}{l_с} \frac{\Delta x}{\Delta x_с}.
\end{equation}

\experiment

Детальная схема измерений петли гистерезиса представлена
на рис.~\figref{tor}.
К блоку питания (источнику постоянного напряжения)
подключён специальный генератор, позволяющий скачками менять токи
в~намагничивающей обмотке.

% Лишнее? // ППВ
%
% Одинаковые скачки $\Delta I$
% (${\propto}\Delta H$) вызовут разные отклонения
% $\Delta x$ ($\propto \Delta B$) на участках $FD'$ и $D'E'$:
% на рис.~\figref{Ferromagnet hysteresis loop} скачок
% $\Delta H_1$ может дать и~$\Delta B_1$ и $\Delta B_2$.
% Поэтому генератор меняет ток неравномерно: большими скачками вблизи насыщения
% и малыми вблизи нуля.

\begin{figure}[h!]
	\pic{\textwidth}{Chapter_4/4_4_4}
	\caption{Схема установки для исследования петли гистерезиса}
	\figmark{tor}
\end{figure}

\begin{figure}[h!]
    \pic{\textwidth}{Chapter_4/4_4_5}
    \caption{Схема установки для калибровки гальванометра}
    \figmark{sol}
\end{figure}

Ток в  намагничивающей  обмотке  измеряется   амперметром. Переключатель
$\text{П}_1$ позволяет менять направление тока в первичной обмотке.
Чувствительность гальванометра $\text{Г}$ во вторичной цепи можно менять  с
помощью  магазина  сопротивлений $R_{М}$. Ключ~К$_2$ предохраняет гальванометр  от
перегрузок и замыкается \emph{только на время измерения отклонений
зайчика}.  Ключ К$_0$  служит  для  мгновенной остановки зайчика  
(короткое замыкание  гальванометра). 
Переключатель~П$_1$ позволяет менять направление тока в~первичной обмотке. 
Переключателем~$\text{П}_2$ можно изменять направление тока через гальванометр.
Ток в намагничивающей обмотке измеряется амперметрами~А$_1$ и~A$_2$
с разными пределами измерения.
% % с~пределом~0,75~А
% % при малых токах или амперметром~А$_2$
% %с~пределом~3~А в~области насыщения. При токах больше 0,75~А амперметр~А$_1$
% При больших токах амперметр с меньшим пределом измерения
% должен быть закорочен (ключ~К$_1$ замкнут).
% %(Сопротивление амперметра мало и сравнимо с~сопротивлением ключа, поэтому
% % показания амперметра~А$_1$ не падают до нуля
% %даже при замкнутом ключе.)

Схема на рис.~\figref{sol} отличается от схемы на
рис.~\figref{tor} только тем, что вместо тороида
подключён калибровочный соленоид.

% Не нужно? // ППВ
%
% Сопротивления измерительных цепей тороида ($R=R_T+ R_{М} + R_0$) и~соленоида
% ($R_{с}=R_C+R'_M+R_0$) должны быть одинаковы
% [см.~замечание после формулы \eqref{6}].
% Сопротивление тороида $R_T\ll R_0$ --- сопротивления гальванометра, поэтому
% сопротивления магазина в~схеме с~тороидом
% и~соленоидом отличаются на величину сопротивления соленоида~$R_C$: $R'_M = R_C -
% R_{М}$.



\paragraph{Размагничивание образца}
Чтобы снять начальную кривую 
\begin{wrapfigure}[8]{r}{0.5\textwidth}
    \centering
    \pic{55mm}{Chapter_4/4_4_6}
    \caption{Схема размагничивания}
    \figmark{Scheme for demagnization}
\end{wrapfigure}
намагничивания, нужно предварительно 
размагнитить образец.
Для этого тороид подключается к цепи переменного тока
(рис.~\figref{Scheme for demagnization}). При уменьшении амплитуды тока через
намагничивающую обмотку от тока насыщения до нуля характеристики сердечника~$B$
и~$H$ <<пробегают>> за секунду 50 петель всё меньшей площади и в~итоге
приходят в~нулевую точку.



\paragraph{Исследование петли}
Измерения начинаются с~максимального тока (точка~$C$ на рис.~\figref{Ferromagnet
hysteresis loop}). Переключая тумблер генератора,
следует фиксировать ток, соответствующий каждому положению тумблера, и
отклонение зайчика~$\Delta x$, соответствующее
каждому щелчку тумблера. Дойдя до нулевого тока
(точка~E), следует при размыкании переключателя~$\text{П}_1$ 
зафиксировать последний отброс гальванометра вблизи точки~E. Следующий
отброс~--- при замыкании переключателя~$\text{П}_1$. 
Ток вблизи нуля меняется мало, но скачки~$\Delta x$ обычно заметны. 
Это соответствует вертикальным участкам петли.

Поменяв направление тока в намагничивающей обмотке переключателем~$\text{П}_1$,
следует, увеличивая ток, пройти участок EC$'$ до
насыщения другого знака. В~точке~C$'$ переключателем~$\text{П}_2$ следует
поменять направление тока в измерительной обмотке, чтобы при
движении по правой ветви петли зайчик отклонялся в ту же сторону. 
В~точке E$'$ при нулевом токе ещё раз ключом~$\text{П}_1$
изменяется направление тока в первичной обмотке, чтобы пройти участок~E$'$F$'$C.
Таким образом, измеряя шаг за шагом отклонения зайчика при изменениях тока, 
нужно пройти всю петлю гистерезиса.

Нельзя при замкнутом ключе~К$_2$ менять ток сразу на несколько щелчков тумблера или
отключать ключ~$\text{П}_1$ при больших токах, так как при резком изменении
тока можно повредить гальванометр.
%нить может перекрутиться.

При движении по петле ток должен меняться \emph{строго монотонно}.
Если случайно пропущен один отброс зайчика, \emph{нельзя вернуться}
назад на один шаг~--- это приведёт к~искажению петли. Следует при разомкнутом
ключе~К$_2$ вернуться к~насыщению и начать обход петли сначала. 
При нарушении монотонности в измерении начальной кривой
намагничивания образец снова надо размагничивать, а для предельной петли
достаточно вернуться к~насыщению (именно поэтому измерения начинают с~предельной
петли).

\begin{lab:task}

\taskpreamble{В работе исследуются начальная (основная) кривая 
намагничивания и предельная петля гистерезиса для образцов тороидальной
формы, изготовленных из чистого железа или стали.}

\tasksection{Подготовка  к работе}

	\item Соберите схему согласно рис.~\figref{tor}.

	\item Не подключая гальванометра, проверьте работу цепи первичной обмотки.
Определите диапазон изменения тока.

	\item Включите осветитель гальванометра. Шкалу можно установить так, чтобы нулевое
    положение зайчика было недалеко от края шкалы.
    
    Чувствительность гальванометра, при которой зайчик не зашкаливает,
можно подобрать, меняя сопротивление магазина~$R_{М}$. Установите начальное значение~$R_М>R_{с}$~--- сопротивления
соленоида. Значения~$R_{М}$ и $R_{с}$ указаны на установке.

	
	\begin{lab:warning}
		Внимательно изучите рекомендации в разделе <<Исследование петли>>.
	\end{lab:warning}

	\item Замкните ключ~К$_2$. Сначала, не проводя записей, наблюдайте за
отклонениями зайчика при каждом щелчке тумблера. Изменят ток следует только
после того, как зайчик вернется к начальному положению.

	Аккуратно обойдите всю петлю, чтобы убедиться, что зайчик нигде не выходит
за пределы шкалы. Как правило, самые большие скачки~$\Delta x$ происходят 
на участках~EF и E$'$F$'$.

	Если зайчик вышел за пределы шкалы~--- разомкните ключ~К$_2$ и, увеличив
сопротивление~$R_{М}$, начните обход петли сначала.

	Если зашкаливания не произошло и максимальное отклонение зайчика близко к
концу шкалы~--- приступайте к измерениям.

	\tasksection{Предельная петля гистерезиса}

	\item Измерение предельной петли начните с~максимального тока
намагничивания. Отмечайте величину тока~$I$, соответствующую
	каждой позиции тумблера генератора, 
    и отклонение зайчика, соответствующие каждому щелчку.
    
	Не забудьте, что при изменении полярности тока вблизи точки~E следует
фиксировать отклонения зайчика как при размыкании, так и при замыкании ключа
$\text{П}_1$.

	Завершив полный  замкнутый цикл,  разомкните ключ К$_2$.  
    Уменьшив ток до нуля, разомкните ключ $\text{П}_1$. Проверьте, что суммы 
    всех отклонений по верхней и нижней частям петли
одинаковы. Если расхождение превышает 10\%, пройдите цикл снова.

	\tasksection{Калибровка гальванометра}

	\item Для калибровки гальванометра соберите схему согласно
рис.~\figref{sol}.

    \item Уменьшите на магазине сопротивлений значение~$R_{с}$ на
	величину~$R_{с}$: $R'_M=R_{М}-R_C$~--- в таком случае сопротивления
    измерительных цепей в схемах рис.~\figref{tor} и рис.~\figref{sol} 
    окажутся одинаковы (в формуле \eqref{6} можно положить $R=R_{с}$).
    
    \item Установите тумблер генератора тока на максимум и, 
    замкнув ключ~$\text{П}_1$, запишите значение тока~$I_{\rm max}$. 
    Подключите гальванометр (ключ~К$_2$). 
    Размыкая переключатель~$\text{П}_1$, измерьте отклонение 
    гальванометра~$\Delta x_{с}$, возникшее при изменении 
    тока~$\Delta I_{с}=I_{\rm max}$. 
    
    По измеренному отношению $\Delta I_{с}/\Delta x_{с}$ 
    формула~\eqref{6} позволяет вычислять изменение магнитной 
    индукции при произвольном отклонении гальванометра~$\Delta x$.

	\tasksection{Начальная кривая намагничивания}

	\item Начальную кривую намагничивания (участок~OAC на
рис.~\figref{Ferromagnet hysteresis loop}) можно снять по той же схеме
(рис.~\figref{tor}), если предварительно
	размагнитить тороид в~цепи переменного тока. Для этого соберите схему,
изображённую на рис.~\figref{Scheme for demagnization}. Включите ЛАТР в~сеть и
	установите ток, соответствующий насыщению~(участок~CD на
рис.~\figref{Ferromagnet hysteresis loop}). 
Ручкой \mbox{ЛАТРа} медленно (за 5--10~с) уменьшайте ток до нуля. Образец размагничен.

	\item Вновь подсоедините тороид к цепи, изображённой 
    на рис.~\figref{tor}. Установите тумблер генератора на минимальный ток и
снимите  начальную кривую намагничивания,  скачками увеличивая ток от нуля 
до~$I_{\rm max}$. Напомним, что первый отброс даёт замыкание 
переключателя~$\text{П}_1$. В~случае сбоя в измерениях образец надо снова размагнитить.
Дойдя до максимального тока, разомкните ключ~К$_2$.

	\item Запишите параметры установки: $R_{М}$ и $R_{М}'$;
сопротивление гальванометра~$R_0$; размеры тороида: $d_{т}$ и~$D$. 
Количество витков тороида и параметры соленоида указаны на установке.

	\tasksection{Обработка результов}

		\item Используя формулы \eqref{2} и \eqref{6}, получите зависимости
        $H(I)$ и $\Delta B(\Delta x)$.

		\item Постройте петлю гистерезиса $B(H)$. Для выбора масштаба
просуммируйте все скачки~$\Delta B$ (или $\Delta x$) по левой части
		петли и все скачки по правой части. Убедитесь, что суммы совпадают.

		Построение удобно начать с~максимального значения~$H$ (точка~C или
C$'$ на рис.~\figref{Ferromagnet hysteresis loop}). Переход к~следующему
значению~$H$ соответствует первому скачку~$\Delta B$ и т.\,д.
Отложив все~$\Delta B$ по одной стороне петли и дойдя до насыщения, постройте
вторую сторону таким же образом.

		Найдите середину петли и проведите ось~$H(I)$.

		\item Постройте начальную кривую намагничивания на том же графике.

		\item Определите по графику коэрцитивное поле~$H_c$,
        остаточную индукцию~$B_r$, индукцию насыщения~$B_s$ и 
        максимальную намагниченность $M_s$.

		\item Проведя касательную к графику в области с наибольшим наклоном 
        начальной кривой намагниченности, определите максимальное значение 
        дифференциальной магнитной проницаемости~$\mu_\text{диф}$.

		\item Оцените погрешности результатов эксперимента,
        и сравните их с табличными значениями.

 
% 		\begin{center}
% 		\begin{tabular}{|c|c|c|}
% 		\hline
% 		&Эксперим.&Табличн.\\
% 		\hline\hline
% 		$H_c\,\frac{A}{\text{м}}$& & \\
% 		$B_s\;$ T & & \\
% 		$\mu_\text{диф}$ & & \\
% 		\hline
% 		\end{tabular}
% 		\end{center}

\end{lab:task}


\begin{lab:questions}

	\item Получите выражение, связывающее заряд $\Delta q$, прошедший через
    измерительную катушку гальванометра, и изменение $\Delta B$ в образце.
    При каких условиях справедливо это соотношение?

	\item Пользуясь теоремой о~циркуляции, получите формулу для напряжённости
магнитного поля в~длинном соленоиде и в торе.

    \item В каких случаях поле $H$ в образце можно считать совпадающим
    с полем~$H_0$, создаваемым токами в обмотке?

    \item Почему рекомендуется начинать обход петли с~насыщения образца?

    \item Как изменится индукция в сердечнике, если в состоянии насыщения
    резко выключить внешнее поле (разорвать цепь)?
    
    \item Каков наклон участков CD и C$'$D$'$ на рис.~\figref{Ferromagnet hysteresis loop}?
    
\end{lab:questions}


\begin{lab:literature}
    \item \Kirichenko~--- \S~4.1, 4.2, 9.3.

	\item \SivuhinIII~--- \S\S~74, 79.

	\item \Kalashnikov~--- \S\S~110, 111, 118, 119.

	\item \KingLokOlh~--- Ч.~II, гл.~5, \S~5.3.
    

\end{lab:literature}
