\lab{Петля гистерезиса (статический метод)}

\begin{lab:aim}
	исследование кривых намагничивания ферромагнетиков с помощью баллистического
гальванометра.
\end{lab:aim}

\begin{lab:equipment}
	источник питания, тороид, соленоид, баллистический гальванометр с
осветителем и шкалой,
	амперметр, магазин сопротивлений, лабораторный автотрансформатор (ЛАТР),
разделительный трансформатор.
\end{lab:equipment}


Перед выполнением работы необходимо ознакомиться с
пп.~\ref{sec:ferromagnetism}--\ref{sec:measure-HB} теоретического введения к
разделу.

Магнитная индукция~${B}$ и напряжённость поля~${H}$
в~ферромагнитном материале неоднозначно связаны между
собой: индукция зависит не только от напряжённости, но и от предыстории образца.
Связь между $B$ и $H$ типичного ферромагнетика иллюстрирует
рис.~\figref{Ferromagnet hysteresis loop}.

\begin{figure}[h]
%     \hfil\pic{0.38\textwidth}{4_4_1}
    \centering
    \pic{0.7\textwidth}{Chapter_4/4-4-hist}
    \caption{Петля гистерезиса ферромагнетика}
    \figmark{Ferromagnet hysteresis loop}
\end{figure}

Если к размагниченному образцу начинают прикладывать магнитное поле,
то его намагничивание следует кривой~OACD (начальная кривая намагничивания).
Кривая OAC практически совпадает с соответствующей зависимостью~$M(H)$,
поскольку для неё второе слагаемое в выражении~\chaptereqref{fieldB}
существенно превосходит первое ($\chi\gg1$). В~точке~C намагниченность
достигает насыщения ($M\to \rm max$), соответствующее значение
индукции (индукцию насыщения) обозначим $B_s$.
Дальнейшее увеличение индукции происходит только вследствие
роста~$H$ (линия CD --- прямая). Нетрудно видеть, что экстраполяция прямой CD до пересечения
с осью ординат $H=0$ позволяет определить намагниченность в насыщении
$M_s$.

При уменьшении $H$ до нуля зависимость~$B(H)$ имеет вид
кривой DCE, в точке E ($H=0$) имеет место некоторая остаточная индукция~$B_r$.
Чтобы размагнитить образец, то есть перевести
его в состояние $B=0$ (точка F), необходимо приложить <<обратное>> магнитное поле~$-H_c$
(коэрцитивным поле или коэрцитивная сила).
Замкнутая кривая CEFC$'$E$'$F$'$C, возникающая при циклическом перемагничивании
образца с достижением насыщения
называется \important{предельной} петлёй гистерезиса. Если начать уменьшать поле
в некоторой промежуточной точке A основной кривой, траектория
системы будет также представлять собой некоторую гистерезисную петлю AA$'$
меньшей площади (обозначена пунктиром).

% Индукция $\vec{B}$ в образце состоит из индукции, связанной с~намагничивающим
% полем~$\vec{H}$, и~индукции, создаваемой самим
% намагниченным образцом. В~системе~СИ эта связь имеет вид
% \begin{equation}
% 	\eqmark{1}
% 	\vec{B}=\mu_0(\vec{H}+\vec{M}),
% \end{equation}
% где $\vec{M}$~--- \term{намагниченность}~--- магнитный момент единичного
% объёма образца, а~$\mu_0$~--- магнитная
% постоянная.

\begin{figure}[h!]
\hfil
\parbox{5cm}{%
	\begin{minipage}{5cm}
		\pic{\linewidth}{Chapter_4/4_4_2}
		\caption{Схема для измерения индукционного тока (или заряда)}
		\figmark{Scheme for induction current}
	\end{minipage}%
}\hfil
\parbox{5cm}{%
	\begin{minipage}{5cm}\centering
		\pic{0.7\linewidth}{Chapter_4/4_4_3}
		\caption{Схема для калибровки гальванометра}
		\figmark{Scheme for galvanometr calibration}
	\end{minipage}
}
\end{figure}

В настоящей работе предлагается измерить начальную кривую намагничивания
и предельную петлю гистерезиса следующим методом.
На тороидальный сердечник (рис.~\figref{Scheme for
induction current}), изготовленный из исследуемого образца,
равномерно намотана \important{намагничивающая} обмотка T$_0$
(с~числом витков~$N_{т0}$), а поверх неё~--- \important{измерительная} обмотка
T$_1$
(число витков~$N_{т1}$).
При скачкообразном изменении тока в~намагничивающей обмотке
в~измерительной обмотке возникает ЭДС индукции. Ток, вызванный
этой ЭДС, регистрируется гальванометром~Г, работающим в баллистическом
(импульсном) режиме: его отклонение пропорционально на полному
заряду $\Delta q$, протекшему через него (подробнее баллистический режим
описан в~работе 3.2.5).
\todo[inline]{Добавить правильный номер и гиперссылку}
Напряжённость поля $H$ в~сердечнике пропорциональна току~$I$ в первичной
обмотке~T$_0$, а~изменение магнитной индукции~$\Delta B$~---
заряду $\Delta q$. Таким образом, измеряя токи,
текущие через обмотку~T$_0$, и суммируя отклонения гальванометра,
подключённого к~обмотке~T$_1$, можно найти
зависимость~$B(H)$ для материала сердечника.

Рассмотрим подробнее, как выразить~$B$ и~$H$ через параметры, измеряемые в
эксперименте. Напряжённость магнитного поля~$H$ в тороиде определяется током,
текущем в~намагничивающей обмотке:
\begin{equation}
	\eqmark{2}
	H\approx\frac{N_{0}}{\pi D}\,I,
\end{equation}
где $D$~--- средний диаметр тора.

Пусть в намагничивающей обмотке ток скачкообразно изменился
на величину~$\Delta I$. При этом пропорционально меняется поле~$H$ в тороиде:
$\Delta H\propto\Delta I$.
Изменение поля $\Delta H$ приводит к~изменению потока магнитной индукции~$\Phi$
в~сердечнике, и в~измерительной обмотке
сечения $S$ c~числом витков~$N$ возникает ЭДС индукции:
\begin{equation*}
	\mathcal{E}=-\frac{d\Phi}{dt}=-S N\frac{dB}{dt}.
\end{equation*}

Через гальванометр Г протекает импульс тока; первый отброс ``зайчика''
гальванометра, работающего в~баллистическом режиме,
пропорционален величине прошедшего через гальванометр заряда~$\Delta q$:
\begin{equation*}
\Delta x=\frac{\Delta q}{b},
\end{equation*}
где $b$~--- константа, называемая \important{баллистической постоянной гальванометра}.
Свяжем отклонение $\Delta x$ с изменением магнитной индукции~$\Delta B$:
\begin{equation}
	\eqmark{3}
|\Delta  x|=\frac{1}{b}\int\! I\,dt= \frac{1}{bR}\int\!
\mathcal{E}\,dt=\frac{S_т N_{т1}}{bR}\left| \Delta B \right|,
\end{equation}
где $R$~--- полное сопротивление измерительной цепи тороида, $S_т$~--- площадь
поперечного сечения сердечника: $S_т=\pi {d_т}^2/4$.

Баллистическую постоянную $b$ можно определить, если провести аналогичные
измерения, взяв вместо тороида с~сердечником
пустотелый соленоид с~числом витков~$N_{с0}$, на который намотана короткая
измерительная катушка с~числом
витков~$N_{с1}$ (рис.~\figref{Scheme for galvanometr calibration}). В длинном
пустом соленоиде имеем
% (практически достаточно, чтобы его длина превышала~6 диаметров: $l_C >
% 6d_C$)
\begin{equation*}
B_c=\mu_0 H_с \approx \frac{N_{с0}}{l_с} I,
\end{equation*}
где $l_с$ --- его длина. Отклонение гальванометра при изменении тока
в соленоиде найдём по формуле, аналогичной \eqref{3}:
\begin{equation}%5
	\eqmark{5}
	\left| \varphi_с \right|=\frac{S_с\,N_{с1}}{bR_1}\,\left|\Delta B_с
    \right| = \frac{\mu_0 S_с\,N_{с1}N_{с0}}{bR_1l_с}\left|\Delta I_с\right|
\end{equation}
Здесь $R_1$~--- полное сопротивление измерительной цепи соленоида, $S_с$~---
площадь поперечного сечения соленоида: $S_с=\pi d_с^2/4$.

Таким образом, выражения \eqref{3} и \eqref{5} позволяют, исключив
баллистическую постоянную $b$, установить связь
между отклонением ``зайчика'' $\Delta x$
и изменением магнитной индукции $\Delta B$ в сердечнике тороида:
\begin{equation}%6
\eqmark{6}
\Delta B=\frac{\mu_0\Delta I_с}{l_с} \frac{R}{R_1}
\frac{d_с^2}{d_т^2} \frac{N_{с0} N_{с1}}{N_{т1}}\frac{\Delta x}{\Delta x_с}.
\end{equation}

% TODO Какая-то ерунда: сопротивления R и R1 входят в полученную формулу.
% Зачем им быть одинаковыми? // ППВ
%
% Строго говоря, величина $b$~--- это не константа. Она зависит не только от
% параметров гальванометра, но и от
% сопротивления $R$ цепи, к~которой подключён гальванометр, поэтому
% формула~\eqref{6} справедлива, если полные сопротивления
% измерительных цепей тороида и соленоида одинаковы: $R = R_1$.

\experiment

Детальная схема измерений петли гистерезиса представлена
на рис.~\figref{Scheme for hysteresis loop study}.
К блоку питания (источнику постоянного напряжения)
подключён специальный генератор, позволяющий скачками менять токи
в~намагничивающей обмотке.

% Лишнее? // ППВ
%
% Одинаковые скачки $\Delta I$
% (${\propto}\Delta H$) вызовут разные отклонения
% $\Delta x$ ($\propto \Delta B$) на участках $FD'$ и $D'E'$:
% на рис.~\figref{Ferromagnet hysteresis loop} скачок
% $\Delta H_1$ может дать и~$\Delta B_1$ и $\Delta B_2$.
% Поэтому генератор меняет ток неравномерно: большими скачками вблизи насыщения
% и малыми вблизи нуля.

\begin{figure}[h!]
	\pic{0.9\textwidth}{Chapter_4/4_4_4}
	\caption{Схема установки для исследования петли гистерезиса}
	\figmark{Scheme for hysteresis loop study}
\end{figure}

\begin{figure}[h!]
    \pic{0.9\textwidth}{Chapter_4/4_4_5}
    \caption{Схема установки для калибровки гальванометра}
    \figmark{Galvanometr calibration}
\end{figure}

Ток в  намагничивающей  обмотке  измеряется   амперметром. Переключатель
$\text{П}_1$ позволяет менять направление тока в первичной обмотке.
Чувствительность гальванометра $\text{Г}$ во вторичной цепи можно менять  с
помощью  магазина  сопротивлений $R_M$.  Ключ $K$ предохраняет гальванометр  от
перегрузок и замыкается \term{только на время измерения отклонений
зайчика}.  Ключ $K_0$  служит  для  мгновенной остановки зайчика  (короткое
замыкание  гальванометра). Переключатель~П$_1$ позволяет менять направление
тока в~первичной обмотке. Переключателем $\text{П}_2$ можно изменять
направление тока через гальванометр.

% Ток в намагничивающей обмотке измеряется амперметрами~А$_1$ и A$_2$
% с разными пределами измерения.
% % с~пределом~0,75~А
% % при малых токах или амперметром~А$_2$
% %с~пределом~3~А в~области насыщения. При токах больше 0,75~А амперметр~А$_1$
% При больших токах амперметр с меньшим пределом измерения
% должен быть закорочен (ключ~К$_1$ замкнут).
% %(Сопротивление амперметра мало и сравнимо с~сопротивлением ключа, поэтому
% % показания амперметра~А$_1$ не падают до нуля
% %даже при замкнутом ключе.)




Схема на рис.~\figref{Galvanometr calibration} отличается от схемы на
рис.~\figref{Scheme for hysteresis loop study} только тем, что вместо тороида
подключён калибровочный соленоид.

% Не нужно? // ППВ
%
% Сопротивления измерительных цепей тороида ($R=R_T+ R_M + R_0$) и~соленоида
% ($R_1=R_C+R'_M+R_0$) должны быть одинаковы
% [см.~замечание после формулы \eqref{6}].
% Сопротивление тороида $R_T\ll R_0$ --- сопротивления гальванометра, поэтому
% сопротивления магазина в~схеме с~тороидом
% и~соленоидом отличаются на величину сопротивления соленоида~$R_C$: $R'_M = R_C -
% R_M$.

\paragraph{Размагничивание образца.}
Чтобы снять начальную кривую намагничивания, нужно размагнитить образец.
Для этого тороид подключается к цепи переменного тока
(рис.~\figref{Scheme for demagnization}). При уменьшении амплитуды тока через
\begin{wrapfigure}[8]{r}{0.45\textwidth}
	\pic{0.45\textwidth}{Chapter_4/4_4_6}
	\caption{Схема установки для размагничивания образца}
	\figmark{Scheme for demagnization}
\end{wrapfigure}
намагничивающую обмотку от тока насыщения до нуля характеристики сердечника~$B$
и~$H$ <<пробегают>> за секунду 50 петель всё меньшей площади и в~итоге
приходят в~нулевую точку.

\paragraph{Исследование петли.}
Измерения начинаются с~максимального тока (точка~$C$ на рис.~\figref{Ferromagnet
hysteresis loop}). Переключая тумблер генератора,
следует фиксировать ток, соответствующий каждому положению тумблера, и
отклонение зайчика~$\Delta x$, соответствующее
каждому щелчку тумблера. Дойдя до нулевого тока
(точка~$E$), следует при размыкании ключа~$\text{П}_1$ зафиксировать последний
отброс гальванометра вблизи точки~$E$. Следующий
отброс~--- при замыкании ключа~$\text{П}_1$. Ток вблизи нуля меняется мало, но
скачки~$\Delta x$ обычно заметны. Это соответствует
вертикальным участкам петли.

Поменяв направление тока в обмотке~Т$_0$ переключателем~$\text{П}_1$,
следует, увеличивая ток, пройти участок $EC'$ до
насыщения другого знака. В~точке $C'$ переключателем~$\text{П}_2$ следует
поменять направление тока в обмотке~$Т_{1}$, чтобы при
движении по правой ветви петли зайчик отклонялся в ту же сторону. В~точке $E'$
при нулевом токе ещё раз ключом~$\text{П}_1$
изменяется направление тока в первичной обмотке, чтобы пройти участок~$E'F'C$.
Таким образом, измеряя шаг за шагом
отклонения зайчика при изменениях тока, нужно пройти всю петлю гистерезиса.

Нельзя при замкнутом ключе~К менять ток сразу на несколько щелчков тумблера или
отключать ключ~$\text{П}_1$ при больших токах, так как при резком изменении
тока можно повредить гальванометр.
%нить может перекрутиться.

При движении по петле ток должен меняться \term{строго монотонно}.
Если случайно пропущен один отброс зайчика, \important{нельзя вернуться}
назад на один шаг~--- это приведёт к~искажению петли. Следует при разомкнутом
ключе~$K$ вернуться к~насыщению и начать
обход петли сначала. При нарушении монотонности в измерении начальной кривой
намагничивания образец снова надо размагничивать, а для предельной петли
достаточно вернуться к~насыщению (именно поэтому измерения начинают с~предельной
петли).

\labtask

В работе исследуются начальная (основная) кривая намагничивания и предельная
петля гистерезиса для образцов тороидальной
формы, изготовленных из чистого железа или стали.

\begin{lab:task}

%	\tasksection{Подготовка  к работе}

	\item Соберите схему согласно рис.~\figref{Scheme for hysteresis loop
study}.

	\item Не подключая гальванометра, проверьте работу цепи первичной обмотки.
Определите диапазон изменения тока.

	\item Чувствительность гальванометра, при которой зайчик не зашкаливает,
можно подобрать, меняя сопротивление
	магазина~$R_М$. Установите начальное значение~$R_М>R_C$~--- сопротивления
соленоида. Значения~$R_M$ и $R_C$ указаны на
	установке.

	Включите осветитель гальванометра. Шкалу можно установить так, чтобы нулевое
положение зайчика было недалеко от края
	шкалы.

	\begin{lab:warning}
		Внимательно перечитайте раздел <<Исследование петли>>.
	\end{lab:warning}

	\item Замкните ключ~$K$. Сначала, не проводя записей, наблюдайте за
отклонениями зайчика при каждом щелчке тумблера. Изменят ток следует только
после того, как зайчик вернется к начального положению.

	Аккуратно обойдите всю петлю, чтобы убедиться, что зайчик нигде не выходит
за пределы шкалы. Как правило, самые большие
	скачки~$\Delta x$ происходят на участках~$EF$ и $E'F'$.

	Если зайчик вышел за пределы шкалы~--- разомкните ключ~$K_2$ и, увеличив
сопротивление~$R_M$, начните обход петли
	сначала.

	Если зашкаливания не произошло и максимальное отклонение зайчика близко к
концу шкалы~--- приступайте к измерениям.

	\tasksection{Предельная петля гистерезиса}

	\item Измерение предельной петли начните с~максимального тока
намагничивания. Отмечайте величину тока~$I$, соответствующую
	каждой позиции тумблера генератора ($I$, а не $\Delta I$), и отклонение
зайчика, соответствующие каждому щелчку.
	Не забудьте, что при изменении полярности тока вблизи точки E следует
фиксировать отклонения зайчика как при размыкании, так и при замыкании ключа
$\text{П}_1$.
	Завершив полный  замкнутый цикл,  разомкните ключ $K$. Уменьшив ток до нуля,
разомкните ключ $\text{П}_1$.
	Проверьте, что суммы всех отклонений по верхней и нижней частям петли
одинаковы. Если расхождение превышает 10\%, пройдите цикл снова.

	\tasksection{Калибровка гальванометра}

	\item Для калибровки гальванометра соберите схему согласно
рис.~\figref{Galvanometr calibration}.

	Уменьшите на магазине сопротивлений значение~$R_M$ на
	величину~$R_C: R'_M=R_M-R_C$. Установите тумблер генератора тока на максимум
и, замкнув ключ~$\text{П}$, запишите значение
	тока~$I_{\rm max}$. Подключите гальванометр (ключ~$K$). Размыкая
ключ~$\text{П}$, измерьте отклонение гальванометра~$\Delta x_1$,
	возникшее при изменении тока~$\Delta I_1=I_{\rm max}$. Формула \eqref{6}
позволяет выразить изменение магнитной индукции через
	отношение~$\Delta I_1/(\Delta x_1)$ и величину~$\Delta x$.

	\tasksection{Начальная кривая намагничивания}

	\item Начальную кривую намагничивания (участок~$OAC$ на
рис.~\figref{Ferromagnet hysteresis loop}) можно снять по той же схеме
(рис.~\figref{Scheme for hysteresis loop study}), если предварительно
	размагнитить тороид в~цепи переменного тока. Для этого соберите схему,
изображённую на рис.~\figref{Scheme for demagnization}. Включите ЛАТР в~сеть и
	установите ток, соответствующий насыщению~(участок~$CD$ на
рис.~\figref{Ferromagnet hysteresis loop}). Ручкой ЛАТРа медленно (за 5 -- 10~с)
уменьшайте ток до
	нуля. Образец размагничен.

	\item Вновь подсоедините тороид к цепи, изображённой на рис.~\figref{Scheme
for hysteresis loop study}. Установите тумблер генератора на минимальный ток и
снимите  начальную кривую намагничивания,  скачками увеличивая ток от нуля до
$I_{\rm max}$. Напомним, что первый отброс даёт замыкание ключа $\text{П}_1$.  В
случае сбоя в измерениях образец надо снова размагнитить.
	Дойдя до максимального тока,  разомкните ключ $K$.

	\item Запишите параметры установки: $R_M$ и $R'_M$~--- для контроля;
сопротивление гальванометра~$R_0$; размеры тороида:
	$d_T$ и $D$. Количество витков тороида и параметры соленоида указаны на
установке.

	\tasksection{Обработка результов}

		\item Используя формулы \eqref{2} и \eqref{6}, получите зависимости
        $H(I)$ и $\Delta B(\Delta x)$.

		\item Постройте петлю гистерезиса $B(H)$. Для выбора масштаба
просуммируйте все скачки~$\Delta B$ (или $\Delta x$) по левой части
		петли и все скачки по правой части. Убедитесь, что суммы совпадают.

		Построение удобно начать с~максимального значения~$H$ (точка $C$ или
$C'$ на рис.~\figref{Ferromagnet hysteresis loop}). Переход к~следующему
значению~$H$ соответствует первому скачку~$\Delta B$ и т.\,д.
Отложив все~$\Delta B$ по одной стороне петли и дойдя до насыщения, постройте
вторую сторону таким же образом.

		Найдите середину петли и проведите ось~$H(I)$.

		\item Постройте начальную кривую намагничивания на том же графике.

		\item Определите по графику коэрцитивную силу~$H_c$ и индукцию
насыщения~$B_s$.

		\item Определите максимальное значение дифференциальной магнитной
проницаемости~$\mu_\text{диф}$ для начальной кривой
		намагничивания:
		\begin{equation*}
			\mu_\text{диф}=\frac{1}{\mu_0}\frac{dB}{dH}.
		\end{equation*}

		\item Оцените погрешности эксперимента и сравните
        результаты с табличными.

% 		\begin{center}
% 		\begin{tabular}{|c|c|c|}
% 		\hline
% 		&Эксперим.&Табличн.\\
% 		\hline\hline
% 		$H_c\,\frac{A}{\text{м}}$& & \\
% 		$B_s\;$ T & & \\
% 		$\mu_\text{диф}$ & & \\
% 		\hline
% 		\end{tabular}
% 		\end{center}

\end{lab:task}


\begin{lab:questions}

	\item Получите выражение, связывающее заряд $\Delta q$, прошедший через
    измерительную катушку гальванометра, и изменение $\Delta B$ в образце.
    При каких условиях справедливо это соотношение?

	\item Пользуясь теоремой о~циркуляции, получите формулу для напряжённости
магнитного поля в~длинном соленоиде и в торе.

    \item В каких случаях поле $H$ в образце можно считать совпадающим
    с полем $H_0$, создаваемым токами в обмотке?

    \item Почему рекомендуется начинать обход петли с~насыщения образца?

    \item Как изменится индукция в сердечнике, если в состоянии насыщения
    резко выключить внешнее поле (разорвать цепь)?
\end{lab:questions}


\begin{lab:literature}

	\item \SivuhinIII~--- \S\S~74, 79.

	\item \Kalashnikov~--- \S\S~110, 111, 118, 119.

	\item \KingLokOlh~--- Ч.~II, гл.~5, \S~5.3.

\end{lab:literature}
