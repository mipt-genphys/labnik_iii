\lab{3.4.4}%{Исследование кривых намагничивания ферромагнетиков баллистическим методом}
{Петля гистерезиса (статический метод)}

\cel{исследование кривых намагничивания ферромагнетиков с помощью баллистического гальванометра.}

\prin {генератор тока с блоком питания, тороид, соленоид, баллистический гальванометр с осветителем и шкалой,
амперметры, магазин сопротивлений, лабораторный автотрансформатор (ЛАТР), разделительный трансформатор.}

\rpic{60mm}{4_4_1}{Петля гистерезиса ферромагнетика}{1}

Магнитная индукция~$\vB$ и напряжённость магнитного поля~$\vH$ в~ферромагнитном материале неоднозначно связаны между
собой: индукция зависит не только от напряжённости, но и от предыстории образца. Связь между индукцией и напряжённостью
поля типичного ферромагнетика иллюстрирует \p{1}. Если к~размагниченному образцу начинают прикладывать магнитное поле,
то его намагничивание следует кривой~$OACD$, выходящей из начала координат. Эту кривую называют \term{основной кривой
намагничивания.}

Индукция $\vB$ в образце состоит из индукции, связанной с~намагничивающим полем~$\vH$, и~индукции, создаваемой самим
намагниченным образцом. В~системе~СИ эта связь имеет вид
\be1
\v{B}=\mu_0(\v{H}+\v{M}),
\ee
где $\v{M}$~--- \emph{намагниченность}~--- магнитный момент единичного объёма образца, а~$\mu_0$~--- магнитная
постоянная. Кривая $OACD$, изображающая зависимость~$B(H)$, практически совпадает с~зависимостью~$M(H)$, поскольку
второй член в~выражении~(\r{1})~--- в малых полях~--- существенно превосходит первый. В~точке~$C$ намагниченность~$M$
достигает насыщения, и дальнейшее медленное увеличение индукции происходит в~основном вследствие роста~$H$.

Намагнитим образец до насыщения~--- до точки~$D$. Соответствующее значение индукции~$B_s$ называют \term{индукцией
насыщения}. При уменьшении поля~$H$ до нуля зависимость~$B(H)$ имеет вид кривой~$DCE$, и при нулевом поле индукция имеет
конечное~--- ненулевое --- значение. Это \term{остаточная индукция}~$B_r$. Чтобы размагнитить образец, то есть перевести
его в~состояние~$F$, необходимо приложить <<обратное>> магнитное поле~$H_c$, которое называют \term{коэрцитивной силой}.

Замкнутая кривая $DEFD'E'F'D$, возникающая при циклическом перемагничивании образца, намагниченного до насыщения,
называется \term{предельной петлёй гистерезиса}.

В работе исследуются ферромагнитные образцы тороидальной формы.

\begin{figure}[hbt]
\hfil
\piccapt{0.45\textwidth}{4_4_2}{\cct Схема для измерения индукционного тока (или заряда)}{2}
\hfil\hfil
\piccapt{0.35\textwidth}{4_4_3}{\cct Схема для калибровки гальванометра}{3}
\end{figure}

Изложим коротко суть метода. На тороидальный сердечник (\p{2}) равномерно намотана \emph{намагничивающая} обмотка
с~числом витков~$N_{T0}$, а поверх неё~--- {измерительная} обмотка с~числом витков~$N_{T1}$.

Если быстро изменить ток в~намагничивающей обмотке, то в~измерительной обмотке возникает ЭДС индукции. Ток, вызванный
этой ЭДС, течёт через гальванометр~Г, который работает в~баллистическом (импульсном) режиме, то~есть реагирует на полный
заряд, протекший через катушку гальванометра (подробнее баллистический режим описан в~работе 2.2.6).

Напряжённость поля $H$ в~сердечнике пропорциональна току~$I$ в~первичной обмотке~$N_{T0}$, а~изменение магнитной
индукции~$B$~--- заряду, протекшему через гальванометр при изменении тока намагничивания. Таким образом, измеряя токи,
текущие через обмотку~$N_{T0}$, и суммируя отклонения гальванометра, подключённого к~обмотке~$N_{T1}$, можно рассчитать
зависимость~$B(H)$ для материала сердечника.

\etp

Рассмотрим подробнее, как выразить~$B$ и~$H$ через параметры, измеряемые в~эксперименте. Напряжённость магнитного
поля~$H$ в~тороиде зависит от тока, текущего в~намагничивающей обмотке:
\be2
H=\frac{N_{T0}}{\pi D}\,I,
\ee
где $D$~--- средний диаметр тора.

Пусть в намагничивающей обмотке ток скачкообразно изменился на величину~$\D I$. При этом меняется поле~$H$ в~тороиде:
$\D H\sim\D I$.

Изменение поля $\D H$ приводит к~изменению потока магнитной индукции~$\Phi$ в~сердечнике, и в~измерительной обмотке
сечения $S_T$ c~числом витков~$N_{T1}$ возникает ЭДС индукции:
\[
\E=-\frac{d\Phi}{dt}=-S_T N_{T1}\frac{dB}{dt}.
\]
Через гальванометр Г протекает импульс тока; первый отброс зайчика гальванометра, работающего в~баллистическом режиме,
пропорционален величине прошедшего через гальванометр заряда~$q$:
\[
\phi=\frac{q}{b}.
\]
Коэффициент пропорциональности $b$ называют \term{баллистической постоянной гальванометра}.

Свяжем отклонение зайчика $\phi$ с~изменением магнитной индукции~$\D B$:
\be3
|\phi|=\frac{q}{b}=\frac{1}{b}\int\! I\,dt= \frac{1}{bR}\int\! \E\,dt=\frac{S_T N_{T1}}{bR}\D B,
\ee
где $R$~--- полное сопротивление измерительной цепи тороида, $S_T$~--- площадь поперечного сечения сердечника: $S_T=\pi
{d_T}^2/4$.

Баллистическую постоянную $b$ можно определить, если провести аналогичные измерения, взяв вместо тороида с~сердечником
пустотелый соленоид с~числом витков~$N_{C0}$, на который намотана короткая измерительная катушка с~числом
витков~$N_{C1}$ (\p{3}). В длинном соленоиде (практически достаточно, чтобы его длина превышала~6 диаметров: $l_C >
6d_C$) поле~$H$ можно рассчитать так же, как для тороида (см.~(\r{2})); $B$ и $H$ в~соленоиде связаны линейно, поэтому
связь между изменением тока~$\D I_1$ в~обмотке~$N_{C0}$ и~изменением магнитной индукции~$\D B_C$ имеет простой вид:
\be4
\D B_C=\frac{\mu_0 N_{C0}}{l_C}\,\D I_1.
\ee

Изменение магнитной индукции в соленоиде связано с~отклонением~$\phi_1$ зайчика гальванометра формулой, аналогичной
формуле~(\r{3}):
\be5
\phi_1=\frac{S_C\,N_{C1}}{bR_1}\,\D B_C.
\ee
Здесь $R_1$~--- полное сопротивление измерительной цепи соленоида, $S_C$~--- площадь поперечного сечения соленоида:
$S_C=\pi d^2_C/4$.

Таким образом, выражения (\r{3}), (\r{4}) и (\r{5}) позволяют, исключив баллистическую постоянную $b$, установить связь
между отклонением зайчика в~делениях $\D x$~($\D x\sim \phi)$ и изменением магнитной индукции $\D x\sim B$ в~сердечнике
тороида:
\be6
\D B~[Т]=\mu_0\left(\frac{d_C}{d_T}\right)^2\frac{R}{R_1}\frac{N_{C0}}{N_{T1}}\frac{N_{C1}}{l_C}\,\D I_1\frac{\D x}{\D
x_1}.
\ee

Строго говоря, величина $b$~--- это не константа. Она зависит не только от параметров гальванометра, но и от
сопротивления цепи, к~которой подключён гальванометр, поэтому формула~(\r{6}) справедлива, если полные сопротивления
измерительных цепей тороида и соленоида одинаковы: $R = R_1$.

\eo Схема для исследования петли гистерезиса представлена на \p{4}. К~блоку питания (источнику постоянного напряжения)
подключён специальный генератор, позволяющий скачками менять токи в~намагничивающей обмотке. Одинаковые скачки $\D I$
(${\sim}\D H$) вызовут разные отклонения $\D x$ ($\sim \D B$) на участках $FD'$ и $D'E'$: на \p{1} скачок $\D H_1$ может
дать и~$\D B_1$ и $\D B_2$. Поэтому генератор меняет ток неравномерно: большими скачками вблизи насыщения и малыми
вблизи нуля.

\cpic[1]{4_4_4}{Схема установки для исследования петли гистерезиса}{4}

Ток в намагничивающей обмотке измеряется амперметром~А$_1$ с~пределом~0,75~А при малых токах или амперметром~А$_2$
с~пределом~3~А в~области насыщения. При токах больше 0,75~А амперметр~А$_1$ должен быть закорочен: ключ~К$_1$ замкнут.
(Сопротивление амперметра мало и сравнимо с~сопротивлением ключа, поэтому показания амперметра~А$_1$ не падают до нуля
даже при замкнутом ключе.) Переключатель~П$_1$ позволяет менять направление тока в~первичной обмотке.

Чувствительность гальванометра~Г во вторичной цепи можно менять с~помощью магазина сопротивлений~$R_M$. Ключ~К$_2$
предохраняет гальванометр от перегрузок и замыкается только~(!) на время измерения отклонений зайчика. Ключ~К$_0$ служит
для мгновенной остановки зайчика (короткое замыкание гальванометра). Переключателем~П$_2$ можно изменять направление тока
через гальванометр.

Схема на \p{5} отличается от схемы на \p{4} только тем, что вместо тороида подключён калибровочный соленоид.

Сопротивления измерительных цепей тороида ($R=R_T+ R_M + R_0$) и~соленоида ($R_1=R_C+R'_M+R_0$) должны быть одинаковы
[см.~замечание после формулы (\r{6})].

\cpic[1]{4_4_5}{Схема установки для калибровки гальванометра}{5}

Сопротивление тороида $R_T\ll R_0$~--- сопротивления гальванометра, поэтому сопротивления магазина в~схеме с~тороидом
и~соленоидом отличаются на величину сопротивления соленоида~$R_C$: $R_M = R_C + R'_M$.

\rpic{5.5cm}{4_4_6}{\cct Схема установки для размагничивания образца}{6}

Чтобы снять начальную кривую намагничивания, нужно размагнитить сердечник. Для этого тороид подключается к~цепи
переменного тока (\p{6}). При уменьшении амплитуды тока через намагничивающую обмотку от тока насыщения до нуля
характеристики сердечника $B$ и $H$ <<пробегают>> за секунду 50 петель всё меньшей площади и в~итоге приходят в~нулевую
точку.

\bfno{Исследование петли} Измерения начинаются с~максимального тока (точка~$C$ на \p{1}). Переключая тумблер генератора,
следует фиксировать ток, соответствующий каждому положению тумблера, и отклонение зайчика~$\D x$, соответствующее
каждому щелчку тумблера. При токах $< 0,75$~А размыканием ключа~К\_1 подключается амперметр~А\_1. Дойдя до нулевого тока
(точка~$E$), следует при размыкании ключа~П\_1 зафиксировать последний отброс гальванометра вблизи точки~$E$. Следующий
отброс~--- при замыкании ключа~П\_1. Ток вблизи нуля меняется мало, но скачки~$\D x$ обычно заметны. Это соответствует
вертикальным участкам петли.

Поменяв направление тока в~обмотке~$N_{T0}$ переключателем~П\_1, следует, увеличивая ток, пройти участок $EC'$ до
насыщения другого знака. В~точке~$C'$ переключателем~П\_2 следует поменять направление тока в~обмотке~$N_{T1}$, чтобы при
движении по правой ветви петли зайчик отклонялся в~ту же сторону. В~точке $E'$ при нулевом токе ещё раз ключом~П\_1
изменяется направление тока в первичной обмотке, чтобы пройти участок~$E'F'C$. Таким образом, измеряя шаг за шагом
отклонения зайчика при изменениях тока, нужно пройти всю петлю гистерезиса.

Нельзя при замкнутом ключе~К\_2 менять ток сразу на несколько щелчков тумблера или отключать ключ~П\_1 при больших
токах, так как при резком изменении тока можно повредить гальванометр.
%нить может перекрутиться.

При движении по петле ток должен меняться строго монотонно. Если случайно пропущен один отброс зайчика, нельзя вернуться
назад на один шаг~--- это приведёт к~искажению петли. Следует при разомкнутом ключе~К\_2 вернуться к~насыщению и начать
обход петли сначала. При нарушении монотонности в~измерении начальной кривой намагничивания образец снова надо
размагничивать, а для предельной петли достаточно вернуться к~насыщению. Вот почему измерения начинают с~предельной
петли.

\zad

В работе исследуются начальная (основная) кривая намагничивания и предельная петля гистерезиса для образцов тороидальной
формы, изготовленных из чистого железа или стали.

\dopop

\n Соберите схему согласно \p{4}.

\n Не подключая гальванометра, проверьте работу цепи первичной обмотки. Определите диапазон изменения тока.

\n Чувствительность гальванометра, при которой зайчик не зашкаливает, можно подобрать, меняя сопротивление
магазина~$R_М$. Установите начальное значение~$R_М>R_C$~--- сопротивления соленоида. Значения~$R_M$ и $R_C$ указаны на
установке.

Включите осветитель гальванометра. Шкалу можно установить так, чтобы нулевое положение зайчика было недалеко от края
шкалы.

\ahtung{Внимательно перечитайте раздел <<Исследование петли>>.}

\n Замкните ключ~К\_2. Сначала, не проводя записей, наблюдайте за отклонениями зайчика при каждом щелчке тумблера.

Аккуратно обойдите всю петлю, чтобы убедиться, что зайчик нигде не выходит за пределы шкалы. Как правило, самые большие
скачки~$\D x$ происходят на участках~$EF$ и $E'F'$.

Если зайчик вышел за пределы шкалы~--- разомкните ключ~К$_2$ и, увеличив сопротивление~$R_M$, начните обход петли
сначала.

Если зашкаливания не произошло и максимальное отклонение $\D x$ близко к концу шкалы~--- приступайте к измерениям.

\n Измерение предельной петли начните с~максимального тока намагничивания. Фиксируйте величину тока~$I$, соответствующую
каждой позиции тумблера генератора ($I$, а не $\D I$), и скачки~$\D x$, соответствующие каждому щелчку.

\n Для калибровки гальванометра соберите схему согласно \p{5}. Уменьшите на магазине сопротивлений значение~$R_M$ на
величину~$R_C$: $R'_M=R_M-R_C$. Установите тумблер генератора тока на максимум и, замкнув ключ~П\_1, запишите значение
тока~$I_{\max}$. Подключите гальванометр (ключ~К\_2). Размыкая ключ~П\_1, измерьте отклонение гальванометра~$\D x_1$,
возникшее при изменении тока~$\D I_1=I_{\max}$. Формула (\r{6}) позволяет выразить изменение магнитной индукции через
отношение~$\D I_1/(\D x_1)$ и величину~$\D x$.

\n Начальную кривую намагничивания (участок~$OAC$ на \p{1}) можно снять по той же схеме (\p{4}), если предварительно
размагнитить тороид в~цепи переменного тока. Для этого соберите схему, изображённую на \p{6}. Включите ЛАТР в~сеть и
установите ток, соответствующий насыщению~(участок~$CD$ на \p{1}). Ручкой ЛАТРа медленно (за 5--10~с) уменьшайте ток до
нуля. Образец размагничен.

\n Вновь подсоедините тороид к~цепи, изображённой на \p{4}, и снимите начальную кривую намагничивания.

\n Запишите параметры установки: $R_M$ и $R'_M$~--- для контроля; сопротивление гальванометра~$R_0$; размеры тороида:
$d_T=1$~см, $D=10$~см. Количество витков тороида и параметры соленоида указаны на установке.

\Obrab

\n Используя формулы (\r{2}) и (\r{6}), получите зависимости
\[
H(А/м)=f_1[I(А)]\quad и\quad \D B(Тл)=f_2[\D x(мм)].
\]

\n Постройте петлю гистерезиса $B=f(H)$. Для выбора масштаба просуммируйте все скачки~$\D B$ (или $\D x$) по левой части
петли и все скачки по правой части. Убедитесь, что суммы совпадают.

Построение удобно начать с~максимального значения~$H$ (точка $C$ или $C'$ на \p{1}). Переход к~следующему значению~$H$
соответствует первому скачку~$\D B$ и т.~д. Отложив все~$\D B$ по одной стороне петли и дойдя до насыщения, постройте
вторую сторону таким же образом.

Найдите середину петли и проведите ось~$H(I)$.

\n Постройте начальную кривую намагничивания на том же графике.

\n Определите по графику коэрцитивную силу~$H_c$ и индукцию насыщения~$B_s$. Сравните результаты с~табличными.

\n Определите максимальное значение дифференциальной магнитной проницаемости~$\mu_{диф}$ для начальной кривой
намагничивания:
\[
\mu_{диф}=\frac{1}{\mu_0}\frac{dB}{dH}.
\]

\n Оцените погрешности. Сведите результаты в таблицу:

\begin{center}
\begin{tabular}{|c|c|c|}
\hline
&Эксперим.&Табличн.\\
\hline\hline
$H_c\,\frac{A}{м}$& & \\
$B_s\;$ T & & \\
$\mu_{диф}$ & & \\
\hline
\end{tabular}
\end{center}

{\small

\kv

\n Почему рекомендуется начинать обход петли с~насыщения образца?

\n Получите выражение, связывающее отклонение рамки гальванометра и изменение индукции образца. При каких условиях
справедливо это соотношение?

\n Пользуясь теоремой о~циркуляции, получите формулу для напряжённости магнитного поля в~длинном соленоиде.

\lit

\n \emph{Сивухин Д. В.} Общий курс физики. Т.~III. Электричество. --- М.: Наука, 1983. \S\S~74, 79.

\n \emph{Калашников С.Г.} Электричество. --- М.: Наука, 1977. \S\S~110, 111, 118, 119.

\n~\emph{Кингсеп А.С., Локшин Г.Р., Ольхов О.А.} Основы Физики. Т.~1. Механика, электричество и магнетизм, колебания и
волны, волновая оптика.~--- М.:~Физматлит, 2001. Ч.~II, гл.~5, \S~5.3.

}
