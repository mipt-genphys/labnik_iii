\lab{Петля гистерезиса(динамический метод)}

\aim{изучение петель гистерезиса ферромагнитных мате­риалов с помощью осциллографа.}

\equip{автотрансформатор, понижающий транс­форматор, интегрирующая цепочка, амперметр, вольтметр, элек­тронный осциллограф, делитель напряжения, тороидальные образцы с двумя обмотками.}

Перед выполнением работы необходимо ознакомиться с
пп.~\ref{sec:ferromagnetism}--\ref{sec:measure-HB} теоретического введения к разделу.

Ферромагнитные материалы часто применяются в трансформаторах, дросселях, машинах переменного тока, то есть в устройствах, где они подвергаются периодическому перемагничиванию. Изучение магнитных характеристик ферромагнетиков в переменных полях представляет поэтому большой практический интерес. Основные характеристики ферромагнетиков~--- их коэрцитивная сила, магнитная проницаемость, мощность, рассеиваемая в виде тепла при перемагничивании, при учёте потерь на вихревые токи и т. д.~--- зависят от частоты перемагничивающего поля. В настоящей работе кривые гистерезиса ферромагнитных материалов изучаются в поле частоты 50~Гц с помощью электронного осциллографа.

Измерение магнитной индукции в образцах. Магнитную индукцию удобно определять с помощью ЭДС, возникающей при изменении магнитного потока $\Phi$ в катушке, намотанной на образец.

Пусть катушка плотно охватывает образец, и индукция $B$ в образце однородна. В этом случае
\begin{equation}
	\eqmark{4.5.1}
	\Phi = BSN_k.
\end{equation}

Тогда при изменении магнитного потока ЭДС в катушке будет равна
\begin{equation*}
	\mathcal{E} = -\frac{d\Phi}{dt} = -SN_k\frac{dB}{dt},
\end{equation*}
и
\begin{equation}
	\eqmark{4.5.2}
	|B| = \frac{1}{SN_k}\int \mathcal{E}dt
\end{equation}

Таким образом, для определения $B$ нужно проинтегрировать сигнал, наведённый меняющимся магнитным полем на измерительную катушку, намотанную на образец.

Для интегрирования используют RC-цепочку (рис.~\figref{integrating curcuit}).
\begin{figure}[h!]
	\pic{0.9\textwidth}{4_5_1}
	\caption{Интегрирующая цепочка}
	\figmark{integrating curcuit}
\end{figure}
Если сопротивление источника напряжения мало по сравнению с $R$, и частота сигнала $\omega$, входное и выходное сопротивление связаны соотношением
\begin{equation}
	\eqmark{4.5.3}
	\begin{aligned}
		U_\text{вых} = \frac{1}{C}&\int Idt = \frac{1}{C}\int \frac{U_\text{вх}}{\sqrt{R^2 + \frac{1}{(\omega C)^2}}} \frac{1}{\omega C} dt =\\
		= \frac{1}{RC}&\int \frac{U_\text{вх}}{\sqrt{1 + \frac{1}{(\omega RC)^2}}} dt \approx \frac{1}{RC}\int U_\text{вх} dt
	\end{aligned}
\end{equation}

Таким образом, если $(\omega RC)^2 \gg 1$, выходное напряжение будет пропорционально интегралу входного, и здесь $\omega$~--- частота самой низкой гармоники~--- частота повторения $\omega = 2\pi/T.$

Тогда из \eqref{4.5.2} получаем
\begin{equation}
	\eqmark{4.5.4}
	|B| = \frac{1}{SN_\text{и}}\int U_\text{вх}dt = \frac{R_\text{и}C_\text{и}}{SN_\text{и}}U_\text{вх}
\end{equation}

Если на вход интегрирующей ячейки подать синусоидальный сигнал, связь выходного сигнала со входным будет такой
\begin{equation}
	\eqmark{4.5.5}
	\frac{U_\text{вых}}{U_\text{вх}} = \frac{1}{2\pi fRC},
\end{equation}
где $f$~--- частота сигнала.

\experiment
 Схема установки изображена на рис.~\figref{experimental environment}. Напряжение сети (220 В, 50 Гц) с помощью регулировочного автотрансформатора $\text{Ат}$ через разделительный понижающий трансформатор $\text{Тр}$ подаётся на намагничивающую обмотку $N_O$ исследуемого образца.

Действующее значение переменного тока в обмотке $N_O$ измеряется амперметром $A$. Последовательно с амперметром включено сопротивление $R_O,$ напряжение с которого подаётся на вход $X$ электронного осциллографа (ЭО). Это напряжение пропорционально току в обмотке $N_q,$ а следовательно, и напряжённости $H$ магнитного поля в образце.

Для измерения магнитной индукции $B$ с измерительной обмотки $N_\text{и}$ на вход $RC-$цепочки подаётся напряжение
$U_\text{и}$ ($U_\text{вх}$), пропорциональное производной $B$, а с интегрирующей ёмкости $C_\text{и}$ снимается напряжение $U_C$ ($U_\text{вых}$), пропорциональное величине $B$, и подаётся на вход $Y$ осциллографа.

\begin{figure}[h!]
	\pic{0.9\textwidth}{4_5_2}
	\caption{Схема установки для исследования намагничивания образцов}
	\figmark{experimental environment}
\end{figure}

Замкнутая кривая, возникающая на экране, воспроизводит в некотором масштабе (различном для осей $X$ и $Y$) петлю гистерезиса. Чтобы придать этой кривой количественный смысл, необходимо установить масштабы изображения, т. е. провести калибровку каналов $X$ и $Y$ ЭО. Для этого, во-первых, надо узнать, каким напряжениям (или токам) соответствуют амплитуды сигналов, видимых на экране, и, во-вторых, каким значениям $B$ и $H$ соответствуют эти напряжения (или токи).

Напряжение, пропорциональное току в намагничивающей обмотке, подаётся на горизонтальную ось $X$. Если ручка усиления оси $X$ в положении калибр, цену деления по горизонтальной оси получим, разделив цену деления в вольтах
на сопротивление $R_0$ в амперах. Значения поля $H$ рассчитываются по теореме о циркуляции (см.~\chaptereqref{H-toroid}). Для
дополнительной проверки с помощью амперметра нужно закоротить обмотку $N_O$, так как катушка с ферромагнитным образцом является нелинейным элементом, ток в ней не имеет синусоидальной формы, и связать амплитуду тока с показаниями амперметра, измеряющим эффективное значение, можно только с большой ошибкой.

При закороченной обмотке $N_O$ показания эффективного тока, умноженные на $2\sqrt{2}$, дадут удвоенное значение амплитуды тока, подаваемого на ось $X$, соответствующего длине горизонтальной развёртки. Калибровка вертикальной
оси, как правило, не нужна.Но она может проводиться с помощью сигнала, снимаемого через делитель напряжения с обмотки 12,6~В понижающего трансформатора (рис.~\figref{experimental environment}). Вольтметр $V$ может достаточно точно измерить напряжение $U_\text{эфф}$, подаваемого на вход осциллографа. После этого можно сравнить показания осциллографа и вольтметра.

Величина индукции $B$ рассчитывается по формуле \eqref{4.5.4}.

Постоянную времени $RC-$цепочки можно определить экспериментально. С обмотки 6,3 В на вход интегрирующей цепочки подаётся синусоидальное напряжение $U_\text{вх}$. На вход $Y$ осциллографа или цифрового вольтметра поочерёдно подаются сигналы со входа ($U_\text{вх}$) и выхода ($U_\text{вых} = U_C$) $RC-$цепочки. Измерив амплитуды этих сигналов, можно рассчитать постоянную времени $\tau = RC$ (формула \eqref{4.5.5}).
\begin{equation}
	\eqmark{4.5.6}
	RC = \frac{U_\text{вх}}{\omega U_\text{вых}}
\end{equation}

\begin{lab:task}

В работе предлагается при помощи ЭО исследовать предельные петли гистерезиса и начальные кривые намагничивания для нескольких ферромагнитных образцов; определить магнитные характеристики материалов, чувствительность каналов $X$ и $Y$ осциллографа и постоянную времени $\tau$ интегрирующей цепочки.
\begin{enumerate}
\item
Для наблюдения петли гистерезиса на экране ЭО соберите схему согласно рис.~\figref{experimental environment}. Подготовьте приборы к работе.

\item
Подберите ток питания в намагничивающей обмотке и коэффициен­ты усиления ЭО так, чтобы предельная петля гистерезиса занимала большую часть экрана (при этом ширина петли при увеличении тока практически не меняется).

\item
Зарисуйте на кальку предельную петлю и оси координат; отметьте на осях деления шкалы. Укажите (на кальке!) материал образца, значения коэффициентов усиления $K_x$ и $K_y$ осциллографа, ток $I_\text{эфф}$ в намагничивающей обмотке, параметры тороида.

\item
Снимите начальную кривую намагничивания: плавно уменьшая ток намагничивания до нуля, отмечайте на кальке вершины наблюдаемых частных петель. Эти вершины лежат на начальной кривой намагничи­вания.

\item
Восстановите предельную петлю. Измерьте на экране (это точнее, чем по кальке) двойные амплитуды для коэрцитивной силы $[2x(c)]$ и индукции насыщения $[2y(s)]$. Запишите соответствующие значения $K_x$ и $K_y$.

\item
Повторите измерения пп.~2~--~5 для двух других катушек.

\item
Прокалибруйте горизонтальную ось ЭО. Для этого отключите на­магничивающую обмотку $N_O$ от цепи и снимите длину развёртки по оси $X$ при токе $I_\text{эфф}$, близком к току насыщения петли гистерезиса.

\item
Для проверки калибровки вертикальной оси ЭО подключите вольтметр и осциллограф к делителю 1:100 (рис.~\figref{experimental environment}) и сравните показания вольтметра и осциллографа при развёртке вертикали почти на весь экран. Оцените погрешность.

\item
Определите $\tau$ -- постоянную времени $RC-$цепочки (см.~\eqref{4.5.6}). Для этого
разберите цепь тороида и подайте на вход $RC-$цепочки синусоидальное напряжение с обмотки 6,3~В трансформатора.

\item
Измерьте отношение $U_\text{вх} / U_\text{вых}$ с помощью осциллографа и вольтметра. Рассчитайте на месте постоянную времени $\tau=RC$ по формуле \eqref{4.5.5} и сравните с расчётом через параметры $R_\text{и}$ и $C_\text{и}$, указанные на установке.

\item
Запишите параметры RС-цепочки, амперметра, вольтметра и значе­ние RC.
\end{enumerate}

\tasksection{Обработка результатов}
\begin{enumerate}
\item
Сравните экспериментальное значение $\tau$ с расчётом через параметры $R_\text{и}$ и $C_\text{и}$, указанные на установке.

\item
Рассчитайте напряжённости поля $H$ в тороиде, поставив в соответствие деления по оси $X$ величине поля в $\text{А} / \text{м}$.

\item
Рассчитайте коэрцитивную силу $H_c$, используя измеренное значение $[2x(c)]$.

\item
Рассчитайте $B_s$ по формуле (?6?),
\todo[author = Andrew]{уточнить на какую формулу ссылка}
взяв значения $R_\text{и}$ и $C_\text{и}$, указанные на установке. Укажите на кальках масштабы для предельных петель: $H~[\text{А} / \text{м}]$ на одно деление; $B_s~[\text{Тл}]$ на одно деление рассчитайте по формуле (?6?), взяв вместо $U_\text{вых}$ .Оцените начальные и максимальные значения $\mu_\text{диф}$ по основным кривым намагничивания.

\item
Оцените погрешности. Сведите результаты в таблицу:
\begin{center}
\begin{tabular}{|c|c|c|c|}
\hline
$\text{Ампл.}$ & $Fe-Ni$ & $Fe-Si$ & $\text{Феррит}$ \\
\hline
$H_c, \frac{\text{А}}{\text{м}}$ & $\frac{\text{эксп.}}{\text{табл.}}$ & & \\
$B_s, \text{Тл}$ & & & \\
$\mu_\text{диф}$ & & & \\
\hline
\end{tabular}
\end{center}

\end{enumerate}
\end{lab:task}

\begin{lab:questions}
\item
При какой форме образцов, помещённых в однородное магнитное поле, их намагниченность постоянна по всему объёму?

\item
Почему для наблюдения петли гистерезиса используются образцы в видетора, а не в виде стержня?

\item
Почему при калибровке горизонтальной оси осциллографа необходимо от­ключать намагничивающую обмотку?

\item
Оцените погрешность, которая возникает при измерении индукции $B$, если измерительная катушка неплотно надета на образец; например, если образец занимает всего половину охватываемой ею площади.
\end{lab:questions}
\begin{lab:literature}
\item
\emph{Сивухин~Д.В.} Общий курс физики. Т.~III. Электричество. --- М.:~Наука, 1983. \S\S~74, 79.

\item
\emph{Калашников~С.Г.} Электричество. --- М.:~Наука, 1977. \S\S~110, 111, 119.

\item
\emph{Кингсеп~А.С., Локшин~Г.Р., Ольхов~О.А.} Основы Физики. Т.~1.~Механика, электричество и магнетизм, колебания и волны, волновая оптика. --- М.:~Физматлит, 2001. Ч. II, гл. 5, \S~5.3.
\end{lab:literature}
