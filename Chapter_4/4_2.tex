\lab{Закон Кюри--Вейсса}

\begin{lab:aim}
изучение температурной зависимости магнитной восприимчивости ферромагнетика
выше точки Кюри.
\end{lab:aim}

\begin{lab:equipment}
катушка самоиндукции с образцом из гадолиния, термостат, частотомер,
цифровой вольтметр, $LC$-автогенератор, термопара медь-константан.
\end{lab:equipment}

Перед выполнением работы рекомендуется ознакомиться с
пп.~\ref{sec:paramagnetism}, \ref{sec:ferromagnetism} теоретического
введения к разделу.

Вещества с отличными от нуля атомными магнитными моментами обладают
парамагнитными свойствами. Внешнее магнитное поле ориентирует магнитные моменты,
которые в~отсутствие поля располагались в~пространстве хаотичным образом.
При повышении температуры~$T$ возрастает дезориентируещее действие теплового
движения частиц, и магнитная восприимчивость парамагнетиков убывает
по \term{закону Кюри} --- обратно пропорционально температуре (см.
вывод формулы \chaptereqref{chi-para} во Введении к разделу):
\begin{equation}%1
	\chi\propto \frac{1}{T}.
	\eqmark{curi}
\end{equation}

\begin{wrapfigure}{r}{0.4\textwidth}
    \pic{0.4\textwidth}{Chapter_4/4_2_1}
    \caption{Зависимость обратной величины магнитной восприимчивости
от~температуры}
    \figmark{muT}
\end{wrapfigure}

Некоторые парамагнетики при понижении температуры испытывают
фазовый переход в ферромагнитное состояние. При малых температурах
тепловое движение всё меньше препятствует магнитным моментам атомов
ориентироваться в~одном направлении при сколь угодно слабом внешнем поле.
Благодаря обменному взаимодействию, имеющему электростатическую природу,
в ферромагнетиках самопроизвольное упорядочение магнитных моментов
возможно при довольно высоких температурах. Температуру фазового перехода
парамагнетик--ферромагнетик называют \term{температурой Кюри} $\Theta_К$.
Температурная зависимость магнитной восприимчивости у ферромагнетиков
выше точки Кюри с удовлетворительной точностью
описывается \term{законом Кюри--Вейсса} (см. также вывод
\chaptereqref{Kuri-Weiss}):
\begin{equation}%2
    \chi\propto \frac{1}{T-\Theta_p},
    \eqmark{curi-veis}
\end{equation}
где $\Theta_p$~--- параметр с размерностью температуры, называемый
иногда \important{парамагнитной точкой Кюри}. Величина $\Theta_p$ близка к
$\Theta_К$, но не совпадает с ней.

% \todo{Переделать рисунок}

Непосредственно вблизи $\Theta_К$ закон Кюри--Вейсса
\eqref{curi-veis} нарушается. На практике наблюдается зависимость, изображенная
на (рис.~\figref{muT}).

\experiment

В работе изучается температурная зависимость~$\chi(T)$ гадолиния при
температурах выше точки Кюри. Выбор материала определяется тем, что его точка
Кюри лежит в диапазоне комнатных температур.

\begin{figure}[h]
    \pic{0.9\textwidth}{Chapter_4/4_2_2}
    \caption{Схема экспериментальной установки}
    \figmark{setup}
\end{figure}


Схема установки для проверки закона Кюри-Вейсса показана на рис.~\figref{setup}.
Исследуемый ферромагнитный образец (гадолиний) расположен внутри пустотелой
катушки самоиндукции, которая служит индуктивностью колебательного контура,
входящего в состав $LC$-авто\-генератора (генератора колебаний с
самовозбуждением).

Гадолиний является хорошим проводником электрического тока, а рабочая частота
генератора достаточно велика (${\sim}50$~кГц), поэтому для уменьшения вихревых
токов образец изготовлен из мелких кусочков размером около 0,5~мм.
Катушка~1 с образцом помещена в стеклянный сосуд~2, залитый трансформаторным
маслом. Масло предохраняет образец от окисления и способствует ухудшению
электрического контакта между отдельными частичками образца. Кроме того, оно
улучшает тепловой контакт между образцом и термостатируемой (рабочей)
жидкостью~3 в термостате. Ртутный термометр~4 используется для приближённой
оценки температуры. Температура образца регулируется с~помощью термостата~5.

Коэффициент самоиндукции катушки $L$ пропорционален магнитной 
проницаемости $\mu$ заполняющей его среды (почему?): $L\propto \mu$.
Тогда разность самоиндукций катушки с~образцом $L$ и без него~$L_0$ 
будет пропорциональна восприимчивости образца~$\chi$:
\begin{equation*}%3
	L-L_0\propto \mu - 1 = \chi.
\end{equation*}
При изменении индуктивности образца меняется период колебаний автогенератора:
\begin{equation*}%4
	\tau=2\pi\sqrt{LC},
	\eqmark{4}
\end{equation*}
где $C$~--- ёмкость контура автогенератора.
Период колебаний в отсутствие образца определяется самоиндукцией пустой катушки:
\begin{equation*}
	\tau_0=2\pi\sqrt{L_0 C}.
\end{equation*}
Отсюда находим
\begin{equation*}
	L-L_0 \propto \tau^2-\tau_0^2
\end{equation*}
и, следовательно,
\begin{equation}%6
	\eqmark{6}
	\chi\propto \tau^2-\tau_0^2.
\end{equation}

Из формул \eqref{curi-veis} и \eqref{6} следует, что закон Кюри-Вейсса
справедлив, если выполнено соотношение
\begin{equation}
\frac{1}{\tau^2-\tau_0^2} \propto T-\Theta_p.
\end{equation}

Измерения проводятся в интервале температур от~$14\oC$ до~$40\oC$.
С~целью экономии времени следует начинать измерения с~низких температур.

% Для охлаждения образца используется холодная водопроводная вода, циркулирующая
% вокруг сосуда с рабочей жидкостью (дистиллированной водой); рабочая жидкость
% постоянно перемешивается.

% Величина стабилизируемой температуры задаётся на дисплее~5 термостата. Для
% нагрева служит внутренний электронагреватель, не показанный на рисунке. Когда
% температура рабочей жидкости в сосуде приближается к заданной, непрерывный режим
% работы нагревателя автоматически переходит в импульсный (нагреватель то
% включается, то выключается)~--- начинается процесс стабилизации температуры.

Температура исследуемого образца всегда несколько отличается от температуры
воды в термостате. После того как вода достигла заданной
температуры, идёт медленный процесс выравнивания температур образца и воды.
Разность их температур контролируется с помощью медно-константановой
термопары~6, один из спаев которой находится в тепловом
контакте с образцом, а другой погружён в воду. Чувствительность термопары
указана на установке. Рекомендуется измерять период колебаний автогенератора в
тот момент, когда указанная разность температур становится меньше $0,5\oC$
(более точному измерению температур мешают паразитные ЭДС, возникающие в цепи
термопары).

\begin{lab:task}

\taskpreamble{В работе предлагается исследовать зависимость периода колебаний
автогенератора от температуры сердечника катушки и по результатам измерений
определить парамагнитную точку Кюри гадолиния.}

\item Подготовьте приборы к работе.

\item Зная температурный коэффициент термопары (указан на установке),
оцените допустимую ЭДС термопары, если допустимая разность
температур образца и рабочей жидкости $\Delta T=0,5\oC$;

		\item Исследуйте зависимость периода колебаний~$LC$-генератора от
температуры образца, отмечая период колебаний~$\tau$
		по частотомеру, а температуру $T$~--- по показаниям дисплея и цифровому
вольтметру ($\Delta U$ с~учётом знака). Термопара
		подключена так, что при знаке~<<+>> на табло вольтметра температура
образца выше температуры рабочей жидкости.

		Измерения проведите в диапазоне от~$14\oC$ до~$40\oC$.

\item Запишите период колебаний $\tau_0$ без образца, указанный на установке.

\item Закончив измерения, охладите термостат, 
руководствуясь техническим описанием.

\tasksection{Обработка результатов}

		\item Рассчитайте температуру $T$ образца с учётом показаний термопары.
Постройте график зависимости $1/(\tau^2-\tau_0^2)=f(T)$.
		Экстраполируя полученную прямую к~оси абсцисс, определите парамагнитную
точку Кюри~$\Theta_p$ для гадолиния.

        \item По участку, отклоняющемуся от линейной зависимости, оцените
        положение ферромагнитной точки Кюри $\Theta_К$.
		\item Оцените погрешности эксперимента и сравните результат с~табличным.

\end{lab:task}


\begin{lab:questions}

    \item Почему приращение индуктивности $L-L_0$ пропорционально магнитной
    восприимчивости образца $\chi$?
	\item Как объяснить явления пара- и диамагнетизма с молекулярной точки
зрения?

	\item Чем отличаются пара- и ферромагнетики в отсутствие магнитного поля?

% 	\item Сформулируйте общий физический принцип, объясняющий явление
% диамагнетизма.

	\item Качественно изобразите на одном графике $B(H)$ для пара-, диа- и
ферромагнетика.

	\item Какой вклад в измерение магнитной восприимчивости вносит проводимость
гадолиния? Как связан этот вклад с размером крупинок, частотой и удельной
проводимостью? Зависит ли этот вклад от температуры? Оцените этот вклад для
крупинок размером 0,5~мм.
% Обратите внимание,
% на рис.~\figref{muT} кривая около точки Кюри идёт практически горизонтально, не
% опускаясь до нуля.
\item Оцените  глубину проникновения магнитного поля около точки Кюри, считая
магнитную проницаемость и проводимость ферромагнетика примерно такими же,
как у железа.
\end{lab:questions}


\begin{lab:literature}
	\item \SivuhinIII~--- \S\S~74, 79.

	\item \Kalashnikov~--- \S\S~110, 111, 119.
    
    \item \Kirichenko~--- \S\S~4.2, 9.3
\end{lab:literature}

