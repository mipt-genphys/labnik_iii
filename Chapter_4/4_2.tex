\lab{3.4.2}{Закон Кюри--Вейсса}

\cel{изучение температурной зависимости магнитной восприимчивости ферромагнетика выше точки Кюри.}

\prin{катушка самоиндукции с образцом из гадолиния, термостат, частотомер, цифровой вольтметр, $LC$-автогенератор,
термопара медь-константан.}

Вещества с отличными от нуля атомными магнитными моментами обладают парамагнитными свойствами. Внешнее магнитное поле
ориентирует магнитные моменты, которые в~отсутствие поля располагались в~пространстве хаотичным образом.

При повышении температуры $T$ возрастает дезориентируещее действие теплового движения частиц, и магнитная
восприимчивость парамагнетиков убывает, в~простейшем случае (в~постоянном магнитном поле)~--- по закону Кюри:
\be1
\chi=\frac{C}{T},
\ee
где $C$~--- постоянная Кюри.

Для парамагнитных веществ, которые при понижении температуры становятся ферромагнитными, формула~(\r{1}) должна быть
видоизменена. Эта формула показывает, что температура~$T=0$ является особой точкой температурной кривой, в~которой
$\chi$ неограниченно возрастает.

\rpic{33mm}{4_2_1}{\cct Зависимость обратной величины магнитной восприимчивости от~температуры}{1}

При $T\to 0$ тепловое движение всё меньше препятствует магнитным моментам атомов ориентироваться в~одном направлении при
сколь угодно слабом внешнем поле. В~ферромагнетиках~--- под влиянием обменных сил~--- это происходит при понижении
температуры не до абсолютного нуля, а до температуры Кюри $\Theta$. Оказывается, что у~ферромагнетиков закон Кюри должен
быть заменён законом Кюри--Вейсса:
\be2
\chi\sim \frac{1}{T-\Theta_p},
\ee
где $\Theta_p$~--- температура, близкая к температуре Кюри.

Эта формула хорошо описывает поведение ферромагнитных веществ после их перехода в~парамагнитную фазу при заметном
удалении температуры от~$\Theta$, но недостаточно точна при $T\approx \Theta$.

Иногда для уточнения формулы~(\r{2}) вводят вместо одной две температуры Кюри, одна из которых описывает точку фазового
перехода~--- ферромагнитная точка Кюри $\Theta$, а другая является параметром в формуле~(\r{2})~--- парамагнитная точка
Кюри~--- $\Theta_p$ (\p{1}).

В нашей работе изучается температурная зависимость~$\chi(T)$ гадолиния при температурах выше точки Кюри. Выбор материала
определяется тем, что его точка Кюри лежит в интервале комнатных температур.

\eo Схема установки для проверки закона Кюри--Вейсса показана на \p{2}. Исследуемый ферромагнитный образец (гадолиний)
расположен внутри пустотелой катушки самоиндукции, которая служит индуктивностью колебательного контура, входящего в
состав $LC$-автогенератора. Автогенератор собран на полевом транзисторе КП-103 и смонтирован в виде отдельного блока.

\fcpic[0.99]{4_2_2}{Схема экспериментальной установки}{2}

Гадолиний является хорошим проводником электрического тока, а рабочая частота генератора достаточно велика
(${\sim}50$~кГц), поэтому для уменьшения вихревых токов образец изготовлен из мелких кусочков размером около 0,5~мм.
Катушка~1 с~образцом помещена в~стеклянный сосуд~2, залитый трансформаторным маслом. Масло предохраняет образец от
окисления и способствует ухудшению электрического контакта между отдельными частичками образца. Кроме того, оно улучшает
тепловой контакт между образцом и термостатируемой (рабочей) жидкостью~3 в~термостате. Ртутный термометр 4 используется
для приближённой оценки температуры. Температура образца регулируется с~помощью термостата.% \mbox{ТЖ-ТС-01НМ.}

Магнитная восприимчивость образца~$\chi$ определяется по изменению самоиндукции катушки. Обозначив через~$L$
самоиндукцию катушки с~образцом и через $L_0$~--- её самоиндукцию в~отсутствие образца, получим
\be3
(L-L_0)\sim\chi.
\ee
При изменении самоиндукции образца меняется период колебаний автогенератора:
\be4
\vartau=2\pi\sqrt{LC},
\ee
где $C$~--- ёмкость контура автогенератора.

Период колебаний в отсутствие образца определяется самоиндукцией пустой катушки:
\be5
\vartau_0=2\pi\sqrt{L_0 C}.
\ee

Из (\r{4}) и (\r{5}) имеем
\[
(L-L_0)\sim (\vartau^2-\vartau_0^2).
\]
Таким образом,
\be6
\chi\sim (\vartau^2-\vartau_0^2).
\ee

Из формул (\r{2}) и (\r{6}) следует, что закон Кюри--Вейсса справедлив, если выполнено соотношение
\be7
\frac{1}{\chi}\sim(T-\Theta_p)\sim\frac{1}{(\vartau^2-\vartau_0^2)}.
\ee

Измерения проводятся в интервале температур от~$14\C$ до~$40\C$. С~целью экономии времени следует начинать измерения
с~низких температур.

Для охлаждения образца используется холодная водопроводная вода, циркулирующая вокруг сосуда с рабочей жидкостью
(дистиллированной водой); рабочая жидкость постоянно перемешивается.

Величина стабилизируемой температуры задаётся на дисплее~5 термостата. Для нагрева служит внутренний электронагреватель,
не показанный на рисунке. Когда температура рабочей жидкости в сосуде приближается к заданной, непрерывный режим работы
нагревателя автоматически переходит в импульсный (нагреватель то включается, то выключается)~--- начинается процесс
стабилизации температуры.

Температура исследуемого образца всегда несколько отличается от температуры дистиллированной воды в сосуде. После того
как вода достигла заданной температуры, идёт медленный процесс выравнивания температур образца и воды. Разность их
температур контролируется с помощью медно-константановой термопары~6 и цифрового вольтметра. Один из спаев термопары
находится в тепловом контакте с образцом, а другой погружён в воду. Концы термопары подключены к цифровому вольтметру.
Чувствительность термопары указана на установке. Рекомендуется измерять период колебаний автогенератора в тот момент,
когда указанная разность температур становится меньше 0,5\C (более точному измерению температур мешают паразитные ЭДС,
возникающие в цепи термопары).

\zad

В работе предлагается исследовать зависимость периода колебаний автогенератора от температуры сердечника катушки и по
результатам измерений определить парамагнитную точку Кюри гадолиния.

\dopop

\n Подготовьте приборы к работе.

\n Оцените допустимую ЭДС термопары, если допустимая разность температур образца и рабочей жидкости $\D T=0,5\C$, а
постоянная термопары $k=24$~град/мВ;

\n Исследуйте зависимость периода колебаний~$LC$-генератора от температуры образца, отмечая период колебаний~$\vartau$
по частотомеру, а температуру $T$~--- по показаниям дисплея и цифровому вольтметру ($\D U$ с~учётом знака). Термопара
подключена так, что при знаке~<<+>> на табло вольтметра температура образца выше температуры рабочей жидкости.

Проведите измерения в диапазоне от $14\C$ до $40\C$ через $2\C$.

Запишите период колебаний $\vartau_0$ без образца, указанный на установке.

\n Закончив измерения, охладите термостат, руководствуясь техническим описанием.


\Obrab

\n Рассчитайте температуру $T$ образца с учётом показаний термопары. Постройте график зависимости
$1/(\vartau^2-\vartau_0^2)=f(T)$. Экстраполируя полученную прямую к~оси абсцисс, определите парамагнитную
точку~Кюри~$\Theta_p$ для гадолиния.

\n Оцените погрешности эксперимента и сравните результат с~табличным.

{\small

\kv

\n Как объяснить явления пара- и диамагнетизма с молекулярной точки зрения?

\n Чем отличаются пара- и ферромагнетики в отсутствие магнитного поля?

\n Сформулируйте общий физический принцип, объясняющий явление диамагнетизма.

\n Качественно изобразите на одном графике $B(H)$ для пара-, диа- и ферромагнетика.

\nz Какой вклад в магнитную восприимчивость образца вносит проводимость гадолиния? Как связан этот вклад с размером
крупинок, частотой и удельной проводимостью? Зависит ли этот вклад от температуры? Оцените этот вклад для крупинок
размером 0,5~мм.

\lit

\n \emph {Сивухин Д.В.} Общий курс физики. Т.~III. Электричество.~--- М.: Наука, 1983. \S\S~74, 79.

\n \emph{Калашников С.Г.} Электричество. --- М.: Наука, 1977. \S\S~110, 111, 119.

}
