\lab{Закон Кюри--Вейсса}

\begin{lab:aim}
	изучение температурной зависимости магнитной восприимчивости ферромагнетика выше точки Кюри.
\end{lab:aim}

\begin{lab:equipment}
	катушка самоиндукции с образцом из гадолиния, термостат, частотомер, цифровой вольтметр, $LC$-автогенератор,
	термопара медь-константан.
\end{lab:equipment}

Вещества с отличными от нуля атомными магнитными моментами обладают парамагнитными свойствами. Внешнее магнитное поле
ориентирует магнитные моменты, которые в~отсутствие поля располагались в~пространстве хаотичным образом.

При повышении температуры $T$ возрастает дезориентируещее действие теплового движения частиц, и магнитная
восприимчивость парамагнетиков убывает, в~простейшем случае (в~постоянном магнитном поле)~--- по закону Кюри:
\begin{equation}%1
	\chi=\frac{C}{T},
	\eqmark{curi}
\end{equation}
где $C$~--- постоянная Кюри.

Для парамагнитных веществ, которые при понижении температуры становятся ферромагнитными, формула~(\eqref{curi}) должна быть
видоизменена. Эта формула показывает, что температура~$T=0$ является особой точкой температурной кривой, в~которой
$\chi$ неограниченно возрастает.

\begin{wrapfigure}{r}{0.4\textwidth}
	\pic{0.38\textwidth}{4_2_1}
	\caption{Зависимость обратной величины магнитной восприимчивости от~температуры}
	\figmark{muT}
\end{wrapfigure}

При $T\to 0$ тепловое движение всё меньше препятствует магнитным моментам атомов ориентироваться в~одном направлении при
сколь угодно слабом внешнем поле. В~ферромагнетиках~--- под влиянием обменных сил~--- это происходит при понижении
температуры не до абсолютного нуля, а до температуры Кюри $\Theta$. Оказывается, что у~ферромагнетиков закон Кюри должен
быть заменён законом Кюри--Вейсса:
\begin{equation}%2
	\chi\sim \frac{1}{T-\Theta_p},
	\eqmark{curi-veis}
\end{equation}
где $\Theta_p$~--- температура, близкая к температуре Кюри.

Эта формула хорошо описывает поведение ферромагнитных веществ после их перехода в~парамагнитную фазу при заметном
удалении температуры от~$\Theta$, но недостаточно точна при $T\approx \Theta$.

Иногда для уточнения формулы~(\eqref{curi-veis}) вводят вместо одной две температуры Кюри, одна из которых описывает точку фазового
перехода~--- ферромагнитная точка Кюри $\Theta$, а другая является параметром в формуле~(\eqref{curi-veis})~--- парамагнитная точка
Кюри~--- $\Theta_p$ (\figref{muT}).

В нашей работе изучается температурная зависимость~$\chi(T)$ гадолиния при температурах выше точки Кюри. Выбор материала
определяется тем, что его точка Кюри лежит в интервале комнатных температур.

\experiment

Схема установки для проверки закона Кюри--Вейсса показана на \figref{setup}. Исследуемый ферромагнитный образец (гадолиний)
расположен внутри пустотелой катушки самоиндукции, которая служит индуктивностью колебательного контура, входящего в
состав $LC$-автогенератора. Автогенератор собран на полевом транзисторе КП-103 и смонтирован в виде отдельного блока.

\begin{figure}
	\pic{0.8\textwidth}{4_2_2}
	\caption{Схема экспериментальной установки}
	\figmark{setup}
\end{figure}

Гадолиний является хорошим проводником электрического тока, а рабочая частота генератора достаточно велика
(${\sim}50$~кГц), поэтому для уменьшения вихревых токов образец изготовлен из мелких кусочков размером около 0,5~мм.
Катушка~1 с~образцом помещена в~стеклянный сосуд~2, залитый трансформаторным маслом. Масло предохраняет образец от
окисления и способствует ухудшению электрического контакта между отдельными частичками образца. Кроме того, оно улучшает
тепловой контакт между образцом и термостатируемой (рабочей) жидкостью~3 в~термостате. Ртутный термометр 4 используется
для приближённой оценки температуры. Температура образца регулируется с~помощью термостата.% \mbox{ТЖ-ТС-01НМ.}

Магнитная восприимчивость образца~$\chi$ определяется по изменению самоиндукции катушки. Обозначив через~$L$
самоиндукцию катушки с~образцом и через $L_0$~--- её самоиндукцию в~отсутствие образца, получим
\begin{equation}%3
	(L-L_0)\sim\chi.
\end{equation}

При изменении самоиндукции образца меняется период колебаний автогенератора:
\begin{equation}%4
	\tau=2\pi\sqrt{LC},
	\eqmark{4}
\end{equation}
где $C$~--- ёмкость контура автогенератора.

Период колебаний в отсутствие образца определяется самоиндукцией пустой катушки:
\begin{equation}%5
	\tau_0=2\pi\sqrt{L_0 C}.
	\eqmark{5}
\end{equation}

Из (\eqref{4}) и (\eqref{5}) имеем
\begin{equation*}
	(L-L_0)\sim (\tau^2-\tau_0^2).
\end{equation*}

Таким образом,
\begin{equation}%6
	\eqmark{6}
	\chi\sim (\tau^2-\tau_0^2).
\end{equation}

Из формул (\eqref{curi-veis}) и (\eqref{6}) следует, что закон Кюри--Вейсса справедлив, если выполнено соотношение
\begin{equation}
	\frac{1}{\chi}\sim(T-\Theta_p)\sim\frac{1}{(\tau^2-\tau_0^2)}.
\end{equation}

Измерения проводятся в интервале температур от~$14^\circ$~C до~$40^\circ$~C. С~целью экономии времени следует начинать измерения
с~низких температур.

Для охлаждения образца используется холодная водопроводная вода, циркулирующая вокруг сосуда с рабочей жидкостью
(дистиллированной водой); рабочая жидкость постоянно перемешивается.

Величина стабилизируемой температуры задаётся на дисплее~5 термостата. Для нагрева служит внутренний электронагреватель,
не показанный на рисунке. Когда температура рабочей жидкости в сосуде приближается к заданной, непрерывный режим работы
нагревателя автоматически переходит в импульсный (нагреватель то включается, то выключается)~--- начинается процесс
стабилизации температуры.

Температура исследуемого образца всегда несколько отличается от температуры дистиллированной воды в сосуде. После того
как вода достигла заданной температуры, идёт медленный процесс выравнивания температур образца и воды. Разность их
температур контролируется с помощью медно-константановой термопары~6 и цифрового вольтметра. Один из спаев термопары
находится в тепловом контакте с образцом, а другой погружён в воду. Концы термопары подключены к цифровому вольтметру.
Чувствительность термопары указана на установке. Рекомендуется измерять период колебаний автогенератора в тот момент,
когда указанная разность температур становится меньше $0,5^\circ$~C (более точному измерению температур мешают паразитные ЭДС,
возникающие в цепи термопары).

\begin{lab:task}

	В работе предлагается исследовать зависимость периода колебаний автогенератора от температуры сердечника катушки и по
	результатам измерений определить парамагнитную точку Кюри гадолиния.
	
	\begin{enumerate}
		\item Подготовьте приборы к работе.
		
		\item Оцените допустимую ЭДС термопары, если допустимая разность температур образца и рабочей жидкости $\Delta T=0,5^\circ$~C, а
		постоянная термопары $k=24$~град/мВ;
		
		\item Исследуйте зависимость периода колебаний~$LC$-генератора от температуры образца, отмечая период колебаний~$\tau$
		по частотомеру, а температуру $T$~--- по показаниям дисплея и цифровому вольтметру ($\Delta U$ с~учётом знака). Термопара
		подключена так, что при знаке~<<+>> на табло вольтметра температура образца выше температуры рабочей жидкости.
		
		Проведите измерения в диапазоне от $14^\circ$~C до $40^\circ$~C через $2^\circ$~C.
		
		Запишите период колебаний $\tau_0$ без образца, указанный на установке.
		
		\item Закончив измерения, охладите термостат, руководствуясь техническим описанием.
	
	\end{enumerate}
	
	\labsection{Обработка}
	\begin{enumerate}
	
		\item Рассчитайте температуру $T$ образца с учётом показаний термопары. Постройте график зависимости $1/(\tau^2-\tau_0^2)=f(T)$. 
		Экстраполируя полученную прямую к~оси абсцисс, определите парамагнитную точку~Кюри~$\Theta_p$ для гадолиния.
	
		\item Оцените погрешности эксперимента и сравните результат с~табличным.
	
	\end{enumerate}

\end{lab:task}


\begin{lab:questions}

	\item Как объяснить явления пара- и диамагнетизма с молекулярной точки зрения?
	
	\item Чем отличаются пара- и ферромагнетики в отсутствие магнитного поля?
	
	\item Сформулируйте общий физический принцип, объясняющий явление диамагнетизма.
	
	\item Качественно изобразите на одном графике $B(H)$ для пара-, диа- и ферромагнетика.
	
	\item Какой вклад в магнитную восприимчивость образца вносит проводимость гадолиния? Как связан этот вклад с размером 
	крупинок, частотой и удельной проводимостью? Зависит ли этот вклад от температуры? Оцените этот вклад для крупинок
	размером 0,5~мм.
\end{lab:questions}


\begin{lab:literature}
	\item \emph {Сивухин Д.В.} Общий курс физики. Т.~III. Электричество.~--- М.: Наука, 1983. \S\S~74, 79.
	
	\item \emph{Калашников С.Г.} Электричество. --- М.: Наука, 1977. \S\S~110, 111, 119.
\end{lab:literature}

