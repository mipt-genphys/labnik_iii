\lab{Свободные колебания в электрическом контуре}

\begin{lab:aim}
исследование свободных колебаний в электрическом колебательном контуре.
\end{lab:aim}

\begin{lab:equipment}
генератор импульсов, электронное реле, магазин сопротивлений, магазин ёмкостей, 
катушка индуктивности, электронный осциллограф, универсальный измерительный мост.
\end{lab:equipment}

Перед выполнением работы необходимо ознакомиться с п.~\ref{sec:free}
введения к данному разделу.

В работе исследуются свободные колебания, возбуждаемые в колебательном $RLC$-контуре. 
Конденсатор контура заряжается поступающими от специального генератора короткими 
одиночными импульсами, после каждого из которых в контуре возникают свободные 
затухающие колебания. По картине колебаний, наблюдаемой на экране электронного осциллографа,
можно определить период свободных колебаний в контуре и коэффициент затухания,
и вычислить параметры колебательного контура. Характеристики контура 
можно также определить, рассматривая затухающие колебания на фазовой плоскости системы 
на экране осциллографа.

\experiment 

%\begin{wrapfigure}[15]{r}{0.45\linewidth}
%	\pic{0.4\textwidth}{Chapter_2/2_4_1}
%	\caption{Схема установки для наблюдения затухающих колебаний на фазовой
%плоскости}
%	\figmark{fig1}
%\end{wrapfigure}

На рис.~\figref{fig1} приведена схема для исследования свободных колебаний в контуре, 
содержащем постоянную индуктивность~$L$ c активным сопротивлением~$R_L$, 
а также переменные сопротивление~$R$ и ёмкость~$C$, 
выбираемые из соответствующих «магазинов». Картина колебаний напряжения 
на ёмкости наблюдается на экране двухканального осциллографа.
Выходные разъёмы схемы и входы каналов осциллографа собраны на 
отдельной панели~П.

Для периодического возбуждения колебаний в контуре используется генератор 
импульсов. С выхода генератора сигналы поступают на колебательный контур 
через электронное реле, которое содержит диодный \emph{тиристор} и ограничительный 
резистор. Тиристор без управляющего электрода представляет собой 
полупроводниковый ключ, открывающийся при напряжении на нём, превышающем 
пороговое значение, и закрывающийся при любом напряжении другого знака. 
Благодаря этому генератор отключается от колебательного контура после каждого 
импульса, и внутреннее сопротивление генератора не влияет на процессы 
в колебательном контуре.

Каждый импульс заряжает конденсатор~$C$, после чего в контуре возникают 
свободные затухающие колебания. Напряжение на конденсаторе поступает на вход 
канала 1($X$) осциллографа, а напряжение на сопротивлении~$R$, пропорциональное 
току $I$ в контуре, поступает на вход канала~2($Y$). В~двухканальном режиме
работы осциллографа на экране отображаются оба сигнала одновременно. Наблюдение 
фазовой плоскости осуществляется в режиме <<$X$--$Y$>>. Входное сопротивление 
осциллографа велико ($\approx 1$~МОм), поэтому его влиянием на контур 
можно пренебречь.

Картину колебаний можно представить не только в координатах $(t,\, U_C)$
(см. рис.~\chapterfigref{fig2}а), но и в координатах $(U_C, dU_C/dt)$~--- на фазовой плоскости 
(см. рис.~\chapterfigref{fig2}б). В этих координатах кривая незатухающих колебаний 
(при $\gamma=0$) имеет вид эллипса, а картина затухающих колебаний (при $\gamma > 0$) 
представляет собой сворачивающуюся спираль. При этом движение точки, отвечающей 
состоянию системы, по фазовой траектории происходит со временем по часовой стрелке, 
что соответствует принятому правилу знаков: ток положителен, 
если заряд конденсатора растёт.

\begin{figure}[h!]
    \centering
	\pic{0.95\textwidth}{Chapter_2/2_4_2}
	\caption{Схема установки для исследования свободных колебаний}
	\figmark{fig1}
\end{figure}

\begin{lab:task}


\taskpreamble{В работе предлагается исследовать зависимость периода свободных колебаний 
контура от ёмкости, зависимость логарифмического декремента затухания от сопротивления, 
а также определить критическое сопротивление и добротность контура. Правила выполнения 
работы изложены в техническом описании (ТО), расположенном на столе установки.}

\tasksection{I. Подготовка приборов к работе}

\item Для настройки генератора импульсов подключите сначала выходные разъёмы генератора, 
отмеченные на схеме цифрами~0 и~1, непосредственно к входу канала 1(X) осциллографа 
через соответствующие клеммы на панели~П.

\item Установите на генераторе длительность импульсов $\tau=5$~мкс, частоту повторения 
импульсов $\nu_0=300$~Гц, выходное напряжение $V = 30$~В.

\item Включите и настройте осциллограф, руководствуясь техническим описанием.

\item Включите генератор, подберите развёртку и усиление канала 1(Х) осциллографа 
в измерительном режиме так, чтобы на экране умещалось несколько импульсов. Проверьте, 
совпадает ли период повторения импульсов на генераторе с периодом повторения импульсов, 
измеренным при помощи горизонтальной шкалы осциллографа. При несовпадении периодов 
прокалибруйте горизонтальную шкалу осциллографа по известному периоду повторения импульсов. 
Зарегистрируйте наблюдаемую картину. 

\item Выключите генератор и осциллограф.

\tasksection{II. Измерение периодов свободных колебаний}

\item Соберите полную схему, представленную на рис.~\figref{fig1}. 
Установите на магазине сопротивлений величину $R = 0$; на магазине ёмкостей~--- 
величину $C = 0,02$~мкФ. Включите генератор и осциллограф. Установите на 
осциллографе канал~1 в рабочее положение при выключенном канале~2. 

\item Подберите частоту внутренней развёртки так, чтобы сигнал на экране 
был представлен на временном интервале, слегка перекрывающем длительность 
импульсов генератора~$\tau$.

\item Измерьте по шкале экрана осциллографа длительность нескольких периодов
колебаний контура. Рассчитайте период~$T$ свободных затухающих 
колебаний.

\item \label{325-p2} Изменяя ёмкость~$C$ от $0,02$~мкФ до~$0,9$~мкФ, проведите измерения 
8--10 значений периодов~$T$ свободных колебаний.


\tasksection{III. Измерение критического сопротивления и декремента затухания}

\item \label{325-p8} Для данной в работе индуктивности~$L$ рассчитайте ёмкость~$C$, при которой
собственная частота колебаний контура составляет $\nu_0=5\;кГц$.
Для выбранных значений~$L$ и~$C$ рассчитайте критическое сопротивление
контура $R_\text{кр}$. Найденное в эксперименте значение $R_{\text{кр}}$
может отличаться от рассчитанного, так как величина~$L$ задана приближённо.

\todo[inline,author=Popov]{Почему бы не измерить значение 
индуктивности $LCR$-метром сразу?}

\item Установите на магазине ёмкость, близкую к рассчитанной. Увеличивая
сопротивление~$R$ от нуля до~$R_\text{кр}$, наблюдайте картину затухающих
колебаний на экране осциллографа. Зафиксируйте сопротивление магазина, при
котором колебательный режим переходит в апериодический. Сравните значения
найденного экспериментально и рассчитанного значения~$R_\text{кр}$.
\todo[inline,author=Popov{В ворд-файле этот пункт удален. Зачем?}

\item Установите сопротивление $R \approx 0,1 R_\text{кр}$. Получите
на экране картину затухающих колебаний. Для расчёта логарифмического декремента
затухания~$\Theta$ по формуле~\chaptereqref{2.27} измерьте амплитуды,
разделенные целым числом периодов. Расчёт будет тем точнее, чем больше
отличаются друг от друга измеряемые амплитуды, однако минимальная амплитуда не должна быть
меньше слишком мала (не менее половины большого деления шкалы экрана).

\item \label{325-p9} Повторите измерения для 6--8 значений~$R$ 
в интервале $(0,1-0,3)R_\text{кр}$.

\tasksection{IV. Свободные колебания на фазовой плоскости}

\item Для наблюдения свободных затухающих колебаний на фазовой плоскости дополнительно 
к каналу 1(X) осциллографа включите канал 2(Y). Нажмите кнопку CH2 INV и выполните 
настройку канала синхронизации по ТО к работе. 
Установите раздельное по высоте расположение осциллограмм на экране, подберите 
амплитуды сигналов и частоту развёртки каналов так, чтобы оба сигнала были представлены 
на одном временном интервале, слегка превышающем длительность $\tau$ импульсов генератора. 
Зарегистрируйте осциллограмму и сравните её с рис.~\chapterfigref{fig2}а.
\todo[inline, author=Popov]{Непонятный пункт! Зачем нажимать кнопку CH2 INV 
(это инвертирует сигнал)? Что такое настройка "канала синхронизации" и зачем она нужна 
(по идее всё должно уже быть синхронизировано)?
Зачем располагать осциллограммы раздельно по высоте, ведь на рисунке они наложены 
друг на друга?} 

\item Для наблюдения свободных затухающих колебаний на фазовой плоскости переведите осциллограф 
в режим <<$X$--$Y$>> (режим внешней развертки). Изменяя чувствительность каналов, подберите 
удобный масштаб спирали. Если спираль стягивается в точку, уменьшите длительность $\tau$ импульсов 
генератора. Зарегистрируйте осциллограмму и сравните её с рис.~\chapterfigref{fig2}б. 
При том же значении ёмкости~$C$, что и в п.~\ref{325-p8}, наблюдайте за изменением спирали 
при увеличении сопротивления~$R$ от~$0,1R_{\text{кр}}$ до~$0,3R_{\text{кр}}$. 
Измерьте максимумы~$X_k$ и~$X_{k+n}$ отклонения витков спирали по 
одной из осей фазовой плоскости, разделённые целым числом периодов $n$, 
для $6$--$8$ значений~$R$ рабочего диапазона и определите логарифмический декремент 
затухания $\Theta$.

\item Выключите питание установки, после чего отсоедините катушку от цепи. 
С помощью универсального измерительного моста ($LCR$-метра) измерьте
омическое сопротивление~$R_L$ и индуктивность~$L$ катушки на частотах $50$~Гц,
$1$~кГц и~$5$~кГц. Почему результат измерения омического
сопротивления катушки зависит от частоты? При необходимости измерьте с помощью 
$LCR$-метра другие параметры установки.


\tasksection{V. Обработка результатов}

	\item Рассчитайте экспериментальные значения периодов по результатам
измерений п.~\ref{325-p2} и теоретические по формуле \chaptereqref{2.7}. 
Постройте график $T_\text{эксп} = f(T_\text{теор})$.
\todo[inline,author=Popov]{Построение графика экспериментальных значений от теоретических ---
довольно не наглядный способ представления результатов. Лучше так:}
Рассчитайте экспериментальные значения периодов по результатам
измерений п.~\ref{325-p2}. Оцените погрешности. 
Постройте график $T_{эксп}(C)$. Нанесите на этот график теоретическую кривую, 
рассчитанную по формуле~\chaptereqref{2.7} и сравните результаты.

\item 
Используя данные измерений в п.~\ref{325-p9}, рассчитайте величину критического 
сопротивления~$R_{кр}$ по формуле \chaptereqref{2.30}. Результат \emph{косвенных измерений в невоспроизводимых условиях} представьте в виде: $R_{кр}=\left<R_{кр}\right> \pm \Delta R_{кр}$, 
где $\Delta R_{кр}$ --- случайная погрешность, вычисленная с учётом коэффициента Стьюдента 
$t_{n,\alpha}$ для $n$ измерений при доверительном уровне $\alpha=0,05$.

\todo[inline,author=Popov]{Применение коэффициентов Стьюдента в данном случае не корректно!
У нас имеется функциональная зависимость и её параметры нужно получать по МНК. 
Необходимо оставить пункт в старой редакции:}

\item Рассчитайте значения логарифмического декремента затухания~$\Theta$ (п.~12) 
и сопротивления контура $R_{\Sigma}=R+R_L$.

Постройте график в координатах $1/\Theta^2 =f[1/R^2_{\Sigma}]$. Убедитесь 
в линейности зависимости и по наклону прямой определите критическое 
сопротивление~$R_\text{кр}$ (см. формулы \chaptereqref{2.30}). 
%С помощью равенств
%\chaptereqref{2.6}, \chaptereqref{2.23}, \chaptereqref{2.27}, и
%\chaptereqref{2.38} 
%и приняв обозначения $1/\Theta^2~=~Y$, $1/
%R^2_\text{конт}~=~X$, покажите, что $R_\text{кр} = 2\pi\sqrt{\Delta Y / \Delta
%X}$.

\item Рассчитайте теоретическое значение $R_\text{кр} = 2\pi\sqrt{L / C}$
и сравните его с измеренным.

\item Рассчитайте добротность контура~$Q$ для максимального и минимального
значений~$\Theta$ по картине затухающих колебаний и сравните с расчётом~$Q$
через параметры контура~$R$, $L$ и~$C$  \chaptereqref{2.35}.

\item Рассчитайте добротность~$Q=\pi/\Theta$ по спирали на фазовой плоскости.

\item Сведите результаты эксперимента и их погрешности в таблицу:
\begin{center}\small
\begin{tabular}{|c|c|c|c|c|c|c|c|}
\hline
& \multicolumn{3}{c|}{$R_{\text{кр}}$} &  & \multicolumn{3}{c|}{$Q$} \\
\cline{2-4}
\cline{6-8}
$L_{\text{кат}}$ & $\text{Теор.}$ & $\text{Подбор}$ & $\text{Граф.}$ & $R$ &
$\text{Теор.}$ & $f(\Theta)$ & $\text{Спираль}$  \\
\hline
& & & & $\text{max}$ & & &  \\
& & & & $\text{min}$ & & &  \\
\hline
\end{tabular}
\end{center}

\item Сравните результаты и оцените, какой из методов определения~$R_\text{кр}$ и~$Q$ 
точнее.

\end{lab:task}


\begin{lab:questions}
	\item Что такое собственная частота, добротность, логарифмический декремент
затухания колебательного контура?
    \item Получите выражение для добротности колебательного контура через его 
    параметры $R$, $L$, $C$.
    \item Как отличаются периоды свободных колебаний с затуханием и без затухания?
	\item Что называют фазовой плоскостью?
	\item  Как определить добротность по картине колебаний в фазовой плоскости?
%	\item  Возможно ли вызвать резонанс в колебательном контуре при помощи
%периодических импульсов?
\end{lab:questions}


\begin{lab:literature}
	\item \SivuhinIII~--- \S\S~123--124.
	\item \KingLokOlh~--- \S~2.4.
	\item \Kirichenko~--- \S~17.1.
%	\item \Kalashnikov~--- \S\S~207--210.
	\item \textit{Горелик Г.С.} Колебания и волны. --- М.\,: Физматлит, 1959.~--- 
	Гл.~II, \S~3.
\end{lab:literature}
