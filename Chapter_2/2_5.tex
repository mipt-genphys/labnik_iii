\lab{Свободные колебания в электрическом контуре}

\begin{lab:aim}
исследование свободных колебаний в колебательном контуре.
\end{lab:aim}

\begin{lab:equipment}
генератор импульсов, электронное реле, магазин сопротивлений, магазин ёмкостей,
катушка индуктивности, электронный осциллограф, универсальный мост.
\end{lab:equipment}

Исследуемый колебательный контур состоит из индуктивности $L$, ёмкости $C$ и
резистора $R$ (рис.~\figref{fig1}). Конденсатор контура заряжается короткими
одиночными импульсами, после каждого из которых в контуре возникают свободные
затухающие колебания. Подав напряжение с конденсатора на осциллограф, можно по
изображению на экране осциллографа определить период свободных колебаний в
контуре, исследовать их затухание и определить основные параметры колебательного
контура.

%\begin{figure}[h]
\begin{wrapfigure}[15]{r}{0.45\linewidth}
	\pic{0.4\textwidth}{Chapter_2/2_4_1}
	\caption{Схема установки для наблюдения затухающих колебаний на фазовой
плоскости}
	\figmark{dumped oscillations}
\end{wrapfigure}
%\end{figure}

Картину колебаний можно представить не только в координатах ($U$,~$t$)
(рис.~\figref{fig2}а), 
но и в координатах $(U,\,I)$ на так называемой \emph{фазовой плоскости} 
(рис.~\figref{fig2}б). В этих координатах кривая
незатухающих колебаний (при $\gamma~=~0$) имеет вид эллипса, а картина реальных
затухающих колебаний представляет собой сворачивающуюся спираль.

Принципиальная схема подключения осциллографа для изучения колебаний на фазовой
плоскости изображена на рис.~\figref{dumped oscillations}. На вертикальный вход
осциллографа подаётся напряжение $U_C$ с конденсатора, а на
горизонтальный~---~напряжение с резистора $U_R$ ($U_R ~ {\sim} ~ I~  {\sim}~
dq/dt ~ {\sim}~  dU_C/dt$).

\experiment  На рис.~\figref{free oscillations} приведена схема для исследования
свободных колебаний в контуре, содержащем постоянную индуктивност $L$ и
переменные ёмкость $C$ и сопротивление $R$. Картина колебаний наблюдается на
экране осциллографа.


Для периодического возбуждения колебаний в контуре используется генератор
импульсов. С выхода генератора сигналы поступают на колебательный контур через
электронное реле, которое содержит диодный тиристор $D$ и ограничительный
резистор $R_1$. Тиристор без управляющего электрода представляет собой
полупроводниковый ключ, открывающийся при напряжении на нём выше порогового, и
закрывающийся при любом напряжении другого знака. Благодаря этому генератор
отключается от колебательного контура после каждого импульса, и внутреннее
сопротивление генератора не влияет на процессы в колебательном контуре.

Каждый импульс заряжает конденсатор $C$, после чего в контуре возникают
свободные затухающие колебания. Входное сопротивление осциллографа велико,
поэтому его влиянием на контур можно пренебречь.
\todo[author=Tiffani]{Картинка отличается от той, что в ворд-файле}
\begin{figure}[h!]
	\pic{0.9\textwidth}{Chapter_2/2_4_2}
	\caption{Схема установки для исследования свободных колебаний}
	\figmark{free oscillations}
\end{figure}

\begin{lab:task}

\taskpreamble{В работе предлагается исследовать зависимость периода свободных
колебаний контура от ёмкости, зависимость логарифмического декремента затухания от
сопротивления, определить критическое сопротивление и добротность контура.}

\tasksection{I. Подготовка приборов к работе}


	\item Соберите схему согласно рис.~\figref{free oscillations}. Подключите
выход генератора через реле к магазину ёмкостей таким образом, чтобы можно было
менять ёмкость в интервале $0 - 1$ мкФ.

	\item Установите на генераторе длительность импульсов $5$~мкс, а частоту
повторения импульсов $\nu_0 = 100$~Гц ($T = 0,01$~c).

	\item Настройте осциллограф, руководствуясь техническим описанием,
расположенным на установке.

\tasksection{II. Измерение периодов свободных колебаний}

	\item Установите на магазине сопротивлений величину $R = 0$; на магазине
ёмкостей~---~величину $C = 0,02$~мкФ.

	\item Подберите частоту развёртки осциллографа, при которой расстояние между
импульсами, поступающими с генератора, занимает почти весь экран.

	\item Измерьте расстояние, которое занимают несколько полных периодов $n$.
Проверьте, совпадают ли период повторения импульсов на генераторе с периодом
повторения импульсов, измеренным при помощи горизонтальной шкалы осциллографа.
При несовпадении периодов прокалибруйте горизонтальную шкалу осциллографа по
известному периоду повторения импульсов.

	\item Измерьте на экране осциллографа расстояние $x$, которое занимают
несколько полных периодов $n$. Рассчитайте период свободных колебаний контура.
Малые расстояния $x$ можно увеличить кнопкой растяжки развёртки.

	\item Изменяя ёмкость от $0,02$~мкФ до $0,9$~мкФ, проведите измерения
периодов свободных колебаний (8~--~10 значений).

\tasksection{III. Измерение критического сопротивления и декремента затухания.}


	\item Для данной в работе индуктивности рассчитайте ёмкость $C$, при которой
собственная частота колебаний контура $\nu_0 = 1/(2\pi\sqrt{LC})$ составляет
$5$~кГц. Для выбранных значений $L$ и $C$ рассчитайте критическое сопротивление
контура $R_\text{кр} = 2\sqrt{L/C}$. \chaptereqref{2.38}

	\item Установите на магазине ёмкость, близкую к рассчитанной. Увеличивая
сопротивление $R$ от нуля до  $R_\text{кр}$, наблюдайте картину затухающих
колебаний на экране осциллографа. Зафиксируйте сопротивление магазина, при
котором колебательный режим переходит в апериодический. Сравните значения
найденного экспериментально и рассчитанного значения  $R_\text{кр}$.

	\item Установите сопротивление $R \simeq 0,1 R_\text{кр}$ (эксп.). Получите
на экране картину затухающих колебаний. Для расчёта логарифмического декремента
затухания $\Theta$ по формуле \chaptereqref{2.27} измерьте амплитуды,
разделенные целым числом периодо $n$. Расчёт будет тем точнее, чем больше
отличаются друг от друга измеряемые амплитуды, а минимальная не должна быть
меньше $5 - 6$~мм.

	\item Повторите измерения для 6~--~8 значений $R$ в интервал
$(0,1-0,3)R_\text{кр}$ .

\tasksection{IV. Свободные колебания на фазовой плоскости.}


	\item Для наблюдения затухающих колебаний на фазовой плоскости подайте на
вход «Х» осциллографа напряжение с магазина сопротивлений. Переведите
осциллограф в режим измерения «X--Y». Изменяя чувствительность каналов,
подберите масштаб спирали. При том же значении ёмкости что и в п.~10, наблюдайте
за изменением спирали при увеличении сопротивления от $0,1R_\text{кр}$  до
$0,3R_\text{кр}$. Для определения $\Theta$~измерьте радиусы витков спирали,
разделённые целым числом периодов $n$, для одного-двух значений $R$ на каждом
краю рабочего диапазона.

	\item Отсоедините катушку от цепи. С помощью измерителя $LCR$ измерьте
омическое сопротивление $R_L$ и индуктивность $L$ катушки на частотах $50$~Гц,
$1$~кГц и $5$~кГц. Подумайте, почему результат измерения омического
сопротивления катушки зависит от частоты.

\tasksection{V. Обработка результатов.}

	\item Рассчитайте экспериментальные значения периодов по результатам
измерений (ч. II) и теоретические по формуле $\nu_0 = 1/(2\pi\sqrt{LC})$
\chaptereqref{2.7}. Постройте график $T_\text{эксп} = f(T_\text{теор})$.

	\item Рассчитайте значения $\Theta$ (п.~12) и $R_\text{конт}$ (сопротивление
контура состоит из сопротивления магазина $R$ и омического сопротивления катушки
$R_L$).

Постройте график в координатах $1/\Theta^2 =f[1/(R^2_\text{конт})]$. Определите
критическое сопротивление $R_\text{кр}$  по наклону прямой. С помощью равенств
\chaptereqref{2.6}, \chaptereqref{2.23}, \chaptereqref{2.27}, и
\chaptereqref{2.38} и приняв обозначения $1/\Theta^2~=~Y$, $1/
R^2_\text{конт}~=~X$, покажите, что $R_\text{кр} = 2\pi\sqrt{\Delta Y / \Delta
X}$.

	\item Рассчитайте теоретическое значение $R_\text{кр} = 2\pi\sqrt{L / C}$
(\chaptereqref{2.38}) и сравните его с измеренным.

	\item Рассчитайте добротность контура $Q$ для максимального и минимального
значений $\Theta$ по картине затухающих колебаний и сравните с расчётом $Q$
через параметры контура $R$, $L$  и $C$  \chaptereqref{2.34}.

	\item Рассчитайте добротность $Q$ по спирали.

	\item Сведите результаты эксперимента в таблицу:

\begin{center}
\begin{tabular}{|c|c|c|c|c|c|c|c|}
\hline
& \multicolumn{3}{c|}{$R_{\text{кр}}$} &  & \multicolumn{3}{c|}{$Q$} \\
\cline{2-4}
\cline{6-8}
$L_{\text{кат}}$ & $\text{Теор.}$ & $\text{Подбор}$ & $\text{Граф.}$ & $R$ &
$\text{Теор.}$ & $f(\Theta)$ & $\text{Спираль}$  \\
\hline
& & & & $\text{max}$ & & &  \\
& & & & $\text{min}$ & & &  \\
\hline
\end{tabular}
\end{center}


	\item Оцените погрешности и сравните результаты. Какой из методов
определения $R_\text{кр}$ и $Q$ точнее?

\end{lab:task}


\begin{lab:questions}
	\item Что такое собственная частота, добротность, логарифмический декремент
затухания колебательного контура?
	\item Что называют фазовой плоскостью колебаний?
	\item  Как определить логарифмический декремент затухания по картине
колебаний в фазовой плоскости?
	\item  Возможно ли вызвать резонанс в колебательном контуре при помощи
периодических импульсов?
\end{lab:questions}


\begin{lab:literature}

	\item \SivuhinIII~--- \S\S~122~--~124.

	\item \textit{Калашников~С.Г.} Электричество. --~М.:~Физматлит, 2008.
\S\S~207~--~210.

\end{lab:literature}
