\lab{Резонанс напряжений в последовательном контуре}

\begin{lab:aim}
     исследование резонанса напряжений в последовательном колебательном контуре
с изменяемой ёмкостью, включающее получение амплитудно-частотных и
фазово-частотных характеристик, а также определение основных параметров контура.
\end{lab:aim}

\begin{lab:equipment}
	генератор сигналов, источник напряжения, нагрузкой которого является
последовательный колебательный контур с переменной ёмкостью, двулучевой
осциллограф, цифровые вольтметры.
\end{lab:equipment}

%\begin{lab:warning}
% \warning{
Перед выполнением работы следует изучить основы теории электрических  колебаний
по вводной части Раздела II настоящего сборника и/или рекомендованной в нём
литературе. Необходимые дополнения будут приведены ниже.
% }
%\end{lab:warning}

\experiment
Схема экспериментального стенда для изучения резонанса напряжений в
последовательном колебательном контуре показана на рис.~\figref{exp}.
Синусоидальный сигнал от генератора поступает на вход \important{управляемого
напряжением источника напряжения} (см., например, [1]), собранного на
операционном усилителе, питание которого осуществляется встроенным
блоком-выпрямителем от сети~$\sim220$~В (цепь питания на схеме не показана).
Источник напряжения, обладающий по определению нулевым внутренним
сопротивлением, фактически обеспечивает с высокой точностью постоянство
амплитуды сигнала~$\varepsilon=\varepsilon_0\cos(\omega t+\varphi_0)$ на
меняющейся по величине нагрузке~---~последовательном колебательном контуре,
изображенном на рис.~\figref{exp} в виде эквивалентной схемы. Источник
напряжения, колебательный контур и блок питания заключены в отдельный корпус с
названием <<Резонанс напряжений>>, отмеченный на рисунке штриховой линией.
\begin{figure}[h!]
	\pic{0.9\textwidth}{Chapter_2/2_2_2}
	\caption{Схема экспериментального стенда}
	\figmark{exp}
\end{figure}
\todo[author=Tiffani]{Рис. 14 не соответствует ворд-файлу, проверить!}
На корпусе имеются коаксиальные разъёмы <<Вход>>, <<U1>> и <<U2>>, а также
переключатель магазина ёмкостей $C_n$ с указателем номера $n=1,~2,~\ldots~7.$
Величины ёмкостей $C_n$ указаны в табличке на крышке корпуса. Напряжение
$\varepsilon$ на контуре через разъём <<$U_1$>> попадает одновременно на канал
1~осциллографа и вход 1-го цифрового вольтметра. Напряжение на конденсаторе
$U_C$ подаётся через разъём <<$U_2$>> одновременно на канал 2 осциллографа и
вход 2-го цифрового вольтметра.

Колебательный контур нашей установки собран из стандартных элементов,
используемых в современных радиоэлектронных цепях. Известно, что в реальных
конденсаторах и, особенно, в катушках индуктивности происходят необратимые
потери энергии, обусловленные различными причинами. К ним относятся: утечки и
диэлектрические потери в конденсаторах, вихревые токи и потери на
перемагничивание в сердечниках катушек индуктивности, омические потери в
проводниках, растущие с частотой за счёт скин-эффекта, и некоторые другие. Рост
потерь приводит к увеличению действительных частей комплексных сопротивлений
элементов контура, и, значит, к изменению его резонансных свойств, в частности,
к уменьшению добротности.

Потери в элементах контура зависят как от частоты, так и от амплитуды тока
(напряжения), температуры и ряда других факторов, например, от вида диэлектрика
конденсатора, сердечника катушки и т.д. От перечисленных факторов в общем случае
зависят и основные параметры контура: индуктивность $L,$ ёмкость $C$ и суммарное
активное сопротивление~$R_{\Sigma}.$

В нашем контуре катушка индуктивности $L$ на ферритовом каркасе обладает малым
сопротивлением по постоянному току и высокой собственной резонансной
частотой~$f_r\ge1,3$~МГц. В общем случае каждая катушка, помимо индуктивности
$L,$ характеризуется также собственной (межвитковой) ёмкостью $C_L$ и активным
сопротивлением потерь $R_L,$ распределёнными по её длине. Принимается, что эти
величины сосредоточены в отдельных элементах схемы, образующих с индуктивностью
$L$ замкнутую колебательную цепь с собственной резонансной частотой
$f_r=1/2\pi\sqrt{LC_L}.$ Вследствие влияния ёмкости $C_L$ при измерении на
частоте $f$ определяется не истинная индуктивность $L,$ а эффективное значение
индуктивности $L_{eff}=L/(1-f^2/f_r^2),$ которое может заметно отличаться от
истинной величины $L.$ В рабочем диапазоне частот нашего контура выполняется
неравенство $f<<f_r,$ так что в эквивалентной схеме контура на рис.~\figref{exp}
индуктивность представлена своим истинным значением $L$ и активным
сопротивлением $R_L.$

Полипропиленовые конденсаторы, входящие в комплект магазина ёмкостей
$C_n,(n=1,~2,~\ldots~7),$ в рабочем диапазоне частот имеют пренебрежимо малые
собственные индуктивности (менее $10^{-5}$~мГн на 1~см общей длины обкладок и
выводов) и относительно малые активные потери. Для оценки возможного вклада
активных потерь в конденсаторах в общий импеданс контура воспользуемся
представлением конденсатора с ёмкостью~$C$ последовательной эквивалентной
схемой, показанной на рис.~\figref{equiv capacitor}а.
\begin{figure}[h!]
	\pic{0.9\textwidth}{Chapter_2/2_2_3}
	\caption{Последовательная эквивалентная схема конденсатора с потерями.}
	\figmark{equiv capacitor}
\end{figure}
\todo[author=Tiffani]{Нет рис 15 из ворд-файла в формате pic, проверить!}
На этой схеме $R_S$~---~так называемое эквивалентное последовательное
сопротивление (ЭПС), обусловленное, главным образом, электрическим
сопротивлением материала обкладок и выводов конденсатора и контактов между ними,
а также потерями в диэлектрике. Из эквивалентной схемы и векторной диаграммы к
ней (рис.~\figref{equiv capacitor}б) видно, что активные потери в конденсаторе,
пропорциональные, как известно, косинусу угла $\varphi$ сдвига фаз между током и
напряжением на ёмкости, убывают с ростом $\varphi$ и, соответственно, с
уменьшением угла $\delta=90^{\circ}-\varphi.$ Потери в конденсаторе принято
характеризовать величиной $\tg\delta,$ обычно приводимой в документации к
изделию. Из рисунка 2 и закона Ома при этом получаем выражение для ЭПС на
циклической частоте $\omega=2\pi f$ в виде
\begin{equation}\eqmark{2.2.1}
R_S=\dfrac{U_{RS}}{I}=\dfrac{U_{RS}}{\omega CU_{CS}}=\dfrac{1}{\omega C}\tg\delta.
\end{equation}

Конденсаторы магазина ёмкостей~$C_n$ в интересующем нас диапазоне частот имеют
$\tg\delta<10^{-3},$ что является очень хорошим (низким!) показателем для
конденсаторов с твёрдым диэлектриком.

В колебательный контур наших установок добавлен постоянный резистор $R$ (см.
рис.~\figref{exp}), снижающий его добротность. Это сделано для упрощения
процедур получения и обработки резонансных кривых. Таким образом, суммарное
активное сопротивление контура принимается равным
\begin{equation}\eqmark{2.2.2}
	R_{\Sigma}=R+R_L+R_S.
\end{equation}
Добротность контуров, тем не менее, остаётся достаточно высокой, чтобы можно
было пользоваться формулами \chaptereqref{2.34} и \chaptereqref{2.54}, в которых
надо учитывать суммарное активное сопротивление контура:
\begin{equation}\eqmark{2.2.3}
	Q=\rho/R_{\Sigma}=\omega_0L/R_{\Sigma}=1/\omega_0CR_{\Sigma}>>1.
\end{equation}
Сильное неравенство в \eqref{2.2.4} в рабочем диапазоне частот выполняется для
всех контуров, используемых в работе.

Для импедансов ёмкости $Z_C,$ индуктивности $Z_L$ и последовательного контура
$Z=Z_L+R+Z_C$ с учётом \eqref{2.2.1}, \eqref{2.2.2} получаем выражения:
\begin{equation}\eqmark{2.2.4}
Z_C=R_S-i\dfrac{1}{\omega C}, \qquad Z_L=R_L+i\omega L,\qquad
Z=R_{\Sigma}+i\left(\omega L-\dfrac{1}{\omega C}\right).
\end{equation}
\important{Комплексные амплитуды} тока в контуре $\vec I=\vec E/Z$ и напряжений
на сопротивлении $\vec U_R=R\vec I,$ ёмкости $\vec U_C=Z_C\vec I$ и
индуктивности $\vec U_L=Z_L\vec I$ при нулевой начальной фазе $\varphi_0$
напряжения на контуре $\vec \varepsilon=\varepsilon_0e^{i\varphi_0}$ по формулам
\chaptereqref{2.48} с учётом \eqref{2.2.1}~--~\eqref{2.2.4} удобно записать в
виде
\begin{subequations}
	\eqmark{2.2.5}
		\begin{equation}
			\eqmark{2.2.5a}
			\vec I=\dfrac{\vec
U_R}{R}=\dfrac{\varepsilon_0}{R_{\Sigma}}\dfrac{1}{
1+iQ(\omega/\omega_0-\omega_0/\omega)}, \end{equation}
		\begin{equation}
			\eqmark{2.2.5b}
			\vec
U_C=-iQ\varepsilon_0\dfrac{\omega_0}{\omega}\dfrac{1+i\tg\delta}{
1+iQ(\omega/\omega_0-\omega_0/\omega)},
		\end{equation}
		\begin{equation}
			\eqmark{2.2.5c}
			\vec
U_L=iQ\varepsilon_0\dfrac{\omega}{\omega_0}\dfrac{1-iR_L/\rho}{
1+iQ(\omega/\omega_0-\omega_0/\omega)}.
		\end{equation}
\end{subequations}
Эти формулы уточняют результаты \chaptereqref{2.48} и \chaptereqref{2.54},
полученные без учёта потерь в конденсаторе и в катушке индуктивности. Отметим,
однако, что указанными потерями, представленными мнимыми добавками в числителях
формул \eqref{2.2.5b} и \eqref{2.2.5c}, при условиях $Q>>1$ и
$tg\delta>>10^{-3}$ можно пренебречь. В то же время вклад этих потерь в
суммарное активное сопротивление контура $R_{\Sigma},$ примерно равный вблизи
резонанса $R_L+\rho\tg\delta,$ можно будет оценить только по результатам
эксперимента.

Основные особенности резонанса в последовательном контуре, называемого также
\important{резонансом напряжений} из-за увеличения в $Q$ раз напряжений на
ёмкости $U_C$ и индуктивности $U_L$ по отношению к внешнему напряжению
$\varepsilon_0,$ следуют из анализа формул \eqref{2.2.5}. Подробно этот вопрос
рассматривался в п.~3.1.
\todo[author=Tiffani]{уточнить номер пункта (п.~3.1?)}

Наибольший практический интерес представляет случай, когда отклонение
$\delta\omega=\omega-\omega_0$ частоты внешней ЭДС от собственной частоты
контура $\omega_0$ удовлетворяет сильному неравенству
%\setcounter{equation}{5}
\begin{equation}\eqmark{2.2.6}
	|\Delta\omega|<<\omega_0.
\end{equation}
При этом в первом порядке малости по относительной расстройке частоты
$\Delta\omega/\omega_0$
\begin{equation}\eqmark{2.2.7}
\dfrac{\omega}{\omega_0}-\dfrac{\omega_0}{\omega}=
\dfrac{2\Delta\omega}{\omega_0},
\end{equation}
что позволяет упростить выражения \eqref{2.2.5} и представить вещественные части
комплексных амплитуд и соответствующих им фаз в виде:
\begin{subequations}
	\eqmark{2.2.8}
		\begin{equation}
			\eqmark{2.2.8a}
			I=\dfrac{U_R}{R}=\dfrac{\varepsilon_0}{R_{\Sigma}}\dfrac{\cos(\omega
t-\psi_I)}{\sqrt{1+(\tau\Delta\omega)^2}},\qquad
\psi_I=\arctg(\tau\Delta\omega),
		\end{equation}
		\begin{equation}
			\eqmark{2.2.8b}
			U_C=Q\varepsilon_0\dfrac{\omega_0}{\omega}\dfrac{\cos(\omega
t-\psi_C)}{\sqrt{1+(\tau\Delta\omega)^2}},\qquad
\psi_C=\arctg(\tau\Delta\omega)+\dfrac{\pi}{2}-\delta,
		\end{equation}
		\begin{equation}
			\eqmark{2.2.8c}
			U_L=Q\varepsilon_0\dfrac{\omega_0}{\omega}\dfrac{\cos(\omega
t-\psi_L)}{\sqrt{1+(\tau\Delta\omega)^2}},\qquad
\psi_L=\arctg(\tau\Delta\omega)-\dfrac{\pi}{2}+\dfrac{R_L}{\rho}.
		\end{equation}
\end{subequations}
В выражениях \eqref{2.2.8b}, \eqref{2.2.8c} мы сохранили в прежнем виде
множители с отношениями частот в амплитудах и учли только линейные по малым
величинам $R_L/\rho$ и $\delta$ поправки в фазах, причём величину $\delta$
сохранили исключительно для общности, положив её, однако, константой.

При резонансе, когда для высокодобротного контура можно положить
$\omega=\omega_0, \Delta\omega=0,$ выражения для модулей комплексных амплитуд
тока и напряжения на ёмкости, их фаз и производных фаз по частоте $\omega$
принимают вид:
\begin{subequations}
	\eqmark{2.2.9}
		\begin{equation}
			\eqmark{2.2.9a}
			I(\omega_0)=\dfrac{U_R}{R}=\dfrac{\varepsilon_0}{R_{\Sigma}},\qquad
\varphi_I(\omega_0)=0,
		\end{equation}
		\begin{equation}
			\eqmark{2.2.9b}
			U_C(\omega_0)=Q\varepsilon_0, \;
\psi_C(\omega_0)=\dfrac{\pi}{2}-\delta;\quad U_L(\omega_0)=Q\varepsilon_0, \;
\psi_L(\omega_0)=-\dfrac{\pi}{2}+\dfrac{R_L}{\rho},
		\end{equation}
		\begin{equation}
			\eqmark{2.2.9c}
			\psi_I'(\omega_0)=\psi_C'(\omega_0)=\psi_L(\omega_0)=\tau.
		\end{equation}
\end{subequations}

Из формул \eqref{2.2.8}, \eqref{2.2.9} следует, что на частоте $\omega_0,$ где
импеданс контура $Z$ становится чисто активным и равным $R_{\Sigma},$ амплитуда
тока достигает максимального значения $I_{max}=\varepsilon_0/R_{\Sigma}.$
Напряжения $U_L$ и $U_C$ на индуктивности и ёмкости на частоте $\omega_0$
находятся почти в противофазе и в $Q$ раз превышают по амплитуде напряжение
$\varepsilon$ внешней ЭДС. Напомним, однако, что максимальные (резонансные)
значения напряжений на индуктивности и ёмкости, как было показано в п.~3.1, не
строго равны $Q\varepsilon_0$ и достигаются не строго на частоте $\omega_0.$
Соответствующие относительные поправки, обусловленные множителями
$(\omega_0/\omega)^{\pm1}$ в амплитудах и малыми добавками к фазам в выражениях
для $U_C$ \eqref{2.2.8b} и $U_L$ \eqref{2.2.8c}, составляют доли $Q^{-2}.$
\todo[author=Tiffani]{уточнить номер пункта (п.~3.1?)}

При отклонении $\Delta\omega$ частоты внешней ЭДС от $\omega_0$ таком, что
выполняется условие
%\setcounter{equation}{9}
\begin{equation}\eqmark{2.2.10}
\tau\Delta\omega=\pm1,
\end{equation}
амплитуда тока $I,$ как видно из формул \eqref{2.2.8}, уменьшается в $\sqrt{2}$
раз относительно своей максимальной (резонансной) величины, а фаза $\psi_I$
изменяется на угол $\pm\pi/4.$ При условии \eqref{2.2.10}, если не учитывать
поправки в его правой части порядка $Q^{-1},$ происходят аналогичные изменения
амплитуд $U_C,~U_L $ и фаз $\psi_C,~\psi_L$ напряжений на ёмкости и
индуктивности: амплитуды уменьшаются в $\sqrt{2}$ раз, а фазы меняются на угол
$\pm\pi/4$ по отношению к своим резонансным значениям.

Величина $\delta\omega=2|\Delta\omega|=2/\tau$ представляет собой важную
характеристику колебательного контура~---~ширину резонансной кривой
$U_C(\omega)$ на уровне $U_C(\omega_0)/\sqrt{2},$ по которой можно определить
время затухания $\tau=2/\delta\omega$ и, зная резонансную частоту $\omega_0,$
найти добротность контура $Q=\omega_0/\delta\omega.$

Эти же параметры можно определить по фазово-частотной характеристике: расстояние
по оси $\omega$ между точками, в которых фаза $\varphi_C$ меняется от $-\pi/4$
до $3\pi/4,$ согласно \eqref{2.2.8b} равно $2/\tau,$ а тангенс угла наклона
функции $\varphi_C(\omega)$ в точке резонанса согласно \eqref{2.2.8c} определяет
время затухания $\tau.$

В нашем эксперименте резонансные явления в последовательном колебательном
контуре исследуются по напряжению на контуре $\varepsilon$ и напряжению на
ёмкости $U_C$ (см. рис.~\figref{exp}), а также по фазовым сдвигам между ними.
Приведённые выше формулы для других характеристик дополняют общую картину
рассматриваемых процессов. В частности, они могут быть использованы для
построения векторных диаграмм.

\begin{lab:task}
% Символом <<*>> отмечены дополнительные задачи эксперимента и, соответственно,
% обработки и представления результатов, а также контрольные вопросы повышенной
% сложности.
% \todo[author=Tiffani]{Нужно ли доп. задания эксперимента отмечать <<*>> или
% достаточно указания ``4. Дополнительное задание''?}

\item Проведите настройку экспериментального стенда по техническому описанию
(ТО), расположенному на рабочем столе. В двухканальном режиме работы
осциллографа установите общее начало отсчёта $X(t)$, $Y(t)$ вблизи левого края
средней линии экрана. В качестве синхронизующего сигнала выберите напряжение
$\varepsilon(t)$ при начальных условиях: $\varepsilon(0)=0,
\dot{\varepsilon}(0)<0.$

    \item Установите с максимальной точностью на выходе генератора (на входе
схемы) эффективное значение напряжения $\varepsilon$, заданное преподавателем (в
пределах $30\div300$~мВ). Меняя частоту $f=\omega/2\pi$ генератора, убедитесь по
осциллографу и вольтметрам, что у синусоиды $U_C(t)$ меняется амплитуда и фаза
относительно начала координат, тогда как синусоида
$\varepsilon(t)$~---~синхронизующий сигнал~---~<<привязана>> к началу отсчёта
при начальных условиях: $\varepsilon(0)=0, \dot{\varepsilon}(0)<0$~---~а её
амплитуда $\varepsilon_0$ остаётся неизменной с относительной погрешностью не
более $1\%$. Теперь можно приступить к измерениям.

    \item Для контуров с семью различными ёмкостями $C_n,$ меняя их с помощью
переключателя на блоке, измерьте резонансные частоты $f_{0n}$ и напряжения
$U_C(f_{0n})$ при установленном в п.~2 напряжении $\varepsilon$ на выходе
генератора. Регистрируйте также напряжения $\varepsilon(f_{0n}),$ игнорируя
отклонения в пределах относительной погрешности $1\%$. Приближение к резонансу
удобно наблюдать по фигуре Лиссажу на экране осциллографа в режиме X-Y (см. ТО).
При этом фигура Лиссажу представляет собой эллипс, оси которого на собственной
частоте $f_{0n}$ направлены вдоль осей X, Y. Напомним, что максимальные значения
напряжения $U_C(f)$ достигаются на частотах, несколько отличных от собственных
частот $f_{0n}.$

    \item *\important{Дополнительное упражнение.} Проделайте измерения п.~3 ещё
для двух напряжений $\varepsilon$ из интервала $30\div 300~мВ,$ существенно
отличающихся друг от друга и от напряжения, использованного в п.~3.

    \item Для контуров с двумя разными ёмкостями (по указанию преподавателя)
снимите амплитудно-частотные характеристики $U_C(f)$ для значений
$U_C(f)\ge0,6U_C(f_{0n})$ (16~--~17 точек в сумме по обе стороны от резонанса)
при том же напряжении $\varepsilon,$ что и в п.~3.

    \item Для тех же двух контуров снимите фазово-частотные характеристики
$\psi_C(f)$ для значений $U_C(f)\ge0,3U_C(f_{0n})$ (16~--~17 точек в сумме по
обе стороны от резонанса) при том же напряжении $\varepsilon,$ что и в п.~3.

		    Перед выполнением этого задания проверьте настройки осциллографа:
синхронизующий сигнал $\varepsilon(t),$ как указано в п.~2, должен был
<<привязан>> к общему началу отсчёта времени и напряжений на экране, лежащему на
оси X координатной сетки экрана. На той же оси должны располагаться нули
сигналов $\varepsilon(t)$ и $U_C(t).$ Если это не так, то следует повторить
процедуру центрировки горизонтальных осей каналов по техническому описанию.

		    Расстояние $x$ от начала отсчёта до точки первого обращения в нуль
напряжения $U_C(t)$ на участке спада характеризует разность фаз $\Delta\varphi$
сигналов $\varepsilon(t)$ и $U_C(t).$ Эта величина, выраженная в радианах,
очевидно, даётся формулой $\Delta\varphi=(x/x_0)\pi,$ где $x_0$~---~расстояние
от начала отсчёта до точки первого обращения в нуль напряжения $\varepsilon(t)$
на участке подъёма, соответствующее полупериоду этого сигнала.

\tasksection{Обработка и представление результатов}

% 		    \important{Настоятельно рекомендуется для обработки и представления
% результатов измерений использовать электронные таблицы.}

		    \item Результаты измерений п.~3 внесите в табл.~\tabref{2.2.1}.
		    \begin{center}
		        \begin{table}[tb!]
		           % \caption{\eqmark{tab:1}
		            \caption{}
		            \tabmark{2.2.1}
		            \begin{center}
		                \begin{tabular}{|c|c|c|c|c|c|c|c|c|c|c|}
		                    \hline
		                    $C_n,$ & $f_{0n},$& $U_C,$& $\varepsilon,$ & $L,$~&
$Q$& $\rho,$ & $R_{\Sigma},$~& $R_{S~\text{max}},$& $R_L,$& $I,$\\

		                    нФ & кГц & В & В & мкГн &  & Ом & Ом & Ом & Ом & мА
\\
		                    \hline
		                    $C_1$ & {---} & --- & --- & --- & --- & --- & --- &
--- & --- & --- \\
		                    \hline
		                    --- & --- & --- & --- & --- & --- & --- & --- & ---
& ---& ---\\
		                    \hline
		                    $C_7$& --- & --- & --- & --- & --- & --- & ---& ---
&--- &---\\
		                    \hline
		                    \multicolumn{4}{|c|}{ Среднее значение} & --- & & &
& &---& \\
		                    \hline
		                    \multicolumn{4}{|c|}{ Среднеквадратичная } & & & & &
& & \\
		                    \multicolumn{4}{|c|}{погрешность} &---& & & & &---&
\\
		                    \multicolumn{4}{|c|}{  среднего значения} &  & & & &
& & \\
		                    \hline
		                    \multicolumn{4}{|c|}{ Коэффициент } & & & & & & & \\
		                    \multicolumn{4}{|c|}{ Стьюдента $t_{n\alpha}$ } &
---& & & & &---& \\
		                    \multicolumn{4}{|c|}{для~$n=7,~\alpha=0,95$} &  & &
& & && \\
		                    \hline
%		                    \multicolumn{4}{|c|}{ Коэффициент Стьюдента
% $t_{n\alpha}$~для~$n=7,~\alpha=0,95$} & ---& & & & &---&\\
%		                    \hline
		                    \multicolumn{4}{|c|}{ Случайная погрешность } & ---&
& & & &---&\\

		                    \hline
		                \end{tabular}
		            \end{center}
		        \end{table}
		    \end{center}

В первый столбец табл.~\tabref{2.2.1} запишите значения ёмкостей $C_n,$
приведённые в табличке на корпусе блока <<Резонанс напряжений>>. Для каждого
значения $C_n$ по формулам вводной части и данным эксперимента проведите
\important{последовательно} расчёт $L$, $Q$, $\rho$, $R_{\Sigma}$,
$R_{S~\text{max}}~=~10^{-3}\rho$, $R_L$, $I$.

		Представьте результат проделанных в работе \important{косвенных
измерений при невоспроизводимых условиях} величин $L$ и $R_L$ в виде: $\langle
L\rangle\pm\Delta L$ и $\langle R_L\rangle\pm\Delta R_L,$ где угловыми скобками
отмечены средние значения, а символом $"\Delta"$~---~случайные погрешности
величин $L$ и $R_L.$

		Оцените относительный вклад активных потерь в конденсаторах,
представленных в табл.~\tabref{2.2.1} сопротивлением $R_{S~{max}},$ рассчитанным
для максимального значения $\tg\delta=10^{-3},$ в суммарное активное
сопротивление контура. Оцените влияние погрешностей приборов на результаты
измерений.

    \item *\important{Дополнительное упражнение.} Выполните задание п.~7 для
данных, полученных в п.~4*. Сравните с результатами п.~7. Объясните причины
расхождения результатов, если они обнаружатся.

    \item По данным измерений п.~5 постройте на одном графике
амплитудно-частотные характеристики в координатах $ f,~U_C(f)$ для выбранных
контуров. Проведите сравнительный анализ характеристик.

    \item По данным измерений п.~5 также постройте на одном графике
ампли-\\*тудно-частотные характеристики в безразмерных координатах $x\equiv
f/f_{0n}$, $y\equiv U_C(x)/U_C(1)$. По ширине резонансных кривых на уровне
$0,707$ определите добротности $Q$ соответствующих контуров. Оцените
погрешности. Сравните эти величины с данными табл.~\tabref{2.2.1} из п.~7.

    \item По данным измерений п.~6 постройте на одном графике
фазово-частот-\\*ные характеристики в координатах $x\equiv f/f_{0n},
y\equiv\varphi_C/\pi$ для выбранных контуров. По этим характеристикам определите
добротности контуров одним из двух способов: по расстоянию между точками по оси
$x,$ в которых $y$ меняется от $1/4$ до $3/4,$ равному $1/Q,$ или по формуле
$Q=0,5d\varphi_C(x)/dx$ при $x=0.$ Оцените погрешности. Сравните с результатами
пп.~7,~10.

    \item По данным табл.~\tabref{2.2.1} постройте зависимость $R_L(f_{0n})$ в
системе координат с началом в точке $(0,6f_{07};0);$ нанесите на график прямую
$\langle R_L\rangle.$ Назовите возможные причины изменения $R_L$ с частотой.

    \item По данным табл.~\tabref{2.2.1} постройте векторную диаграмму тока и
напряжений для контура с наименьшей добротностью в резонансном состоянии. Ось
абсцисс направьте по вектору $\vec \varepsilon.$ Масштаб изображения по этой оси
для напряжения сделайте в 3 раза более крупным, чем по оси абсцисс.

\end{lab:task}


\begin{lab:questions}
	\item Приведите определение добротности колебательного контура в
<<энергетических>> терминах.

	\item Объясните, почему оси эллипса на экране осциллографа (п.~3) при
условии $\omega=\omega_0$ ориентированы вдоль направлений X, Y.

	\item Как оценить добротность контура по векторной диаграмме в п.~13?

	\item По каким причинам потери в контуре зависят от частоты?

	\item Зависит ли добротность контура от амплитуды сигнала и, если зависит,
то по каким причинам?

	\item Получите выражение для частоты $\omega_m,$ на которой напряжение $U_C$
достигает максимума. Чему равно $U_C(\omega_m)?$
\end{lab:questions}


\begin{lab:literature}
	\item \textit{Титце~У., Шенк~К.} Полупроводниковая схемотехника.  – Т.~II. –
М.: ДМК Пресс,~2007.~12.1.
\end{lab:literature}
