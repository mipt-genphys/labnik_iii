\introsection{Автоколебания}

\introsubsection{Автоколебания в системах с одной степенью свободы}
В предыдущих разделах были рассмотрены свободные и вынужденные колебания в
диссипативных системах с одной степенью свободы, подчиняющихся дифференциальным
уравнениям второго порядка вида \chaptereqref{2.5} и \chaptereqref{2.45}
соответственно. Диссипация энергии, обусловленная наличием резистивных элементов
в этих системах, в первом случае приводила к затуханию колебаний, а во
втором~--- компенсировалась энергией, поступающей от
внешнего источника синусоидального напряжения (или тока). Однако колебания в
системе с одной степенью свободы при определённых условиях можно поддерживать,
используя постоянный (не синусоидальный) источник энергии, который периодически
компенсирует потери колебательной энергии по входящей в систему цепи обратной
связи. Такие системы называются \important{автоколебательными}, а протекающие в
них процессы~--- \important{автоколебаниями.} Форма и период автоколебаний
определяются свойствами самой системы, чем автоколебания существенно отличаются
от колебаний вынужденных.

Для определения условий возбуждения автоколебаний в диссипативной системе с
одной степенью свободы запишем уравнение \chaptereqref{2.5} с учётом формул
\chaptereqref{2.2}, \chaptereqref{2.3} в виде
\begin{equation}
	\eqmark{auto-1}
	\frac{dW}{dt}=-P(t),
\end{equation}
где $W={LI^2}/{2}\;+{q^2}/{2C}$~--- энергия, запасённая в колебательном контуре,
а $P(t)=R{{I}^{2}}(t)$~--- мощность потерь. Интегрируя уравнение \eqref{auto-1}
по периоду колебаний $T$, приходим к равенству
\begin{equation}
	\eqmark{auto-2}
	W=W_0-\int\limits_{0}^{T}P(t)dt,
\end{equation}
где $W_0$~---~энергия системы в некоторый момент времени, принятый за начало
отсчёта периода колебаний $T$. В обычной~--- диссипативной~--- системе $P(t)>0$,
так что автоколебания невозможны. Если же мощность потерь $P(t)=R{{I}^{2}}(t)$ в
системе \important{знакопеременна}, то подбором режима работы системы можно
обеспечить энергетический баланс:
\begin{equation}
	\eqmark{auto-3}
	\int\limits_{0}^{T}{R{{I}^{2}}(t)dt}=0,
\end{equation}
и, следовательно, возбудить в системе автоколебания.

Выполнение условия \eqref{auto-3} возможно, например, в \important{нелинейной}
колебательной системе, в которой сопротивление $R$ является функцией тока:
$R=R(I)$, причём~--- \important{знакопеременной}. Необходимым для
автоколебательного режима отрицательным «сопротивлением» ${dV}/{dI}$ на
«падающих» участках своих вольт-амперных характеристик $I(V)$, представленных на
рис.~\figref{auto-1}(а, б), обладают, например, газоразрядная лампа (а) и туннельный
диод (б). Обычно характеристики вида~(а) называют $S$-образными, а
вида~(б)~---~$N$-образными.
\begin{figure}
	\centering
	\pic{0.9\textwidth}{Chapter_2/auto-1}
	%\import{Images/Chapter_2/}{auto-1.pdf_tex}
	\caption{Вольт-амперные характеристики с «падающими» участками: а)~$S$-образным, б) $N$-образным.}
	\figmark{auto-1}
\end{figure}

Форма автоколебаний зависит от добротности колебательного контура. При большой
добротности характер протекающих процессов почти не изменяется по сравнению с
тем, как они протекали бы в системе без поступления
энергии от источника: период и форма автоколебаний будут близки к периоду и
форме собственных колебаний. Это связано с тем, что в этом случае от постоянного
источника поступает энергия, составляющая малую
долю полной энергии колебательной системы. При малой добротности контура (в
общем случае~---~колебательной системы) для поддержания колебаний от постоянного
источника должна поступать энергия, сопоставимая с энергией колебаний. В этом
случае форма автоколебаний может значительно отличаться от синусоидальной.
Наконец, в апериодической системе, в которой за период автоколебаний теряется вся
накопленная энергия, автоколебания становятся \important{релаксационными} и
могут по форме очень сильно отличаться от колебаний синусоидальных.

\introsubsection{Автоколебания в вырожденных колебательных системах}
Автоколебательная система, не содержащая одного из накопителей колебательной
энергии, называется \important{вырожденной}. Колебания в такой системе
описываются дифференциальным уравнением первого порядка и, очевидно, могут быть
только релаксационными. В рассматриваемом здесь случае электрических колебаний
речь идёт об отсутствии в системе одного из реактивных элементов: индуктивности
или ёмкости.

В качестве примера рассмотрим представленную на рис.~\figref{auto-2}(а, б) и вырожденную автоколебательную систему, содержащую источник
постоянного напряжения $U$, ёмкость $C$, сопротивление $R$ и нелинейный элемент
с $S$-образной вольт-амперной характеристикой $I_S(V)$. Как видно, в системе
отсутствует второй накопитель колебательной энергии~---~индуктивность.

\begin{figure}[h]
	\begin{minipage}[h]{0.49\linewidth}
		\centering
		\pic{0.79\textwidth}{Chapter_2/auto-2}
		%\center{\resizebox{0.79\textwidth}{!}{\import{Images/Chapter_2/}{auto-2.pdf_tex}} \\ а) Схема автоколебательной $RC$-системы.}
	\end{minipage}
	\hfill
	\begin{minipage}[h]{0.49\linewidth}
		\centering
		\pic{0.79\textwidth}{Chapter_2/auto-3}
		%\center{\resizebox{0.79\textwidth}{!}{\import{Images/Chapter_2/}{auto-3.pdf_tex}} \\ б) Вольт-амперная характеристика и нагрузочная прямая $RC$-системы.}
	\end{minipage}
	\caption{Вырожденная автоколебательная RC-система.}
	\figmark{auto-2}
\end{figure}

Уравнения, описывающие поведение этой системы релаксационного типа,
имеют вид:
\begin{equation}
	\eqmark{auto-4}
	RI+V=RI_0=U,\quad I=I_C+I_S,\quad I_C=C\frac{dV}{dt},\quad I_S=I_S\left( V \right).
\end{equation}
Следовательно,
\begin{equation}
	\eqmark{auto-5}
	RC\frac{dV}{dt}=U-V-R{{I}_{S}}\left( V \right).
\end{equation}

В стационарном состоянии, когда $dV / dt = 0$, должно выполняться равенство
\begin{equation}
	\eqmark{auto-6}
	I_S(V)=(U-V)/R.
\end{equation}
Правая часть здесь представляет \important{нагрузочную} прямую, точки
пересечения которой с вольт-амперной характеристикой ${{I}_{S}}\left( V \right)$
определяют стационарные состояния системы. На рис.~\figref{auto-2}б) параметры $U$
и $R$ выбраны так, чтобы стационарное состояние $A(V_A,~I_A)$ лежало на падающей
ветви вольт-амперной характеристики, где, как говорилось выше, возможен
автоколебательный режим. Покажем, что состояние $I_A=I(V_A)$ может быть
\important{неустойчивым}. Для этого дадим малое приращение $v$ переменной $V$ в
точке
$V_A$ и представим в линейном приближении по $v$ вольт-амперную характеристику
$I_S(V)$ вблизи стационарного состояния $V_A$:
\begin{equation}
	\eqmark{auto-7}
	I_S(V)=I_S(V_A + v)\approx I_S\left( V_A \right)+{I}_S'\left( V_A \right)v,
\end{equation}
где «штрих» означает производную по $V$, а величина ${I}_S'\left( V_A \right)$ --- так называемая «крутизна» вольт-амперной характеристики в точке $V_A$. Подстановка этого выражения в
\eqref{auto-5} приводит к уравнению
\begin{equation}
	\eqmark{auto-8}
	RC\frac{dv}{dt}=-\left[ 1+R{I}_S'\left( V_A \right) \right]v,
\end{equation}
из которого следует, что при условии
\begin{equation}
	\eqmark{auto-9}
	I_S'\left( V_A \right) <- {1}/{R}
\end{equation}
возмущение $v$ со временем экспоненциально нарастает, и, значит, стационарное
состояние $I_A=I_S(V_A)$ является неустойчивым. Система при этом будет совершать
\important{релаксационные автоколебания}, замкнутая фазовая траектория которых
на рис.~\figref{auto-2}б) состоит из плавных участков $da$ и $bc$ вольт-амперной
характеристики между напряжениями $V_1$ и $V_2$, соединённых двумя вертикальными
участками $ab$ и $cd$, показанными на рисунке штриховыми линиями. Формально
вертикальные участки соответствуют \important{скачкам тока}, которые возможны
только при отсутствии индуктивностей в системе, исходно заложенной в данной
идеализированной модели. Учёт малой «паразитной» индуктивности элементов схемы, приводящей к конечной скорости скачков, добавляет ещё одно дифференциальное уравнение первого порядка. Система, таким образом, перестаёт быть вырожденной, а в такой системе заведомо возможен автоколебательный процесс.

Аналогичным образом можно показать, что условие возбуждения автоколебаний в вырожденной $RL$-системе: соединённых последовательно с источником постоянного напряжения резистора, индуктивности и элемента с $N$-образной вольт-амперной характеристикой, --- имеет вид неравенства \eqref{auto-9}, в котором вместо $I_S'\left(V_A\right)$ стоит $I_N'\left(V_A\right)$ --- крутизна падающего  участка $N$-образной вольт-амперной характеристики в рабочей точке $\left(V_A\right)$.
