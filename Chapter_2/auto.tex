\introsection{Автоколебания}

\introsubsection{Автоколебания в системах с одной степенью свободы}
В предыдущих разделах были рассмотрены свободные и вынужденные колебания в
диссипативных системах с одной степенью свободы, подчиняющихся дифференциальным
уравнениям второго порядка вида \chaptereqref{2.8} и \chaptereqref{2.45}
соответственно. Диссипация энергии, обусловленная наличием резистивных элементов
в этих системах, в первом случае приводила к затуханию колебаний, а во
втором~--- компенсировалась энергией, поступающей от
внешнего источника синусоидального напряжения (или тока). Однако колебания в
системе с одной степенью свободы при определённых условиях можно поддерживать,
используя и \emph{постоянный} (не синусоидальный) источник энергии, 
который периодически компенсирует потери колебательной энергии по входящей 
в систему цепи \term{обратной связи}. 
Такие системы называются \term{автоколебательными}, а протекающие в
них процессы~--- \term{автоколебаниями}. Форма и период автоколебаний
определяются свойствами самой системы, чем автоколебания существенно отличаются
от обычных вынужденных колебаний

Для определения условий возбуждения автоколебаний в диссипативной системе с
одной степенью свободы воспользуемся уравнением изменения энергии 
системы \chaptereqref{JL}:
\begin{equation}
	\eqmark{auto-1}
	\frac{dW}{dt}=-P(t),
\end{equation}
где $W=\frac12 LI^2+ \frac{q^2}{2C}$~--- энергия, запасённая в колебательном контуре,
а $P(t)$~--- мощность потерь. 
Интегрируя \eqref{auto-1} по периоду колебаний $T$, приходим 
к равенству
\begin{equation}
	\eqmark{auto-2}
	W=W_0-\int\limits_{0}^{T}P(t)dt,
\end{equation}
где $W_0$~---~энергия системы в некоторый момент времени, принятый за начало
отсчёта. В обычной~--- диссипативной~--- системе $P(t)=RI^{2}(t)>0$, 
так что автоколебания невозможны. 
Если же мощность потерь $P(t)$ в системе можно сделать \important{знакопеременной}, 
то подбором режима работы системы можно обеспечить энергетический баланс:
\begin{equation}
	\eqmark{auto-3}
	\int\limits_{0}^{T} P(t) dt = 0,
\end{equation}
и, следовательно, возбудить в системе автоколебания. Отрицательные
<<потери>>, естественно, реализуются за счёт совершения 
работы внешним источником над системой.

Выполнение условия \eqref{auto-3} возможно, например, в \important{нелинейной}
колебательной системе, в которой сопротивление $R=\frac{dU}{dI}$ является 
функцией тока: $R=R(I)$, причём \emph{знакопеременной}. То есть для 
автоколебательного режима необходима возможность реализации
отрицательного \term{дифференциального сопротивления} цепи:
\[
R_{диф} = \frac{dU}{dI} < 0.
\]
Иными словами, вольт-амперная характеристика $I(U)$ элемента должна
обладать <<спадающими>> участками (участками с отрицательным наклоном). 
Примеры таких характеристик приведены на
рис.~\figref{auto-1}(а, б). Ими обладают, например, газоразрядная лампа 
(а) и туннельный диод (б). Обычно характеристики вида~(а) называют $S$-образными, 
а вида~(б)~---~$N$-образными.
\begin{figure}
	\centering
	\pic{0.8\textwidth}{Chapter_2/auto-1}
	%\import{Images/Chapter_2/}{auto-1.pdf_tex}
	\caption{Вольт-амперные характеристики с «падающими» участками: а)~$S$-образным, б) $N$-образным.}
	\figmark{auto-1}
\end{figure}

Форма автоколебаний зависит от добротности колебательного контура. При большой
добротности характер протекающих процессов почти не изменяется по сравнению с
тем, как они протекали бы в системе без поступления
энергии от источника: период и форма автоколебаний будут близки к периоду и
форме собственных колебаний. Это связано с тем, что в этом случае от постоянного
источника поступает энергия, составляющая малую
долю полной энергии колебательной системы. При малой добротности колебательной системы 
для поддержания колебаний от постоянного
источника должна поступать энергия, сопоставимая с энергией колебаний. В этом
случае форма автоколебаний может значительно отличаться от синусоидальной.
Наконец, в апериодической системе, в которой за период автоколебаний теряется вся
накопленная энергия, автоколебания становятся \term{релаксационными} и
могут по форме очень сильно отличаться от колебаний синусоидальных.

\introsubsection{Автоколебания в вырожденных системах}
Автоколебательная система, не содержащая одного из накопителей колебательной
энергии (например, индуктивности или емкости), называется \term{вырожденной}. Колебания в такой системе
описываются дифференциальным уравнением первого порядка и могут быть
только \emph{релаксационными}.

В качестве примера рассмотрим представленную на рис.~\figref{auto-2}(а,~б) 
вырожденную автоколебательную систему, содержащую источник
постоянного напряжения $U$, ёмкость $C$, сопротивление $R$ и нелинейный элемент
с $S$-образной вольт-амперной характеристикой $I_S(U)$. 

\begin{figure}[h]
	\begin{minipage}[h]{0.49\linewidth}
		\centering
		\pic{0.9\textwidth}{Chapter_2/auto-2}
		%\center{\resizebox{0.79\textwidth}{!}{\import{Images/Chapter_2/}{auto-2.pdf_tex}} \\ а) Схема автоколебательной $RC$-системы.}
	\end{minipage}
	\hfill
	\begin{minipage}[h]{0.49\linewidth}
		\centering
		\pic{0.79\textwidth}{Chapter_2/auto-3}
		%\center{\resizebox{0.79\textwidth}{!}{\import{Images/Chapter_2/}{auto-3.pdf_tex}} \\ б) Вольт-амперная характеристика и нагрузочная прямая $RC$-системы.}
	\end{minipage}
	\caption{Вырожденная автоколебательная $RC$-система}
	\figmark{auto-2}
\end{figure}

Уравнения, описывающие поведение этой системы релаксационного типа,
имеют вид:
\begin{equation}
	\eqmark{auto-4}
	RI+U=\mathcal{E},\quad I=I_C+I_S,\quad I_C=C\frac{dU}{dt},\quad I_S=I_S\left(U \right).
\end{equation}
Следовательно,
\begin{equation}
	\eqmark{auto-5}
	\tau_С\frac{dU}{dt}=\mathcal{E}-U-R{{I}_{S}}\left( U \right),
\end{equation}
где $\tau_С = RC$ --- характерное время разрядки конденсатора.

В стационарном состоянии, когда $dU / dt = 0$, должно выполняться равенство
\begin{equation}
	\eqmark{auto-6}
	I_S(U)=\frac{\mathcal{E}-U}{R}.
\end{equation}
Правая часть здесь представляет \important{нагрузочную прямую}, точки
пересечения которой с вольт-амперной характеристикой 
${I}_{S}\left( U \right)$ определяют стационарные состояния системы. 
На рис.~\figref{auto-2}б) параметры $\mathcal{E}$ и $R$ выбраны так, 
чтобы стационарное состояние $A(U_{А},\,I_{А})$ лежало на падающей
ветви вольт-амперной характеристики, где возможен автоколебательный режим. 

Покажем, что состояние $I_{А}=I(U_{А})$ может быть
\important{неустойчивым}. Для этого дадим малое приращение $\Delta U$ 
переменной $U$ в точке $U_{А}$ и представим в линейном приближении по 
$\delta U$ вольт-амперную характеристику $I_S(V)$ вблизи стационарного 
состояния $U_{А}$:
\begin{equation}
	\eqmark{auto-7}
I_S(U_{А} + \delta U)\approx 
    I_S(U_{А}) + {I}_S' \delta U,
\end{equation}
где ${I}_S' = \left.\frac{dI_S}{dU}\right|_{U_{А}}$ --- 
наклон вольт-амперной характеристики в точке~$U_{А}$. 
Подстановка этого выражения в \eqref{auto-5} приводит к уравнению
\begin{equation}
	\eqmark{auto-8}
\tau_С \frac{d\delta U}{dt}=-\left( 1+R{I}_S' \right) \delta U,
\end{equation}
из которого видно, что при условии
\begin{equation}
	\eqmark{auto-9}
	I_S' < - \frac{1}{R}
\end{equation}
возмущение $\delta U$ со временем экспоненциально нарастает, 
и, значит, стационарное состояние $I_{А}=I_S(U_{А})$ является неустойчивым. 

Если выполнено условие неустойчивости, система будет совершать
\important{релаксационные автоколебания}. 
Их фазовая траектория на рис.~\figref{auto-2}б) является замкнутой и 
состоит из плавных участков $da$ и $bc$ вольт-амперной
характеристики между напряжениями $U_1$ и $U_2$, соединённых двумя вертикальными
участками $ab$ и $cd$, показанными на рисунке штриховыми линиями. Формально
вертикальные участки соответствуют \important{скачкам тока}, которые возможны
только при отсутствии индуктивностей в системе, исходно заложенной в данной
идеализированной модели. Учёт малой «паразитной» индуктивности элементов схемы, 
приводящей к конечной скорости скачков, добавляет ещё одно дифференциальное 
уравнение первого порядка. Система, таким образом, перестаёт быть вырожденной, 
а в такой системе заведомо возможен автоколебательный процесс.

Аналогичным образом можно показать, что условие возбуждения автоколебаний 
в вырожденной $RL$-системе: соединённых последовательно с источником постоянного 
напряжения резистора, индуктивности и элемента с $N$-образной вольт-амперной 
характеристикой, --- имеет вид неравенства \eqref{auto-9}, 
в котором вместо $I_S'\left(U_{А}\right)$ стоит $I_N'\left(U_{А}\right)$ --- 
крутизна падающего  участка $N$-образной вольт-амперной характеристики 
в рабочей точке $\left(U_{А}\right)$.
