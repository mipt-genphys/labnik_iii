\lab{Релаксационные колебания}

\aim{изучение вольт-амперной характеристики нормального тлеющего разряда;
исследование релаксационного генератора на стабилитроне.}

\equip{стабилитрон СГ-2 (газонаполненный диод) на монтажной панели, амперметр,
вольтметр, магазин сопротивлений, магазин ёмкостей, источник питания,
осциллограф (ЭО), генератор звуковой частоты (ЗГ).}

Перед выполнением работы необходимо ознакомиться с п.~\ref{sec:auto} Введения
к разделу.

Колебательные системы, как правило, имеют два накопителя энергии, между
которыми происходит её перекачка. Например, в контуре,
содержащем конденсатор и катушку индуктивности, электрическая энергия переходит
в магнитную и обратно.
Встречаются, однако, колебательные системы, содержащие всего один накопитель
энергии. Рассмотрим в качестве примера
электрическую цепь, содержащую конденсатор и сопротивление без самоиндукции.
Разряд конденсатора через сопротивление
представляет собой апериодический процесс. Разряду, однако, можно придать
периодический характер, возобновляя заряд
конденсатора через постоянные промежутки времени. Колебания в этом случае
являются совокупностью двух апериодических
процессов~--- процесса зарядки конденсатора и процесса его разрядки. Такие
колебания называются \term{релаксационными}.


В данной работе роль <<ключа>>, обеспечивающего попеременную зарядку и
разрядку конденсатора, играет газоразрядный диод. 
Зависимость тока от напряжения для газоразрядной лампы не подчиняется
закону Ома и характеризуется рядом
особенностей (рис.~\figref{VACH stabilitron}). При малых напряжениях лампа
практически не пропускает тока (участок OAB на рис.~\figref{VACH stabilitron}).
Ток в лампе возникает только в том случае, если разность потенциалов на её
электродах достигает \important{напряжения зажигания}~$U_1$. 
При этом скачком (участок BC) устанавливается конечная сила тока~$I_1$~--- в лампе возникает
\important{нормальный тлеющий разряд} (см. подробнее Приложение к Разделу~V,
стр.~\pageref{sec:discharge}). 
При дальнейшем незначительном увеличении напряжения сила тока заметно возрастает 
по закону, близкому к линейному (участок CD). Нормальный тлеющий разряд~---
стабилизатор напряжения, отсюда второе название лампы~--- \important{стабиловольт}.


Если начать уменьшать напряжение на горящей лампе, то при напряжении, равном
$U_1$, лампа ещё не гаснет, и сила тока
продолжает уменьшаться (участок CE на рис.~\figref{VACH stabilitron}). 
Лампа перестанет пропускать ток лишь при \important{напряжении гашения}~$U_2$, 
которое обычно
существенно меньше $U_1$. Сила тока при этом скачком падает от значения~$I_2$
($I_2<I_1$) до нуля (участок EA).

\begin{figure}[h!]
    \centering
    \pic{0.4\textwidth}{Chapter_2/5_1_1}
    \caption{Вольт-амперная характеристика стабилитрона с последовательно
        включённым резистором}
    \figmark{VACH stabilitron}
\end{figure}
%\rpic{38mm}{5_1_1}{\cct Вольт-амперная характеристика стабилитрона с
%последовательно включённым резистором}{1}

Характеристика, изображённая на рис.~\figref{VACH stabilitron}, несколько
идеализирована. У~реальной лампы зависимость~$I(U)$ не вполне линейна.
При $U>U_1$ графики, соответствующие возрастанию и убыванию напряжения, 
могут не совпадать. Эти отличия, впрочем, носят второстепенный характер 
и для нашей задачи несущественны.


Рассмотрим схему релаксационного генератора, представленную
на рис.~\figref{generator scheme}. Пусть напряжение батареи $\mathcal{E}$ больше
напряжения зажигания $U_1$. В обозначениях, принятых на схеме, справедливо уравнение
\begin{equation*}
I_C+I(U)=\frac{\mathcal{E}-U}{R}
\end{equation*}
или
\begin{equation}
	\eqmark{2.8.1}
	C\frac{dU}{dt}+I(U)=\frac{\mathcal{E}-U}{R}.
\end{equation}
В стационарном режиме работы, когда напряжение $U$ на конденсаторе постоянно и
$dU/dt = 0$, ток через лампу равен
\begin{equation}
	\eqmark{2.8.2}
	I_{\text{ст}}=\frac{\mathcal{E}-U}{R}.
\end{equation}

\begin{figure}[h!]
    \centering
    \pic{0.4\textwidth}{Chapter_2/5_1_2}
    \caption{Принципиальная схема релаксационного генератора}
    \figmark{generator scheme}
\end{figure}
%\rpic{4.0cm}{5_1_2}{\cct Принципиальная схема релаксационного генератора}{2}

Равенство \eqref{2.8.2} может быть представлено графически
(рис.~\figref{generator work}).
При разных $R$ графики имеют вид прямых, пересекающихся в точке $U=\mathcal{E}$,\\*$I=0$.
Область, где эти \important{нагрузочные прямые}
пересекают вольт-амперную характеристику лампы, соответствует стационарному
режиму~---~при малых $R$ (прямая~1) лампа
горит постоянно, колебания отсутствуют. Прямая~2, проходящая через точку
($I_2$,~$U_2$), соответствует критическому
сопротивлению
\begin{equation}
	\eqmark{2.8.3}
	R_{\text{кр}}= \frac{\mathcal{E}-U_2}{I_2}.
\end{equation}
При сопротивлении $R>R_{\text{кр}}$ нагрузочная прямая~3 не пересекает
характеристику лампы, поэтому стационарный режим
невозможен. В этом случае в системе устанавливаются колебания.

\begin{figure}[h!]
    \centering
    \pic{0.4\textwidth}{Chapter_2/5_1_3}
    \caption{Режимы работы релаксационного генератора}
    \figmark{generator work}
\end{figure}
%\rpic[14]{4.0cm}{5_1_3}{\cct Режимы работы релаксационного генератора}{3}

Рассмотрим, как происходит колебательный процесс. Пусть в начале опыта ключ~К
разомкнут (рис.~\figref{generator scheme}) и $U=0$. Замкнём ключ.
Конденсатор~$C$ начинает заряжаться через сопротивление~$R$, напряжение на нём
увеличивается (рис.~\figref{relax osc}). Как только оно
достигнет напряжения зажигания $U_1$, лампа начинает проводить ток, причём
прохождение тока сопровождается разрядкой
конденсатора. В самом деле, батарея~$\mathcal{E}$, подключённая через большое
сопротивление~$R$, не может поддерживать необходимую
для горения лампы величину тока. Во время горения лампы конденсатор разряжается,
и когда напряжение на нём достигнет
потенциала гашения, лампа перестанет проводить ток, а конденсатор вновь начнёт
заряжаться. Возникают релаксационные
колебания с~амплитудой, равной $(U_1-U_2)$.

\begin{figure}[h!]
    \centering
	\pic{0.8\textwidth}{Chapter_2/5_1_4}
	\caption{Осциллограмма релаксационных колебаний}
	\figmark{relax osc}
\end{figure}
%\rpic{5.0cm}{5_1_4}{\cct Осциллограмма релаксационных колебаний}{4}

Рассчитаем период колебаний. Полное время одного периода колебаний $T$ состоит
из суммы времени зарядки $\tau_{\text{з}}$ и
времени разрядки $\tau_{\text{р}}$, но если сопротивление $R$ существенно
превосходит сопротивление зажжённой лампы, то
$\tau_{\text{з}}\gg \tau_{\text{р}}$ и $T\approx\tau_{\text{з}}$ (этим случаем
мы и ограничимся).

Во время зарядки конденсатора лампа не горит ($I(U)=0$), и уравнение
\eqref{2.8.1} приобретает вид
\begin{equation}
	\eqmark{2.8.4}
	RC\frac{dU}{dt}=\mathcal{E}-U.
\end{equation}
Будем отсчитывать время с момента гашения лампы, так что $U=U_2$ при $t=0$
(рис.~\figref{relax osc}). Решив уравнение \eqref{2.8.4}, найдём
\begin{equation}
	\eqmark{2.8.5}
	U=\mathcal{E}-(\mathcal{E}-U_2)e^{-t/RC}.
\end{equation}
В момент зажигания $t=\tau_{\text{з}}$, $U=U_1$, поэтому
\begin{equation}
	\eqmark{2.8.6}
	U_1=\mathcal{E}-(\mathcal{E}-U_2)e^{-\tau_{\text{з}}/RC}.
\end{equation}
Из уравнений \eqref{2.8.5} и \eqref{2.8.6} нетрудно найти период колебаний:
\begin{equation}
	\eqmark{2.8.7}
	T\approx \tau_{\text{з}}=RC\ln\frac{\mathcal{E}-U_2}{\mathcal{E}-U_1}.
\end{equation}

Развитая выше теория является приближённой. Ряд принятых при расчётах упрощающих
предположений оговорен в тексте.
Следует иметь в виду, что мы полностью пренебрегли паразитными емкостями и
индуктивностями схемы. Не рассматривались
также процессы развития разряда и деионизация при гашении. Поэтому теория
справедлива лишь в тех случаях, когда в схеме
установлена достаточно большая ёмкость и когда период колебаний существенно
больше времени развития разряда и времени
деионизации (практически $\gg10^{-5}$~с). 
Кроме того, потенциал гашения $U_2$, взятый из статической вольт-амперной
характеристики, может отличаться от потенциала гашения лампы, работающей в
динамическом режиме релаксационных колебаний.

\begin{lab:task}

\taskpreamble{В работе предлагается получить вольт-амперную характеристику
стабилитрона и познакомиться с работой релаксационного генератора: определить
критическое сопротивление, исследовать зависимость периода колебаний от
сопротивления при фиксированной ёмкости и от ёмкости при фиксированном
сопротивлении.}

\tasksection{Измерение ВАХ стабилитрона}

		\item Соберите схему, изображённую на рис.~\figref{stabilitron scheme
charact}. Добавочное сопротивление~$r$ подпаяно между ножкой лампы и
соответствующей клеммой для того, чтобы предохранить стабилитрон от перегорания. 
Это сопротивление остаётся включённым при всех
измерениях. Запишите величину~$r$, указанную на установке.

\begin{figure}[h!]
    \centering
    \pic{0.45\textwidth}{Chapter_2/5_1_5} % закоммент. т.к. не работает
    \caption{Схема установки для~изучения характеристик стабилитрона}
    \figmark{stabilitron scheme charact}
\end{figure}
%\rpic{5.0cm}{5_1_5}{\cct Схема установки для~изучения характеристик
%стабилитрона}{5}

		\item Установите регулятор источника питания на минимум напряжения и
включите источник в сеть.

		\item Снимите вольт-амперную характеристику стабилитрона с
сопротивлением $r$ при возрастании и убывании напряжения. При
этом как можно точнее определите потенциалы зажигания и гашения $U_1$ и $U_2$ и
соответствующие токи $I_1$ и $I_2$.

\tasksection{Осциллограммы релаксационных колебаний}

		\item Соберите релаксационный генератор согласно
         рис.~\figref{scheme exp osc}.

		\item Установите на магазине ёмкостей значение $C=0,05$~мкФ, а на
магазине сопротивлений $R=900$~кОм.

		\item Включите в сеть осциллограф, звуковой генератор и источник питания
и установите напряжение $\mathcal{E}\approx 1,2\,U_1$.

\begin{figure}[h!]
    \centering
	\pic{0.9\textwidth}{Chapter_2/5_1_6}
	\caption{Схема установки для исследования релаксационных колебаний}
	\figmark{scheme exp osc}
\end{figure}
%\fcpic[1]{5_1_6}{Схема установки для исследования релаксационных колебаний}{6}

		\item Подберите частоту развёртки ЭО, при которой на экране видна
картина пилообразных колебаний (рис.~\figref{relax osc}).

		\item Получив пилу на экране, оцените соотношение между временем
зарядки~$\tau_{\text{з}}$ и временем разрядки $\tau_{\text{р}}$.
Зарисуйте картину колебаний.

		\item Уменьшая сопротивление магазина, определите $R_{\text{кр}}$, при
котором пропадают колебания, и сравните его с величиной,
рассчитанной по формуле~\eqref{2.8.3}. Это сравнение полезно сделать в процессе
работы и подумать о причинах расхождения
результатов.

Убедитесь, что колебания пропадают не только при уменьшении $R$ при постоянном
$\mathcal{E}$, но и при увеличении $\mathcal{E}$ при
постоянном $R$, когда это $R$ не слишком превышает $R_{\text{кр}}$.

\tasksection{Фигуры Лиссажу и частота колебаний}

		\item Восстановите исходные параметры релаксационного
генератора: $C=5\cdot 10^{-2}$~мкФ, $R=900$~кОм, $\mathcal{E}\approx 1,2 \cdot
U_1$. Подайте сигнал с генератора на вход $X$ осциллографа. Меняя частоту ЗГ,
получите на экране фигуру Лиссажу без
самопересечений, соответствующую отношению частот 1:1.

		\item Не меняя параметров релаксационного генератора, уменьшите частоту
ЗГ вдвое (втрое) и получите фигуры Лиссажу при
соотношении частот 2:1 (3:1). Зарисуйте эти кривые в тетрадь.

Получите и зарисуйте фигуры Лиссажу при увеличении частоты ЗГ в два и три раза
(1:2 и 1:3).

		\item При любом целом значении $R$ из интервала (2~--~4)~$R_{\text{кр}}$
снимите с помощью фигур Лиссажу 1:1 зависимость частоты
колебаний $\nu$ от ёмкости~$C$, меняя величину ёмкости в пределах от
$5\cdot10^{-2}$ до $5\cdot10^{-3}$~мкФ.

Напряжение $\mathcal{E}$, необходимое для расчёта теоретического значения периода по
формуле \eqref{2.8.7}, следует поддерживать
постоянным.

		\item Проведите серию измерений $\nu(R)$ при постоянной ёмкости 
$C=5\cdot10^{-2}$, меняя величину $R$ от максимального
значения до критического.

\tasksection{Обработка результатов}

		\item Постройте графики вольт-амперных характеристик~$I(U)$ для системы, 
        состоящей из стабилитрона и дополнительного сопротивления $r$ 
        (по результатам измерений) и 
        для стабилитрона без сопротивления~$r$ (вычитая падение напряжения
на сопротивлении~$r$ при каждом токе).
Сравните относительные изменения тока и напряжения на стабилитроне.

		\item Рассчитав экспериментальные и теоретические значения периодов,
постройте графики $T_{\text{эксп}}(C)$ и $T_{\text{теор}}(C)$ на
одном листе. На другом листе постройте графики $T_{\text{эксп}}(R)$ и~$T_{\text{теор}}(R)$.

		\item Если наклоны теоретической и экспериментальной прямых заметно
отличаются, рассчитайте из экспериментальной прямой
динамический потенциал гашения. Потенциалы зажигания можно считать одинаковыми.
\end{lab:task}


\begin{lab:questions}
	\item Какие колебания называются релаксационными?

	\item От каких параметров газа зависит напряжение зажигания стабиловольта?

	\item Почему напряжение гашения существенно меньше напряжения зажигания?

%	\item Какие режимы работы релаксационного генератора Вы знаете?

	\item Как по вольт-амперной характеристике стабиловольта и известным
параметрам генератора найти ток в лампе в стационарном
режиме?

	\item Что такое критическое сопротивление релаксационного генератора? От
чего оно зависит?

	\item Почему критическое сопротивление зависит от величины напряжения $\mathcal{E}$ на
входе генератора? Рассмотрите рис.~\figref{generator work}.

	\item Почему при малой ёмкости колебания не возникают (лампа не гаснет) даже
при $R>R_{\text{кр}}$? Оцените <<малость>> ёмкости,
сравнив время релаксации и время деионизации.
\end{lab:questions}


\begin{lab:literature}

	\item \SivuhinIII~--- \S~134.

	\item \Kalashnikov~--- \S~244.

	\item \emph{Горелик~Г.С.} Колебания и волны.~--- М.:~Физматгиз, 1959.
Гл.~IV, \S~6.
\end{lab:literature}
