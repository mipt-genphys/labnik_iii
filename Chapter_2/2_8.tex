\lab{Релаксационные колебания}

\aim{изучение вольт-амперной характеристики нормального тлеющего разряда; исследование релаксационного генератора на
стабилитроне.}

\equip{стабилитрон СГ-2 (газонаполненный диод) на монтажной панели, амперметр, вольтметр, магазин сопротивлений, магазин
ёмкостей, источник питания, осциллограф (ЭО), генератор звуковой частоты (ЗГ).}

Колебательные системы, как правило, имеют два накопителя энергии, между которыми происходит её перекачка. В контуре,
содержащем конденсатор и катушку индуктивности, электрическая энергия переходит в магнитную и обратно.

\rpic{38mm}{5_1_1}{\cct Вольт-амперная характеристика стабилитрона с последовательно включённым резистором}{1}

Встречаются, однако, колебательные системы, содержащие всего один накопитель энергии. Рассмотрим в качестве примера
электрическую цепь, содержащую конденсатор и сопротивление без самоиндукции. Разряд конденсатора через сопротивление
представляет собой апериодический процесс. Разряду, однако, можно придать периодический характер, возобновляя заряд
конденсатора через постоянные промежутки времени. Колебания в этом случае являются совокупностью двух апериодических
процессов~--- процесса зарядки конденсатора и процесса его разрядки. Такие колебания называются релаксационными.

В нашей установке роль <<ключа>>, обеспечивающего попеременную зарядку и разрядку конденсатора, играет газоразрядный
диод. Зависимость тока от напряжения для газоразрядной лампы не подчиняется закону Ома и характеризуется рядом
особенностей (\p{1}). При малых напряжениях лампа практически не пропускает тока (см. участок ОАБ на рис.~\oref{v5_r2} и
\oref{v5_r6}). Ток в лампе возникает только в том случае, если разность потенциалов на её электродах достигает
\term{напряжения зажигания} $V_1$. При этом скачком устанавливается конечная сила тока $I_1$~--- в лампе возникает
\term{нормальный тлеющий разряд}. При дальнейшем незначительном увеличении напряжения сила тока заметно возрастает по
закону, близкому к линейному. Нормальный тлеющий разряд~--- стабилизатор напряжения, отсюда второе название лампы ---
\term{стабиловольт}.

\rpic{4.0cm}{5_1_2}{\cct Принципиальная схема релаксационного генератора}{2}

Если начать уменьшать напряжение на горящей лампе, то при напряжении, равном $V_1$, лампа ещё не гаснет, и сила тока
продолжает уменьшаться. Лампа перестанет пропускать ток лишь при \term{напряжении гашения}~$V_2$, которое обычно
существенно меньше $V_1$. Сила тока при этом скачком падает от значения~$I_2$ ($I_2<I_1$) до нуля.

Характеристика, изображённая на \p{1}, несколько идеализирована. У~реальной лампы зависимость $I(V)$ не вполне линейна.
При $V>V_1$ графики, соответствующие возрастанию и убыванию напряжения, не всегда совпадают. Эти отличия, впрочем, носят
второстепенный характер и для нашей задачи несущественны.

\rpic[14]{4.0cm}{5_1_3}{\cct Режимы работы релаксационного генератора}{3}

Рассмотрим схему релаксационного генератора, представленную на \p{2}. Пусть напряжение батареи $U$ больше напряжения
зажигания $V_1$. В обозначениях, принятых на схеме, справедливо уравнение
\[
I_C+I(V)=\frac{U-V}{R}
\]
или
\be1
C\frac{dV}{dt}+I(V)=\frac{U-V}{R}.
\ee

В стационарном режиме работы, когда напряжение $V$ на конденсаторе постоянно и $dV/dt = 0$, ток через лампу равен
\be2
I_{ст}=\frac{U-V}{R}.
\ee

Равенство (\r2) может быть представлено графически (\p{3}).

При разных $R$ графики имеют вид прямых, пересекающихся в точке $V=U$, $I=0$. Область, где эти \term{нагрузочные прямые}
пересекают вольт-амперную характеристику лампы, соответствует стационарному режиму~--- при малых $R$ (прямая~1) лампа
горит постоянно, колебания отсутствуют. Прямая~2, проходящая через точку ($I_2$, $V_2$), соответствует критическому
сопротивлению
\be3
R_{кр}= \frac{U-V_2}{I_2}.
\ee

При сопротивлении $R>R_{кр}$ нагрузочная прямая~3 не пересекает характеристику лампы, поэтому стационарный режим
невозможен. В этом случае в системе устанавливаются колебания.

Рассмотрим, как происходит колебательный процесс. Пусть в~начале опыта ключ~К разомкнут (\p{2}) и $V=0$. Замкнём ключ.
Конденсатор~$C$ начинает заряжаться через сопротивление~$R$, напряжение на нём увеличивается (\p{4}). Как только оно
достигнет напряжения зажигания $V_1$, лампа начинает проводить ток, причём прохождение тока сопровождается разрядкой
конденсатора. В самом деле, батарея~$U$, подключённая через большое сопротивление~$R$, не может поддерживать необходимую
для горения лампы величину тока. Во время горения лампы конденсатор разряжается, и когда напряжение на нём достигнет
потенциала гашения, лампа перестанет проводить ток, а конденсатор вновь начнёт заряжаться. Возникают релаксационные
колебания с~амплитудой, равной $(V_1-V_2)$.

\rpic{5.0cm}{5_1_4}{\cct Осциллограмма релаксационных колебаний}{4}

Рассчитаем период колебаний. Полное время одного периода колебаний $T$ состоит из суммы времени зарядки $\vartau_з$ и
времени разрядки $\vartau_р$, но если сопротивление $R$ существенно превосходит сопротивление зажжённой лампы, то
$\vartau_з\gg \vartau_р$ и $T\approx\vartau_з$ (этим случаем мы и ограничимся).

Во время зарядки конденсатора лампа не горит ($I(V)=0$), и уравнение (\r1) приобретает вид
\be4
RC\frac{dV}{dt}=U-V.
\ee

Будем отсчитывать время с момента гашения лампы, так что $V=V_2$ при $t=0$ (\p{4}). Решив уравнение (\r4), найдём
\be5
V=U-(U-V_2)e^{-t/RC}.
\ee

В момент зажигания $t=\vartau_з$, $V=V_1$, поэтому
\be6
V_1=U-(U-V_2)e^{-\vartau_з/RC}.
\ee

Из уравнений (\r5) и (\r6) нетрудно найти период колебаний:
\be7
T\approx \vartau_з=RC\ln\frac{U-V_2}{U-V_1}.
\ee

Развитая выше теория является приближённой. Ряд принятых при расчётах упрощающих предположений оговорен в~тексте.
Следует иметь в~виду, что мы полностью пренебрегли паразитными емкостями и индуктивностями схемы. Не рассматривались
также процессы развития разряда и деионизация при гашении. Поэтому теория справедлива лишь в~тех случаях, когда в схеме
установлена достаточно большая ёмкость и когда период колебаний существенно больше времени развития разряда и времени
деионизации (практически $\gg10^{-5}$~с). Кроме того, потенциал гашения $V_2$, взятый из статической вольт-амперной
характеристики, может отличаться от потенциала гашения лампы, работающей в динамическом режиме релаксационных колебаний.

\zad

В работе предлагается снять вольт-амперную характеристику стабилитрона и познакомиться с работой релаксационного
генератора: определить критическое сопротивление, исследовать зависимость периода колебаний от сопротивления при
фиксированной ёмкости и от ёмкости при фиксированном сопротивлении.

\zn Характеристика стабилитрона

\rpic{5.0cm}{5_1_5}{\cct Схема установки для~изучения характеристик стабилитрона}{5}

\n Соберите схему, изображённую на \p{5}. Добавочное сопротивление $r$ подпаяно между ножкой лампы и соответствующей
клеммой для того, чтобы предохранить стабилитрон от перегорания. Это сопротивление остаётся включённым при всех
измерениях. Запишите величину $r$, указанную на панели лампы.

\n Установите регулятор источника питания на минимум напряжения и включите источник в сеть.

\n Снимите вольт-амперную характеристику стабилитрона с сопротивлением $r$ при возрастании и убывании напряжения. При
этом как можно точнее определите потенциалы зажигания и гашения $V_1$ и $V_2$ и соответствующие токи $I_1$ и $I_2$.

\zn Осциллограммы релаксационных колебаний

\n Соберите релаксационный генератор согласно \p{6}.

\n Установите на магазине ёмкостей значение $C=0,05$~мкФ, а на магазине сопротивлений $R=900$~кОм.

\n Включите в сеть осциллограф, звуковой генератор и источник питания и установите напряжение $U\approx 1,2\,V_1$.

\fcpic[1]{5_1_6}{Схема установки для исследования релаксационных колебаний}{6}

\n Подберите частоту развёртки ЭО, при которой на экране видна картина пилообразных колебаний (\p{4}).

\n Получив пилу на экране, оцените соотношение между временем зарядки~$\vartau_з$ и временем разрядки $\vartau_р$.
Зарисуйте картину колебаний.

\n Уменьшая сопротивление магазина, определите $R_{кр}$, при котором пропадают колебания, и сравните его с величиной,
рассчитанной по формуле~(\r3). Это сравнение полезно сделать в процессе работы и подумать о причинах расхождения
результатов.

Убедитесь, что колебания пропадают не только при уменьшении $R$ при постоянном $U$, но и при увеличении $U$ при
постоянном $R$, когда это $R$ не слишком превышает $R_{кр}$.

\zn Фигуры Лиссажу и частота колебаний

\n Восстановите исходные параметры релаксационного генератора: $C=5\cdot 10^{-2}$~мкФ, $R=900$~кОм, $U\approx 1,2 \cdot
V_1$. Подайте сигнал с генератора на вход $X$ осциллографа. Меняя частоту ЗГ, получите на экране фигуру Лиссажу без
самопересечений, соответствующую отношению частот 1:1.

\n Не меняя параметров релаксационного генератора, уменьшите частоту ЗГ вдвое (втрое) и получите фигуры Лиссажу при
соотношении частот 2:1 (3:1). Зарисуйте эти кривые в тетрадь.

Получите и зарисуйте фигуры Лиссажу при увеличении частоты ЗГ в два и три раза (1:2 и 1:3).

\n При любом целом значении $R$ из интервала (2--4)~$R_{кр}$ снимите с помощью фигур Лиссажу 1:1 зависимость частоты
колебаний $\nu$ от ёмкости~$C$, меняя величину ёмкости в пределах от $5\cdot10^{-2}$ до $5\cdot10^{-3}$~мкФ.

Напряжение $U$, необходимое для расчёта теоретического значения периода по формуле (\r7), следует поддерживать
постоянным.

\n Проведите серию измерений $\nu=f(R)$ при постоянной ёмкости $C=5\cdot10^{-2}$, меняя величину $R$ от максимального
значения до критического.

\znr Обработка результатов

\n Постройте графики $I=f(V)$ для системы, состоящей из стабилитрона и дополнительного сопротивления $r$ (по результатам
измерений) и для стабилитрона без сопротивления~$r$ (вычитая падение напряжения на сопротивлении~$r$ при каждом токе).
Сравните относительные изменения тока и напряжения на стабилитроне.

\n Рассчитав экспериментальные и теоретические значения периодов, постройте графики $T_{эксп}=f(C)$ и $T_{теор}=f(C)$ на
одном листе.

На другом листе постройте графики $T_{эксп}$ и $T_{теор}=f(R)$.

\n Если наклоны теоретической и экспериментальной прямых заметно отличаются, рассчитайте из экспериментальной прямой
динамический потенциал гашения. Потенциалы зажигания можно считать одинаковыми.

{\small

\kv

\n Какие колебания называются релаксационными?

\n От каких параметров газа зависит напряжение зажигания стабиловольта?

\n Почему напряжение гашения существенно меньше напряжения зажигания?

% \n Какие режимы работы релаксационного генератора Вы знаете?

\n Как по вольт-амперной характеристике стабиловольта и известным параметрам генератора найти ток в лампе в стационарном
режиме?

\n Что такое критическое сопротивление релаксационного генератора? От чего оно зависит?

\n Почему критическое сопротивление зависит от величины напряжения $U$ на входе генератора? Рассмотрите рис.~\r{r3}.

\n Почему при малой ёмкости колебания не возникают (лампа не гаснет) даже при $R>R_{кр}$? Оцените <<малость>> ёмкости,
сравнив время релаксации и время деионизации.

\lit

\n \emph{Сивухин Д.В.} Общий курс физики. --- Т.~III. Электричество.~--- М.: Наука, 1983. \S~134.

\n \emph{Калашников С.Г.} Электричество.~--- М.: Наука, 1974. \S~244.

\n \emph{Горелик Г.С.} Колебания и волны.~--- М.: Физматгиз, 1959. Гл.~IV, \S~6.

}
