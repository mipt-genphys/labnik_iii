\lab{Релаксационные колебания}

\aim{изучение вольт-амперной характеристики нормального тлеющего разряда;
исследование релаксационного генератора на стабилитроне.}

\equip{стабилитрон СГ-2 (газонаполненный диод) на монтажной панели, 
    магазин ёмкостей, магазин сопротивлений, источник питания,
    амперметр, вольтметр, осциллограф.}

Перед выполнением работы необходимо ознакомиться с п.~\ref{sec:auto} 
введения к разделу.

В работе исследуются релаксационные колебания, возбуждаемые в электрическом
контуре, состоящем из ёмкости $C$, резистора $R$ и газоразрядного диода 
с $S$-образной вольт-амперной характеристикой. 
Принципиальная схема такого релаксационного генератора, представляющего 
частный случай рассмотренных в п.~\ref{subsec:auto:vyr}
\emph{вырожденных автоколебательных систем},
изображена на рис.~\chapterfigref{auto-2}. \emph{Релаксационные} колебания в этом
случае являются совокупностью двух апериодических процессов~--- зарядки 
конденсатора и его разрядки. В~нашей установке роль <<ключа>>, обеспечивающего
попеременную зарядку и разрядку конденсатора, играет газоразрядный диод.

\begin{figure}[h!]
    \centering
    \pic{0.4\textwidth}{Chapter_2/5_1_1}
    \caption{Упрощённая вольт-амперная характеристика стабилитрона $I_S(U)$
        и нагрузочная прямая $I=I_0(1-U/\mathcal{E})$}
    \figmark{fig1}
\end{figure}

Зависимость тока от напряжения для газоразрядной лампы не подчиняется
закону Ома и характеризуется рядом особенностей, представленных
в упрощённом виде на рис.~\figref{fig1} вместе с \emph{нагрузочной} прямой
$I=I_0 (1-U/\mathcal{E})$, где $\mathcal{E}$~--- постоянное напряжение 
внешнего источника питания.


\etp{0.5}

При малых напряжениях лампа практически не пропускает тока 
(участок $0da$ на рис.~\figref{fig1}).
Ток в лампе возникает только в том случае, если разность потенциалов на её
электродах достигает \important{напряжения зажигания}~$U_1$. 
При этом скачком (участок $ab$) устанавливается конечная 
сила тока~$I_1$~--- в лампе возникает
\important{нормальный тлеющий разряд} (см. подробнее приложение к Разделу~V,
стр.~\pageref{sec:discharge}). 
При дальнейшем незначительном увеличении напряжения сила тока заметно возрастает 
по закону, близкому к линейному. 
%Нормальный тлеющий разряд~---
%стабилизатор напряжения, отсюда второе название лампы~--- \important{стабиловольт}.

Если начать уменьшать напряжение на горящей лампе, то при напряжении~$U_1$
лампа ещё не гаснет, и сила тока продолжает уменьшаться 
(участок $bc$ на рис.~\figref{fig1}). 
Лампа перестаёт пропускать ток лишь при \important{напряжении гашения}~$U_2$, 
которое обычно существенно меньше $U_1$. 
Сила тока при этом скачком падает от значения~$I_2 < I_1$ до нуля (участок $cd$).


%Характеристика, изображённая на рис.~\figref{fig1}, несколько
%идеализирована. У~реальной лампы зависимость~$I(U)$ не вполне линейна.
%При $U>U_1$ графики, соответствующие возрастанию и убыванию напряжения, 
%могут не совпадать. Эти отличия, впрочем, носят второстепенный характер 
%и для нашей задачи несущественны.

Схема экспериментального стенда для изучения релаксационных колебаний
представлена на рис.~\figref{fig2}. 
Штриховой линией на схеме выделена панель, на которой установлен газонаполненный
диод (стабилитрон) и последовательно с ним~--- сопротивление $r$, позволяющее 
предохранить этот диод от перегорания, а также по напряжению на нём
измерить ток разряда. Это сопротивление остаётся включённым при всех измерениях.

\begin{figure}[h!]
    \centering
    \pic{0.7\textwidth}{Chapter_2/5_1_6}
    \caption{Схема установки для исследования релаксационных колебаний}
    \figmark{fig2}
\end{figure}

Выясним, при каком условии возможен колебательный процесс. 
В стационарном режиме, когда напряжение~$U$ на конденсаторе постоянно
и $dU/dt=0$, ток через лампу равен
\begin{equation}\eqmark{eq1}
I_{ст} = \frac{\mathcal{E}-U}{R + r}.
\end{equation}
Равенство \eqref{eq1} представлено графически на рис.~\figref{fig1}
для важного случая, когда нагрузочная прямая $I=I_0 (1 - U/\mathcal{E})$, 
имеющая отрицательный наклон, пересекает <<падающий>> участок вольт-амперной
характеристики стабилитрона, где $I'_S(U) < 0$.
Если при этом выполняется условие 
\begin{equation}
R + r < - \frac{1}{I'_S(U_A)},
\end{equation}
то, как показано в разделе \ref{subsec:auto:vyr} (см. \chaptereqref{auto-9}), 
в системе возможно возбуждение автоколебаний.

%
%Пусть напряжение батареи $\mathcal{E}$ больше
%напряжения зажигания $U_1$. В обозначениях, принятых на схеме, справедливо уравнение
%\begin{equation*}
%I_C+I(U)=\frac{\mathcal{E}-U}{R}
%\end{equation*}
%или
%\begin{equation}
%	\eqmark{2.8.1}
%	C\frac{dU}{dt}+I(U)=\frac{\mathcal{E}-U}{R}.
%\end{equation}
%В стационарном режиме работы, когда напряжение $U$ на конденсаторе постоянно и
%$dU/dt = 0$, ток через лампу равен
%\begin{equation}
%	\eqmark{2.8.2}
%	I_{\text{ст}}=\frac{\mathcal{E}-U}{R}.
%\end{equation}

%Равенство \eqref{2.8.2} может быть представлено графически
%(рис.~\figref{generator work}).
%При разных $R$ графики имеют вид прямых, пересекающихся в точке $U=\mathcal{E}$,\\*$I=0$.
%Область, где эти \important{нагрузочные прямые}
%пересекают вольт-амперную характеристику лампы, соответствует стационарному
%режиму~---~при малых $R$ (прямая~1) лампа
%горит постоянно, колебания отсутствуют. Прямая~2, проходящая через точку
%($I_2$,~$U_2$), соответствует критическому
%сопротивлению
%\begin{equation}
%	\eqmark{2.8.3}
%	R_{\text{кр}}= \frac{\mathcal{E}-U_2}{I_2}.
%\end{equation}
%При сопротивлении $R>R_{\text{кр}}$ нагрузочная прямая~3 не пересекает
%характеристику лампы, поэтому стационарный режим
%невозможен. В этом случае в системе устанавливаются колебания.

%\begin{figure}[h!]
%    \centering
%    \pic{0.4\textwidth}{Chapter_2/5_1_3}
%    \caption{Режимы работы релаксационного генератора}
%    \figmark{generator work}
%\end{figure}
%%\rpic[14]{4.0cm}{5_1_3}{\cct Режимы работы релаксационного генератора}{3}

Рассмотрим, как происходит колебательный процесс, отсчитывая время с того 
момента, когда напряжение на конденсаторе~$C$ равно~$U_2$. 
При зарядке конденсатора через сопротивление~$R$ напряжение на нём
увеличивается (рис.~\figref{relax osc}).
Как только оно достигнет напряжения зажигания~$U_1$, 
лампа начинает проводить ток, причём прохождение тока сопровождается разрядкой
конденсатора. В самом деле, батарея~$\mathcal{E}$, подключённая через большое
сопротивление~$R$, не может поддерживать необходимую
для горения лампы величину тока. Во время горения лампы конденсатор разряжается,
и когда напряжение на нём достигнет потенциала гашения~$U_2$, 
лампа перестанет проводить ток, а конденсатор вновь начнёт заряжаться. 
Возникают релаксационные колебания с~амплитудой $U_1-U_2$.

\begin{figure}[h!]
    \centering
	\pic{0.5\textwidth}{Chapter_2/5_1_4}
	\caption{Осциллограмма релаксационных колебаний}
	\figmark{relax osc}
\end{figure}
%\rpic{5.0cm}{5_1_4}{\cct Осциллограмма релаксационных колебаний}{4}

Рассчитаем период колебаний. Полное время одного периода колебаний~$T$ состоит
из суммы времени зарядки $\tau_{\text{з}}$ и
времени разрядки $\tau_{\text{р}}$. Однако если сопротивление~$R$ существенно
превосходит сопротивление зажжённой лампы, то
$\tau_{\text{з}}\gg \tau_{\text{р}}$ и $T\approx\tau_{\text{з}}$,
что подтверждается в нашем эксперименте.
Во время зарядки конденсатора лампа не горит ($I(U)=0$), и уравнение
цепи приобретает вид
\begin{equation}
	\eqmark{2.8.4}
	RC\frac{dU}{dt}=\mathcal{E}-U.
\end{equation}
Будем отсчитывать время с момента гашения лампы, так что $U=U_2$ при $t=0$
(рис.~\figref{relax osc}). Решив уравнение \eqref{2.8.4}, найдём
\begin{equation}
	\eqmark{2.8.5}
	U=\mathcal{E}-(\mathcal{E}-U_2)e^{-t/RC}.
\end{equation}
В момент зажигания $t=\tau_{\text{з}}$, $U=U_1$, поэтому
\begin{equation}
	\eqmark{2.8.6}
	U_1=\mathcal{E}-(\mathcal{E}-U_2)e^{-\tau_{\text{з}}/RC}.
\end{equation}
Из уравнений \eqref{2.8.5} и \eqref{2.8.6} нетрудно найти период колебаний:
\begin{equation}
	\eqmark{2.8.7}
	T\approx \tau_{\text{з}}=RC\ln\frac{\mathcal{E}-U_2}{\mathcal{E}-U_1}.
\end{equation}

Развитая выше теория является приближенной. Ряд принятых при расчётах упрощающих
предположений оговорён в тексте.
Следует иметь в виду, что мы полностью пренебрегли паразитными ёмкостями и
индуктивностями схемы. Не рассматривались
также процессы развития разряда и деионизация при гашении. Поэтому теория
справедлива лишь в тех случаях, когда в схеме
установлена достаточно большая ёмкость и когда период колебаний существенно
больше времени развития разряда и времени
деионизации (практически $\gg10^{-5}$~с). 
Кроме того, потенциал гашения $U_2$, взятый из статической вольт-амперной
характеристики, может отличаться от потенциала гашения лампы, работающей в
динамическом режиме релаксационных колебаний.

%\rpic{4.0cm}{5_1_2}{\cct Принципиальная схема релаксационного генератора}{2}

\begin{lab:task}

\taskpreamble{В работе предлагается получить растущий участок 
    вольт-амперной характеристики стабилитрона и познакомиться с работой 
    релаксационного генератора: исследовать осциллограммы колебаний тока
    и напряжения в цепи генератора, а также их совместное представление
    на фазовой плоскости системы; определить критическое сопротивление,
    исследовать зависимость периода колебаний от сопротивления при 
    фиксированной ёмкости и от ёмкости при фиксированном сопротивлении.}

\tasksection{Характеристика стабилитрона}

		\item Соберите схему, изображённую на рис.~\figref{fig4}. 
        Запишите величину~$r$, указанную на установке.

%\begin{figure}[h!]
%    \centering
%    \pic{0.4\textwidth}{Chapter_2/5_1_2}
%    \caption{Принципиальная схема релаксационного генератора}
%    \figmark{generator scheme}
%\end{figure}

\begin{figure}[h!]
    \centering
    \pic{0.45\textwidth}{Chapter_2/5_1_5} % закоммент. т.к. не работает
    \caption{Схема установки для~изучения характеристик стабилитрона}
    \figmark{fig4}
\end{figure}

		\item Установите регулятор источника питания на минимум напряжения и
включите источник в сеть.

		\item Получите доступную для измерений в данной схеме
        часть вольт-амперной характеристики стабилитрона с
сопротивлением~$r$ при возрастании и убывании напряжения~$U$. 
При этом как можно точнее определите потенциалы зажигания 
и гашения~$U_1$ и~$U_2$, а также соответствующие им токи~$I_1$ и~$I_2$.

\tasksection{Осциллограммы релаксационных колебаний}

		\item Соберите схему для исследования релаксационных колебаний 
        согласно рис.~\figref{fig2}.

		\item Установите на магазине ёмкостей значение $C=50$~нФ, а на
магазине сопротивлений $R=900$~кОм.

		\item Включите в сеть осциллограф в измерительном режиме
        раздельно для каналов~1 и~2 и источник питания, установив
        на нём выходное напряжение $\mathcal{E}\approx 1,2U_1$.


		\item Подберите частоту развёртки осциллографа и коэффициенты 
        усиления каналов так, чтобы на экране были видны раздельно картины
        колебаний напряжения~$U$ на конденсаторе (канал~1) и пропорционального
        току в стабилитроне~$I$ напряжения $U_r=rI$ на сопротивлении~$r$
        (канал~2).

		\item Оцените соотношение времён зарядки~$\tau_{\text{з}}$ 
        и разрядки~$\tau_{\text{р}}$. Зарегистрируйте картину колебаний.
        Оцените по осциллограмме времена $\tau_{з}$ и $\tau_{р}$, 
        период $T=\tau_{з}+\tau_{р}$ и частоту повторения
        процессов зарядки/разрядки $\nu = 1/T$.

		\item Уменьшая сопротивление магазина $R$, определите критическое 
        сопротивление~$R_{\text{кр}}$, при котором пропадают колебания.

        \item Убедитесь, что колебания пропадают не только 
        при уменьшении~$R$ при постоянном~$\mathcal{E}$, 
        но и при увеличении~$\mathcal{E}$ при постоянном~$R$, 
        когда это~$R$ не слишком превышает~$R_{\text{кр}}$.
        
        \item При значении $R$ из интервала 2--4$R_{кр}$ проведите серию 
        измерений периодов колебаний $T(C)$, меняя величину
        ёмкости~$C$ в пределах от 50~нф до 2~нф, поддерживая постоянным
        напряжение~$\mathcal{E}$.
        
        \item Проведите серию измерений периодов колебаний $T(R)$ при
        ёмкости $C=50$~нФ, меняя величину сопротивления~$R$ в пределах
        от максимального значения до критического~$R_{кр}$,
        поддерживая напряжение~$\mathcal{E}$.

\tasksection{Фазовые траектории релаксационных колебаний}

		\item Восстановите работу релаксационного генератора 
        на стенде рис.~\figref{fig2} с настройками, рекомендованными 
        в п.~5--6. 
        
        \item Переведите осциллограф в измерительный двухканальный режим.
        Установите по осям координат сдвиги и коэффициенты усиления,
        подходящие для наблюдения фазовой траектории релаксационных
        колебаний.
        
        \item Зарегистрируйте фазовую траекторию с координатной сеткой.
        Запишите коэффициенты усиления по осям координат, по которым
        можно будет восстановить количественные характеристики траектории.

\tasksection{Обработка результатов}

		\item По результатам п.~1--3 постройте графики $I(U)$ для системы,
        состоящей из стабилитрона и дополнительного сопротивления~$r$ и для
        стабилитрона без сопротивления~$r$, вычитая падение напряжения
        на~$r$ при каждом токе. Сравните относительные изменения тока и 
        напряжения на стабилитроне.
        
        \item Рассчитав экспериментальные и теоретические значения периодов,
        постройте графики~$T_{\text{эксп}}(C)$ и~$T_{\text{теор}}(C)$ на
        одном рисунке. На другом рисунке постройте графики 
        $T_{\text{эксп}}(R)$ и~$T_{\text{теор}}(R)$.

		\item Если наклоны теоретической и экспериментальной прямых заметно
        отличаются, оцените из экспериментальной прямой динамический потенциал 
        гашения~$U_2$. Потенциалы зажигания можно считать одинаковыми.
        
        \item По результатам п.~15 постройте фазовую траекторию релаксационных
        колебаний $I(U)$. Сопоставьте её с фазовой траекторией $dabc$,
        представленной на рис.~\figref{fig1}.
        
        \item Сравните полученную в п.~3 часть вольт-амперной характеристики
        стабилитрона с соответствующей частью фазовой траектории
        релаксационных колебаний. Укажите возможные причины расхождения.
        
\end{lab:task}


\begin{lab:questions}
	\item Какие колебания называются релаксационными?

	\item От каких параметров схемы зависит напряжение зажигания стабиловольта?

	\item Почему напряжение гашения существенно меньше напряжения зажигания?

%	\item Какие режимы работы релаксационного генератора Вы знаете?

	\item Как по вольт-амперной характеристике стабиловольта и известным
параметрам генератора найти ток в лампе в стационарном режиме?

	\item Что такое критическое сопротивление релаксационного генератора? От
чего оно зависит?

%	\item Почему критическое сопротивление зависит от величины 
%    напряжения $\mathcal{E}$ на входе генератора? Рассмотрите рис.~\figref{generator work}.

	\item *Почему при малой ёмкости колебания не возникают (лампа не гаснет) даже
при $R>R_{\text{кр}}$? Оцените <<малость>> ёмкости,
сравнив время релаксации и время деионизации.

\end{lab:questions}

\pagebreak

\begin{lab:literature}

	\item \SivuhinIII~--- \S~134.

	\item \Kalashnikov~--- \S~244.

	\item \emph{Горелик~Г.С.} Колебания и волны.~--- Москва\,: Физматлит, 2001.~---
Гл.~IV, \S~6.
\end{lab:literature}
