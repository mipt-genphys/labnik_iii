\lab{Резонанс токов в параллельном контуре}

\begin{lab:aim}
	исследование резонанса токов в параллельном колебательном контуре с
изменяемой ёмкостью, получение амплитудно-частотных и
фазово-частотных характеристик, определение основных параметров контура.
\end{lab:aim}

\begin{lab:equipment}
	генератор сигналов, источник напряжения, нагрузкой которого является
последовательный колебательный контур с переменной ёмкостью, двухканальный
осциллограф, цифровые вольтметры.
\end{lab:equipment}

% \warning{
Перед выполнением работы необходимо ознакомиться с основами теории электрических  
колебаний (см. введение к разделу п. \ref{sec:forced}, \ref{sec:ires}).
Необходимые дополнения применительно к реальным элементам колебательного
контура будут приведены ниже.
% }

\experiment
В данной работе изучаются резонансные явления в параллельном колебательном
контуре (резонанс токов). Блок-схема экспериментального стенда 
показана на рис.~\figref{exp schem}. Синусоидальный сигнал
от генератора поступает на вход \important{управляемого напряжением источника
тока} (см., например, [1]), собранного на операционном усилителе с полевым
транзистором, питание которого осуществляется встроенным блоком-выпрямителем от
сети~$\sim220$~В (цепи питания на схеме не показаны).  Внутреннее (выходное)
сопротивление источника тока, бесконечно большое в идеальном случае, в нашей
схеме составляет несколько ГОм. Это обеспечивает постоянство амплитуды тока~$I$
на меняющейся нагрузке~--- параллельном контуре, изображенном на 
рис.~\figref{exp schem} в виде эквивалентной схемы.

\begin{figure}[h!]
    \centering
	\pic{0.65\textwidth}{Chapter_2/2_3_1}
	\caption{Блок-схема экспериментального стенда}
	\figmark{exp schem}
\end{figure}


Источник тока, колебательный контур и блок питания заключены в отдельный корпус,
%с названием <<Резонанс токов>> на верхней крышке, 
отмеченный на рисунке штриховой линией. 
На корпусе имеются коаксиальные разъёмы <<Вход>>,
<<${U}_1$>> и <<${U}_2$>>, а также переключатель магазина ёмкостей~$C_n$,
$n=1\ldots7$. Величины ёмкостей $C_n$ и сопротивления $R_1$ указаны на установке.
Напряжение $\mathcal{E}=\mathcal{E}_0\cos(\omega t+\varphi_0)$ от генератора поступает на
вход источника тока. Это же напряжение через разъём <<${U}_1$>> подаётся на
канал~1 осциллографа и на вход вольтметра~1. Переменное напряжение на
сопротивлении~$R_1$ в используемой схеме равно напряжению~$\mathcal{E}$ 
на выходе генератора и совпадает с ним по фазе. 
Cледовательно, ток $I$ во внешней цепи параллельного контура 
определяется формулами
\begin{equation}\eqmark{2.3.1}
	I=\frac{\mathcal{E}}{R_1}=I_0\cos(\omega t+\varphi_0), \qquad 
    I_0=\frac{\mathcal{E}_0}{R_1}.
\end{equation}

Напряжение на контуре $U$, совпадающее с напряжением на конденсаторе $U_C$,
поступает \emph{со знаком} <<$–$>> через разделительный конденсатор и разъём
<<${U}_2$>> на канал~2~осциллографа, а также на вход вольтметра~2.


Колебательный контур нашей установки собран из стандартных элементов,
используемых в современных радиоэлектронных цепях. Характеристики реальных элементов
представлены в описании работы \ref{lab:322}. \emph{Для понимания дальнейшего
изложения читателю рекомендуется ознакомиться с разделом
<<Особенности реальных элементов цепи>> на стр.~\pageref{par:real_elements}.} 
С учётом приведённых в указанном материале результатов получаем выражения
для импедансов ёмкостной $Z_C$ и индуктивной $Z_L$ ветвей параллельного 
колебательного контура:
\begin{equation}\eqmark{2.3.2}
	Z_C=R_S-\dfrac{i}{\omega C}, \qquad Z_L=R+R_L+i\omega L,
\end{equation}
где $R_S\equiv\dfrac{\tg\delta}{\omega C}$ и $R_L$~--- активные части
импедансов конденсатора и катушки индуктивности соответственно, а $R$~--- величина
постоянного активного сопротивления, добавленного в индуктивную ветвь
колебательного контура для снижения его добротности с целью упрощения  процедур
получения и обработки резонансных кривых. 
Конденсаторы магазина ёмкостей~$C_n$ в интересующем нас диапазоне
частот имеют относительно малые потери: для них $\tg\delta<10^{-3}.$

Добротность $Q$ контуров в наших установках достаточно высока, поэтому
можно пользоваться формулами \chaptereqref{2.34} и \chaptereqref{2.64}, в
которых, однако, надо учитывать, что суммарное активное сопротивление контура в
этом случае даётся формулой
\begin{equation}\eqmark{2.3.3}
	R_{\Sigma}=R+R_L+R_S
\end{equation}
и, следовательно,
\begin{equation}\eqmark{2.3.4}
	Q=\frac{\rho}{R_{\Sigma}}=
    \frac{\omega_0 L}{R_{\Sigma}}=
    \frac{1}{\omega_0 CR_{\Sigma}} \gg 1.
\end{equation}
Сильное неравенство \eqref{2.3.4} в рабочем диапазоне частот выполняется для
всех контуров, используемых в работе.

\important{Комплексные амплитуды} токов в ёмкостной $\vec{I}_{\! C}$ и индуктивной
$\vec{I}_{\! L}$ ветвях контура, а также напряжения $\vec U$ на контуре, положив без
ограничения общности $\varphi_0=0$ в выражении для внешнего тока $\vec
I=I_0e^{i\varphi_0}$ и используя формулы \chaptereqref{2.64} с учётом
\eqref{2.3.1}–-\eqref{2.3.4}, удобно представить в виде
\begin{align}
%	\eqmark{2.3.5}
%		\begin{equation}
			\eqmark{2.3.5a}
			\vec{I}_{\! C} &= \vec{I}\dfrac{Z_{LR}}{Z_C+Z_{LR}}=
                iQI_0\dfrac{\frac{\omega}{\omega_0}-i \frac{R+R_L}{\rho}}{%
                        1+iQ(\frac{\omega}{\omega_0}-\frac{\omega_0}{\omega})}, \\
%		\end{equation}
%		\begin{equation}
			\eqmark{2.3.5b}
			\vec{I}_{\! L}&=\vec{I}\dfrac{Z_C}{Z_C+Z_LR}=
                -iQI_0\dfrac{\frac{\omega_0}{\omega}(1+\tg\delta)}{%
                    1+iQ(\frac{\omega}{\omega_0}-\frac{\omega_0}{\omega})}, \\
%		\end{equation}
%		\begin{equation}
			\eqmark{2.3.5c}
			\vec U&=\vec{I}\dfrac{Z_C Z_LR}{Z_C+Z_LR}= 
                Q\rho I_0 \dfrac{[1-i\frac{\omega_0}{\omega}\frac{R+R_L}{\rho}](1+i\tg\delta)}{%
                    1+iQ(\frac{\omega}{\omega_0}-\frac{\omega_0}{\omega})}.
%		\end{equation}
\end{align}
Из формул \eqref{2.3.5b}, \eqref{2.3.5c} следует, что потерями в конденсаторах,
явно представленных величиной $\tg\delta<10^{-3}$, можно пренебречь. 
В~то же время необходимость учёта вклада этих потерь в
суммарное активное сопротивление контура~$R_{\Sigma}$ 
вблизи резонанса, примерно равного $\rho\tg\delta,$ можно будет оценить только по
результатам эксперимента.

Наибольший практический интерес для контуров с \important{высокой добротностью}
($Q\gg1$) представляет случай, когда отклонение $\Delta\omega=\omega-\omega_0$ частоты
внешней ЭДС от собственной частоты контура удовлетворяет сильному неравенству
\begin{equation}\eqmark{2.3.6}
|\Delta\omega|\ll \omega_0.
\end{equation}
При этом в первом порядке малости по относительной расстройке частоты 
$\Delta\frac{\omega}{\omega_0}$ выполняется соотношение
\begin{equation}\eqmark{2.3.7}
\frac{\omega_0}{\omega_0}-\frac{\omega}{\omega}\approx\frac{2\Delta\omega}{\omega_0},
\end{equation}
которое позволяет упростить выражения \eqref{2.3.5a}--\eqref{2.3.5c} 
и представить вещественные части комплексных амплитуд в виде
\begin{align}
%	\eqmark{2.3.8}
%		\begin{equation}
			\eqmark{2.3.8a}
			I_C(t) &= QI_0\dfrac{\omega}{\omega_0}
                \dfrac{\cos(\omega t-\psi_C)}{R\sqrt{1+(\tau\Delta\omega)^2}}, &
            \psi_C &=
                \arctg(\tau\Delta\omega)-\dfrac{\pi}{2}+\dfrac{R+R_L}{\rho}, \\
%		\end{equation}
%		\begin{equation}
			\eqmark{2.3.8b}
			I_L(t) &= QI_0\dfrac{\omega_0}{\omega} 
                \dfrac{\cos(\omega t-\psi_L)}{\sqrt{1+(\tau\Delta\omega)^2}}, &
            \psi_L &= \arctg(\tau\Delta\omega)+\dfrac{\pi}{2}-\delta, \\
%		\end{equation}
%		\begin{equation}
			\eqmark{2.3.8c}
			U(t) &= Q\rho I_0 \dfrac{\cos(\omega t-\psi_U)}{% 
                \sqrt{1+(\tau\Delta\omega)^2}}, &
            \psi_U &=\arctg(\tau\Delta\omega)+
                \dfrac{\omega_0}{\omega}\dfrac{R+R_L}{\rho}-\delta,
%		\end{equation}
\end{align}
%$$
%	I_C(t)=QI_0\dfrac{\omega}{\omega_0}\dfrac{\cos(\omega
% t-\psi_C)}{R\sqrt{1+(\tau\Delta\omega)^2}},~~~~
%	\psi_C=\arctg(\tau\Delta\omega)-\dfrac{\pi}{2}+\dfrac{R+R_L}{\rho},
% \eqno{(8a)}
%$$
%$$
%	I_L(t)=
%	QI_0\dfrac{\omega_0}{\omega}
%	\dfrac{\cos(\omega
% t-\psi_L)}{\sqrt{1+(\tau\Delta\omega)^2}},
% ~~~~\psi_L=\arctg(\tau\Delta\omega)+\dfrac{\pi}{2}-\delta, \eqno{(8\text{б})}
%$$
%$$
%	U(t)=
%	Q\rho I_0
%	\dfrac{\cos(\omega t-\psi_{{U}})}
%	{\sqrt{1+(\tau\Delta\omega)^2}},~~~~
%
% \psi_{{U}}=\arctg(\tau\Delta\omega)+\dfrac{\omega_0}{\omega}\dfrac{R+R_L}{\rho}
% -\delta \eqno{(8\text{в})}
%$$
где $\tau=2L/R_{\Sigma}=2Q/\omega_0$ --- время затухания
колебательного контура. В выражениях \eqref{2.3.8a}--\eqref{2.3.8c} 
мы сохранили в прежнем виде множители с отношениями частот в амплитудах и учли только линейные по малым
величинам $(R~+~R_L)/\rho$ и $\delta$ поправки в фазах, причём величину $\delta$
сохранили исключительно для общности, полагая её, однако, константой.

Как видно из выражений \eqref{2.3.8a}--\eqref{2.3.8c}, 
вблизи частоты~$\omega_0$ зависимости амплитуд токов и напряжения на контуре 
от частоты~$\omega$ несколько различаются, что надо иметь в виду при экспериментальном 
исследовании резонанса токов \important{по напряжению на контуре}~$U.$

Отдельно обратим внимание на тот факт, что зависимость \eqref{2.3.8c} амплитуды
напряжения $U$ на параллельном контуре от частоты $\omega$ вблизи резонанса в
принятом приближении совпадает с аналогичной зависимостью \chaptereqref{2.60b}
амплитуды тока $I_{\omega}$ для последовательного контура в том же приближении.

В резонансе, когда для высокодобротного контура можно положить
$\omega=\omega_0$, амплитуды токов и напряжения \eqref{2.3.8a}--\eqref{2.3.8c}
и их фазы 
%и производная фазы $\psi_U$ по частоте $\omega$ 
принимают вид
\begin{equation}
	\eqmark{2.3.9}
    \begin{aligned}
%			\eqmark{2.3.9a}
			 I_C(\omega_0) & = QI_0, & 
                 \psi_C(\omega_0) & = -\dfrac{\pi}{2}+Q^{-1}-\tg\delta, \\
			 I_L(\omega_0) & = QI_0, & \psi_L(\omega_0) & =
                 \dfrac{\pi}{2}-\delta, \\
%			\eqmark{2.3.9b}
			U(\omega_0) &= Q\rho I_0, & \psi(\omega_0) & = 
                    Q^{-1}-\tg\delta-\delta.
%                    \qquad \psi'_U(\omega_0)=\tau.
    \end{aligned}
\end{equation}
%$$
%	I_C(\omega_0)=
%	QI_0,~~\psi_C(\omega_0)=
%	-\dfrac{\pi}{2}+Q^{-1}-\tg\delta,~~~~~~
%	I_L(\omega_0)=
%	QI_0,~~
%	\psi_L(\omega_0)=
%	\dfrac{\pi}{2}-\delta, \eqno{(9a)}
%$$
%$$
%	U(\omega_0)=
%	Q\rho I_0,~~~~
%	\psi(\omega_0)=
%	Q^{-1}-\tg\delta-\delta,~~~~
%	\psi'_{{U}}(\omega_0)=\tau. \eqno{(9\text{б})}
%$$
%\setcounter{equation}{9}
В последнем равенстве мы пренебрегли относительными поправками порядка $Q^{-2}$
и $Q^{-1}\tg\delta$. Из формул \eqref{2.3.9} следует, что на частоте $\omega_0$
токи $\vec{I}_{\! C}$ и~$\vec{I}_{\! L}$ в ёмкостной и индуктивной ветвях контура в~$Q$ раз
превышают по амплитуде ток~$\vec{I}$ во внешней цепи. При этом ток~$\vec{I}_{\! C}$
опережает внешний ток~$\vec{I}$ по фазе почти на $\pi/2$, 
а ток~$\vec{I}_{\! L}$ отстаёт от~$\vec{I}$ почти на~$\pi/2$. 
Между собой токи~$\vec{I}_{\! C}$ и~$\vec{I}_{\! L}$ сдвинуты по фазе на угол, 
близкий к  $\pi.$ Можно сказать, что токи $\vec{I}_{\! C}$ и $\vec{I}_{\! L}$ 
образуют контурный ток, последовательно обтекающий элементы контура и 
в~$Q$ раз превышающий внешний ток $\vec{I}$. Именно
последнее обстоятельство послужило поводом назвать резонанс в 
параллельном контуре <<резонансом токов>>.

Отметим также, что максимальные (резонансные) значения токов в контуре 
не строго равны $QI_0$ и достигаются не строго на частоте $\omega_0$.
Соответствующие относительные поправки составляют доли малой 
величины $Q^{-2}$ и связаны с входящим в выражения 
\eqref{2.3.8a}, \eqref{2.3.8b} для вещественных амплитуд 
токов $\vec{I}_{\! C}$, $\vec{I}_{\! L}$ множителем $\frac{\omega}{\omega_0}$.

Из формул \eqref{2.3.9} вытекает, что на частоте $\omega_0$ импеданс контура
$Z(\omega_0)=\vec U(\omega_0)/I_0$ является почти чисто активным. В~пренебрежении 
относительными поправками порядка $Q^{-2}$ его модуль и фаза относительно внешнего 
тока определяются формулами
\begin{equation}\eqmark{2.3.10}
|Z(\omega_0)|=Q\rho=Q^2R_{\Sigma}, \qquad
\psi_Z(\omega_0)=\dfrac{R+R_L}{\rho}-\delta,
\end{equation}
которые дополняют формулы \chaptereqref{2.67} учётом активных потерь в катушке
индуктивности и в конденсаторе.

При отклонении $\Delta\omega$ частоты внешней ЭДС от частоты $\omega_0$, таком,
что выполняется условие
\begin{equation}\eqmark{2.3.11}
\tau\Delta\omega=\pm1,
\end{equation}
амплитуда напряжения $U,$ как видно из формул \eqref{2.3.8c}, уменьшается в
$\sqrt{2}$ раз относительно своей резонансной величины, а фаза $\psi_U$
изменяется примерно на угол $\pm\pi/4.$
Величина $\delta\omega\equiv2|\Delta\omega_{\gamma}|=2\gamma=2/\tau$
представляет собой важную характеристику колебательного
контура~---~\important{ширину резонансной кривой} $U(\omega),$ по которой с
учётом соотношений $Q=\omega_0/2\gamma=\tau\omega_0/2,$ зная частоту $\omega_0,$
можно найти добротность контура
\begin{equation}\eqmark{2.3.12}
Q=\dfrac{\omega_0}{\delta\omega}.
\end{equation}

Эти же параметры можно определить по фазово-частотной характеристике: тангенс
угла наклона $\psi_U$ в точке $\omega=\omega_0$ согласно \chaptereqref{2.66c}
определяет время затухания $\tau,$ а расстояние по оси $\omega$ между точками, в
которых фаза $\psi_U(\omega)$ меняется от $-\pi/4$ до $\pi/4,$ равно $2/\tau$ с
относительной погрешностью порядка $Q^{-2}.$

\begin{lab:task}
    \item Проведите настройку экспериментального стенда по техническому описанию
установки.

    \item Меняя частоту $\nu$ генератора, убедитесь по осциллографу и вольтметрам,
что у синусоиды $U(t)$ меняется амплитуда и фаза относительно начала координат,
тогда как синусоида $\mathcal{E}(t)$~---~синхронизующий сигнал~---~<<привязана>>
к началу отсчёта при начальных условиях: $\mathcal{E}(0)=0$,
$\dot{\mathcal{E}}(0)=0$, а её амплитуда остаётся неизменной с относительной
погрешностью  $\le1\%$. После этого можно приступить к измерениям.

    \item Для контуров с семью различными ёмкостями $C_n$, меняя их с помощью
переключателя на блоке, измерьте резонансные частоты~$\nu_{0n}$ и напряжения
$U(\nu_{0n})$. Регистрируйте для контроля также напряжения~$\mathcal{E}$,
игнорируя отклонения в пределах относительной погрешности~$1\%$. Состояние
резонанса определяйте по максимуму напряжения $U(\nu_{0n})$, измеряемого
вольтметром и наблюдаемого на экране осциллографа. Приближение к резонансу
удобно наблюдать по фигуре Лиссажу на экране осциллографа в режиме $X$--$Y$.
При этом фигура Лиссажу представляет собой эллипс, вырождающийся в прямую линию
с положительным наклоном \important{почти} на частоте $\nu_{0n}$.

    \item \footnote{Дополнительное упражнение, выполняется по указанию
        преподавателя.}Проделайте измерения п.~3 для напряжения, существенно 
    отличающегося от использованного в п.~3, но лежащего в диапазоне $100\div500$~мВ 
    по амплитуде.

    \item Для контуров с двумя разными ёмкостями (по указанию преподавателя)
снимите амплитудно-частотные характеристики $U(f)$ для значений
$U(f)\ge0,6U(\nu_{0n})$ (всего 16--18 точек по обе стороны от резонанса) при
том же напряжении $\mathcal{E}$, что и в п.~3.

    \item Для тех же двух контуров измерьте фазово-частотные характеристики
$\psi_U(f)$ для значений $U(f)\ge0,3U(\nu_{0n})$ (всего 16--18 точек по обе
стороны от резонанса) при том же напряжении, что и в п.~3. Перед выполнением
этой части работы измените с помощью ручек горизонтальной развёртки настройки
осциллографа таким образом, чтобы синхронизующий сигнал $\mathcal{E}(t)$ был
<<привязан>> к общему началу отсчёта времени и напряжений на экране, лежащему на
оси X координатной сетки экрана (см. п.~2), и оба сигнала были симметричны
относительно этой оси. Если это не так, то следует повторить процедуру
центрирования горизонтальных осей каналов по техническому описанию.

    Расстояние $x$ от начала отсчёта до точки первого обращения в нуль
напряжения $U(t)$ на участке спада на осциллограмме характеризует разность фаз
$\Delta\varphi$ сигналов $U(t)$ и $\mathcal{E}(t).$ Эта величина, выраженная в
радианах, очевидно, даётся формулой $\Delta\varphi=\frac{x}{x_0}\pi$, где
$x_0$~--- расстояние от начала отсчёта до точки первого обращения в нуль
напряжения~$\mathcal{E}(t)$ на участке подъёма, соответствующее полупериоду
этого сигнала.

	\tasksection{Обработка и представление результатов}
% 	\important{Настоятельно рекомендуется для обработки и представления
% результатов измерений использовать электронные таблицы.}

	\item Результаты измерений п.~3 внесите в таблицу:\par
%	    \begin{table}[h!]
	        %\caption{\label{tab:1}}
%	        \caption{}
%	        \tabmark{2.3.1}
%	        \begin{center}
\begingroup
\noindent\small \begin{tabular}{|c|c|c|c|c|c|c|c|c|c|c|}
	                \hline
	                $C,$ & $f,$ & $U,$ & $\mathcal{E},$ & $L,$ &
$\rho,$ & $|Z_{\text{рез}}|,$& $Q$ & $R_{\Sigma},$ & $R_{S
\text{max}},$& $R_L,$\\

	                нФ & кГц & В & В & мкГн & Ом & Ом &  & Ом & Ом & Ом \\
	                \hline
	                $C_1$ & --- & --- & --- & --- & --- & --- & --- & --- & --- & --- \\
	                \hline
	                $\cdots$ & --- & --- & --- & --- & --- & --- & --- & ---& --- & --- \\
	                \hline
	                $C_7$& --- & --- & --- & --- & --- & --- & ---& --- &--- &---\\
	                \hline
	                \multicolumn{4}{|c|}{Среднее значение} & --- & & & & & & --- \\
	                \hline
	                \multicolumn{4}{|c|}{Среднекв. отклонение} & ---& & & & & &--- \\
			                    \hline
%\multicolumn{4}{|c|}{Коэфф. Стьюдента} & & & & & & & \\
%\multicolumn{4}{|c|}{$t_{n\alpha}$, $n=7$, $\alpha=0,05$} & & & & & & & \\ \hline
	                \multicolumn{4}{|c|}{Случ. погрешность} & ---& & & & & &---\\
	                \hline
	            \end{tabular}\par
\endgroup

В первый столбец этой таблицы запишите значения ёмкостей $C_n$. 
Для каждого значения $C_n$ по формулам вводной части и данным 
эксперимента проведите \important{последовательно} расчёт~$L$, 
$\rho$, $|Z_{\text{рез}}|$, $Q$, $R_{\Sigma}$, 
$R_{S \text{max}}=10^{-3}\rho$, $R_L$.
Затем рассчитайте средние значения $\left< L \right>$ и $\left< R_L \right>$ 
и их случайные погрешности~$\Delta L$ и~$\Delta R_L$.

%Представьте результат \important{косвенных измерений при невоспроизводимых
%условиях,} проделанных в работе, в виде: $\langle L \rangle\pm\Delta L$ и
%$\langle R_L\rangle\pm\Delta R_L,$ где угловыми скобками отмечено среднее
%значение, а символом $"\Delta"$ – случайная погрешность величин $L$ и $R_L.$

Оцените относительный вклад активных потерь в конденсаторах 
(представленных в таблице сопротивлением $R_{S\text{max}}$, рассчитанным для
максимального значения $\tg\delta=10^{-3}$) в суммарное активное сопротивление
контура.

\item *Выполните задание п.~7 для данных,
полученных в п.~4. Сравните с результатами п.~7. Объясните причины расхождения
результатов, если они обнаружатся.

\item По данным измерений п.~5 постройте на одном графике амплитудно-частотные
характеристики $U(\nu)$  для выбранных контуров. Проведите сравнительный 
анализ характеристик.

\item По данным измерений п.~5 постройте на одном графике амплитудно-частотные
характеристики в безразмерных координатах $x\equiv \nu/\nu_{0n}$, $y\equiv
U(x)/U(1).$ По ширине резонансных кривых на уровне 0,707 определите добротности
$Q$ соответствующих контуров. Оцените погрешности. Сравните эти величины с
расчётами п.~7.

\item По данным измерений п.~6 постройте на одном графике фазово-частотные
характеристики $\psi_U(\nu)$ в координатах $x\equiv \nu/\nu_{0n}$, $y\equiv\psi_U/\pi$
для выбранных контуров. По этим характеристикам определите добротности контуров
одним из двух способов: по формуле $Q=\frac12 \frac{d\varphi_U(x)}{dx}$ при $x=1$ или по
расстоянию $1/Q$ между точками оси $x,$ в которых  меняется от $-1/4$ до $1/4$. 
Оцените погрешности. Сравните с результатами п.~7 и~10.

\item Постройте зависимость $R_L(\nu_{0n})$ в системе координат с началом 
в точке $(0,6\nu_{07};0)$ нанесите на график прямую $\langle R_L \rangle$. 
Назовите возможные причины изменения~$R_L$ с частотой.

\item Постройте векторную диаграмму токов и напряжений для контура 
с наименьшей добротностью в резонансном состоянии. Ось ординат направьте по 
вектору $\vec{I}$.
\end{lab:task}


\begin{lab:questions}
    \item Что такое импеданс электрической цепи?
    Как складываются импедансы при последовательном и параллельном
    соединении элементов?
    
    \item Дайте определение добротности колебательного контура.
    Опишите известные вам способы измерения добротности.
    
    \item Дайте энергетическое определение добротности колебательного контура.

    \item  Получите выражение для напряжения на катушке индуктивности  в
резонансе.

    \item Дайте обоснование способам определения добротности по
фазово-частотной характеристике.

    \item По каким причинам потери в контуре зависят от частоты?

    \item *Зависят ли потери в контуре от амплитуды сигнала, и если зависят, то
по каким причинам?

    \item *Оцените, на какой частоте $\omega_m$ эллипс на экране осциллографа в
п.~3 вырождается в прямую линию с положительным наклоном.
\end{lab:questions}


\begin{lab:literature}
        \item \SivuhinIII~--- \S~126, 127.
        \item \Kirichenko~--- \S~17.1--17.3.
    \item *\emph{Титце~У., Шенк~К.} Полупроводниковая схемотехника.  --- Т.~II. ---
М.: ДМК~Пресс, 2007.~--- \S~12.1.
\end{lab:literature}

