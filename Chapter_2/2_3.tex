\lab{Резонанс токов в параллельном контуре}

\begin{lab:aim}
	исследование резонанса токов в параллельном колебательном контуре с изменяемой ёмкостью, включающее получение амплитудно-частотных и фазово-частотных характеристик, а также определение основных параметров контура.
\end{lab:aim}

\begin{lab:equipment}
	генератор сигналов, источник напряжения, нагрузкой которого является последовательный колебательный контур с переменной ёмкостью, двулучевой осциллограф, цифровые вольтметры.
\end{lab:equipment}

\warning{
	Перед выполнением работы следует изучить основы теории электрических  колебаний по вводной части Раздела II настоящего сборника и/или рекомендованной в нём литературе. Необходимые дополнения будут приведены ниже.
}
\experiment
Блок-схема экспериментального стенда для изучения резонанса токов в параллельном колебательном контуре показана на рис.~\figref{exp schem}. Синусоидальный сигнал от генератора поступает на вход \important{управляемого напряжением источника тока} (см., например, [1]), собранного на операционном усилителе с полевым транзистором, питание которых осуществляется встроенным блоком-выпрямителем от сети $\sim220$~В (цепи питания на схеме не показаны).  Внутреннее (выходное) сопротивление источника тока, бесконечно большое в идеальном случае, в нашей схеме составляет несколько ГОм. Это обеспечивает постоянство амплитуды тока $I$ на меняющейся, но значительно меньшей по величине, нагрузке – параллельном контуре, изображенном на рис.~\figref{exp schem} в виде эквивалентной схемы.

\todo[author=Tiffani]{Сделать нормальный рисунок. Этот даже не отображается}
\begin{figure}[h!]
%	\pic{0.38\textwidth}{2_3_1}% закомментированно, т.к. не работает (нужны нормальные картинки)
	\caption{Блок-схема экспериментального стенда}
	\figmark{exp schem}
\end{figure}


Источник тока, колебательный контур и блок питания заключены в отдельный корпус с названием <<Резонанс токов>> на верхней крышке, отмеченный на рисунке штриховой линией. На корпусе имеются коаксиальные разъёмы <<Вход>>, <<$\text{U}_1$>> и <<$\text{U}_2$>>, а также переключатель магазина ёмкостей $C_n$ с указателем номера $n=1,~2~\ldots7.$ Величины ёмкостей $C_n$ и сопротивления $R_1$ указаны в табличке на крышке корпуса. Напряжение $\varepsilon=\varepsilon_0\cos(\omega t+\varphi_0)$ от генератора поступает на вход источника тока. Это же напряжение через разъём <<$\text{U}_1$>> подаётся на канал 1 осциллографа и на вход вольтметра 1. Переменное напряжение на сопротивлении $R_1$ в используемой схеме равно напряжению $\varepsilon$ на выходе генератора и совпадает с ним по фазе. Cледовательно, ток $I$ во внешней цепи параллельного контура определяется формулами:

\begin{equation}\eqmark{2.3.1}
	I=\varepsilon/R_1=I_0\cos(\omega t+\varphi_0),~~~~I_0=\varepsilon_0/R_1.
\end{equation}

Напряжение на контуре $U,$ совпадающее с напряжением на конденсаторе $U_C,$ поступает \important{со знаком <<–>>} через разделительный конденсатор и разъём <<$\text{U}_2$>> на канал 2 осциллографа, а также на вход вольтметра 2.%!!!!!!!!!!!!!!!!!!

Колебательный контур нашей установки собран из стандартных элементов, используемых в современных радиоэлектронных цепях. Характеристики этих элементов представлены в описании работы <<Резонанс напряжений в последовательном контуре>>. \underline{Соответствующий материал читателю необходимо освоить для} \underline{понимания дальнейшего изложения и выполнения лабораторной работы.} С учётом приведённых в указанном материале результатов для импедансов ёмкостной $Z_{\scriptscriptstyle{C}}$ и индуктивной $Z_{\scriptscriptstyle{L}}$ ветвей параллельного колебательного контура получаем выражения:

\begin{equation}\eqmark{2.3.2}
	Z_{\scriptscriptstyle{C}}=R_{\scriptscriptstyle{S}}-\dfrac{i}{\omega C},~~~~Z_{\scriptscriptstyle{L}}=R+R_{\scriptscriptstyle{L}}+i\omega L,
\end{equation}
где $R_{\scriptscriptstyle{S}}\equiv\dfrac{1}{\omega C}\tg\delta$ и $R_{\scriptscriptstyle{L}}$ - активные части импедансов конденсатора и катушки индуктивности соответственно, а $R$  величина постоянного активного сопротивления, добавленного в индуктивную ветвь колебательного контура для снижения его добротности с целью упрощения  процедур получения и обработки резонансных кривых (см. рис.~\figref{exp schem} и табличку на корпусе). Конденсаторы магазина ёмкостей $C_n$ в интересующем нас диапазоне частот имеют относительно малые потери: для них $\tg\delta<10^{-3}.$

Добротность $Q$ контуров в наших установках является достаточно высокой, чтобы можно было пользоваться формулами \eqref{2.34} и \eqref{2.64}, в которых, однако, надо учитывать, что суммарное активное сопротивление контура в этом случае даётся формулой
\begin{equation}\eqmark{2.3.3}
	R_{\scriptscriptstyle{\sum}}=R+R_{\scriptscriptstyle{L}}+R_{\scriptscriptstyle{S}}
\end{equation}
и, следовательно,
\begin{equation}\eqmark{2.3.4}
	Q=\rho/R_{\scriptscriptstyle{\sum}}=\omega_{\scriptscriptstyle{0}}L/R_{\scriptscriptstyle{\sum}}=1/\omega_{\scriptscriptstyle{0}}CR_{\scriptscriptstyle{\sum}}>>1.
\end{equation}
Сильное неравенство в \eqref{2.3.4} в рабочем диапазоне частот выполняется для всех контуров, используемых в работе.

\underline{Комплексные амплитуды} токов в ёмкостной $\vec I_{\scriptscriptstyle{C}}$ и индуктивной $\vec I_{\scriptscriptstyle{L}}$ ветвях контура, а также напряжения $\vec U$ на контуре, положив без ограничения общности $\varphi_{\scriptscriptstyle{0}}=0$ в выражении для внешнего тока $\vec I=I_{\scriptscriptstyle{0}}e^{i\varphi_{\scriptscriptstyle{0}}}$ и используя формулы \eqref{2.64} с учётом \eqref{2.3.1}~–~\eqref{2.3.4}, удобно представить в виде:
\begin{subequations}
\renewcommand{\theequation}{\theparentequation \asbuk {equation}} %%данная строчка позволяет делать автоматическую подчиненную нумерацию русскими буквами.
	\eqmark{2.3.5}
		\begin{equation}
			\eqmark{2.3.5а}
			\vec I_{\scriptscriptstyle{C}}=\vec I\dfrac{Z_{\scriptscriptstyle{LR}}}{Z_{\scriptscriptstyle{C}}+Z_{\scriptscriptstyle{LR}}}=iQI_{\scriptscriptstyle{0}}\dfrac{\omega/\omega_{\scriptscriptstyle{0}}-i(R+R_{\scriptscriptstyle{L}})/\rho}{1+iQ(\omega/\omega_{\scriptscriptstyle{0}}-\omega_{\scriptscriptstyle{0}}/\omega)},
		\end{equation}
		\begin{equation}
			\eqmark{2.3.5б}
			\vec I_{\scriptscriptstyle{L}}=\vec I\dfrac{Z{\scriptscriptstyle{C}}}{Z_{\scriptscriptstyle{C}}+Z_{\scriptscriptstyle{LR}}}=
-iQI_{\scriptscriptstyle{0}}\dfrac{(\omega_{\scriptscriptstyle{0}}/\omega)(1+\tg\delta)}{1+iQ(\omega/\omega_{\scriptscriptstyle{0}}-\omega_{\scriptscriptstyle{0}}/\omega)},
		\end{equation}
		\begin{equation}
			\eqmark{2.3.5в}
			\vec U=\vec I\dfrac{Z_{\scriptscriptstyle{C}}Z_{\scriptscriptstyle{LR}}}
{Z_{\scriptscriptstyle{C}}+Z_{\scriptscriptstyle{LR}}}=
Q\rho I_{\scriptscriptstyle{0}}
\dfrac{[1-i\omega_{\scriptscriptstyle{0}}(R+R{\scriptscriptstyle{L}})/\omega\rho]
(1+i\tg\delta)}
{1+iQ(\omega/\omega_{\scriptscriptstyle{0}}-\omega_{\scriptscriptstyle{0}}/\omega)}.
		\end{equation}
\end{subequations}
%$$
%\vec I_{\scriptscriptstyle{C}}=\vec I\dfrac{Z_{\scriptscriptstyle{LR}}}{Z_{\scriptscriptstyle{C}}+Z_{\scriptscriptstyle{LR}}}=iQI_{\scriptscriptstyle{0}}\dfrac{\omega/\omega_{\scriptscriptstyle{0}}-i(R+R_{\scriptscriptstyle{L}})/\rho}{1+iQ(\omega/\omega_{\scriptscriptstyle{0}}-\omega_{\scriptscriptstyle{0}}/\omega)} \eqno{(5a)}
%$$
%$$
%\vec I_{\scriptscriptstyle{L}}=\vec I\dfrac{Z{\scriptscriptstyle{C}}}{Z_{\scriptscriptstyle{C}}+Z_{\scriptscriptstyle{LR}}}=
%-iQI_{\scriptscriptstyle{0}}\dfrac{(\omega_{\scriptscriptstyle{0}}/\omega)(1+\tg\delta)}{1+iQ(\omega/\omega_{\scriptscriptstyle{0}}-\omega_{\scriptscriptstyle{0}}/\omega)} \eqno{(5\text{б})}
%$$
%$$
%\vec U=\vec I\dfrac{Z_{\scriptscriptstyle{C}}Z_{\scriptscriptstyle{LR}}}
%{Z_{\scriptscriptstyle{C}}+Z_{\scriptscriptstyle{LR}}}=
%Q\rho I_{\scriptscriptstyle{0}}
%\dfrac{[1-i\omega_{\scriptscriptstyle{0}}(R+R{\scriptscriptstyle{L}})/\omega\rho]
%(1+i\tg\delta)}
%{1+iQ(\omega/\omega_{\scriptscriptstyle{0}}-\omega_{\scriptscriptstyle{0}}/\omega)}. \eqno{(5\text{в})}
%$$
%\setcounter{equation}{5}
Из формул \eqref{2.3.5б}, \eqref{2.3.5в} следует, что потерями в конденсаторах, явно представленных величиной $\tg\delta,$ в нашем случае меньшей $10^{-3},$ можно пренебречь. В то же время необходимость учёта вклада этих потерь в суммарное активное сопротивление контура $R_{\scriptscriptstyle{\sum}}$ вблизи резонанса, примерно равного $\rho\tg\delta,$ можно будет оценить только по результатам эксперимента.

Наибольший практический интерес для контуров с \underline{высокой добротностью} представляет случай, когда отклонение $\Delta\omega=\omega-\omega_{\scriptscriptstyle{0}}$ частоты внешней ЭДС от собственной частоты контура удовлетворяет сильному неравенству
\todo[author=Tiffani]{Красный цвет в ур. \eqref{2.3.6} << - так задумано или исправлять?}
\begin{equation}\eqmark{2.3.6}
|\Delta\omega|\textcolor{red}{\text{<<}}~\omega_{\scriptscriptstyle{0}}
\end{equation}
При этом в первом порядке малости по \underline{относительной расстройке частоты} $\Delta\omega/\omega_{\scriptscriptstyle{0}}$ выпол\-няется соотношение
\begin{equation}\eqmark{2.3.7}
\dfrac{\omega_{\scriptscriptstyle{0}}}{\omega_{\scriptscriptstyle{0}}}-\dfrac{\omega}{\omega}=\dfrac{2\Delta\omega}{\omega_{\scriptscriptstyle{0}}},
\end{equation}
которое позволяет упростить выражения \eqref{2.3.5} и представить вещественные части комплексных амплитуд в виде:
\begin{subequations}
\renewcommand{\theequation}{\theparentequation \asbuk {equation}} %%данная строчка позволяет делать автоматическую подчиненную нумерацию русскими буквами.
	\eqmark{2.3.8}
		\begin{equation}
			\eqmark{2.3.8а}
			I_{\scriptscriptstyle{C}}(t)=QI_{\scriptscriptstyle{0}}\dfrac{\omega}{\omega_{\scriptscriptstyle{0}}}\dfrac{\cos(\omega t-\psi_{\scriptscriptstyle{C}})}{R\sqrt{1+(\tau\Delta\omega)^2}},~~~~
	\psi_{\scriptscriptstyle{C}}=\arctg(\tau\Delta\omega)-\dfrac{\pi}{2}+\dfrac{R+R_{\scriptscriptstyle{L}}}{\rho},
		\end{equation}
		\begin{equation}
			\eqmark{2.3.8б}
			I_{\scriptscriptstyle{L}}(t)=
	QI_{\scriptscriptstyle{0}}\dfrac{\omega_{\scriptscriptstyle{0}}}{\omega}
	\dfrac{\cos(\omega t-\psi_{\scriptscriptstyle{L}})}{\sqrt{1+(\tau\Delta\omega)^2}},~~~~\psi_{\scriptscriptstyle{L}}=\arctg(\tau\Delta\omega)+\dfrac{\pi}{2}-\delta,
		\end{equation}
		\begin{equation}
			\eqmark{2.3.8в}
			U(t)=
	Q\rho I_{\scriptscriptstyle{0}}
	\dfrac{\cos(\omega t-\psi_{\scriptscriptstyle{U}})}
	{\sqrt{1+(\tau\Delta\omega)^2}},~~~~
	\psi_{\scriptscriptstyle{U}}=\arctg(\tau\Delta\omega)+\dfrac{\omega_{\scriptscriptstyle{0}}}{\omega}\dfrac{R+R_{\scriptscriptstyle{L}}}{\rho}-\delta,
		\end{equation}
\end{subequations}
%$$
%	I_{\scriptscriptstyle{C}}(t)=QI_{\scriptscriptstyle{0}}\dfrac{\omega}{\omega_{\scriptscriptstyle{0}}}\dfrac{\cos(\omega t-\psi_{\scriptscriptstyle{C}})}{R\sqrt{1+(\tau\Delta\omega)^2}},~~~~
%	\psi_{\scriptscriptstyle{C}}=\arctg(\tau\Delta\omega)-\dfrac{\pi}{2}+\dfrac{R+R_{\scriptscriptstyle{L}}}{\rho}, \eqno{(8a)}
%$$
%$$
%	I_{\scriptscriptstyle{L}}(t)=
%	QI_{\scriptscriptstyle{0}}\dfrac{\omega_{\scriptscriptstyle{0}}}{\omega}
%	\dfrac{\cos(\omega t-\psi_{\scriptscriptstyle{L}})}{\sqrt{1+(\tau\Delta\omega)^2}},~~~~\psi_{\scriptscriptstyle{L}}=\arctg(\tau\Delta\omega)+\dfrac{\pi}{2}-\delta, \eqno{(8\text{б})}
%$$
%$$
%	U(t)=
%	Q\rho I_{\scriptscriptstyle{0}}
%	\dfrac{\cos(\omega t-\psi_{\scriptscriptstyle{U}})}
%	{\sqrt{1+(\tau\Delta\omega)^2}},~~~~
%	\psi_{\scriptscriptstyle{U}}=\arctg(\tau\Delta\omega)+\dfrac{\omega_{\scriptscriptstyle{0}}}{\omega}\dfrac{R+R_{\scriptscriptstyle{L}}}{\rho}-\delta \eqno{(8\text{в})}
%$$
где $\tau=2L/R_{\scriptscriptstyle{\sum}}=2Q/\omega_{\scriptscriptstyle{0}}$ – время затухания колебательного контура. В выражениях \eqref{2.3.8} мы сохранили в прежнем виде множители с отношениями частот в амплитудах и учли только линейные по малым величинам $(R+R_{\scriptscriptstyle{L}})/\rho$ и $\delta$ поправки в фазах, причём величину $\delta$ сохранили исключительно для общности, положив её, однако, константой.

Как видно из выражений \eqref{2.3.8}, вблизи частоты $\omega_{\scriptscriptstyle{0}}$ зависимости амплитуд токов и напряжения на контуре от частоты $\omega$ несколько различаются, что надо иметь в виду при экспериментальном исследовании резонанса токов \underline{по напряжению на контуре} $U.$

Отдельно обратим внимание на тот факт, что зависимость \eqref{2.3.8в} амплитуды напряжения $U$ на параллельном контуре от частоты $\omega$ вблизи резонанса в принятом приближении совпадает с аналогичной зависимостью \eqref{2.60б} амплитуды тока $I_{\scriptscriptstyle{\omega}}$ для последовательного контура в том же приближении.

При резонансе, когда для высокодобротного контура можно положить $\omega=\omega_{\scriptscriptstyle{0}},~~\Delta\omega=0,$ амплитуды токов и напряжения \eqref{2.3.8}, их фазы и производная фазы $\psi_{\scriptscriptstyle{U}}$ по частоте $\omega$ принимают вид:
\begin{subequations}
\renewcommand{\theequation}{\theparentequation \asbuk {equation}} %%данная строчка позволяет делать автоматическую подчиненную нумерацию русскими буквами.
	\eqmark{2.3.9}
		\begin{equation}
			\eqmark{2.3.9а}
			\begin{gathered}[c]
			 I_{\sc{C}}(\omega_{\sc{0}})= QI_{\sc{0}},~~\psi_{\sc{C}}(\omega_{\sc{0}})= -\dfrac{\pi}{2}+Q^{-1}-\tg\delta, \\
			 I_{\sc{L}}(\omega_{\scriptscriptstyle{0}})= QI_{\sc{0}},~~\psi_{\sc{L}}(\omega_{\sc{0}})=\dfrac{\pi}{2}-\delta,	
			\end{gathered}
		\end{equation}
		\begin{equation}
			\eqmark{2.3.9б}
			U(\omega_{\sc{0}})=Q\rho I_{\sc{0}},~~~~\psi(\omega_{\sc{0}})=
	Q^{-1}-\tg\delta-\delta,~~~~\psi'_{\sc{U}}(\omega_{\sc{0}})=\tau.
		\end{equation}
\end{subequations}
%$$
%	I_{\sc{C}}(\omega_{\sc{0}})=
%	QI_{\sc{0}},~~\psi_{\sc{C}}(\omega_{\sc{0}})=
%	-\dfrac{\pi}{2}+Q^{-1}-\tg\delta,~~~~~~
%	I_{\sc{L}}(\omega_{\scriptscriptstyle{0}})=
%	QI_{\sc{0}},~~
%	\psi_{\sc{L}}(\omega_{\sc{0}})=
%	\dfrac{\pi}{2}-\delta, \eqno{(9a)}
%$$
%$$
%	U(\omega_{\sc{0}})=
%	Q\rho I_{\sc{0}},~~~~
%	\psi(\omega_{\sc{0}})=
%	Q^{-1}-\tg\delta-\delta,~~~~
%	\psi'_{\sc{U}}(\omega_{\sc{0}})=\tau. \eqno{(9\text{б})}
%$$
%\setcounter{equation}{9}
В последнем равенстве мы пренебрегли относительными поправками порядка $Q^{-2}$ и $Q^{-1}\tg\delta.$ Из формул \eqref{2.3.9а} следует, что на частоте $\omega_{\sc{0}}$ токи $\vec I_{\sc{C}}$ и $\vec I_{\sc{L}}$ в ёмкостной и индуктивной ветвях контура в $Q$ раз превышают по амплитуде ток $\vec I$ во внешней цепи. При этом ток $\vec I_{\sc{C}}$ опережает внешний ток $\vec I$ по фазе почти на $\pi/2,$ а ток $\vec I_{\sc{L}}$ – отстаёт от тока $\vec I$ почти на $\pi/2.$ Между собой токи $\vec I_{\sc{C}}$ и $\vec I_{\sc{L}}$ сдвинуты по фазе на угол, близкий к  $\pi.$ Можно сказать, что токи $\vec I_{\sc{C}}$ и $\vec I_{\sc{L}}$ образуют контурный ток, последовательно обтекающий элементы контура и в $Q$ раз превышающий внешний ток $\vec I.$ Как уже отмечалось в п. 3.2, последнее обстоятельство послужило поводом назвать резонанс в параллельном контуре <<резонансом токов>>.

Отметим также, что максимальные (резонансные) значения токов в контуре, как и в п.3.2, не строго равны $QI_{\sc{0}}$ и достигаются не строго на частоте $\omega_{\sc{0}}.$ Соответствующие относительные поправки составляют доли малой величины $Q^{-2}$ и связаны с входящим в выражения \eqref{2.3.8а}, \eqref{2.3.8б} для вещественных амплитуд токов $\vec I_{\sc{C}}$, $\vec I_{\sc{L}}$ отношением $\omega/\omega_{\sc{0}}.$

Из формул \eqref{2.3.9б} вытекает, что на частоте $\omega_{\sc{0}}$ импеданс контура $Z(\omega_{\sc{0}})=\vec U(\omega_{\sc{0}})/I_{\sc{0}}$ является почти чисто активным. В пренебрежении относительными поправками порядка $Q^{-2}$ его модуль и фаза относительно внешнего тока определяются формулами:
\begin{equation}\eqmark{2.3.10}
|Z(\omega_{\sc{0}})|=Q\rho=Q^2R_{\sc{\sum}},~~~~~\psi_{\sc{Z}}(\omega_{\sc{0}})=\dfrac{R+R_{\sc{L}}}{\rho}-\delta,
\end{equation}
которые дополняют формулы \eqref{2.67} учётом активных потерь в катушке индуктивности и в конденсаторе.

При отклонении $\Delta\omega$ частоты внешней ЭДС от частоты $\omega_{\sc{0}}$ таком, что выполняется условие
\begin{equation}\eqmark{2.3.11}
\tau\Delta\omega=\pm1,
\end{equation}
амплитуда напряжения $U,$ как видно из формул \eqref{2.3.8в}, уменьшается в $\sqrt{2}$ раз относительно своей резонансной величины, а фаза $\psi_{\sc{U}}$ изменяется примерно на угол $\pm\pi/4.$

Величина $\delta\omega\equiv2|\Delta\omega_{\sc{\gamma}}|=2\gamma=2/\tau$ представляет собой важную характеристику колебательного контура – \textsf{ширину резонансной кривой} $U(\omega),$ по которой с учётом соотношений $Q=\omega_{\sc{0}}/2\gamma=\tau\omega_{\sc{0}}/2,$ зная частоту $\omega_{\sc{0}},$ можно найти добротность контура
\begin{equation}\eqmark{2.3.12}
Q=\dfrac{\omega_{\sc{0}}}{\delta\omega}
\end{equation}

Эти же параметры можно определить по фазово-частотной характеристике: тангенс угла наклона $\psi_{\sc{U}}$ в точке $\omega=\omega_{\sc{0}}$ согласно \eqref{2.3.9б} определяет время затухания $\tau,$ а расстояние по оси $\omega$ между точками, в которых фаза $\psi_{\sc{U}}(\omega)$ меняется от $-\pi/4$ до $\pi/4,$ равно $2/\tau$ с относитель\-ной погрешностью порядка $Q^{-2}.$

%\begin{center}
%\underline{Выполнение эксперимента}
%\end{center}



%\newcounter{N}
\begin{lab:task}
\emph{Символом << * >> отмечены дополнительные задачи эксперимента и, соответственно, обработки и представления результатов, а также контрольные вопросы повышенной сложности.}
	\begin{enumerate}
%\begin{list}{\#\arabic{N}}{\usecounter{N}}
    \item[1.] Проведите настройку экспериментального стенда по техническому описанию (ТО), расположенному на рабочем столе.

    \item[2.] Меняя частоту $f$ генератора, убедитесь по осциллографу и вольтметрам, что у синусоиды $U(t)$ меняется амплитуда и фаза относительно начала координат, тогда как синусоида $\varepsilon(t)$ – синхронизующий сигнал – <<привязана>> к началу отсчёта при начальных условиях: $\varepsilon(0)=0,~~\dot{\varepsilon}(0)=0,$ – а её амплитуда остаётся неизменной с относительной погрешностью  $\le1$~\%. После этого можно приступить к измерениям.

    \item[3.] Для контуров с семью различными ёмкостями $C_n,$ меняя их с помощью переключателя на блоке, измерьте резонансные частоты $f_{0n}$ и напряжения $U(f_{0n}).$ Регистрируйте для контроля также напряжения $\varepsilon$, игнорируя отклонения в пределах относительной погрешности 1\%. Состояние резонанса определяйте по максимуму напряжения $U(f_{0n}),$ измеряемого вольтметром и наблюдаемого на экране осциллографа. Приближение к резонансу удобно наблюдать по фигуре Лиссажу на экране осциллографа в режиме X-Y (см. ТО). При этом фигура Лиссажу представляет собой эллипс, вырождающийся в прямую линию с положительным наклоном \underline{почти} на частоте $f_{0n}.$

    \item[4.*] \emph{\underline{Дополнительное упражнение.}} Проделайте измерения п.3 для напряжения, существенно отличающегося от использованного в п.3, но лежащего в диапазоне $100\div500$~мВ по амплитуде.

    \item[5.] Для контуров с двумя разными ёмкостями (по указанию преподавателя) снимите амплитудно-частотные характеристики $U(f)$ для значений $U(f)\ge0,6U(f_{0n})$ (16-17 точек в сумме по обе стороны от резонанса) при том же напряжении $\varepsilon$, что и в п.3.

    \item[6.] Для тех же двух контуров снимите фазово-частотные характеристики $\psi_{\sc{U}}(f)$ для значений $U(f)\ge0,3U(f_{0n})$ (16-17 точек в сумме по обе стороны от резонанса) при том же напряжении, что и в п.3. Перед выполнением этой части работы измените с помощью ручек горизонтальной развёртки настройки осциллографа таким образом, чтобы синхронизующий сигнал $\varepsilon(t)$ был <<привязан>> к общему началу отсчёта времени и напряжений на экране, лежащему на оси X координатной сетки экрана (см. п.2), и оба сигнала были симметричны относительно этой оси. Если это не так, то следует повторить процедуру центрировки горизонтальных осей каналов по техническому описанию.

    Расстояние $x$ от начала отсчёта до точки первого обращения в нуль напряжения $U(t)$ на участке спада на осциллограмме характеризует разность фаз $\Delta\varphi$ сигналов $U(t)$ и $\varepsilon(t).$ Эта величина, выраженная в радианах, очевидно, даётся формулой $\Delta\varphi=(x/x_{\sc{0}})\pi,$ где $x_{\sc{0}}$ – расстояние от начала отсчёта до точки первого обращения в нуль напряжения $\varepsilon(t)$ на участке подъёма, соответствующее полупериоду этого сигнала.

\begin{center}
    \underline{Обработка и представление результатов}
\end{center}

\emph{Настоятельно рекомендуется для обработки и представления результатов измерений использовать электронные таблицы.}

\item[7.] Результаты измерений п.3 внесите в таблицу~\tabref{2.3.1}.
\begin{center}
    \begin{table}[h!]
        %\caption{\label{tab:1}}
        \caption{}
        \tabmark{2.3.1}
        \begin{center}
            \begin{tabular}{|c|c|c|c|c|c|c|c|c|c|c|}
                \hline
                $C,$~нФ& $f,$~кГц& $U,$~В & $\varepsilon\varepsilon,$~В& $L,$~& $\rho,$~Ом& $|Z_{\text{рез}}|,$& $Q$~& $R_{\sc{\sum}},$& $R_{\sc{S}\text{max}},$& $R_{\sc{L}},$\\

                &  & & & мкГн &  &Ом&  & ~Ом & Ом & Ом \\
                \hline
                $C_1$ & {---} & --- & --- & --- & --- & --- & --- & --- & --- & --- \\
                \hline
                --- & --- & --- & --- & --- & --- & --- & --- & --- & ---& ---\\
                \hline
                $C_7$& --- & --- & --- & --- & --- & --- & ---& --- &--- &---\\
                \hline
                \multicolumn{4}{|c|}{ Среднее значение} & --- & & & & & &--- \\
                \hline
                \multicolumn{4}{|c|}{ Среднеквадратичная погрешность } & ---& & & & & &--- \\\multicolumn{4}{|c|}{  среднего значения} &  & & & & && \\
                \hline
                \multicolumn{4}{|c|}{ Коэффициент Стьюдента $t_{n\alpha}$ } & ---& & & & &---& \\\multicolumn{4}{|c|}{для~$n=7,~\alpha=0,95$} &  & & & & && \\
		                    \hline
 %               \multicolumn{4}{|c|}{ Коэффициент Стьюдента ~для~$n=7,~\alpha=0,95$} & ---& & & & & &---\\
 %               \hline
                \multicolumn{4}{|c|}{ Случайная погрешность } & ---& & & & & &---\\

                \hline
            \end{tabular}
        \end{center}
    \end{table}
\end{center}

В первый столбец этой таблицы запишите значения ёмкостей $C_n,$ приведённые в табличке на корпусе блока <<Резонанс токов>>. Для каждого значения $C_n$ по формулам вводной части и данным эксперимента проведите \underline{последовательно} расчёт $L,$ $\rho,~~|Z_{\text{рез}}|,~~Q,~~R_{\sc{\sum}},~~R_{\sc{S}\text{max}}=10^{-3}\rho,~~R_{\sc{L}}.$

Представьте результат \underline{косвенных измерений при невоспроизводимых условиях,} проделанных в работе, в виде:$\langle L \rangle\pm\Delta L$ и $\langle R_{\sc{L}}\rangle\pm\Delta R_{\sc{L}},$ где угловыми скобками отмечено среднее значение, а символом $"\Delta"$ – случайная погрешность величин $L$ и $R_{\sc{L}}.$

Оцените относительный вклад активных потерь в конденсаторах, представленных в табл.~\tabref{2.3.1}. сопротивлением $R_{\sc{S}\text{max}},$ рассчитанным для максимального значения $\tg\delta=10^{-3},$ в суммарное активное сопротивление контура.

\item[8.*] \emph{Дополнительное упражнение.} Выполните задание п.7 для данных, полученных в п.4*. Сравните с результатами п.7. Объясните причины расхождения результатов, если они обнаружатся.

\item[9.] По данным измерений п.5 постройте на одном графике амплитудно-частотные характеристики в координатах $f,~~U(f)$   для выбранных контуров. Проведите сравнительный анализ характеристик.

\item[10.] По данным измерений п.5 постройте на одном графике амплитудно-частотные характеристики в безразмерных координатах $x\equiv f/f_{0n},~~y\equiv U(x)/U(1).$ По ширине резонансных кривых на уровне 0,707 определите добротности $Q$ соответствующих контуров. Оцените погрешности. Сравните эти величины с данными табл.~\tabref{2.3.1}.

\item[11.] По данным измерений п.6 постройте на одном графике фазово-частотные характеристики $\psi_{\sc{U}}(f)$ в координатах $x\equiv f/f_{0n},~~y\equiv\psi_{\sc{U}}/\pi$ для выбранных контуров. По этим характеристикам определите добротности контуров одним из двух способов: по формуле $Q=(1/2)d\varphi_{\sc{U}}(x)/dx$ при $x=1$ или по расстоянию $1/Q$ между точками оси $x,$ в которых  меняется от $-1/4$ до $1/4$ (см. формулы \eqref{2.3.9б} и сопровождающий их текст). Оцените погрешности. Сравните с результатами табл.~\tabref{2.3.1} и п.10.

\item[12.] По данным табл.~\tabref{2.3.1} постройте зависимость $R_{\sc{L}}(f_{\sc{0}}n)$ в системе координат с началом в точке $(0,6f_{\sc{07}};0)$ нанесите на график прямую $\rangle R_{\sc{L}}\langle .$ Назовите возможные причины изменения $R_{\sc{L}}$ с частотой.

\item[13.] По данным табл.~\tabref{2.3.1} постройте векторную диаграмму токов и напряжений для контура с наименьшей добротностью в резонансном состоянии. Ось ординат направьте по вектору $\vec I.$ Масштаб по этой оси \underline{для тока} сделайте в 3 раза более крупным, чем по оси абсцисс.
	\end{enumerate}
\end{lab:task}

%\begin{center}
%    \textbf{Контрольные вопросы}
%\end{center}
%\normalsize

\begin{lab:questions}

%\newcounter{B}
%\begin{list}{\#\arabic{B}}{\usecounter{B}}
    \item[1.]   Приведите определение добротности колебательного контура в <<энергетических>> терминах.

    \item[2.]   Получите выражение для напряжения на катушке индуктивности  в резонансе.

    \item[3.]Дайте обоснование способам определения добротности по фазово-частотной характеристике.

    \item[4.] По каким причинам потери в контуре зависят от частоты?

    \item[5.*] Зависят ли потери в контуре от амплитуды сигнала и, если зависят, то по каким причинам?

    \item[6.*] Оцените, на какой частоте $\omega_m$ эллипс на экране осциллографа в п.3 вырождается в прямую линию с положительным наклоном.
%\end{list}
\end{lab:questions}


%\begin{center}
%    \large{ДОПОЛНИТЕЛЬНАЯ ЛИТЕРАТУРА}
%\end{center}
%\newcounter{M}
%\begin{list}{\#\arabic{M}}{\usecounter{M}}

\begin{lab:literature}
    \item[1.] \emph{Титце~У., Шенк~К.} Полупроводниковая схемотехника.  – Т.~II. – М.:~ДМК~Пресс, 2007. 12.1.
\end{lab:literature}
%\end{list}
