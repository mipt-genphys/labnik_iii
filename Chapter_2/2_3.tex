\lab{Резонанс токов в параллельном контуре}

\begin{lab:aim}
	исследование резонанса токов в параллельном колебательном контуре с изменяемой ёмкостью, включающее получение амплитудно-частотных и фазово-частотных характеристик, а также определение основных параметров контура.
\end{lab:aim}

\begin{lab:equipment}
	генератор сигналов, источник напряжения, нагрузкой которого является последовательный колебательный контур с переменной ёмкостью, двулучевой осциллограф, цифровые вольтметры.
\end{lab:equipment}

\warning{
	Перед выполнением работы следует изучить основы теории электрических  колебаний по вводной части Раздела II настоящего сборника и/или рекомендованной в нём литературе. Необходимые дополнения будут приведены ниже.
}
\experiment
Блок-схема экспериментального стенда для изучения резонанса токов в параллельном колебательном контуре показана на рис.~\figref{exp schem}. Синусоидальный сигнал от генератора поступает на вход \important{управляемого напряжением источника тока} (см., например, [1]), собранного на операционном усилителе с полевым транзистором, питание которых осуществляется встроенным блоком-выпрямителем от сети~$\sim220$~В (цепи питания на схеме не показаны).  Внутреннее (выходное) сопротивление источника тока, бесконечно большое в идеальном случае, в нашей схеме составляет несколько ГОм. Это обеспечивает постоянство амплитуды тока $I$ на меняющейся, но значительно меньшей по величине, нагрузке – параллельном контуре, изображенном на рис.~\figref{exp schem} в виде эквивалентной схемы.

\todo[author=Tiffani]{Рисунок 16 не соответствует рисунку в ворд-файле, в формате pic рисунка нет}
\begin{figure}[h!]
	\pic{0.9\textwidth}{Chapter_2/2_3_1}
	\caption{Блок-схема экспериментального стенда}
	\figmark{exp schem}
\end{figure}


Источник тока, колебательный контур и блок питания заключены в отдельный корпус с названием <<Резонанс токов>> на верхней крышке, отмеченный на рисунке штриховой линией. На корпусе имеются коаксиальные разъёмы <<Вход>>, <<$\text{U}_1$>> и <<$\text{U}_2$>>, а также переключатель магазина ёмкостей $C_n$ с указателем номера $n=1,~2~\ldots7.$ Величины ёмкостей $C_n$ и сопротивления $R_1$ указаны в табличке на крышке корпуса. Напряжение $\varepsilon=\varepsilon_0\cos(\omega t+\varphi_0)$ от генератора поступает на вход источника тока. Это же напряжение через разъём <<$\text{U}_1$>> подаётся на канал~1~осциллографа и на вход вольтметра~1. Переменное напряжение на сопротивлении $R_1$ в используемой схеме равно напряжению $\varepsilon$ на выходе генератора и совпадает с ним по фазе. Cледовательно, ток $I$ во внешней цепи параллельного контура определяется формулами:

\begin{equation}\eqmark{2.3.1}
	I=\varepsilon/R_1=I_0\cos(\omega t+\varphi_0), \qquad I_0=\varepsilon_0/R_1.
\end{equation}

Напряжение на контуре $U,$ совпадающее с напряжением на конденсаторе $U_C,$ поступает \important{со знаком <<–>>} через разделительный конденсатор и разъём <<$\text{U}_2$>> на канал~2~осциллографа, а также на вход вольтметра~2.%!!!!!!!!!!!!!!!!!!

Колебательный контур нашей установки собран из стандартных элементов, используемых в современных радиоэлектронных цепях. Характеристики этих элементов представлены в описании работы <<Резонанс напряжений в последовательном контуре>>. \important{Соответствующий материал читателю необходимо освоить для понимания дальнейшего изложения и выполнения лабораторной работы.} С учётом приведённых в указанном материале результатов для импедансов ёмкостной $Z_C$ и индуктивной $Z_L$ ветвей параллельного колебательного контура получаем выражения:

\begin{equation}\eqmark{2.3.2}
	Z_C=R_S-\dfrac{i}{\omega C}, \qquad Z_L=R+R_L+i\omega L,
\end{equation}
где $R_S\equiv\dfrac{1}{\omega C}\tg\delta$ и $R_L$~---~активные части импедансов конденсатора и катушки индуктивности соответственно, а $R$  величина постоянного активного сопротивления, добавленного в индуктивную ветвь колебательного контура для снижения его добротности с целью упрощения  процедур получения и обработки резонансных кривых (см. рис.~\figref{exp schem} и табличку на корпусе). Конденсаторы магазина ёмкостей $C_n$ в интересующем нас диапазоне частот имеют относительно малые потери: для них $\tg\delta<10^{-3}.$

Добротность $Q$ контуров в наших установках является достаточно высокой, чтобы можно было пользоваться формулами \chaptereqref{2.34} и \chaptereqref{2.64}, в которых, однако, надо учитывать, что суммарное активное сопротивление контура в этом случае даётся формулой
\begin{equation}\eqmark{2.3.3}
	R_{\scriptscriptstyle \sum}=R+R_L+R_S
\end{equation}
и, следовательно,
\begin{equation}\eqmark{2.3.4}
	Q=\rho/R_{\scriptscriptstyle \sum}=\omega_0 L/R_{\scriptscriptstyle \sum}=1/\omega_0 CR_{\scriptscriptstyle \sum}>>1.
\end{equation}
Сильное неравенство в \eqref{2.3.4} в рабочем диапазоне частот выполняется для всех контуров, используемых в работе.

\important{Комплексные амплитуды} токов в ёмкостной $\vec I_C$ и индуктивной $\vec I_L$ ветвях контура, а также напряжения $\vec U$ на контуре, положив без ограничения общности $\varphi_0=0$ в выражении для внешнего тока $\vec I=I_0e^{i\varphi_0}$ и используя формулы \chaptereqref{2.64} с учётом \eqref{2.3.1}~–-~\eqref{2.3.4}, удобно представить в виде:
\begin{subequations}
	\eqmark{2.3.5}
		\begin{equation}
			\eqmark{2.3.5a}
			\vec I_C=\vec I\dfrac{Z_{LR}}{Z_C+Z_{LR}}=iQI_0\dfrac{\omega/\omega_0-i(R+R_L)/\rho}{1+iQ(\omega/\omega_0-\omega_0/\omega)},
		\end{equation}
		\begin{equation}
			\eqmark{2.3.5b}
			\vec I_L=\vec I\dfrac{Z_C}{Z_C+Z_LR}=
-iQI_0\dfrac{(\omega_0/\omega)(1+\tg\delta)}{1+iQ(\omega/\omega_0-\omega_0/\omega)},
		\end{equation}
		\begin{equation}
			\eqmark{2.3.5c}
			\vec U=\vec I\dfrac{Z_C Z_LR}
{Z_C+Z_LR}=
Q\rho I_0
\dfrac{[1-i\omega_0(R+R_L)/\omega\rho]
(1+i\tg\delta)}
{1+iQ(\omega/\omega_0-\omega_0/\omega)}.
		\end{equation}
\end{subequations}
Из формул \eqref{2.3.5b}, \eqref{2.3.5c} следует, что потерями в конденсаторах, явно представленных величиной $\tg\delta,$ в нашем случае меньшей $10^{-3},$ можно пренебречь. В то же время необходимость учёта вклада этих потерь в суммарное активное сопротивление контура $R_{\scriptscriptstyle \sum}$ вблизи резонанса, примерно равного $\rho\tg\delta,$ можно будет оценить только по результатам эксперимента.

Наибольший практический интерес для контуров с \important{высокой добротностью} представляет случай, когда отклонение $\Delta\omega=\omega-\omega_0$ частоты внешней ЭДС от собственной частоты контура удовлетворяет сильному неравенству
\todo[author=Tiffani]{Красный цвет в ур. \eqref{2.3.6} << --- так задумано или исправлять?}
\begin{equation}\eqmark{2.3.6}
|\Delta\omega|\textcolor{red}{\text{<<}}~\omega_0
\end{equation}
При этом в первом порядке малости по \important{относительной расстройке частоты} $\Delta\omega/\omega_0$ выполняется соотношение
\begin{equation}\eqmark{2.3.7}
\dfrac{\omega_0}{\omega_0}-\dfrac{\omega}{\omega}=\dfrac{2\Delta\omega}{\omega_0},
\end{equation}
которое позволяет упростить выражения \eqref{2.3.5} и представить вещественные части комплексных амплитуд в виде:
\begin{subequations}
	\eqmark{2.3.8}
		\begin{equation}
			\eqmark{2.3.8a}
			I_C(t)=QI_0\dfrac{\omega}{\omega_0}\dfrac{\cos(\omega t-\psi_C)}{R\sqrt{1+(\tau\Delta\omega)^2}}, \quad \psi_C=\arctg(\tau\Delta\omega)-\dfrac{\pi}{2}+\dfrac{R+R_L}{\rho},
		\end{equation}
		\begin{equation}
			\eqmark{2.3.8b}
			I_L(t)=
	QI_0\dfrac{\omega_0}{\omega}
	\dfrac{\cos(\omega t-\psi_L)}{\sqrt{1+(\tau\Delta\omega)^2}}, \quad \psi_L=\arctg(\tau\Delta\omega)+\dfrac{\pi}{2}-\delta,
		\end{equation}
		\begin{equation}
			\eqmark{2.3.8c}
			U(t)=
	Q\rho I_0
	\dfrac{\cos(\omega t-\psi_U)}
	{\sqrt{1+(\tau\Delta\omega)^2}}, \quad \psi_U=\arctg(\tau\Delta\omega)+\dfrac{\omega_0}{\omega}\dfrac{R+R_L}{\rho}-\delta,
		\end{equation}
\end{subequations}
%$$
%	I_C(t)=QI_0\dfrac{\omega}{\omega_0}\dfrac{\cos(\omega t-\psi_C)}{R\sqrt{1+(\tau\Delta\omega)^2}},~~~~
%	\psi_C=\arctg(\tau\Delta\omega)-\dfrac{\pi}{2}+\dfrac{R+R_L}{\rho}, \eqno{(8a)}
%$$
%$$
%	I_L(t)=
%	QI_0\dfrac{\omega_0}{\omega}
%	\dfrac{\cos(\omega t-\psi_L)}{\sqrt{1+(\tau\Delta\omega)^2}},~~~~\psi_L=\arctg(\tau\Delta\omega)+\dfrac{\pi}{2}-\delta, \eqno{(8\text{б})}
%$$
%$$
%	U(t)=
%	Q\rho I_0
%	\dfrac{\cos(\omega t-\psi_{{U}})}
%	{\sqrt{1+(\tau\Delta\omega)^2}},~~~~
%	\psi_{{U}}=\arctg(\tau\Delta\omega)+\dfrac{\omega_0}{\omega}\dfrac{R+R_L}{\rho}-\delta \eqno{(8\text{в})}
%$$
где $\tau=2L/R_{\scriptscriptstyle \sum}=2Q/\omega_0$ – время затухания колебательного контура. В выражениях \eqref{2.3.8} мы сохранили в прежнем виде множители с отношениями частот в амплитудах и учли только линейные по малым величинам $(R~+~R_L)/\rho$ и $\delta$ поправки в фазах, причём величину $\delta$ сохранили исключительно для общности, положив её, однако, константой.

Как видно из выражений \eqref{2.3.8}, вблизи частоты $\omega_0$ зависимости амплитуд токов и напряжения на контуре от частоты $\omega$ несколько различаются, что надо иметь в виду при экспериментальном исследовании резонанса токов \important{по напряжению на контуре} $U.$

Отдельно обратим внимание на тот факт, что зависимость \eqref{2.3.8c} амплитуды напряжения $U$ на параллельном контуре от частоты $\omega$ вблизи резонанса в принятом приближении совпадает с аналогичной зависимостью \chaptereqref{2.60b} амплитуды тока $I_{\omega}$ для последовательного контура в том же приближении.

При резонансе, когда для высокодобротного контура можно положить $\omega=\omega_0$, $\Delta\omega=0,$ амплитуды токов и напряжения \eqref{2.3.8}, их фазы и производная фазы $\psi_U$ по частоте $\omega$ принимают вид:
\begin{subequations}
	\eqmark{2.3.9}
		\begin{equation}
			\eqmark{2.3.9a}
			\begin{gathered}[c]
			 I_C(\omega_0)= QI_0, \qquad \psi_C(\omega_0)= -\dfrac{\pi}{2}+Q^{-1}-\tg\delta, \\
			 I_L(\omega_0)= QI_0, \qquad \psi_L(\omega_0)=\dfrac{\pi}{2}-\delta,
			\end{gathered}
		\end{equation}
		\begin{equation}
			\eqmark{2.3.9b}
			U(\omega_0)=Q\rho I_0, \qquad \psi(\omega_0)=
	Q^{-1}-\tg\delta-\delta, \qquad \psi'_U(\omega_0)=\tau.
		\end{equation}
\end{subequations}
%$$
%	I_C(\omega_0)=
%	QI_0,~~\psi_C(\omega_0)=
%	-\dfrac{\pi}{2}+Q^{-1}-\tg\delta,~~~~~~
%	I_L(\omega_0)=
%	QI_0,~~
%	\psi_L(\omega_0)=
%	\dfrac{\pi}{2}-\delta, \eqno{(9a)}
%$$
%$$
%	U(\omega_0)=
%	Q\rho I_0,~~~~
%	\psi(\omega_0)=
%	Q^{-1}-\tg\delta-\delta,~~~~
%	\psi'_{{U}}(\omega_0)=\tau. \eqno{(9\text{б})}
%$$
%\setcounter{equation}{9}
В последнем равенстве мы пренебрегли относительными поправками порядка $Q^{-2}$ и $Q^{-1}\tg\delta.$ Из формул \eqref{2.3.9a} следует, что на частоте $\omega_0$ токи $\vec I_C$ и $\vec I_L$ в ёмкостной и индуктивной ветвях контура в $Q$ раз превышают по амплитуде ток $\vec I$ во внешней цепи. При этом ток $\vec I_C$ опережает внешний ток $\vec I$ по фазе почти на $\pi/2,$ а ток $\vec I_L$ – отстаёт от тока $\vec I$ почти на $\pi/2.$ Между собой токи $\vec I_C$ и $\vec I_L$ сдвинуты по фазе на угол, близкий к  $\pi.$ Можно сказать, что токи $\vec I_C$ и $\vec I_L$ образуют контурный ток, последовательно обтекающий элементы контура и в $Q$ раз превышающий внешний ток $\vec I.$ Как уже отмечалось в п.~3.2, последнее обстоятельство послужило поводом назвать резонанс в параллельном контуре <<резонансом токов>>.
\todo[author=Tiffani]{Уточнить номер пункта (п. 3.2)}

Отметим также, что максимальные (резонансные) значения токов в контуре, как и в п.~3.2, не строго равны $QI_0$ и достигаются не строго на частоте $\omega_0.$ Соответствующие относительные поправки составляют доли малой величины $Q^{-2}$ и связаны с входящим в выражения \eqref{2.3.8a}, \eqref{2.3.8b} для вещественных амплитуд токов $\vec I_C$, $\vec I_L$ отношением $\omega/\omega_0.$

Из формул \eqref{2.3.9b} вытекает, что на частоте $\omega_0$ импеданс контура $Z(\omega_0)=\vec U(\omega_0)/I_0$ является почти чисто активным. В пренебрежении относительными поправками порядка $Q^{-2}$ его модуль и фаза относительно внешнего тока определяются формулами:
\begin{equation}\eqmark{2.3.10}
|Z(\omega_0)|=Q\rho=Q^2R_{\scriptscriptstyle \sum}, \qquad \psi_Z(\omega_0)=\dfrac{R+R_L}{\rho}-\delta,
\end{equation}
которые дополняют формулы \chaptereqref{2.67} учётом активных потерь в катушке индуктивности и в конденсаторе.

При отклонении $\Delta\omega$ частоты внешней ЭДС от частоты $\omega_0$ таком, что выполняется условие
\begin{equation}\eqmark{2.3.11}
\tau\Delta\omega=\pm1,
\end{equation}
амплитуда напряжения $U,$ как видно из формул \eqref{2.3.8c}, уменьшается в $\sqrt{2}$ раз относительно своей резонансной величины, а фаза $\psi_U$ изменяется примерно на угол $\pm\pi/4.$

Величина $\delta\omega\equiv2|\Delta\omega_{\gamma}|=2\gamma=2/\tau$ представляет собой важную характеристику колебательного контура~---~\important{ширину резонансной кривой} $U(\omega),$ по которой с учётом соотношений $Q=\omega_0/2\gamma=\tau\omega_0/2,$ зная частоту $\omega_0,$ можно найти добротность контура
\begin{equation}\eqmark{2.3.12}
Q=\dfrac{\omega_0}{\delta\omega}
\end{equation}

Эти же параметры можно определить по фазово-частотной характеристике: тангенс угла наклона $\psi_U$ в точке $\omega=\omega_0$ согласно \eqref{2.3.9b} определяет время затухания $\tau,$ а расстояние по оси $\omega$ между точками, в которых фаза $\psi_U(\omega)$ меняется от $-\pi/4$ до $\pi/4,$ равно $2/\tau$ с относительной погрешностью порядка $Q^{-2}.$

\begin{lab:task}
	\begin{enumerate}

    \item Проведите настройку экспериментального стенда по техническому описанию (ТО), расположенному на рабочем столе.

    \item Меняя частоту $f$ генератора, убедитесь по осциллографу и вольтметрам, что у синусоиды $U(t)$ меняется амплитуда и фаза относительно начала координат, тогда как синусоида $\varepsilon(t)$~---~синхронизующий сигнал~---~<<привязана>> к началу отсчёта при начальных условиях: $\varepsilon(0)=0$, $\dot{\varepsilon}(0)=0,$~---~а её амплитуда остаётся неизменной с относительной погрешностью  $\le1~\%$. После этого можно приступить к измерениям.

    \item Для контуров с семью различными ёмкостями $C_n,$ меняя их с помощью переключателя на блоке, измерьте резонансные частоты $f_{0n}$ и напряжения $U(f_{0n}).$ Регистрируйте для контроля также напряжения $\varepsilon$, игнорируя отклонения в пределах относительной погрешности $1\%$. Состояние резонанса определяйте по максимуму напряжения $U(f_{0n}),$ измеряемого вольтметром и наблюдаемого на экране осциллографа. Приближение к резонансу удобно наблюдать по фигуре Лиссажу на экране осциллографа в режиме X-Y (см. ТО). При этом фигура Лиссажу представляет собой эллипс, вырождающийся в прямую линию с положительным наклоном \important{почти} на частоте $f_{0n}.$

    \item \important{Дополнительное упражнение.} Проделайте измерения п.~3 для напряжения, существенно отличающегося от использованного в п.~3, но лежащего в диапазоне $100\div500$~мВ по амплитуде.

    \item Для контуров с двумя разными ёмкостями (по указанию преподавателя) снимите амплитудно-частотные характеристики $U(f)$ для значений $U(f)\ge0,6U(f_{0n})$ (16~--~17 точек в сумме по обе стороны от резонанса) при том же напряжении $\varepsilon$, что и в п.~3.

    \item Для тех же двух контуров снимите фазово-частотные характеристики $\psi_U(f)$ для значений $U(f)\ge0,3U(f_{0n})$ (16~--~17 точек в сумме по обе стороны от резонанса) при том же напряжении, что и в п.~3. Перед выполнением этой части работы измените с помощью ручек горизонтальной развёртки настройки осциллографа таким образом, чтобы синхронизующий сигнал $\varepsilon(t)$ был <<привязан>> к общему началу отсчёта времени и напряжений на экране, лежащему на оси X координатной сетки экрана (см. п.~2), и оба сигнала были симметричны относительно этой оси. Если это не так, то следует повторить процедуру центрировки горизонтальных осей каналов по техническому описанию.

    Расстояние $x$ от начала отсчёта до точки первого обращения в нуль напряжения $U(t)$ на участке спада на осциллограмме характеризует разность фаз $\Delta\varphi$ сигналов $U(t)$ и $\varepsilon(t).$ Эта величина, выраженная в радианах, очевидно, даётся формулой $\Delta\varphi=(x/x_0)\pi,$ где $x_0$~---~расстояние от начала отсчёта до точки первого обращения в нуль напряжения $\varepsilon(t)$ на участке подъёма, соответствующее полупериоду этого сигнала.

	\tasksection{Обработка и представление результатов}
	\important{Настоятельно рекомендуется для обработки и представления результатов измерений использовать электронные таблицы.}

	\item Результаты измерений п.~3 внесите в таблицу~\tabref{2.3.1}.
	\begin{center}
	    \begin{table}[h!]
	        %\caption{\label{tab:1}}
	        \caption{}
	        \tabmark{2.3.1}
	        \begin{center}
	            \begin{tabular}{|c|c|c|c|c|c|c|c|c|c|c|}
	                \hline
	                $C,$ & $f,$ & $U,$ & $\varepsilon\varepsilon,$ & $L,$ & $\rho,$ & $|Z_{\text{рез}}|,$& $Q$ & $R_{\scriptscriptstyle \sum},$ & $R_{S \text{max}},$& $R_L,$\\

	                нФ & кГц & В & В & мкГн & Ом & Ом &  & Ом & Ом & Ом \\
	                \hline
	                $C_1$ & {---} & --- & --- & --- & --- & --- & --- & --- & --- & --- \\
	                \hline
	                --- & --- & --- & --- & --- & --- & --- & --- & --- & ---& ---\\
	                \hline
	                $C_7$& --- & --- & --- & --- & --- & --- & ---& --- &--- &---\\
	                \hline
	                \multicolumn{4}{|c|}{ Среднее значение} & --- & & & & & &--- \\
	                \hline
	                \multicolumn{4}{|c|}{ Среднеквадратичная } & & & & & & & \\
	                \multicolumn{4}{|c|}{ погрешность } & ---& & & & & &--- \\
	                \multicolumn{4}{|c|}{ среднего значения } &  & & & & && \\
	                \hline
	                \multicolumn{4}{|c|}{ Коэффициент } & & & & & & & \\
	                \multicolumn{4}{|c|}{ Стьюдента $t_{n\alpha}$ } & ---& & & & &---& \\
	                \multicolumn{4}{|c|}{для~$n=7,~\alpha=0,95$} &  & & & & && \\
			                    \hline
	 %               \multicolumn{4}{|c|}{ Коэффициент Стьюдента ~для~$n=7,~\alpha=0,95$} & ---& & & & & &---\\
	 %               \hline
	                \multicolumn{4}{|c|}{ Случайная погрешность } & ---& & & & & &---\\

	                \hline
	            \end{tabular}
	        \end{center}
	    \end{table}
	\end{center}
\todo[author=Tiffani]{Проверить таблицу, есть сомнения в правильности столбцов}
В первый столбец этой таблицы запишите значения ёмкостей $C_n,$ приведённые в табличке на корпусе блока <<Резонанс токов>>. Для каждого значения $C_n$ по формулам вводной части и данным эксперимента проведите \important{последовательно} расчёт $L,$ $\rho$, $|Z_{\text{рез}}|$, $Q$, $R_{\scriptstyle \sum}$, $R_{S \text{max}}=10^{-3}\rho$, $R_L$.

Представьте результат \important{косвенных измерений при невоспроизводимых условиях,} проделанных в работе, в виде: $\langle L \rangle\pm\Delta L$ и $\langle R_L\rangle\pm\Delta R_L,$ где угловыми скобками отмечено среднее значение, а символом $"\Delta"$ – случайная погрешность величин $L$ и $R_L.$

Оцените относительный вклад активных потерь в конденсаторах, представленных в табл.~\tabref{2.3.1}. сопротивлением $R_{S\text{max}},$ рассчитанным для максимального значения $\tg\delta=10^{-3},$ в суммарное активное сопротивление контура.

\item \important{Дополнительное упражнение.} Выполните задание п.~7 для данных, полученных в п.~4*. Сравните с результатами п.~7. Объясните причины расхождения результатов, если они обнаружатся.

\item По данным измерений п.~5 постройте на одном графике амплитудно-частотные характеристики в координатах $f,~U(f)$   для выбранных контуров. Проведите сравнительный анализ характеристик.

\item По данным измерений п.~5 постройте на одном графике амплитудно-частотные характеристики в безразмерных координатах $x\equiv f/f_{0n}$, $y\equiv U(x)/U(1).$ По ширине резонансных кривых на уровне 0,707 определите добротности $Q$ соответствующих контуров. Оцените погрешности. Сравните эти величины с данными табл.~\tabref{2.3.1}.

\item По данным измерений п.~6 постройте на одном графике фазово-частот-\\*ные характеристики $\psi_U(f)$ в координатах $x\equiv f/f_{0n}$, $y\equiv\psi_U/\pi$ для выбранных контуров. По этим характеристикам определите добротности контуров одним из двух способов: по формуле $Q=(1/2)d\varphi_U(x)/dx$ при $x=1$ или по расстоянию $1/Q$ между точками оси $x,$ в которых  меняется от $-1/4$ до $1/4$ (см. формулы \eqref{2.3.9b} и сопровождающий их текст). Оцените погрешности. Сравните с результатами табл.~\tabref{2.3.1} и п.~10.

\item По данным табл.~\tabref{2.3.1} постройте зависимость $R_L(f_{0n})$ в системе координат с началом в точке $(0,6f_{07};0)$ нанесите на график прямую $\langle R_L \rangle.$ Назовите возможные причины изменения $R_L$ с частотой.

\item По данным табл.~\tabref{2.3.1} постройте векторную диаграмму токов и напряжений для контура с наименьшей добротностью в резонансном состоянии. Ось ординат направьте по вектору $\vec I.$ Масштаб по этой оси \important{для тока} сделайте в 3 раза более крупным, чем по оси абсцисс.
	\end{enumerate}
\end{lab:task}

\begin{lab:questions}

    \item   Приведите определение добротности колебательного контура в <<энергетических>> терминах.

    \item  Получите выражение для напряжения на катушке индуктивности  в резонансе.

    \item Дайте обоснование способам определения добротности по фазово-час-\\*тотной характеристике.

    \item По каким причинам потери в контуре зависят от частоты?

    \item * Зависят ли потери в контуре от амплитуды сигнала и, если зависят, то по каким причинам?

    \item * Оцените, на какой частоте $\omega_m$ эллипс на экране осциллографа в п.~3 вырождается в прямую линию с положительным наклоном.
\end{lab:questions}

\begin{lab:literature}
    \item \emph{Титце~У., Шенк~К.} Полупроводниковая схемотехника.  – Т.~II. – М.: ДМК~Пресс, 2007. 12.1.
\end{lab:literature}
