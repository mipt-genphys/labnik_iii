\lab{Сдвиг фаз в цепи переменного тока}

\aim{изучить влияние активного сопротивления, индуктивности и ёмкости на сдвиг
фаз между током и напряжением в цепи
переменного тока.}

\equip{генератор звуковой частоты (ЗГ), двухканальный осциллограф (ЭО), магазин
ёмкостей, магазин сопротивлений, катушка
индуктивности, резисторы, мост переменного тока.}

Удобным, хотя и не очень точным прибором для измерения фазовых соотношений
служит электронный осциллограф. Пусть нужно
измерить сдвиг фаз между двумя напряжениями~$U_1$ и~$U_2$. Подадим эти
напряжения на горизонтальную и вертикальную
развёртки осциллографа. Смещение луча по горизонтали и вертикали определяется
выражениями
\begin{equation*}
x=x_0\cos\Omega t,\qquad y=y_0\cos(\Omega t+\alpha),
\end{equation*}
где $\alpha$~--- сдвиг фаз между напряжениями $U_1$ и $U_2$, а $x_0$ и~$y_0$~---
амплитуды напряжений, умноженные на
коэффициенты усиления соответствующих каналов осциллографа. Исключив время,
после несложных преобразований найдём:
\begin{equation*}
\left(\frac{x}{x_0}\right)^2+ \left(\frac{y}{y_0}\right)^2+ \frac{2xy}{x_0 y_0}
\cos\alpha=\sin^2 \alpha.
\end{equation*}

\begin{wrapfigure}[13]{r}{0.45\linewidth}
	\pic{0.4\textwidth}{Chapter_2/2_1_1}
	\caption{ Эллипс на экране осциллографа}
	\figmark{elips}
\end{wrapfigure}

Полученное выражение определяет эллипс, описываемый электронным лучом на экране
осциллографа (рис.~\figref{r1}). Ориентация
эллипса зависит как от искомого угла $\alpha$, так и от усиления каналов
осциллографа. Для расчёта сдвига фаз можно
измерить отрезки $2y_{x=0}$ и $2y_0$ (или $2x_{y=0}$ и $2x_0$, на рисунке не
указанные) и, подставляя эти значения в
уравнение эллипса, найти
\begin{equation*}
\alpha=\pm\arcsin\left(\frac{y_{x=0}}{y_0}\right).
\end{equation*}

\begin{wrapfigure}[13]{r}{0.45\linewidth}
	\pic{0.4\textwidth}{Chapter_2/2_1_2}
	\caption{ Принципиальная схема фазовращателя}
	\figmark{scheme}
\end{wrapfigure}

\important{Для правильного измерения отрезка $2y_{x=0}$ важно, чтобы центр
эллипса лежал на оси~$y$.}

На практике часто используются устройства, позволяющие в широких пределах
изменять фазу напряжения ($0<\psi<\pi$). Такие
устройства называются \textit{фазовращателями}. Схема простого фазовращателя
приведена на \figref{scheme}. Она включает в себя два
одинаковых резистора~$R_1$, ёмкость~$C$ и переменное сопротивление~$R$.

Используя метод комплексных амплитуд, найдём зависимость сдвига фаз между
входным напряжением $U_{вх}=U_0\cos\Omega t$ и
выходным~$U_{вых}$ от соотношения между импедансами сопротивления~$R$ и
ёмкости~$C$. Для этого выразим выходное
напряжение~$U_{вых}$ через~$U_{вх}$, параметры контура и частоту внешнего
источника~$\Omega$:
$U_{34}=f(U_{12},\,R,\,C,\,\Omega)$.

\begin{figure}[h!]
	\pic{0.9\textwidth}{2_1_2}
	\caption{Схема установки для исследования сдвига фаз между током и
напряжением}
	\figmark{shift}
\end{figure}

Обозначим комплексную амплитуду входного напряжения через~$\widehat{U}_0$. Тогда
напряжение между точками 1 и 3 в силу
равенства сопротивлений~$R_1$
\begin{equation*}
\widehat{U}_{13}=\frac{\widehat{U}_0}{2}.
\end{equation*}
Если фазу напряжения $\widehat{U}_{вх}$ положить равной нулю, то $\widehat{U}_0$
будет действительной величиной:
$\widehat{U}_0=U_0$. Приняв напряжение в точке 1 равным нулю, получим амплитуду
напряжения в точке~3:
\begin{equation*}
\widehat{U}_{03}=\frac{U_0}{2}.
\end{equation*}
Рассчитаем $\widehat{U}_{04}$~--- амплитуду напряжения в точке~4. Импеданс~$Z$
последовательно соединённых сопротивления
$R$ и ёмкости $C$ равен
\begin{equation*}
Z=R-\frac{i}{\Omega C}.
\end{equation*}
Для комплексной амплитуды тока $\widehat{I}_0$, проходящего через~$R$ и~$C$,
имеем
\begin{equation*}
\widehat{I}_0=\frac{U_0}{Z}=\frac{U_0}{R-i/(\Omega C)},
\end{equation*}
а для комплексной амплитуды напряжения в точке~4~---
\begin{equation*}
\widehat U_{04}=\widehat I_0 R=U_0\frac{R}{R-i/(\Omega C)}.
\end{equation*}
Выходное напряжение $\widehat{U}_{вых}$ равно разности напряжений в точках 3
и~4:
\begin{equation*}
\widehat{U}_{вых}=\widehat{U}_{04}-\widehat{U}_{03}=\widehat{U}_{04}-U_0/2=
\frac{U_0}{2}\;\frac{R+i/(\Omega
C)}{R-i/(\Omega C)}.
\end{equation*}
В числитель и знаменатель последнего выражения входят комплексно-сопряжённые
величины, модули которых одинаковы, поэтому
величина выходного напряжения не меняется при изменении~$R$. Модуль~$U_{вых}$
всегда равен $U_0/2$~---
половине~$U_{вх}$. Сдвиг фаз между входным и выходным напряжениями равен
$2\arctan[1/(\Omega RC)]$ и меняется от~$\pi$
(при $R\to 0$) до~$0$ (при $R\to\infty$).

\experiment  Схема для исследования сдвига фаз между током и напряжением в цепи
переменного тока представлена на \figref{shift}. Эталонная
катушка~$L$, магазин ёмкостей~$C$ и магазин сопротивлений $R$ соединены
последовательно и через дополнительное
сопротивление~$r$ подключены к источнику синусоидального напряжения~---
звуковому генератору.

Сигнал, пропорциональный току, снимается c сопротивления~$r$, пропорциональный
напряжению~--- с генератора. Оба сигнала
подаются на универсальный осциллограф. Этот осциллограф имеет два канала
вертикального отклонения, что позволяет
одновременно наблюдать на экране два сигнала. В нашей работе это две синусоиды,
смещённые друг относительно друга в
зависимости от сдвига фаз между током и напряжением в цепи. На
рис.~\figref{shift} синусоиды на экране ЭО сдвинуты по фазе
на~$\pi/2$.


\begin{figure}[h!]
	\pic{0.9\textwidth}{2_1_4}
	\caption{Схема установки для исследования фазовращателя}
	\figmark{rotator}
\end{figure}

Схема фазовращателя, изображённая на \figref{rotator}, содержит два одинаковых
резистора~$R_1$, смонтированных на отдельной плате,
магазин сопротивлений~$R$ и магазин ёмкостей~$C$.

\begin{lab:task}

	В работе предлагается исследовать зависимости сдвига фаз между током и
напряжением от сопротивления в~$RC$- и
	в~$RL$-цепи; определить добротность колебательного контура, сняв зависимость
сдвига фаз от частоты вблизи резонанса;
	оценить диапазон работы фазовращателя.

	\begin{enumerate}

	\tasksection{Подготовка приборов к работе}

	\item Соберите схему, изображённую на рис.~\figref{shift}. Установите на
катушке индуктивности максимальное значение, указанное на установке (например,
~$L=50$~мГн).

	Подключите магазин ёмкостей так, чтобы ёмкость можно было менять в интервале
$0\div1$~мкФ. Установите значение $C=0,5$~мкФ.

	Подключите магазин сопротивлений таким образом, чтобы работали все декады.
Установите $R=0$.

	Установите рабочую частоту звукового генератора $\nu=1$~кГц;
переключатель--- нагрузка
	генератора~--- поставьте в положение, указанное на установке (например,
5~Ом). Включите генератор. Величину выходного напряжения можно менять с помощью
потенциометра.
	Настройте осциллограф согласно техническому описанию.
	Величину сигнала на экране можно регулировать как регулятором выходной
мощности генератора, так и регуляторами
	коэффициента усиления каналов. Внешний регулятор осуществляет дискретное
переключение, внутренний~--- плавное.
	Установите амплитуды обоих сигналов примерно одинаковыми.

	\item \label{i2} Измерение сдвига фаз удобно проводить следующим образом:

	1)~подобрать частоту развёртки, при которой на экране осциллографа
укладывается чуть больше половины периода синусоиды;

	2)~отцентрировать горизонтальную ось (см. п.~\ref{i2});

	3) измерить расстояние $x_0$ (рис.~\figref{shift}) между нулевыми значениями
одного из сигналов, что соответствует смещению по
	фазе на $\pi$;

	4) измерить расстояние $x$ между нулевыми значениями двух синусоид и
пересчитать в сдвиг по фазе: $\psi=\pi\cdot x/x_0$.

	\tasksection{Исследование зависимости сдвига фаз между током и напряжением
от $R$ в $RC$-цепи}

	\item В схеме, собранной согласно \figref{shift}, закоротите катушку,
подключив оба провода, идущих к катушке, на одну клемму.
	Установите $C=0,5$~мкФ, $\nu=1$~кГц и рассчитайте реактивное сопротивление
цепи $X_1=1/(\Omega C)$. Циклическая частота
	$\Omega=2\pi\nu$.

	\item \label{i5} Увеличивая сопротивление $R$ от нуля до $10\cdot Z_1$,
проведите измерения сдвига фаз $\psi$ (см. п.~\r{p3}).
	Предварительно подберите шаги $\Delta R$, для которых приращения $x$ будут
примерно одинаковы. Периодически проверяйте
	положение нулевой линии синусоиды.

	\tasksection{Исследование зависимости сдвига фаз от $R$ в $RL$-цепи}

	\item В схеме, собранной согласно \figref{shift}, закоротите магазин
ёмкостей. Установите  $\nu=1$~кГц и значение индуктивности согласно
рекомендациям на установке. Рассчитайте реактивное сопротивление цепи
$X_2=\Omega L$.

	\item Меняя сопротивление от 0 до $10\cdot Z_2$, проведите измерения сдвига
фаз $\psi$ для $6\div8$ значений $R$ (см.
	п.~\ref{i5}).

	\tasksection{Исследование зависимости сдвига фаз от частоты в $RCL$-цепи}

	\item В цепи, собранной согласно рис.~\figref{shift}, установите значения
емкости и индуктивности, указанные на установке, сопротивление магазина
выставите на ноль
	$R=0$. Рассчитайте резонансную частоту $\nu_0=1/(2\pi\sqrt{LC})$.

	\item Подбирая частоту ЗГ, добейтесь резонанса. При резонансе сдвиг фаз
$\psi=0$ и нулевые значения двух синусоид должны
	совместиться, а при равенстве амплитуд синусоиды полностью совпадают.

	\item \label{i10} Меняя частоту \important{в обе стороны} от резонансного
значения, снимите зависимость сдвига фаз от частоты. С
	изменением частоты меняется расстояние $x_0$, которое занимает половина
периода синусоиды, поэтому разумно каждый раз
	фиксировать отношение $x/x_0$. Вблизи резонанса ($|\psi|<\pi/3$) точки
должны лежать чаще.

	\item Повторите измерения п.~\ref{i10} для сопротивления $R=100$~Ом.

	\item Запишите значения $r$ и $R_L$~--- активное сопротивление катушки,
указанное на её крышке. Проверьте значения $r$, $L$ и $R_L$ с
	помощью моста переменного тока.

	\tasksection{Исследование работы фазовращателя}

	\item Соберите схему по \figref{rotator}. Убедитесь, что выход ЗГ не
заземлён. Установите $C=0,5$~мкФ, $\nu=1$~кГц. Оцените визуально
	диапазон изменения сдвига фаз при изменении $R$ от~$0$ до $10$~кОм.
Подберите сопротивление, при котором сдвиг фаз равен $\pi/2$.

	\tasksection{Обработка результатов}

	\item По результатам измерений сдвига фаз в $RC$-цепи постройте график:
$\tan\psi=f[1/(\Omega C R_{\Sigma})]$. Здесь
	$R_{\Sigma}$~--- суммарное активное сопротивление цепи: $R_{\Sigma}=R+r$;
$r=10$~Ом~--- сопротивление резистора.
	Постройте теоретический график на этом же листе.

	\item Постройте график $\tan\psi=f(\Omega L/R_{\Sigma})$ для $RL$-цепи.
Здесь $R_{\Sigma}=R+r+R_L$. Сравните график с
	теоретическим.

	\item Постройте на одном листе графики $|\psi|=f(\nu/\nu_0)$ для $R=0$
и~$100$~Ом (величину $\psi$ удобно откладывать в
	долях $\pi$).

	Определите по графикам добротность контура: $Q=\nu_0/(2\Delta\nu)$, где
$2\Delta\nu/\nu_0$~--- ширина графика при сдвиге
	фаз $\psi=\pi/4$.

	\item Сравните добротность, определённую графически, с расчётом через
параметры~$L$, $C$ и~$R$.

	\item Постройте векторную диаграмму для фазовращателя. С её помощью найдите
значение сопротивления, которое соответствует сдвигу фаз между входным и
выходным напряжением $\pi/2$. Сравните результат с экспериментальным.

	\item Оцените погрешности и сравните результаты.
	\end{enumerate}


\end{lab:task}


\begin{lab:questions}
	\item Что называется импедансом электрической цепи?
	\item Как складываются импедансы при последовательном и параллельном
соединении элементов электрической цепи?
	%\item Можно ли описывать процессы затухания и установления колебаний,
%     пользуясь понятием импеданса?
\end{lab:questions}


\begin{lab:literature}
	\item \emph {Сивухин Д.В.} Общий курс физики. --- T.~III. Электричество.~---
М.:~Наука, 1983. \S\S~129,~130.
	\item \emph {Калашников С.Г.} Электричество.~---~М.:~Наука, 1977. \S\S~220,
227, 228.
\end{lab:literature}
