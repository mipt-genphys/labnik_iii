\lab{Сдвиг фаз в цепи переменного тока}
\label{lab:phase}

\aim{изучить влияние активного сопротивления, индуктивности и ёмкости на сдвиг
фаз между током и напряжением в цепи
переменного тока.}

\equip{генератор звуковой частоты (ЗГ), двухканальный осциллограф (ЭО), магазин
ёмкостей, магазин сопротивлений, катушка
индуктивности, резисторы, универсальный измеритель импеданса ($LCR$-метр).}

Перед выполнением данной работы необходимо ознакомиться с теоретическим
Введением к разделу (пп. \ref{sec:forced}, \ref{sec:ures}).

Удобным, хотя и не очень точным, прибором для измерения фазовых соотношений 
служит электронный осциллограф. Можно предложить два способа измерения разности фаз.

В первом способе два сигнала~$U_1$ и~$U_2$ подаются 
на горизонтальную (канал~$X$) и вертикальную (канал~$Y$) развёртки осциллографа. 
Смещение луча по горизонтали и вертикали определяется выражениями
\begin{equation*}
x=x_0\cos\omega t,\qquad y=y_0\cos(\omega t+\psi),
\end{equation*}
где $\psi$~--- сдвиг фаз между напряжениями $U_1$ и $U_2$, а $x_0$ и~$y_0$~---
амплитуды напряжений, умноженные на
коэффициенты усиления соответствующих каналов осциллографа. Исключив время,
после несложных преобразований найдём:
\begin{equation*}
\left(\frac{x}{x_0}\right)^2+ \left(\frac{y}{y_0}\right)^2+ \frac{2xy}{x_0 y_0}
\cos\psi=\sin^2 \psi.
\end{equation*}

\begin{wrapfigure}[13]{r}{0.45\linewidth}
	\pic{0.4\textwidth}{Chapter_2/2_1_1}
	\caption{Эллипс на экране осциллографа}
	\figmark{elips}
\end{wrapfigure}

Полученное выражение определяет эллипс, описываемый электронным лучом на 
экране осциллографа (рис.~\figref{elips}). Ориентация эллипса зависит как от искомого 
угла~$\psi$, так и от усиления каналов осциллографа. Для расчёта сдвига фаз 
можно измерить отрезки $2y_{x=0}$ и $2y_0$ (или $2x_{y=0}$ и $2x_0$, 
на рисунке не указанные) и, подставляя эти значения в уравнение эллипса, найти
\begin{equation}
\eqmark{phasexy}
\psi=\pm\arcsin\left(\frac{y_{x=0}}{y_0}\right).
\end{equation}
Для правильного измерения отрезка $2y_{x=0}$ важно, чтобы 
\emph{центр     эллипса лежал на оси~$y$}.

Второй способ заключается в непосредственном измерении сдвига 
фаз между сигналами на экране двухканального осциллографа.
Напряжения $U_1$ и $U_2$ одновременно подаются на входные каналы ЭО
при включенной внутренней горизонтальной развертке. При этом
сигналы одновременно отображаются на экране.
Измерение разности фаз в таком случае удобно проводить следующим образом:
\begin{enumerate}[label=\arabic*),itemsep=0pt]
    \item подобрать частоту горизонтальной развёртки, при которой на экране 
    укладывается чуть больше половины периода синусоиды;
    \item отцентрировать горизонтальную ось;
    \item измерить расстояние $x_0$ (см. рис.~\figref{shift}) между нулевыми значениями 
    \emph{одного} из сигналов, что соответствует разности фаз $\pi$;
    \item измерить расстояние $x$ между нулевыми значениями двух синусоид 
    и пересчитать в сдвиг по фазе: $\psi=\pi x/x_0$. На рис.~\figref{shift} 
    синусоиды на экране ЭО сдвинуты по фазе на~$\pi/2$.
\end{enumerate}

\experiment 
Схема установки для исследования сдвига фаз между током и напряжением 
в цепи переменного тока представлена на рис.~\figref{shift}. Эталонная катушка~$L$, 
магазин ёмкостей~$C$ и магазин сопротивлений~$R$ соединены последовательно 
и через дополнительное сопротивление~$r$ подключены к источнику 
синусоидального напряжения~--- звуковому генератору (ЗГ).

\begin{figure}[hb]
    \centering\small
    \pic{0.8\textwidth}{Chapter_2/2_1_3}
    \caption{Схема установки для исследования сдвига фаз между током и
        напряжением}
    \figmark{shift}
\end{figure}

Сигнал, пропорциональный току, снимается c сопротивления~$r$, 
пропорциональный напряжению~--- с генератора. Оба сигнала подаются 
на осциллограф (ЭО), имеющий два канала вертикального 
отклонения. Измерение разности фаз можно проводить одним из двух
описанных выше способов.

%, что позволяет одновременно наблюдать на экране два сигнала. 
%В нашей работе это две синусоиды, смещённые друг относительно друга 
%на некоторое расстояние, зависящее от сдвига фаз между 
%током и напряжением в цепи.


%\begin{wrapfigure}[13]{r}{0.45\linewidth}
%    \pic{0.4\textwidth}{Chapter_2/2_1_2}
%    \caption{Принципиальная схема фазовращателя}
%    \figmark{scheme}
%\end{wrapfigure}

На практике часто используются устройства, 
называемые \term{фазовращателями}, 
которые позволяют изменять фазу напряжения в широких пределах ($0<\psi<\pi$). 
Схема фазовращателя, применяемого в данной работе, 
изображена на рис.~\figref{rotator}. Она содержит два одинаковых 
резистора~$R_1$, смонтированных на отдельной плате,
магазин сопротивлений~$R$ и магазин ёмкостей~$C$. 

Найдём, как зависит сдвиг фаз между входным 
напряжением $U_{вх}=U_0\cos\omega t$ (точки~1 и~2 на рис.~\figref{rotator}) 
и выходным напряжением $U_{вых}$ (точки~3 и~4)
от соотношения между импедансами сопротивления~$R$ и ёмкости~$C$.
Для соответствующих комплексных амплитуд имеет место соотношение
(получите самостоятельно):
\begin{equation}
\eqmark{cplx}
\vec{U}_{вых}= \frac{\vec{U}_{вх}}{2}\dfrac{R+\frac{i}{\omega C}}{R-\frac{i}{\omega C}}.
\end{equation}
Числитель и знаменатель \eqref{cplx} --- комплексно-сопряжённые 
величины, модули которых одинаковы. Поэтому амплитуда выходного напряжения 
не зависит от~$R$, и всегда равна $U_{0}/2$.
Сдвиг фаз между выходным и входным напряжениями равен 
\begin{equation}
\eqmark{psi}
\psi = \arg \frac{\vec{U}_{вых}}{\vec{U}_{вх}} = 2\arctg \frac{1}{\omega RC}.
\end{equation}
Он может меняться от~$\psi=\pi$ при $R\to 0$ до~$\psi=0$ при $R\to\infty$.

\begin{figure}[h!]
    \centering
    \pic{0.8\textwidth}{Chapter_2/2_1_4}
    \caption{Схема установки для исследования фазовращателя}
    \figmark{rotator}
\end{figure}


\begin{lab:task}
\taskpreamble{В работе предлагается исследовать зависимости сдвига фаз между 
    током и напряжением от сопротивления в $RC$- и в $RL$-цепи; определить 
    добротность колебательного контура, сняв зависимость сдвига фаз от частоты 
    вблизи резонанса, определить диапазон работы фазовращателя.}

\tasksection{Исследование сдвига фаз в $RC$-цепи}

\item Ознакомьтесь с устройством используемых в работе приборов по техническому
описанию. Соберите схему согласно рис.~\figref{shift}.

\item В схеме, собранной по рис.~\figref{shift}, закоротите катушку, подключив оба провода, 
идущих к катушке, на одну клемму. Установите $C=0,5$~мкФ, $\nu=1$~кГц 
(см. рекомендации на установке) и 
рассчитайте реактивное сопротивление цепи $X_1=1/(\omega C)$
(циклическая частота $\omega=2\pi\nu$).

\item Увеличивая сопротивление $R$ от нуля до $\sim 10 X_1$, проведите 
измерения сдвига фаз $\psi$ для 6--8 значений $R$.
Предварительно подберите шаги $\Delta R$, для которых приращения 
сдвиги синусоид на экране осциллографа $x$ (см. рис.~\figref{shift}) 
будут примерно равномерно лежать в диапазоне от~0 до~$x_0$. 

При изменениях параметров цепи периодически проверяйте положение 
нулевой линии синусоиды.

\tasksection{Исследование сдвига фаз в $RL$-цепи}

\item В схеме, собранной согласно \figref{shift}, закоротите магазин ёмкостей. 
Установите $L=50$~мГн, $\nu=1$~кГц (см. рекомендации на установке). 
Рассчитайте реактивное сопротивление цепи $X_2=\omega L$.

\item Меняя сопротивление $R$ от 0 до $\sim 10X_2$ 
(или до $R_{\rm max}$, указанного на установке), 
проведите измерения сдвига фаз $\psi$ для 6--8 значений $R$.

\tasksection{Исследование зависимости свдига фаз от частоты в $RLC$-цепи}

\item В цепи, собранной согласно рис.~\figref{shift}, установите значения 
$R=0$, $L=50$~мГн, $C=0,5$~мкФ (см. рекомендации на установке). 
Рассчитайте резонансную частоту $\nu_0=1/(2\pi\sqrt{LC})$.
 
\item Подбирая частоту ЗГ, добейтесь резонанса. 
При резонансе должен наблюдаться нулевой сдвиг фаз $\psi=0$ (почему?).
При этом нулевые значения двух 
синусоид должны совместиться (а при равенстве амплитуд синусоиды 
полностью совпадают).
 
\item Меняя частоту \emph{в обе стороны} от резонансного значения, 
снимите зависимость сдвига фаз от частоты. С изменением частоты меняется 
расстояние $x_0$, которое занимает половина периода синусоиды, поэтому 
разумно каждый раз фиксировать отношение $x/x_0$. 
Вблизи резонанса ($|\psi|<\pi/3$) точки должны лежать чаще.
 
\item Повторите измерения для сопротивления $R=100$~Ом.
 
\item С помощью лабораторного $LCR$-метра измерьте сопротивление $r$,
а также индуктивность $L$ и активное сопротивление катушки $R_L$. 
Сравните измеренные значения с указанными на установке.


\tasksection{Исследование работы фазовращателя}

\item Соберите схему, изображённую на рис.~\figref{rotator}. Убедитесь, что 
выход ЗГ заземлён. Установите $C=0,5$~мкФ,  $\nu=1$~кГц
(см. рекомендации на установке). Оцените визуально диапазон изменения 
сдвига фаз при изменении $R$ от 0 до 10~кОм.
Подберите сопротивление $R$, при котором сдвиг фаз равен $\pi/2$.


\tasksection{Обработка результатов}

\item По результатам измерений сдвига фаз в $RC$-цепи постройте график
$\ctg\psi=f\left[\omega C R_{\Sigma}\right]$. Здесь $R_{\Sigma}=R+r$~--- 
суммарное активное сопротивление цепи. Получите теоретическую 
зависимость и изобразите её на том же графике. Проанализируйте
соответствие теории и результатов измерения.
%По графику определите значение $R$, при котором $\psi=\pi/2$ и сравните 
%результат с теоретическим.

\item Постройте график зависимости $\ctg\psi=f(R_{\Sigma}/\omega L)$ 
для $RL$-цепи (здесь $R_{\Sigma}=R+r+R_L$). Сравните график с теоретическим.

\item Постройте на одном листе графики $|\psi|=f(\nu/\nu_0)$ 
(фазово-частотные характеристики контура) для $R=0$ и~$100$~Ом (величину $\psi$ удобно откладывать в долях $\pi$).

Определите по графикам добротность контура: $Q=\nu_0/(2\Delta\nu)$, 
где $2\Delta\nu$~--- ширина графика при сдвиге фаз $\psi=\pi/4$.

\item Сравните добротность, определённую графически, с расчётом 
через параметры~$L$, $C$ и~$R$.

\item Постройте векторную диаграмму для фазовращателя; 
с её помощью рассчитайте сопротивление магазина $R$, при котором 
сдвиг фаз между входным и выходным напряжениями равен $\pi/2$. 
Сравните расчёт с измеренным значением.

%\n Сведите результаты эксперимента в таблицу:

\item Оцените погрешности измерений и 
сделайте выводы по результатам эксперимента.

\end{lab:task}


\begin{lab:questions}
	\item Что такое импеданс электрической цепи?
	Как складываются импедансы при последовательном и параллельном
соединении элементов?
    \item Получите формулу \eqref{phasexy} для измерения разности 
          фаз по форме эллипса.
    \item Получите выражение \eqref{cplx} для комплексной амплитуды напряжения на
    выходе фазовращатея.
    \item Дайте определение добротности колебательного контура.
    Опишите известные вам способы измерения добротности.
    \item Получите связь добротности $Q$ колебательного контура c шириной
    $\Delta \nu/\nu_0$ его фазово-частотной характеристики $\psi(\nu/\nu_0)$.
	%\item Можно ли описывать процессы затухания и установления колебаний,
%     пользуясь понятием импеданса?
    \item Как связаны фазы колебаний токов и напряжений 
    при резонансе а) в последовательном контуре, б) в параллельном контуре?
\end{lab:questions}


\begin{lab:literature}
	\item \SivuhinIII.~--- \S\S~129,~130.
	\item \Kalashnikov.~--- \S\S~220, 227, 228.
\end{lab:literature}
