% !TeX spellcheck = russian-aot
% !TeX encoding = UTF-8
\todo[inline]{Основной недостаток текста: слишком мало полезной информации.
Нужно определиться со списком вопросов, которые возникают у студентов
и дать здесь развернутые ответы на них}

\labsection{Измерительные приборы}


Наиболее часто встречающиеся измеримые величины~--- электрический ток $I$ и
разность потенциалов $\Delta U$. Прибор для измерения электрического тока
называется амперметром (название сохраняется, даже когда ток измеряется не в
амперах). Прибор для измерения потенциалов называют вольтметром. Существует
огромное количество типов вольтметров и амперметров, действующих на основе
разнообразных принципов, но все они сохраняют некоторые общие черты.
\todo[inline]{Все знают, как называют приборы измерения тока и напряжения! Абзац не нужен.}

Амперметр включается непосредственно в электрическую цепь таким образом, чтобы
измеряемый ток проходил через него. Идеальным при этом считается такой,
сопротивление которого равно нулю. Исключение из этой общей схемы~---
бесконтактные амперметры, которые используются в основном для измерения больших
токов.
%Несмотря на то, что бесконтактный амперметр не включен в сеть, он все равно
% создает некоторую разность потенциалов за счет электромагнитной индукции. Как
% следствие, даже у такого амперметра есть небольшое сопротивление.
\todo[inline]{Этот абзац, наоборот, нужно существенно расширить. Почему бы не перечислить кратко физические принципы работы амперметров?
В частности, упомянуть, что бесконтактный амперметр работает на законе электромагнитной 
индукции.}

Вольтметр двумя своими контактами подключается к двум точкам цепи, напряжение на
которых надо измерить. Идеальным вольтметром называется такой, сопротивление
которого равно бесконечности и через который не течет ток.
\todo[inline]{Опять-таки, перечисление некоторых физических принципов измерения
напряжения лишним не будет. Можно ли на практике сделать идеальный вольтметр?}

При измерении переменного тока приборы, как правило, показывают не мгновенные
значения тока или напряжения (которые меняются гораздо быстрее, чем человек
может отследить), а их действующее значение, равное среднему квадратичному (для
гармонических сигналов среднее квадратичное равно $\frac{A}{\sqrt{2}}$, где
$A$~--- амплитуда сигнала).
\todo[inline]{Здесь стоит добавить формулу для действующего значения тока 
и напряжения (полную, с интегралом)}

Еще одна важная для электрических цепей величина~--- сопротивление, может быть
получена путем комбинированных измерений напряжения и тока. Самый простой
способ~--- подать на элемент некоторое напряжение и измерить ток, протекающий
через него. 
\todo[inline]{Такой способ ведет в никуда, если на элементе не выполняется
закон Ома! Кроме того, этот способ уж слишком примитивный. Надо хотя бы
вольтметр к измеряемому элементу подключить}
На практике осуществлять точные измерения таким образом не так
просто: у любого источника питания есть собственное сопротивление, в результате
измеренное значение будет смещенным. 
\todo[inline]{От этого недостатка избавляет простой вольтметр}
Смещение можно вычислить, зная
сопротивление источника и контактов, но на практике для точных измерений
используют схему балансировки моста, где сопротивление не измеряется напрямую, а
сравнивается с известным эталонным сопротивлением.
\todo[inline]{Про контакты (и соединительные провода) следует сказать отдельно, 
поскольку именно они больше всего мешают измерять сопротивление, а
вовсе не внутреннее сопротивление источника}
\todo[inline]{Мостовую схему следует привести явно и объяснить, чем она
лучше}
\todo[inline]{Следует также рассказать про 4-х точечную схему подключения
для измерения сопротивлений (например, как в новом описании работы 2.2.2
по термодинамике, см. сайт)}

Помимо тока и напряжения в некоторых случаях измеряют полный протекший заряд и
магнитное поле. Для этих величин методика измерения зависит от того, какой
именно тип прибора используется.
\todo[inline]{Общая фраза. Конкретной информации - ноль}

Измерительные приборы в электричестве можно разделить на два класса: стрелочные
и цифровые. В первом случае всегда присутствует некоторый механический индикатор
(стрелка) и шкала, по которым можно определить интересующее значение.  Во втором
случае важную роль играет так называемый аналогово-цифровой преобразователь
(АЦП), который превращает уровень сигнала (как правило, напряжения) в числовое
значение, которое потом высвечивается на табло или считывается при помощи
компьютера. Важное преимущество стрелочных приборов~--- их точность зависит
только от качества изготовления прибора, в то время как точность цифрового
прибора жестко ограничена разрядностью (диапазоном) АЦП. Еще одно полезное
свойство стрелочных приборов~--- их инерционность: из-за конечной скорости
движения стрелки и конечной жесткости пружины, при быстрых колебаниях измеряемых
величин, стрелочный прибор показывает усредненное значение. Последнее свойство в
некоторых случаях является одновременно и недостатком. В настоящее время
стрелочные приборы выходят из употребления, поскольку точность
среднестатистических АЦП превосходит возможности визуального измерения, а
эффекты усреднения можно имитировать на уровне цифровой электроники или при
компьютерной обработке сигналов.
\todo[inline]{Этот абзац нужно превратить в подраздел. И дать больше полезной
информации. Как считать погрешность прибора? Где полцены деления у цифрового
прибора? И т.д.}


\todo[inline]{Куда копать дальше? Ссылки!}