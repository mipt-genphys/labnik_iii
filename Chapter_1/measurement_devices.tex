% !TeX spellcheck = russian-aot
% !TeX encoding = UTF-8
\labsection{Измерительные приборы}

Наиболее часто встречающиеся измеримые величины~--- электрический ток $I$ и разность потенциалов $\Delta U$. Прибор для измерения электрического тока называется амперметром (название сохраняется, даже когда ток измеряется не в амперах). Прибор для измерения потенциалов называют вольтметром. Существует огромное количество типов вольтметров и амперметров, действующих на основе разнообразных принципов, но все они сохраняют некоторые общие черты.

Амперметр включается непосредственно в электрическую цепь таким образом, чтобы измеряемый ток проходил через него. Идеальным при этом считается такой, сопротивление которого равно нулю. Исключение из этой общей схемы~--- бесконтактные амперметры, которые используются в основном для измерения больших токов. 
%Несмотря на то, что бесконтактный амперметр не включен в сеть, он все равно создает некоторую разность потенциалов за счет электромагнитной индукции. Как следствие, даже у такого амперметра есть небольшое сопротивление.

Вольтметр двумя своими контактами подключается к двум точкам цепи, напряжение на которых надо измерить. Идеальным вольтметром называется такой, сопротивление которого равно бесконечности и через который не течет ток.

При измерении переменного тока приборы, как правило, показывают не мгновенные значения тока или напряжения (которые меняются гораздо быстрее, чем человек может отследить), а их действующее значение, равное среднему квадратичному (для гармонических сигналов среднее квадратичное равно $\frac{A}{\sqrt{2}}$, где $A$~--- амплитуда сигнала).

Еще одна важная для электрических цепей величина~--- сопротивление, может быть получена путем комбинированных измерений напряжения и тока. Самый простой способ~--- подать на элемент некоторое напряжение и измерить ток, протекающий через него. На практике осуществлять точные измерения таким образом не так просто: у любого источника питания есть собственное сопротивление, в результате измеренное значение будет смещенным. Смещение можно вычислить, зная сопротивление источника и контактов, но на практике для точных измерений используют схему балансировки моста, где сопротивление не измеряется напрямую, а сравнивается с известным эталонным сопротивлением.

Помимо тока и напряжения в некоторых случаях измеряют полный протекший заряд и магнитное поле. Для этих величин методика измерения зависит от того, какой именно тип прибора используется.

Измерительные приборы в электричестве можно разделить на два класса: стрелочные и цифровые. В первом случае всегда присутствует некоторый механический индикатор (стрелка) и шкала, по которым можно определить интересующее значение. Во втором случае важную роль играет так называемый аналогово-цифровой преобразователь (АЦП), который превращает уровень сигнала (как правило, напряжения) в числовое значение, которое потом высвечивается на табло или считывается при помощи компьютера. Важное преимущество стрелочных приборов~--- их точность зависит только от качества изготовления прибора, в то время как точность цифрового прибора жестко ограничена разрядностью (диапазоном) АЦП. Еще одно полезное свойство стрелочных приборов~--- их инерционность: из-за конечной скорости движения стрелки и конечной жесткости пружины, при быстрых колебаниях измеряемых величин, стрелочный прибор показывает усредненное значение. Последнее свойство в некоторых случаях является одновременно и недостатком. В настоящее время стрелочные приборы выходят из употребления, поскольку точность среднестатистических АЦП превосходит возможности визуального измерения, а эффекты усреднения можно имитировать на уровне цифровой электроники или при компьютерной обработке сигналов.