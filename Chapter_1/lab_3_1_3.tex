\lab{Измерение магнитного поля Земли}

\begin{lab:aim}
исследовать свойства постоянных неодимовых магнитов; 
измерить с помощью них горизонтальную и вертикальную составляющие индукции магнитного поля
Земли и магнитное наклонение.
\end{lab:aim}

\begin{lab:equipment}
неодимовые магниты; тонкая нить для изготовления 
крутильного маятника; медная проволока; электронные весы; секундомер; 
измеритель магнитной индукции; штангенциркуль; 
брусок, линейка и штатив из немагнитных материалов; 
набор гирь и разновесов.
\end{lab:equipment}


\labsection{Свойства точечного магнитного диполя}

Простейший магнитный диполь может быть образован витком с током или постоянным
магнитом. По определению, магнитный момент  $\vec{\mm}$
тонкого витка площадью $S$ с током $I$ равен (в системе СИ):
\begin{equation}\eqmark{p}
\vec{\mm}=I\vec S,
\end{equation}
где $\vec S = S\vec n$~--- вектор площади контура, образующий 
с направлением тока правовинтовую систему,  
$\vec n$~--- единичный вектор нормали к площадке
(это же направление $\vec{\mm}$ принимается за направление 
S\,$\to$\,N от южного S к северному N полюсу магнита). 
Если размеры контура с током
или магнитной стрелки малы по сравнению расстоянием до диполя, то
соответствующий магнитный диполь $\vec{\mm}$ называют 
\emph{элементарным} или \emph{точечным}. 

Магнитное поле точечного диполя определяется по формуле, 
аналогичной формуле для поля элементарного электрического диполя:
\begin{equation}\eqmark{Bp}
\vec B_{дип}= \frac{\mu_0}{4\pi}
\left(\frac{3(\vec{\mm}\cdot\vec r)\vec r}{r^5} - \frac{\vec{\mm}}{r^3}\right)
\end{equation}
(здесь $\mu_0/4\pi = 10^{-7}\;Н/А^2$).

Во внешнем магнитном поле с индукцией $\vec{B}$
на точечный магнитный диполь $\vec{\mm}$
действует механический момент сил:
\begin{equation}\eqmark{Mp}
\vec{\mathcal{M}}=\left[\vec{\mm}\times \vec B\right].
\end{equation}
При этом потенциальная энергия, которой обладает диполь с постоянным $\vec{\mm}$, 
равна 
\begin{equation}\eqmark{Wp}
W = -\left(\vec{\mm} \cdot \vec B\right).
\end{equation}
Когда диполь ориентирован вдоль внешнего поля ($\vec{\mm} \parallel \vec B$),
он находится состояния \emph{равновесия} ($\vec{\mathcal{M}} = 0$). При этом \emph{устойчивым}
будет только состояние, в котором диполь \emph{сонаправлен} с полем 
$\vec{\mm} \upuparrows \vec B$,
поскольку его потенциальная энергия достигает \emph{минимума} ($W_{\rm min} = - \mm B$).
При противоположной ориентации энергия будет иметь максимум 
($W_{\rm max}=\mm B$) и состояние равновесия будет неустойчивым.

\begin{lab:note}
        В системе СГС формулы \eqref{p}, \eqref{Bp} имеют соответственно вид
        \[
        \vec{\mm}=\frac{1}{c}I\vec S, \qquad 
        \vec B_{дип}= \frac{3(\vec{\mm}\cdot\vec r)\vec r}{r^5} - \frac{\vec{\mm}}{r^3}.
        \]
        Формулы \eqref{Mp} и \eqref{Wp} в СИ и СГС совпадают.
        
Последнее обстоятельство делает выражение \eqref{Wp} удобным для сопоставления единиц измерения 
магнитного момента $\mm$:
\begin{itemize}
\item СИ:  $\left[\mm\right]=\left[W\right]/[B]=\frac{Дж}{Тл}$;
\item СГС: $\left[\mm\right]=\left[W\right]/[B]=\frac{эрг}{Гс} = 
10^{-3}\frac{Дж}{Тл}$.
\end{itemize}
\end{lab:note}

В \emph{неоднородном} внешнем поле выражение для энергии постоянного диполя~\eqref{Wp} 
сохраняется. При этом кроме момента сил на диполь действует ещё и сила:
\begin{equation}\eqmark{Fp}
\vec F= -\nabla W = \left(\vec{\mm} \cdot \nabla\right)\vec B,
\end{equation}
где $\nabla=\left(\frac{{\partial}}{{\partial}x},
\,\frac{{\partial}}{{\partial}y},\,\frac{{\partial}}{{\partial}z}\right)$
--- векторный оператор <<набла>> (оператор Гамильтона). В частности, проекция
\eqref{Fp} на ось $x$ имеет вид
\[
F_x=\mm_{x} \frac{\partial B_{x}}{\partial x}+
\mm_{y} \frac{\partial B_{x}}{\partial y}+
\mm_{z} \frac{\partial B_{x}}{\partial z}.
\]

Таким образом, из \eqref{Mp}---\eqref{Fp} следует, что 
\emph{свободный} магнитный диполь в неоднородном магнитном поле
ориентируется вдоль силовых линий магнитного поля и втягивается в
область более сильного поля, поскольку это ведёт к уменьшению
энергии диполя.

Выражения~\eqref{Bp} и~\eqref{Fp} позволяют рассчитать силу взаимодействия
магнитов с моментами $\vec{\mm}_1$ и $\vec{\mm}_2$ в рамках модели точечных
диполей. В~частном случае, когда моменты двух небольших магнитов 
направлены вдоль соединяющей их прямой $\vec\mm_{1,2} \parallel \vec{r}$, 
где $\vec{r}$ --- радиус-вектор между ними, 
магниты  взаимодействуют с силой
\begin{equation} \eqmark{Fpp}
F_{12}=\mm_1\ppd{B_2}{r}=\mm_1\ppd{(2\mm_2/r^3)}{r}=-\frac{6\mm_1\mm_2}{r^4}
\quad (\text{ед. СГС}).
\end{equation}
(при использовании системы СИ \eqref{Fpp} нужно домножить на $\mu_0/4\pi$).
Здесь магниты притягиваются, если их магнитные моменты сонаправлены 
($\vec\mm_1 \upuparrows \vec\mm_2$) и отталкиваются, 
если направлены противоположно ($\vec\mm_1 \uparrow\downarrow \vec\mm_2$).

Если магнитные моменты направлены перпендикулярно соединяющей их прямой 
$\vec\mm_{1,2}\perp \vec{r}$, то нетрудно показать, что сила
их взаимодействия окажется в два раза меньшей и будет 
иметь противоположный знак: 
\[
F_{12}=\frac{3\mm_1\mm_2}{r^4} \qquad (\text{ед. СГС})
\]
(диполи притягиваются при  $\mm_1\uparrow \downarrow \mm_2$
 и отталкиваются при  $\mm_1\upuparrows \mm_2$).


\experiment

В работе используются неодимовые магниты
шарообразной формы.

\begin{lab:note}
<<Неодимовые>> магниты, состоящие из соединения
неодим-железо-бор (NdFeB), впервые были получены в 1983 году и 
на сегодняшний день являются самыми мощными постоянными магнитами, 
с помощью которых можно поднимать грузы порядка тонны. 
Они нашли широкое применение в изготовлении 
жёстких дисков, DVD-приводов, звуковых динамиков, магнитных томографов и др.
\end{lab:note}

Для проведения эксперимента важно, что а) вещество, из которого изготовлены магниты, 
является \emph{магнитожёстким} материалом; б) шары намагничены однородно.

<<Магнитожёсткость>> материала (см. подробнее Раздел~IV) 
означает, что магнитные моменты шаров 
в процессе работы не изменяются под действием внешних магнитных полей, 
т.\,е. шар ведёт себя как постоянный (<<жёсткий>>) диполь. 
В том числе, магнитные моменты не изменяются при контакте магнитов друг
с другом.
%Поэтому, при расчетах можно считать, что шары взаимодействуют как
%жёсткие точечные магнитные диполи, расположенные в центрах шаров.

Магнитное поле однородно намагниченного шара радиусом~$R$ может быть 
вычислено точно. На расстояниях $r{\geq}R$ от центра шара оно 
совпадает с полем \emph{точечного} магнитного диполя \eqref{Bp}, 
расположенного в центре, магнитный момент~$\vec\mm$ которого совпадает
с полным моментом шара. 
Внутри шара магнитное поле однородно: c помощью формулы 
\eqref{Bp} и условия непрерывности нормальной компоненты индукции
на поверхности шара нетрудно получить, что при $r < R$
\begin{equation}\eqmark{Bin}
\vec B_0 = \frac{\mu_0\vec \mm}{2\pi R^3}
\end{equation}
(в ед. СГС $\vec{B}_0 = 2 \vec \mm / R^3$).

В качестве ещё одной характеристики материала магнита используют остаточную
\term{намагниченность}~$\vec M$.
По определению, намагниченность равна \emph{объёмной плотности магнитного момента},
поэтому для однородно намагниченного шара
\begin{equation}
\vec\mm = \vec M V,
\end{equation}
где $V=\frac{4\pi}{3}R^3$ --- объём магнита. 
Величину $\vec B_r=\mu_0 \vec M$ называют \emph{остаточной индукцией} 
материала (в~ед.~СГС $\vec B_r = 4\pi \vec M$).

Из \eqref{Bp} нетрудно видеть, что индукция~$\vec B_p$ \emph{на полюсах} 
однородно намагниченного шара направлена по нормали к поверхности и
совпадает поэтому с индукцией внутри шара~$\vec B_p = \vec B_0$. 
Как следует из \eqref{Bin}, величина $B_p$ связана с остаточной 
индукцией $B_r$ соотношением
\begin{equation}\eqmark{BpBr}
B_p = B_0 = \frac{2}{3} B_r.
\end{equation}

\begin{lab:note}
Электрическим аналогом магнитного шара из магнитожёсткого материала является
диэлектрический шар, изготовленный из \textit{электрета}~--- материала с
<<замороженной>> поляризацией. По своей топологии внешнее поле шара из электрета
не отличается от поля постоянного шарообразного магнита. В электрическом поле 
он ведёт себя точно также, как магнитный шар в магнитном поле. 
Формулы, описывающие взаимодействие
постоянных шарообразных магнитов между собой и с магнитным полем, идентичны 
соотношениям для электрических диполей в электрическом поле.
\end{lab:note}

\labsection{Определение магнитного момента магнитных
шариков}

\labsubsection{Метод А}

\begin{wrapfigure}{o}{0.25\textwidth}
    \pic{\linewidth}{Chapter_1/1_3_1}
    \caption{Измерение магнитных моментов шариков}\figmark{1}
\end{wrapfigure}

Величину магнитного момента $\mm$ двух одинаковых шариков можно рассчитать, 
зная их массу $m$ и определив максимальное
расстояние  $r_{\mathrm{max}}$, на котором они ещё удерживают друг 
друга в поле тяжести (см. рис.~\figref{1}). 
При максимальном расстоянии сила тяжести шариков $mg$ равна силе их 
магнитного притяжения. Когда векторы двух магнитных моментов
ориентированы вертикально, из \eqref{Fpp} имеем:
\begin{equation}\eqmark{mmA}
\mm = \sqrt{\frac{mgr^4_{\rm max}}{6}}\qquad (\text{ед.~СГС}).
\end{equation}

По величине $\mm$ с помощью \eqref{Bp} 
можно рассчитать величину индукции~$\vec B$ вблизи любой точки на поверхности 
шара радиуса~$R$. Максимальная величина индукции наблюдаются на
полюсах (см.~\eqref{BpBr}).

\labsubsection{Метод Б}

\begin{wrapfigure}{i}{0.35\textwidth}
    \pic{\linewidth}{Chapter_1/1_3_2}
    \caption{Альтернативный метод измерения магнитных моментов шариков}\figmark{2}
\end{wrapfigure}
Величину магнитного момента шариков можно определить также 
по силе их сцепления. Она определяется как сила, необходимая для разрыва 
двух сцепившихся магнитных шариков. Сила сцепления максимальна, 
если шары  соединяются своими противоположными полюсами 
(магнитные моменты сонаправлены). 

Максимальную силу сцепления можно определить по весу магнитной цепочки, которую
способен удержать самый верхний магнитный шарик. Если цепь состоит из
одинаковых магнитных шариков (см. рис.~\figref{2}а), то при определённой 
длине она отрывается от верхнего шарика. При этом, учитывая, 
что сила притяжения убывает как  $F\propto 1/r^4$, где $r$~--- 
расстояния между центрами шаров, для рассчёта прочности цепочки 
достаточно учитывать силу взаимодействия верхнего шара с 3--4 ближайшими соседями. 

Сила сцепления \eqref{Fpp} двух одинаковых шаров радиусами~$R$ 
c магнитными моментами~$\mm$ равна 
\begin{equation}\eqmark{F0}
F_0=\frac{6\mm^2}{(2R)^4}=\frac{3\mm^2}{8R^4}\qquad (ед.~СГС).
\end{equation}
Тогда минимальный вес цепочки, 
при которой она оторвётся от верхнего шарика равен:
\begin{equation}\eqmark{FF}
F=F_0 \left(1 + \frac{1}{2^4} + \frac{1}{3^4}+ \frac{1}{4^4}+\ldots\right)\approx 1,08 F_0. 
\end{equation}
(мы ограничились четырьмя членами ряда; точность такого приближения предлагается
оценить самостоятельно). 
Отметим, что не обязательно составлять цепочку только из одинаковых шариков: на
расстояниях, превышающих 20--30 диаметров шариков, можно подцепить любой
груз, притягиваемый магнитом (см. рис.\figref{2}б),~--- 
на результат это не повлияет, в чём несложно убедиться экспериментально.

\labsection{Измерение горизонтальной составляющей индукции 
    магнитного поля Земли}

Магнитное поле Земли в настоящей работе измеряется 
по периоду крутильных колебаний <<магнитной стрелки>> вокруг вертикальной оси.

\begin{wrapfigure}{i}{0.35\textwidth}
    \pic{\linewidth}{Chapter_1/1_3_3}
    \caption{Крутильный маятник во внешнем магнитном поле}\figmark{3}
\end{wrapfigure}
<<Магнитная стрелка>> образована сцепленными друг
с другом $n$ намагниченными шариками. С~помощью $\Lambda$-образного подвеса стрелка 
подвешена в горизонтальном положении (см. рис.~\figref{3}). 
Для крепления нити в работе используется штатив, изготовленный из немагнитного
материала. 

Магнитные моменты всех шариков направлены в одну сторону вдоль оси <<стрелки>>. Под действием
механического момента сил \eqref{Mp}, действующего на стрелку со стороны
поля Земли, стрелка стремится повернуться по горизонтальной составляющей 
магнитного поля Земли $\vec B_{\parallel}$ в направлении Юг---Север.


При отклонении стрелки на угол $\theta$ от равновесного положения 
в горизонтальной плоскости возникают крутильные колебания вокруг 
вертикальной оси, проходящей через середину стрелки. Если пренебречь 
упругостью нити, то уравнение крутильных колебаний такого маятника
определяется возвращающим моментом сил 
\[
\mathcal{M}=-\mm_n B_{\parallel}\sin \theta
\]
и моментом инерции $J_n$ <<стрелки>> относительно оси вращения. 
При малых амплитудах  $(\sin \theta \approx \theta)$ уравнение колебаний
стрелки имеет вид: 
\begin{equation*}
J_n \ddot{\theta} + \mm_n B_{\parallel} \theta = 0.
\end{equation*}
Отсюда находим период малых колебаний
\begin{equation}
T=2\pi \sqrt{\frac{J_n}{\mm_n B_{\parallel}}}.
\end{equation}

Здесь $\mm_n = n \mm$ --- полный магнитный момент магнитной <<стрелки>>, 
составленной из $n$ шариков.
Момент инерции $J_n$ стрелки из $n$ шариков с хорошей точностью равен 
моменту инерции тонкого однородного стержня массой $m_n = nm$
и длиной $\ell_n = n \cdot 2R$:
\begin{equation}\eqmark{Jn}
J_n \approx \frac{1}{12} m_n \ell_n^2 = \frac{1}{3}n^3 m R^2.
\end{equation}
%Даже для трёх шариков момент инерции, рассчитанный по приближённой формуле,
%отличается от точного результата (см. контрольный вопрос № 14) примерно на 2
%\%, а для  $n{\geq}5$~--- различие не превышает процента; если же учесть, что 
%$T\ \sqrt{I_n}$, то для всех  $n{\geq}3$\textit{ }погрешность наших расчетов
%для периода колебаний  $T$ не превысит процента, что освобождает нас от
%необходимости вводить поправочные коэффициенты.
Отсюда находим, что период колебаний маятника 
пропорционален числу шаров~$n$, составляющих <<стрелку>>:
\begin{equation}\eqmark{Tn}
T_n=k n,
\end{equation}
где $k= 2\pi \sqrt{mR^2/3\mm B_{\parallel}}$.

При выводе \eqref{Tn} предполагалось, что магнитный момент~--- величина
\emph{аддитивная}: полный магнитный момент системы магнитов равен
векторной сумме магнитных моментов шариков.
Экспериментальное подтверждение этой зависимости $(T \propto n)$ будет являться
подтверждением справедливости предположений о магнитожёсткости материала
магнитов и, следовательно, аддитивности их магнитных моментов.

\labsection{Измерение вертикальной составляющей
индукции магнитного поля Земли. Магнитное наклонение.}

Для измерения вертикальной  $B_{\perp}$ составляющей вектора индукции поля Земли
используется та же установка, что и для измерения горизонтальной составляющей с
тем лишь отличием, что подвешенная магнитная <<стрелка>> 
закрепляется на нити в одной точке. 
В этом случае стрелка, составленная из чётного числа одинаковых шариков
и подвешенная за середину, расположится не горизонтально, 
а под некоторым углом к горизонту (см. рис.~\figref{4}а). 
Это связано с тем, что вектор~$\vec B$ индукции магнитного поля Земли 
не горизонтален, а образует с горизонтом некоторый угол~$\beta$, 
зависящий от географической широты  $\varphi$ места, где проводится опыт. 
Величина угла  $\beta $ называется \term{магнитным наклонением}.

\begin{figure}[h!]
    \centering
        \pic{8cm}{Chapter_1/1_3_4}
        \caption{Измерение вертикальной составляющей поля и магнитного наклонения}\figmark{4}
\end{figure}
      
Измерить магнитное наклонение непосредственно по положению подвешенной <<стрелки>>
затруднительно из-за механического момента нити в точке подвеса,
неизбежно возникающем при наклоне <<стрелки>>.
Избавиться от этого можно, если выровнять её горизонтально
с помощью небольшого дополнительного грузика (см. рис.~\figref{4}б).
В~этом случае момент силы тяжести
груза относительно точки подвеса будет равен моменту сил, действующих на
<<стрелку>> со стороны вертикальной составляющей магнитного поля Земли. 
Если масса уравновешивающего груза равна $m_{гр}$, плечо силы тяжести 
$r_{гр}$, а полный магнитный момент стрелки $\mm_n = n\mm$, то в равновесии: 
\begin{equation}\eqmark{mgr}
\mathcal{M}_n = m_{гр}gr_{гр}= n\mm B_{\perp}.
\end{equation}
Видно, что момент $\mathcal{M}_n$ силы тяжести уравновешивающего 
груза пропорционален числу $n$ шариков, образующих
магнитную <<стрелку>>: $\mathcal{M}\propto n$.

\begin{lab:warning}
    Магнитные предметы (ножницы, металлическая линейка, корпуса приборов,
    мобильные телефоны и др.), близко расположенные к экспериментальной установке, 
    могут существенно искажать результаты опыта по измерению индукции 
    магнитного поля Земли.
\end{lab:warning}


\begin{lab:task}

\taskpreamble{~}
\vspace*{-8ex} % workaround

\tasksection{I.~Определение магнитного момента, намагниченности и
остаточной магнитной индукции вещества магнитных шариков}

\item Измерьте диаметр шариков и взвесьте их на весах. Имейте ввиду, что весы могут давать
некорректные показания, если магниты класть непосредственно на платформу весов!

\item С помощью магнетометра измерьте индукцию поля $B_p$ на полюсах шарика. 
 
\item Проложите между двумя магнитными шариками брусок из немагнитного
материала. Подкладывая между бруском и верхним магнитиком листы бумаги 
(см. рис.~\figref{1}), определите, на каком максимальном 
расстоянии~$r_{\mathrm{max}}$ шарики удерживают друг друга в поле тяжести Земли. 

\item Рассчитайте величину магнитного момента магнитика $\mm$
(см. описание метода А). Оцените погрешность измерений.

\item Используя дополнительные шарики, составьте цепочку из 20--30 шариков и, 
с помощью неодимовых магнитов в форме параллелепипедов, подсоедините цепочку к
гире и разновесам, так, чтобы общая масса системы составила $\sim500$~г 
(рис.~\figref{2}б). 
Добавляя или удаляя шарики (шарики можно примагничивать непосредственно к
гире), подберите минимальный вес системы цепочки с гирей, 
при котором она отрывается от верхнего шарика. 

\item С помощью весов определите вес оторвавшейся цепочки с гирей.

\item Рассчитайте силу сцепления двух шаров 
и по ней определите магнитный момент шарика~$\mm$ (см. описание метода Б). 
Оцените погрешность результата.

\item Сравните значения магнитных моментов, полученные двумя методами. 
Какой метод даёт более точный результат?

\item Рассчитайте величину намагниченности материала шариков~$M$
и остаточную индукцию магнитного поля~$B_r$. 
Сравните значение~$B_r$ c табличными для соединения NdFeB.

\item Рассчитайте индукцию $B_p$ у полюсов шарика (см. \eqref{BpBr}). 
Сравните расчетное~$B_p$ значение с измеренным. 
При сильном расхождении результатов повторите измерения.

\tasksection{II.~Определение горизонтальной составляющей магнитного
поля Земли}

\item Соберите крутильный маятник (рис.~\figref{3}) из 12 магнитных шариков
и подвесьте его к немагнитному штативу.
Используя $\Lambda$-образный подвес, установите <<магнитную стрелку>>
 в горизонтальное положение (юстировка системы). 

\begin{figure}[h!]
   \centering
   \pic{1.9cm}{Chapter_1/1_3_5}
   \caption{Магнитная <<стрелка>>, свернутая в кольцо}\figmark{6}       
\end{figure}

\item Возбудите крутильные колебания маятника вокруг вертикальной оси 
и определите их период. Оцените влияние упругости (модуля кручения) нити 
на период колебаний, 
возбудив крутильные колебания <<стрелки>>, свёрнутой в кольцо 
(магнитный момент такого кольцеобразного маятника равен нулю) 
(см. рис.~\figref{6}). Убедитесь, что упругость нити при расчете 
периода колебаний можно не учитывать.

\item Исследуйте зависимость периода $T$ крутильных колебаний <<стрелки>> 
от количества магнитных шариков $n$, составляющих <<стрелку>>. Измерения
проведите для значений  $n=3,\,4,\,5,\,\dots,\,12$. Не забывайте для каждого
значения~$n$ юстировать систему, выставляя перед каждым измерением <<стрелку>>
горизонтально!

\item Постройте график экспериментальной зависимости $T(n)$.
Убедитесь в линейности этой зависимости. 
По значению углового коэффициента рассчитайте 
величину горизонтальной составляющей магнитного поля Земли (см. \eqref{Tn}).  
Оцените погрешность результата.

\tasksection{III. Определение вертикальной составляющей магнитного поля
Земли}

\item Изготовьте магнитную <<стрелку>> из $n=10$ шариков 
и подвесьте её за середину с помощью нити на штативе (см. рис.~\figref{4}а).
 
\item Определите механический момент сил, действующий со стороны магнитного поля
Земли на \emph{горизонтально} расположенную магнитную <<стрелку>>. 
Для этого, с помощью одного или нескольких кусочков проволоки, 
уравновесьте <<стрелку>> в горизонтальном положении
(см. рис.~\figref{4}б). 

\item С помощью весов определите массу уравновешивающего груза $m_{гр}$

\item Из условия равновесия рассчитайте механический момент сил~$\mathcal{M}$,
действующих на горизонтальную <<стрелку>> со стороны поля Земли. Измерения
момента сил проведите для чётных значений  $n= 4,\, 6,\, 8,\, 10,\, 12$.

\item  Постройте график экспериментальной зависимости
$\mathcal{M}(n)$. Убедитесь в линейности этой зависимости. 
Сделайте вывод о применимости приближения аддитивности
магнитных моментов для используемых в работе магнитов.
По значению углового коэффициента аппроксимирующей прямой 
рассчитайте величину вертикальной составляющей~$B_{\bot}$ магнитного
поля Земли. Оцените погрешность результата.

\item Используя результаты измерений  $B_{\perp}$ и  $B_{\parallel}$, 
определите магнитное наклонения~$\beta$ и полную величину индукции магнитного поля Земли на широте Долгопрудного. 

\item Сравните полученное значение наклонения $\beta$ с расчётным, 
в предположении, что магнитное поле Земли есть
поле однородно намагниченного вдоль оси вращения шара.
Оцените также полный магнитный момент~$\mm_{З}$ Земли.

\item Сравните полученные в работе результаты со справочными данными 
параметров магнитного поля Земли в Московском регионе.

\end{lab:task}

\begin{lab:questions}

\item Что такое магнитный момент? 
В~каких единицах он измеряется (в СИ и СГС)?

\item Какой ток $I$ должен циркулировать по плоскому витку радиусом, равным радиусу
используемых в работе магнитных неодимовых шариков, чтобы его магнитный момент
оказался равен магнитному моменту этих шариков? Расчёт проведите 
в системах СИ и СГС.

\item Какие силы действуют на точечный магнитный диполь 
в неоднородном внешнем магнитном поле?

\item По найденному в эксперименте значению магнитного момента неодимовых 
шариков, рассчитайте энергию и силу взаимодействия двух находящихся в контакте 
шариков, моменты которых ориентированных а)~вдоль, или б)~перпендикулярно соединяющих их прямой.

\item Как сила сцепления двух одинаковых неодимовых шарообразных магнитов 
зависит от их диаметров? Оцените силу сцепления двух 
неодимовых шаров диаметром $D=3$~см каждый. 

\item Покажите, что поле внутри однородно намагниченного шара 
радиуса  $R$ однородно и равно $\vec B_0=2\vec{\mm}/R^3$, 
где $\vec{\mm}$~--- полный магнитный момент шара. 

\item Найдите зависимость индукции магнитного поля  $B(\theta )$ 
от угла~$\theta $ вблизи поверхности однородно намагниченного шара 
(полярный угол $\theta $ отсчитывается от северного полюса шара).

%Найдите распределения токов намагничивания  $i(\theta )$ на поверхности
%однородно намагниченного шара ( $i$~--- линейная плотность поверхностных токов, 
%$\theta $~--- полярный угол). 

\item Приведите примеры материалов, используемых для изготовления 
постоянных магнитов. Каковы их характеристики? Используя табличные данные,
покажите, что намагниченность используемых в работе шариков можно 
считать практически постоянной.

\item Изменяется ли намагниченность постоянных магнитов с течением времени? 
Как размагнитить постоянный магнит? 
Как размагнитить намагниченный в поле магнита стальной инструмент 
(отвёртку,  пинцет)?

\item Получите точные значение момента инерции крутильного маятника  $J_n$
относительно оси вращения, составленного из  $n$ шариков и сравните
их с соответствующими значениями момента инерции, рассчитанными по приближённой
формуле для тонкого стержня \eqref{Jn}. При каком числе $n$
отличие будет составлять менее 1\%?

\item Для чего в конструкции крутильного маятника используется $\Lambda$-образный
подвес?

\item Получите формулу для периода крутильных колебаний горизонтальной магнитной
стрелки вблизи положения равновесия.

\item Поясните опыт, с помощью которого выясняется влияние упругости нити на
период крутильных колебаний.

\item Неодимовый шарообразный магнит диаметром $D=6$ см разрезали пополам. Какую
минимальную силу надо приложить, чтобы оторвать (без сдвига) одну половинку от
другой? 


\end{lab:questions}

\begin{lab:literature}
    \item \SivuhinIII~--- \S52, 57, 58, 74.
    \item \KingLokOlh~--- Ч.~II \S4.4, 5.3, 7.2.
    \item \Kirichenko~--- \S3.4, 4.2.
\end{lab:literature}


\end{document}
