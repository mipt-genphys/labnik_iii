\introsection{Единицы измерения в электродинамике}
\label{sec:app_units}

При изучении нового класса физических явлений перед исследователем
неизбежно встаёт проблема выбора единиц измерения (эталонов) новых
физических величин.

Проводя измерения, исследователь в первую очередь пользуется некоторым
базовым набором эталонов (например, секунда, метр и килограмм), и
в идеале новые единицы измерения должны определяться через базовые.
Однако пока законы, описывающие новое явление, остаются не понятыми,
выбор новых единиц измерения происходит по большому счету \emph{произвольно}.
Когда связь между <<новыми>> и <<старыми>>
явлениями устанавливается прочно, эти искусственно введённые эталоны
становятся избыточными, поскольку могут быть сведены к базовым. Однако,
поскольку главным условием выбора единиц служит, как правило,
их \emph{практическое удобство} и непосредственная
\emph{привязка к методикам измерения}~--- они часто остаются в употреблении,
а порой заносятся и в международные стандарты.

Например, на заре исследования тепловых явлений было создано множество
температурных шкал. Лишь позднее, когда была установлена связь температуры
с энергией идеального газа, все эти шкалы оказались по сути избыточны:
в качестве единиц измерения температуры можно использовать единицы
измерения энергии (например, джоули). Тем не менее, по историческим
причинам в международную систему СИ (от фр.~Système international d’Unités) вошла <<независимая>>
единица измерения температуры~--- \emph{кельвин}. Из-за этого во
всех термодинамических формулах нам приходится иметь дело с <<лишней>>
переводной константой~--- \emph{постоянной Больцмана}:
\[
k_{\text{Б}}\approx 1{,}38\cdot10^{-23}\;\text{Дж}/\text{К}.
\]

Аналогичная ситуация имела место и при исследовании электричества
и магнетизма. Более того, пока не было установлено единство электро-магнитных
явлений, в каждой из этих областей была внедрена своя, независимая
система единиц. Последующие попытки переработать и оптимизировать
систему электромагнитных единиц, чтобы избавить её от нефизичных
констант, имели весьма ограниченный успех, поскольку явление быстро
нашло широкое практическое применение и поменять что-либо радикально
было уже невозможно. В результате в качестве международного стандарта
была принята система (СИ), содержащая <<лишнюю>>
базовую единицу~--- \emph{ампер}. При этом в теоретических работах,
не привязанных непосредственно к технике, сохранилась в употреблении
гауссова система (СГС), не содержащая избыточных эталонов (такие системы
единиц называют \emph{абсолютными}). При это возникает весьма неприятная
особенность: в разных системах различными оказываются не только \emph{единицы
измерения} (как в механике), но и \emph{формулы}, их связывающие.

Практическая направленность данной книги вынуждает и нас придерживаться
системы СИ (все лабораторные приборы проградуированы именно в СИ).

\introsubsection{Абсолютная (гауссова) система единиц}

Построим абсолютную систему единиц электродинамики, базирующуюся исключительно
на механических эталонах. За основу примем два твёрдо установленных
закона: Кулона и Ампера.

По сравнению с механикой, существенно новым понятием является \term{электрический
заряд}. Согласно закону Кулона, сила взаимодействия двух одинаковых
точечных зарядов пропорциональна квадрату заряда~$q$ и обратно пропорциональна
квадрату расстояния~$r$ между ними:
\begin{equation}
F_{\text{К}}=k_{q}\cdot\frac{q^{2}}{4\pi r^{2}},\label{eq:coulomb}
\end{equation}
где $k_{q}$~--- некоторая константа. Коэффициент $4\pi$ в знаменателе
поставлен из эстетических соображений,
чтобы подчеркнуть, что зависимость $\propto1/r^{2}$ в законе Кулона
(и в законе всемирного тяготения) возникает как площадь сферы в трёхмерном
пространстве.

На закон (\ref{eq:coulomb}) можно смотреть как на \emph{определение}
того, что такое электрический заряд.

Константа $k_{q}$ может быть выбрана \emph{произвольным} образом.
При этом, если мы не хотим вводить никаких дополнительных эталонов,
она должна быть \emph{безразмерна}. В гауссовой системе
полагают $k_{q}=4\pi$. Неплох также вариант $k_{q}=1$, сохраняющий
<<сферичность>> закона Кулона. Он предлагался в своё время
Лоренцем и другими известными физиками, но не прижился.

Тогда (\ref{eq:coulomb}) позволяет ввести единицу измерения заряда
с размерностью
\[
\left[q\right]=\left[F^{1/2}r\right]=\frac{M^{1/2}L^{3/2}}{T}.
\]
Здесь $\left(M,L,T\right)$~--- размерности базовых единиц массы,
длины и времени соответственно. По историческим причинам наибольшее
распространение получила абсолютная система с эталонами \emph{сантиметр--грамм--секунда}
(СГС). Тогда единицу заряда можно определить как
\[
\left[q\right]=\frac{\text{г}^{1/2}\cdot\text{см}^{3/2}}{\text{с}}.
\]

Закон Ампера говорит о силовом взаимодействии \emph{токов}~--- то
есть \emph{движущихся} зарядов. Его можно сформулировать следующим
образом: сила взаимодействия двух параллельных проводников на единицу их длины
пропорциональна квадрату заряда, проходящему через них в единицу времени
(т.\,е. току $I=q/t$), и обратно пропорциональна расстоянию между ними:
\begin{equation}
\frac{F_{\text{А}}}{l}=k_{I}\frac{I^{2}}{2\pi r}\label{eq:ampere}
\end{equation}
($2\pi r$ --- длина окружности радиуса~$r$ вокруг провода).

Здесь уже все входящие в формулу величины вполне определены,
поэтому константа $k_{I}$ не может быть произвольной~--- её значение
должно быть получено \emph{из опыта}. Найдём её размерность:
\[
\left[k_{I}\right]=\left[\frac{F}{q^{2}}t^{2}\right]=\frac{T^{2}}{L^{2}}.
\]
То есть $k_{I}$ есть величина, обратная квадрату некоторой скорости.
Положим
\begin{equation}
k_{I}=\frac{k_{q}}{c^{2}},\label{eq:kikq}
\end{equation}
где $c$~--- константа, которую можно назвать \term{электродинамической
постоянной}, так как она связывает взаимодействия неподвижных и движущихся
зарядов. Отметим, что соотношение \ref{eq:kikq} должно выполняться
для всех систем измерения.

Как мы знаем сегодня, электродинамическая постоянная,
равная $c\approx3\cdot10^{10}\;\text{см}/\text{с}$,~--- не что иное
как \term{скорость света}, то есть максимальная скорость
распространения всех известных науке взаимодействий. В~настоящее время
эта константа относится к числу \emph{фундаментальных} и ей приписывают
\emph{точное} значение $c\equiv299\,792\,458\;\text{м/с}$
и используют для определения эталона метра через эталон секунды.

Таким образом, выбранные нами в качестве базовых законы Кулона и Ампера
выглядят как
\begin{equation}
F_{\text{К}}=\frac{q_{1}q_{2}}{r^{2}},\qquad F_{\text{А}}=\frac{2I_{1}I_{2}}{rc^{2}}l.\label{eq:col-amp}
\end{equation}

Теперь можно определить и остальные электродинамические величины.

\term{Напряжённость электрического поля} определяют (во всех системах)
как отношение силы Кулона к заряду:
\[
E=\frac{F}{q},\qquad\left[E\right]=\frac{M^{1/2}}{L^{1/2}T}=\frac{\text{г}^{1/2}}{\text{см}^{1/2}\cdot\text{с}}.
\]

\term{Индукцию магнитного поля} можно определить как отношение силы
на единицу длины провода к току в нём:
\begin{equation}
B=k_{B}\cdot\frac{F_{\text{А}}/l}{I}.\label{eq:kB}
\end{equation}
Выбор коэффициента $k_{B}$ опять-таки за нами. Зная о единстве электрического
и магнитного полей, его стоит выбрать таким, чтобы их размерности
\emph{совпадали}:
\[
\left[B\right]=\left[k_{B}\right]\cdot\left[\frac{F}{Il}\right]=\left[k_{B}\right]\cdot\frac{T}{L}\left[E\right]
\]
Видно что $k_{B}$ в таком случае должен иметь размерность скорости:
$\left[k_{B}\right]=L/T$, поэтому уместно положить $k_{B}=c$. Тогда
формулу для силы Ампера в терминах $B$ можно записать как
\begin{equation}
F_{\text{А}}=\frac{1}{c}IBl.
\end{equation}
Закон Ампера (\ref{eq:col-amp}) получится, если поле прямого провода
вычисляется как
\[
B_{\text{пр}}=\dfrac{2I}{cr}.
\]

Итак, на основе законов Кулона и Ампера мы определили единицы измерения
всех основных электродинамических величин: электрического заряда,
электрического и магнитного полей. Кроме того, \emph{теория размерностей}
позволила нам попутно установить существование фундаментальной константы~---
электродинамической постоянной~$c$.

\introsubsection{Система СИ}

Абсолютная система, предложенная выше, не могла быть разработана во
времена Ампера и Кулона, когда еще не было осознано единство электрических
и магнитных явлений.
% (исследования Ампера были \emph{первым} шагом
%в этом направлении).

Следуя системе СИ, выберем закон Ампера в качестве определения того,
что такое \term{электрический ток}. Помимо базовых единиц (\emph{метр--килограмм--секунда}),
СИ определяет \emph{независимый} эталон тока: при токе в 1~\emph{ампер} два
тонких провода на расстоянии 1~м взаимодействую с силой $2\cdot10^{-7}$~Н
на 1~м их длины. Это значит, что константа $k_{I}$ в законе (\ref{eq:ampere}),
которую называют \term{магнитной постоянной} и традиционно обозначают
как~$\mu_{0}$, выбрана равной
\[
\mu_{0}=4\pi\cdot10^{-7}\;\text{Н}/\text{А}^{2}.
\]
Закон Ампера в этих единицах:
\begin{equation}
F_{\text{А}}=\mu_{0}\frac{I_{1}I_{2}}{2\pi r}l.
\end{equation}

Последствием введения лишнего эталона будет то, что во все <<магнитные>> формулы
войдёт дополнительная \emph{размерная константа} $\mu_0$, заданная \emph{точно}.
Она не имеет явного физического смысла, а её единственная функция~---
перевод между единицами измерения. Зато <<ампер>>~--- очень удобная
единица измерения тока: в частности, характерные значения токов в
большинстве бытовых электроприборов варьируются в пределах $10^{-2}\div10^2\;\text{А}$.

Имея эталон тока, можно определить единицу измерения заряда~--- \emph{кулон},
$1\;\text{Кл}=1\;\text{А}\cdot\text{с}$. При таком подходе константа~$k_{q}$ в законе Кулона (\ref{eq:coulomb}) может быть определена
из опыта. Как мы уже обсуждали выше, опыт даёт результат (\ref{eq:kikq}),
поэтому $k_{q}=\mu_{0}c^{2}$. Традиционно вводят обозначение
\begin{equation}
\varepsilon_{0}=\frac{1}{\mu_{0}c^{2}}\approx8{,}85\cdot10^{-12}\;\frac{\text{Кл}^{2}}{\text{Н}\cdot\text{м}^{2}}.
\end{equation}
Величину $\varepsilon_{0}$ называют \term{электрической постоянной}.
Она также является просто переводной константой, не имеющей
явного физического смысла. Физический смысл имеет электродинамическая
постоянная (она же скорость
света в вакууме):
\[
с = \frac{1}{\sqrt{\varepsilon_{0}\mu_{0}}}.
\]
%\begin{lab:note}
%Исторически использовались термины <<диэлектрическая
%проницаемость вакуума>> и <<магнитная проницаемость
%вакуума>>, которые ещё можно встретить в старых учебниках.
%Эти термины достались нам в наследство от теории эфира:
%вакуум считался заполненным некоторой упругой средой. По современным
%представлениям, отвергающим примитивную концепцию эфира, говорить
%о проницаемости вакуума бессмысленно.
%\end{lab:note}


Таким образом, закон Кулона в системе СИ имеет вид
\begin{equation}
F_{\text{К}}=\frac{1}{\varepsilon_{0}}\frac{q_{1}q_{2}}{4\pi r^{2}}.
\end{equation}

Наконец, константа $k_{B}$ в определении индукции магнитного поля
(\ref{eq:kB}) принимается равной единице. Поэтому справедливы формулы
\begin{equation}
F_{\text{А}}=IBl.
\end{equation}
Поле прямого провода вычисляется как
\[
B_{\text{пр}}=\mu_{0}\dfrac{I}{2\pi r}.
\]

Наконец, выразим единицы измерения измерения полей:
\[
\left[E\right]=\frac{\text{Н}}{\text{Кл}}=\frac{\text{В}}{\text{м}}=\frac{\text{кг}\cdot\text{м}}{\text{А}\cdot\text{с}^{3}}.
\]
Здесь $1\;\text{В}=1\;\text{Дж}/1\;\text{Кл}$ --- единица напряжения
\emph{вольт}. Единица индукции магнитного поля \emph{тесла}:
\[
\left[B\right]\equiv\text{Тл}=\frac{\text{Н}}{\text{А}^{2}}\cdot\frac{\text{А}}{\text{м}}=\frac{\text{кг}}{\text{А}\cdot\text{с}^{2}}.
\]
Видно, что размерности $E$ и $B$ в системе СИ разные. Их отношение имеет размерность
скорости: $\left[E/B\right]=\text{м/c}$.

\begin{lab:note}
Согласно решению международной комиссии по стандартам с 2019 года
планируется принять новые определения базовых единиц СИ, в том числе и ампера.
Его величина будет устанавливаться фиксацией численного значения
\emph{элементарного электрического заряда}, которое
планируется принять равным
\[
e = 1,602\,176\,634\;\text{А}\cdot\text{с}\;(точно).
\]
При таком определении электрическая постоянная~$\varepsilon_0$ и, следовательно,
магнитная постоянная~$\mu_0$ станут \emph{измеряемыми} величинами.
\end{lab:note}

\introsubsection{Уравнения электродинамики в произвольной системе единиц\protect\footnote{Если читатель ещё не знаком с системой уравнений Максвелла, этот параграф
можно пропустить.}}

Основными уравнениями электродинамики являются уравнения Максвелла
для электромагнитного поля, а также выражение для силы Лоренца, связывающее
эту теорию с механикой. Запишем эти уравнения для вакуума в произвольной
системе единиц (ограничимся дифференциальной формой):
\begin{align}
\Div \vec{E} & =k_{q}\rho,\label{eq:gauss}\\
\Div \vec{B} & =0,\\
\Rot \vec{E} & =-\frac{1}{k_{B}}\frac{\partial\vec{B}}{\partial t},\label{eq:faradey}\\
\frac{1}{k_B}\Rot \vec{B} & =k_{I}\left(\vec{j}+\frac{1}{k_{q}}\frac{\partial\vec{E}}{\partial t}\right);\label{eq:ampere-dif}
\end{align}
% \begin{comment}
% \begin{align*}
% \oint_{S}\vec{E}\cdot d\vec{S} & =k_{q}q, & \mathop{\mathrm{div}\vec{E}} & =k_{q}\rho,\\
% \oint_{S}\vec{B}\cdot d\vec{S} & =0, & \mathop{\mathrm{div}\vec{B}} & =0,\\
% \oint_{\Gamma}\vec{E}\cdot d\vec{l} & =-\frac{\partial}{\partial t}\int_{S}\frac{\vec{B}}{k_{B}}\cdot d\vec{S}, & \mathop{\mathrm{rot}}\vec{E} & =-\frac{1}{k_{B}}\frac{\partial\vec{B}}{\partial t},\\
% \oint_{\Gamma}\frac{\vec{B}}{k_{B}}\cdot d\vec{l} & =k_{I}\left(I+\frac{\partial}{\partial t}\int_{S}\frac{\vec{E}}{k_{q}}\cdot d\vec{S}\right), & \mathop{\mathrm{rot}}\vec{B} & =k_{B}k_{I}\left(\vec{j}+\frac{1}{k_{q}}\frac{\partial\vec{E}}{\partial t}\right).
% \end{align*}
% \end{comment}
\begin{equation}
\vec{F}_Л=q\left(\vec{E}+\frac{1}{k_{B}}\vec{v}\times\vec{B}\right).\label{eq:lorentz}
\end{equation}

Поясним кратко, почему коэффициенты будут именно такими.

Из уравнения (\ref{eq:gauss}) (\emph{теорема Гаусса}) непосредственно
следует закон Кулона $E=k_{q}\frac{q}{r^{2}}$.

Одинаковые коэффициенты $1/k_{B}$ в уравнениях (\ref{eq:faradey})
(\emph{закон электромагнитной индукции Фарадея}) и (\ref{eq:lorentz})
(\emph{сила Лоренца}) отражают известную связь этих явлений
(см. [1, \S64]).

В постоянном поле ($\partial/\partial t=0$) уравнение (\ref{eq:ampere-dif})
представляет собой \emph{теорему о циркуляции магнитного поля}. Из
неё, в частности, может быть выведен \emph{закон Био--Савара}
 (см. [2, \S3.3]):
\[
\vec{B}=k_{B}k_{I}\oint\frac{\vec{j}\times\vec{r}dV}{4\pi r^{3}}=k_{B}k_{I}\oint\frac{Id\vec{l}\times\vec{r}}{4\pi r^{3}}.
\]
Поле прямого провода равно $B=k_{B}k_{I}\frac{I}{2\pi r}$, что в
совокупности с (\ref{eq:lorentz}) даёт закон Ампера для двух параллельных
токов:
\[
F_{\text{А}}=\frac{1}{k_{B}}I_{1}B_{2}l=k_{I}\frac{2I_{1}I_{2}}{4\pi r}l.
\]

Наконец, второе слагаемое в правой части (\ref{eq:ampere-dif}) (\emph{ток
смещения}) вводится в теорию для того, чтобы выполнялся
\emph{закон сохранения заряда}. Если взять дивергенцию от (\ref{eq:ampere-dif}),
должно получиться \emph{уравнение непрерывности}:
\[
\frac{\partial\rho}{\partial t}+\mathop{\mathrm{div}}\vec{j}=0
\]
(так как $\mathop{\mathrm{div}}\mathop{\mathrm{rot}}\vec{B}\equiv 0$),
что с учётом (\ref{eq:gauss}) и объясняет выбор множителя~$1/k_{q}$.

Если заряды и токи в системе отсутствуют, то система уравнений Максвелла
переходит, как известно, в \emph{волновое уравнение} для электрического
и магнитного полей. В частности,
\[
\frac{\partial^{2}E}{\partial t^{2}}=\frac{k_{q}}{k_{I}}\nabla^{2}\vec{E}.
\]
В вакууме электромагнитные волны распространяются со скоростью $c$,
поэтому, как уже говорилось выше, в любой системе единиц должно выполняться
соотношение (\ref{eq:kikq}): $k_{q}/k_{I}=c^{2}$.

\begin{table}[h!]
    \small
    \renewcommand{\arraystretch}{2}
    \centering
    \begin{tabular}{cccccc}
        \toprule
        & $k_{q}$ & $k_{I}$ & $k_{B}$ & $k_{B}k_{I}$
        & $\dfrac{k_{B}k_{I}}{k_{q}}$\\
        \midrule
        СИ & $\dfrac{1}{\varepsilon_{0}}$ & $\mu_{0}$ & 1 & $\mu_{0}$
        & $\varepsilon_{0}\mu_{0}$\\[1ex] \hline
        СГС (гауссова) & $4\pi$ & $\dfrac{4\pi}{c^{2}}$ & $c$ & $\dfrac{4\pi}{c}$ & $\dfrac{1}{c}$ \\[1ex] \hline
        СГСЭ & $4\pi$ & $\dfrac{4\pi}{c^{2}}$ & 1 & $\dfrac{4\pi}{c^{2}}$
        & $\dfrac{1}{c^{2}}$ \\[1ex] \hline
        СГСМ & $4\pi c^{2}$ & $4\pi$ & 1 & $4\pi$ & $\dfrac{1}{c^{2}}$ \\[1ex] \hline
        МКС (рационализированная) & 1 & $\dfrac{1}{c^{2}}$ & $c$ & $\dfrac{1}{c}$ & $\dfrac{1}{c}$\\
        \bottomrule
    \end{tabular}
    \caption{Коэффициенты в различных системах единиц}
    \tabmark{coeffs}
\end{table}

В таблице~\tabref{coeffs} перечислены возможные комбинации коэффициентов $k_{q}$, $k_{I}=k_{q}/c^{2}$
и $k_{B}$ для наиболее часто встречающихся систем единиц измерения
(также приведены коэффициенты перед током в правой части (\ref{eq:ampere-dif})
$k_{B}k_{I}$ и перед током смещения $k_{B}k_{I}/k_{q}$).
Помимо рассмотренных выше СИ и СГС, здесь для справки представлены системы:
\begin{itemize}
\item СГСЭ --- система c эталонами \emph{сантиметр-грамм-секунда},
    в которой закон Кулона и сила Лоренца не содержат дополнительных постоянных:
    $F_К = q_1q_2/r^2$, $F_Л = q(\vec{E} + \vec{v}\times \vec{B})$.
\item СГСМ --- система c эталонами \emph{сантиметр-грамм-секунда},
в которой закон Ампера и сила Лоренца не содержат дополнительных постоянных:
$F_А = I_1I_2/r^2$, $F_Л = q(\vec{E} + \vec{v}\times \vec{B})$.
\item рационализированная МКС --- предложенная в начале XX в. Лоренцем система
с эталонами \emph{метр-килограмм-секунда}, в которой уравнения
Максвелла содержат единственную константу --- скорость света $c$.
\end{itemize}
Эти системы на сегодня практически полностью вытеснены системами~СИ и
СГС, причём использование последней ограничено в основном
теоретической физикой.


\introsubsection{Перевод между системами единиц}

Установим связи между наиболее часто употребляемыми единицами в системах
СИ и гауссовой СГС. Прежде напомним соответствия между
важнейшими механическими единицами:
\[
1\,\text{Н}=10^{5}\,\frac{\text{г}\cdot\text{см}}{\text{с}^{2}}\equiv10^{5}\,\text{дин},\qquad1\,\text{Дж}=10^{7}\,\text{дин}\cdot\text{см}\equiv10^{7}\,\text{эрг}.
\]


\paragraph{Сила тока}

По определению, провод\'{а} с током $I=1\;\text{А}$ на расстоянии
$r=1\,\text{м}$ взаимодействуют с силой $F/l=2\cdot10^{-7}\,\frac{\text{Н}}{\text{м}}$.
Пользуясь формулой (\ref{eq:col-amp}), находим соответствующий ток
в единицах СГС:
\[
I=c\sqrt{\frac{F}{2l}r}=3\cdot10^{8}\frac{\text{м}}{\text{с}}\cdot\sqrt{10^{-7}\text{Н}}=3\cdot10^{10}\frac{\text{см}}{\text{с}}\cdot\sqrt{10^{-2}\text{дин}}=3\cdot10^{9}\;\text{ед. СГС}.
\]
Таким образом,
\[
1\;\text{А}=3\cdot10^{9}\;\text{ед. СГС},\qquad\text{или}\qquad1\;\text{ед. СГС}=\frac{1}{3}10^{-9}\;\text{А}.
\]

Все единицы измерения абсолютной системы могут быть выражены через
базовые (сантиметр, грамм, секунда). В частности, для тока
\[
1\;\text{ед. СГС}=\small\frac{\text{г}^{1/2}\cdot\text{см}^{3/2}}{\text{с}^{2}}.
\]
Видно, что базовые размерности СГС не очень эстетичны, поэтому, как правило,
пишут просто: <<ед. СГС>>. Иногда для единицы тока используется
название \emph{статампер} (статА, statA).

\paragraph{Заряд}

Поскольку заряд в обеих системах определяется по одной и той же формуле
$dq=Idt$, имеем очевидную связь
\[
1\;\text{Кл}=3\cdot10^{9}\;\text{ед. СГС},\qquad\text{или}\qquad1\;\text{ед. СГС}=\frac{1}{3}10^{-9}\;\text{Кл}.
\]
Элементарный заряд (заряд электрона):
\[
e\approx1{,}6\cdot10^{-19}\;\text{Кл}=4{,}8\cdot10^{-10}\;\text{ед. СГС}.
\]

В иностранной литературе для абсолютной единицы заряда используется
название \emph{франклин}~(Фр, Fr) или \emph{статкулон} (statC).

\paragraph{Потенциал}

В системе СИ разность потенциалов (напряжение)
между точками равно $U[\text{СИ}]=1\,\text{В}$,
если работа $A=qU$ по перемещению заряда 1~Кл равна 1~Дж. Поэтому
\[
1\,\text{В}=1\;\frac{\text{Дж}}{\text{Кл}}=\frac{1}{300}\;\text{ед. СГС},\qquad1\,\text{ед. СГС}=300\,\text{В}.
\]
Абсолютную единицу напряжения называют иногда \emph{статвольт} (статВ, statV).


\paragraph{Магнитное поле}

Единицы измерения магнитных полей (магнитной индукции $B$) в СИ и СГС называются соответственно
\emph{тесла} (Тл) и \emph{гаусс}~(Гс).

Для их связи воспользуемся
законом Ампера: в СИ $F=IBl$, и в СГС $F=\frac{1}{c}IBl$. Тогда
полагая $l=1\;\text{м}$, $I=1\;\text{А}$, $F=1\;\text{Н}$ и, соответственно,
$B=1\;\text{Тл}=1\;\frac{\text{Н}}{\text{А}\cdot\text{м}}$, находим
\[
B=\frac{cF}{Il}=\frac{3\cdot10^{10}\cdot10^{5}}{3\cdot10^{9}\cdot10^{2}}=10^{4}\;\text{ед. СГС}\cdot
\]
Таким образом,
\[
1\;\text{Тл}=10^{4}\;\text{Гс}.
\]

\emph{Напряжённость} магнитного поля~$H$ в вакууме выражается как $H=B/\mu_0$ 
(ед. СИ) и $H=B$ (ед. СГС). Для единиц измерения в СИ имеем: 
$[H] = \frac{Тл}{Н/А^2}=\frac{А}{м}$. В СГС единицы измерения напряжённости
и индукции совпадают, но по историческим причинам имеют разные названия:
$H$ измеряется в \emph{эрстедах} (Э), причём $1\;\text{Э} \equiv 1\;\text{Гс}$.
Связь между единицами в разных системах:
\[
1\;\tfrac{А}{м} = 4\pi \cdot 10^{-7}\cdot 10^4~Э \approx 12,6\cdot 10^{-3}~Э.
\]


\paragraph{Ёмкость}

Определения ёмкости в обеих системах одинаковы\footnote{Но формулы для
ёмкости конденсатора заданной формы разные! Например, для плоского
кодненсатора
\[
C=\frac{\varepsilon_{0}S}{d}\qquad\mathrm{vs.}\qquad C=\frac{S}{4\pi d}.
\]
}: $C=q/U$. Единица СИ измерения ёмкости --- \emph{фарад} (Ф), в
СГС ёмкость выражается в \emph{сантиметрах} (проверьте!):
\[
1\;\text{Ф}=1\;\frac{\text{Кл}}{\text{В}}=\frac{3\cdot10^{9}}{1/300}=9\cdot10^{11}\;\text{см},\qquad1\;\text{см}\approx1{,}1\;\text{пФ}.
\]


\paragraph{Индуктивность}

Индуктивность --- коэффициент пропорциональности между током $I$
в контуре и потоком магнитного поля $\Phi$ через него. В СИ и СГС
индуктивность определяется \emph{по-разному}(!):
\[
\text{СИ}:\quad\Phi=LI,\qquad\text{СГС}:\quad\Phi=\frac{1}{c}LI.
\]
Единица измерения в СИ называется \emph{генри} (Гн). Задавая, как
обычно, единичные значения величин в системе СИ, пересчитаем их в
СГС:
\[
L=\frac{c\Phi}{I}=\frac{3\cdot10^{10}\frac{\text{см}}{\text{с}}\cdot1\;\text{Тл}\cdot\text{м}^{2}}{1\;\text{А}}=\frac{3\cdot10^{10}\cdot10^{4}\cdot10^{4}}{3\cdot10^{9}}=10^{9}\;\text{ед. СГС}.
\]
Нетрудно проверить (проверьте), что индуктивности в СГС также имеет
размерность длины:
\[
1\;\text{Гн}=10^{9}\;\text{см},\qquad1\;\text{см}=1\;\text{нГн}.
\]

Связь между другими часто встречающимися величинами и основные
электродинамические формулы в системах СИ и СГС можно найти в
таблицах~2 и~3 Приложения к сборнику (стр.~??).

\textbf{Упражнение.} Избавьтесь от <<лишней>> переводной
константы $G=6{,}67\cdot10^{-11}\;\frac{\text{м}^{3}}{\text{кг}\cdot\text{с}^{2}}$
в законе всемирного тяготения $F=G\frac{m_{1}m_{2}}{r^{2}}$ и постройте
абсолютную единицу измерения массы, основанную только на эталонах
длины и времени.

\begin{lab:literature}
    \item \SivuhinIII~--- \S~85.
    \item \Kirichenko~--- стр.~407--413.
    \item *\textit{Каршенбойм С.Г.} Фундаментальные физические константы: роль
    в физике и метрологии и рекомендованные значения // \emph{УФН}.~--- 2005.~---
    Т.~75, \textnumero~3.~--- С.~271.
\end{lab:literature}



% Абсолютная система единиц в механике
%
% Посмотрим на закон всемирного тяготения Ньютона:
% \[
% F=G\frac{m_{1}m_{2}}{r^{2}}.
% \]
% Какой физический смысл имеет гравитационная постоянная $G=6{,}67\cdot10^{-11}\;\frac{\text{м}^{3}}{\text{кг}\cdot\text{с}^{2}}$?
% Если сравнить эту формулу с тем, что мы проделали ранее для электродинамики,
% то нетрудно понять, что $G$ --- это \guillemotleft лишняя\guillemotright{}
% переводная константа, возникшая в результате введения \emph{искусственного}
% эталона массы. Действительно, эталон килограмма --- это просто болванка,
% размер и материал которой был выбран \emph{совершенно произвольно}.
%
% Закон Ньютона позволяет ввести абсолютный эталон массы. Положим
% \begin{equation}
% F=\frac{m_{1}m_{2}}{4\pi r^{2}}.\label{eq:newton}
% \end{equation}
% С учётом 2-го закона $F=ma$, эталон масс можно было бы определить
% так: два одинаковых тела имеют единичную массу --- назовём её \guillemotleft абсолютным
% килограммом\guillemotright{} (аКг), --- если находясь на расстоянии
% 1~м они испытывают ускорение $1\;\text{м}/\text{с}^{2}$.
%
% В таком случае масса имеет размерность:
% \[
% \left[m\right]=\frac{L^{3}}{T}=\frac{\text{м}^{3}}{\text{с}^{2}}.
% \]
%
% Свяжем привычные килограммы с \guillemotleft абсолютными\guillemotright .
% При $m=1\;\text{кг}$ и $r=1\;\text{м}$ имеем ускорение $a=6,67\cdot10^{-11}\;\text{м}/\text{с}^{2}$.
% Поэтому
% \[
% m\left[\text{абс.}\right]=4\pi ar^{2}\approx8{,}4\cdot10^{-10}\;\frac{\text{м}^{3}}{\text{с}^{2}}.
% \]
% То есть
% \[
% 1\;\text{кг}=8{,}4\cdot10^{-10}\;\frac{\text{м}^{3}}{\text{с}^{2}},\qquad1\;\text{аКг}\approx1{,}19\cdot10^{9}\;\text{кг}.
% \]
%
% Почему бы не провести \emph{такую} рационализацию? Конечно, $10^{9}$
% кг --- это не очень удобно для повседневных нужд. Но это не серьезный
% аргумент --- новую единицу можно назвать \guillemotleft абсолютным
% гигакилограммом\guillemotright{} (или тераграммом), тогда привычные
% для нас веса будут по-прежнему измеряться в \guillemotleft килограммах\guillemotright{}
% (но зато абсолютных!).
%
% Гораздо более серьезным возражением должен служить тот факт, что изготовить
% надёжные эталоны и методику измерения массы с использованием формулы
% (\ref{eq:newton}) \emph{весьма} затруднительно, ввиду крайней малости
% гравитационных сил для тел \guillemotleft разумных\guillemotright{}
% размеров. Гораздо проще вырезать произвольную болванку и пытаться
% всеми силами не давать ей испаряться или адсорбировать вещество из
% окружающей среды!
