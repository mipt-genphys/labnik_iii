% !TeX spellcheck = russian-aot
% !TeX encoding = UTF-8
\section{Единицы измерения в электричестве}

\subsection{Измеряемые величины}

Наряду с уже знакомыми по курсу механики и термодинамики величинами: массой, расстоянием, временем и температурой, в электростатике и электродинамике возникает еще одна одна фундаментальная величина -- электрический заряд, который традиционно обозначается символом $q$. Определением электрического заряда можно считать закон Кулона:

\begin{equation}
	\eqmark
	\vec{F} = k \frac{q_1 q_2}{r^3}\vec{r},
\end{equation}
где $q_1$ и $q_2$ - величины взаимодействующих зарядов, $r$ - расстояние между ними, а $k$ - некоторая (в общем случае -- размерная) константа, которая зависит от системы единиц. Так как понятия силы и расстояния уже определены в механике, это соотношение позволяет определить величину заряда при любой заданной константе $k$ или, наоборот, установить константу для любого определения заряда. В классической теории электричества конкретный выбор единиц ничем не ограничен и определяется исключительно удобством использования. Исследования, которые были проведены в конце XIX и начале XX вв., показали, что электрический заряд в действительности имеет дискретную природу: все наблюдаемые в природе заряды кратны заряду электрона. В связи с этим очевидным решением было бы измерять все заряды в зарядах электрона. Однако, заряд электрона настолько мал по сравнению с зарядами, встречающимися в повседневной жизни, что пришлось бы постоянно работать с очень большими числами, что не удобно.

Важная производная величина -- напряженность электрического поля $\vec{E}$. По определению напряженность -- это электростатическая сила, действующая на пробный заряд и отнесенная к величине этого заряда: $\vec{E} = \vec{F}/q$. Пробным считается заряд такой величины, что его присутствие не меняет пространственного расположения других зарядов в системе. Несмотря на то, что в чисто механическом смысле напряженность - производная от силы характеристика, в электричестве эта величина применяется более широко, чем сила.

Кулоновская сила потенциальна, как следствие, можно определить потенциальную энергию и, что более важно, потенциал (потенциальную энергию, отнесенную к величине заряда): $\varphi = \Pi/q$. Потенциал определяется с точностью до некоторой константы, которая зависит от системы отсчета, поэтому физически измеряемая величина -- разность потенциалов между двумя точками. Когда говорят о потенциале отдельно взятой точки пространства, подразумевают, что потенциал бесконечно удаленной точки равен нулю. Другими словами, \textbf{потенциал точки пространства -- это разность потенциалов между этой точкой и бесконечно удаленной точкой}. Но такое определение не всегда имеет смысл. К примеру, при расчете потенциала бесконечной плоскости напряженность поля на бесконечности не будет равна нулю.

При переходе к электродинамике важную роль начинают играть скорости движения зарядов. Для характеристики этих скоростей вводят понятие плотности тока -- количества заряда, проходящего через площадку $\sigma$ за единицу времени:
\begin{equation}
	\eqmark
	j = \frac{dq}{\sigma dt}.
\end{equation}

Для проводников конечного размера говорят также о токе -- заряде, протекшем через сечение проводника за единицу времени:
\begin{equation}
	\eqmark[current]
	I = \frac{dq}{dt}.
\end{equation}

Опытным путем установлено, что движущиеся заряды, или токи, взаимодействуют между собой (такое взаимодействие называют магнитным). Сила, действующая на движущийся заряд (сила Лоренца), в общем виде выражается следующим образом:
\begin{equation}
	\vec{F} = q \left( \vec{E} + \vec{v} \times \vec{B} \right).
\end{equation}

Таким образом вводится величина $\vec{B}$, которая называется индукцией магнитного поля. Данное выражение можно считать определением вектора индукции. Величину этого вектора можно найти по закону Био — Савара — Лапласа:
\begin{equation}
	\vec{B} = k_{\mu}\int_{V}{\frac{\vec{j} \times \vec{r} dV}{r^3}},
\end{equation}
где $k_{\mu}$ -- размерная константа, зависящая от системы единиц.

\subsection{Системы единиц}

Рассмотрим вопрос о системах единиц в электродинамике. Законы макроскопической электродинамики определяются ее фундаментальными аксиомами -- уравнениями Максвелла, концентрированным обобщением экспериментальных фактов из области электричества и магнетизма. Запишем уравнения Максвелла для вакуума в произвольной системе единиц:

%\def\urtab#1#2#3{%
%	kip-1ex
%	\be#3
%	\begin{array}{ll}
%	\hbox to 0.5\textwidth{$\ds #1$\hfil}&\hbox to 0.25\textwidth{$\ds #2$\hfil}
%	\end{array}
%	\ee
%	kip-1ex
%}


\begin{gather}
	\eqmark[maxwell]
	\begin{aligned}
		\oint_S \vec{E}d\vec{S} &= \alpha\int_V\rho\,dV, & \Div \vec{E} &= \alpha\rho,\\
		\oint_S\vec{B}d\vec{S} &= 0, & \Div\vec{B} &= 0\\
		\oint_L\vec{E}d\vec{l} &= -\beta\int_S\frac{\partial{\vec{B}}} {\partial t}\,d\vec{S}, & \Rot\vec{E} &= -\beta\frac{\partial\vec B}{\partial t},\\
		\oint_L\vec{B}d\vec{l} &= \gamma\int_S\vec{j}\,d\vec{S}+\delta\int_S\frac{\partial{\vec{E}}}{\partial t}\,d\vec{S}, & \Rot\vec{B} &= \gamma\vec{j}+\delta\frac{\partial\vec E}{\partial t};
	\end{aligned}
\end{gather}

Здесь приняты стандартные обозначения. Множество коэффициентов ($\alpha$, $\beta$, $\gamma$, $\delta$, $\xi$, $\eta$) свидетельствует о том, что для
каждой физической величины, входящей в систему уравнений \eqref{maxwell}, принята собственная единица измерения,
независимая от единиц других величин.


В таблице \tabref{systems} показано, как коэффициенты, введенные в \eqref{maxwell} выглядят в различных системах единиц. В
настоящее время принято считать, что $c=299\,792\,458$~м/с (точно). Это означает, что базисные единицы <<привязаны>> к
этой величине. В рамках данного лабораторного практикума это значение можно округлить: $c \approx 3 \cdot 10^8$~м/с.

\def\pp{\textbf}

\begin{table}[h!]
	\centering
	\caption{Некоторые системы единиц, используемые при изучении макроскопической электродинамики}
	\tabmark[systems]
	\begin{tabular}{|l|c|c|c|c|c|c|c|c|c|}
		\hline
		     &  \pp{$\alpha$}   & \pp{$\beta$}  &    \pp{$\gamma$}    &  \pp{$\delta$}  & \pp{$\xi$} &  \pp{$\eta$}  & \pp{$\frac{\alpha\delta}{\gamma}$} & \pp{$\frac{\xi\beta}{\eta}$} & \pp{$\delta\beta$} \\[2ex] \hline
		СГСЭ &      $4\pi$      &       1       & $\frac{4\pi}{c^2}$  & $\frac{1}{c^2}$ &     1      &       1       &                 1                  &              1               &  $\frac{1}{c^2}$   \\[2ex] \hline
		СГСМ &    $4\pi c^2$    &       1       &       $4\pi$        & $\frac{1}{c^2}$ &     1      &       1       &                 1                  &              1               &  $\frac{1}{c^2}$   \\[2ex] \hline
		СГС  &      $4\pi$      & $\frac{1}{c}$ &  $\frac{4\pi}{c}$   &  $\frac{1}{c}$  &     1      & $\frac{1}{c}$ &                 1                  &              1               &  $\frac{1}{c^2}$   \\[2ex] \hline
		СИ   & $\frac{1}{\varepsilon_0}$ &       1       & $\frac{1}{\varepsilon_0c^2}$ & $\frac{1}{c^2}$ &     1      &       1       &                 1                  &              1               &  $\frac{1}{c^2}$   \\[2ex] \hline
		МКС  &        1         &       1       &   $\frac{1}{c^2}$   & $\frac{1}{c^2}$ &     1      &       1       &                 1                  &              1               &  $\frac{1}{c^2}$   \\[2ex] \hline
		\multicolumn{10}{|l|}{\hskip 1cm $c=299\,792\,458$~м/с;}                                                                                                                                 \\[2ex]
		\multicolumn{10}{|l|}{\hskip 1cm $\varepsilon_0=8,854\cdot10^{-12}$~Ф/м;~~~$\mu_0=\frac{1}{\varepsilon_0c^2}=4\pi\cdot10^{-7}$~Гн/м.}                                                                                     \\[2ex] \hline
	\end{tabular}
\end{table}


В электродинамике почти всегда используется Международная система единиц (СИ), поэтому ток измеряется в амперах, а разность потенциалов -- в вольтах. Причины выбора СИ довольно очевидны: ток в 1 ампер довольно характерен для радиотехники, в то время как единицы СГС на много порядков меньше. В электростатике, напротив, для рассчетов наиболее удобна система СГС.
