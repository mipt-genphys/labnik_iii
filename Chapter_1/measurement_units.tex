% !TeX spellcheck = russian-aot
% !TeX encoding = UTF-8
\labsection{Единицы измерения в электричестве}

%\labsection{Измеряемые величины}

Наряду с уже знакомыми по курсу механики и термодинамики величинами: массой, расстоянием, временем и температурой, в электростатике и электродинамике возникает еще одна одна фундаментальная величина~--- электрический заряд, который традиционно обозначается символом $q$. Определением электрического заряда можно считать закон Кулона:

\begin{equation}
	\vec{F} = k \frac{q_1 q_2}{r^3}\vec{r},
\end{equation}
где $q_1$ и $q_2$~--- величины взаимодействующих зарядов, $r$~--- расстояние между ними, а $k$~--- некоторая (в общем случае~--- размерная) константа, которая зависит от системы единиц. Так как понятия силы и расстояния уже определены в механике, это соотношение позволяет определить величину заряда при любой заданной константе $k$ или, наоборот, установить константу для любого определения заряда. В классической теории электричества конкретный выбор единиц ничем не ограничен и определяется исключительно удобством использования. Исследования, которые были проведены в конце XIX и начале XX вв., показали, что электрический заряд в действительности имеет дискретную природу: все наблюдаемые в природе заряды кратны заряду электрона. В связи с этим очевидным решением было бы измерять все заряды в зарядах электрона. Однако, заряд электрона настолько мал по сравнению с зарядами, встречающимися в повседневной жизни, что пришлось бы постоянно работать с очень большими числами, что не удобно.

Важная производная величина -- напряженность электрического поля $\vec{E}$. По определению напряженность~--- это электростатическая сила, действующая на пробный заряд и отнесенная к величине этого заряда: $\vec{E} = \vec{F}/q$. Пробным считается заряд такой величины, что его присутствие не меняет пространственного расположения других зарядов в системе. Несмотря на то, что в чисто механическом смысле напряженность~--- производная от силы характеристика, в электричестве эта величина применяется более широко, чем сила.

Кулоновская сила потенциальна, как следствие, можно определить потенциальную энергию и, что более важно, потенциал (потенциальную энергию, отнесенную к величине заряда): $\varphi = \Pi/q$. Потенциал определяется с точностью до некоторой константы, которая зависит от системы отсчета, поэтому физически измеряемая величина~--- разность потенциалов между двумя точками. Когда говорят о потенциале отдельно взятой точки пространства, подразумевают, что потенциал бесконечно удаленной точки равен нулю. Другими словами, \important{потенциал точки пространства~--- это разность потенциалов между этой точкой и бесконечно удаленной точкой}. Но такое определение не всегда имеет смысл. К примеру, при расчете потенциала бесконечной плоскости напряженность поля на бесконечности не будет равна нулю.

При переходе к электродинамике важную роль начинают играть скорости движения зарядов. Для характеристики этих скоростей вводят понятие плотности тока~--- количества заряда, проходящего через площадку $\sigma$ за единицу времени:
\begin{equation}
	j = \frac{dq}{\sigma dt}.
\end{equation}

Для проводников конечного размера говорят также о токе~--- заряде, протекшем через сечение проводника за единицу времени:
\begin{equation}
	\eqmark{current}
	I = \frac{dq}{dt}.
\end{equation}

Опытным путем установлено, что движущиеся заряды, или токи, взаимодействуют между собой (такое взаимодействие называют магнитным). Сила, действующая на движущийся заряд (сила Лоренца), в общем виде выражается следующим образом:
\begin{equation}
	\vec{F} = q \left( \vec{E} + \vec{v} \times \vec{B} \right).
\end{equation}

Таким образом вводится величина $\vec{B}$, которая называется индукцией магнитного поля. Данное выражение можно считать определением вектора индукции. Величину этого вектора можно найти по закону Био~-~Савара~-~Лапласа:
\begin{equation}
	\vec{B} = k_{\mu}\int_{V}{\frac{\vec{j} \times \vec{r} dV}{r^3}},
\end{equation}
где $k_{\mu}$~--- размерная константа, зависящая от системы единиц.

\labsection{Системы единиц}

В электродинамике почти всегда используется Международная система единиц (СИ), поэтому ток измеряется в амперах, а разность потенциалов~--- в вольтах. Причины выбора СИ довольно очевидны: ток в 1 ампер довольно характерен для радиотехники, в то время как единицы СГС на много порядков меньше. В электростатике, напротив, для расчетов наиболее удобна система СГС. Более подробное описание различных система единиц и историческая справка об их создании приведена в приложении \ref{sec:app_units}.

\labsubsection{СИ}
Система уравнений Максвелла в международной системе единиц СИ выглядит следующим образом:
\begin{gather}
	\eqmark{maxwell-si}
	\begin{aligned}
		\oint_S\vec{D}d\vec{S} &= \int_V\rho\,dV, 																	& \Div\vec{E} 				&= \rho,												\\
		\oint_S\vec{B}d\vec{S} &= 0, 																				& \Div\vec{B}				&= 0,													\\
		\oint_L\vec{E}d\vec{l} &= -\int_S\frac{\partial{\vec{B}}} {\partial t}\,d\vec{S}, 							& \Rot\vec{E} 	   			&= -\frac{\partial\vec B}{\partial t},					\\
		\oint_L\vec{H}d\vec{l} &= \int_S\vec{j}\,d\vec{S} + \int_S\frac{\partial{\vec{E}}}{\partial t}\,d\vec{S}, 	& \Rot\vec{H} 				&= \vec{j} + \frac{\partial\vec E}{\partial t}.
	\end{aligned}
\end{gather}

При этом не следует забывать, что в системе СИ Индукция электрического поля связана с напряженностью этого поля соотношением:
\begin{equation}
	\vec{D} = \varepsilon_0 \vec{E},~где~\varepsilon_0 = 8,854\cdot10^{-12}~\mathtext{Ф/м}
\end{equation}
и
\begin{equation}
	\vec{B} = \mu_0 \vec{H},~где~\mu_0 = \frac{1}{\varepsilon_0c^2}=4\pi\cdot10^{-7}~\mathtext{Гн/м} = 1.256\cdot10^{-6}~\mathtext{Гн/м}
\end{equation}

При этом также соблюдается соотношение
\begin{equation}
	\varepsilon_0 \mu_0 = \frac{1}{c^2},
\end{equation}
где постоянная $c=299\,792\,458$~м/с - константа, равная скорости света в вакууме.

\labsubsection{СГС}
Система уравнений Максвелла в системе единиц СГС выглядит следующим образом:
\begin{gather}
	\eqmark{maxwell-sgs}
	\begin{aligned}
		\oint_S\vec{D}d\vec{S} &= 4\pi \int_V\rho\,dV, 																& \Div\vec{E} 				&= 4\pi \rho,											\\
		\oint_S\vec{B}d\vec{S} &= 0, 																				& \Div\vec{B}				&= 0,													\\
		\oint_L\vec{E}d\vec{l} &= -\frac{1}{c}\int_S\frac{\partial{\vec{B}}} {\partial t}\,d\vec{S}, 				& \Rot\vec{E} 	   			&= -\frac{1}{c}\frac{\partial\vec B}{\partial t},		\\
		\oint_L\vec{H}d\vec{l} &= \frac{4\pi}{c}\int_S\vec{j}\,d\vec{S} + \frac{1}{c}\int_S\frac{\partial{\vec{D}}}{\partial t}\,d\vec{S}, 	
				& \Rot\vec{H} 				&= \frac{4\pi}{c}\vec{j} + \frac{1}{c}\frac{\partial\vec D}{\partial t};
	\end{aligned}
\end{gather}