% !TeX spellcheck = russian-aot
% !TeX encoding = UTF-8
\labsection{Единицы измерения в электричестве}

%\labsection{Измеряемые величины}

Наряду с уже знакомыми по курсу механики и термодинамики величинами: массой,
расстоянием, временем и температурой, в электростатике и электродинамике
возникает еще одна одна фундаментальная величина~--- электрический заряд,
который традиционно обозначается символом $q$. Определением электрического
заряда можно считать закон Кулона:
\begin{equation}
	\vec{F} = k \frac{q_1 q_2}{r^3}\vec{r},
\end{equation}
где $q_1$ и $q_2$~--- величины взаимодействующих зарядов, $r$~--- расстояние
между ними, а $k$~--- некоторая (в общем случае~--- размерная) константа,
которая зависит от выбора системы единиц. Так как понятия силы и расстояния уже
определены в механике, это соотношение позволяет определить величину заряда при
любой заданной константе~$k$ или, наоборот, установить константу для любого
определения заряда. В~классической теории электричества конкретный выбор единиц
% ничем не ограничен и
определяется исключительно удобством использования.
Исследования, проводимые в конце~XIX и начале~XX вв., показали, что
электрический заряд имеет дискретную природу: все наблюдаемые
в природе заряды кратны заряду электрона. В~связи с этим, естесственно было
бы измерять все заряды в зарядах электрона. Однако, заряд электрона настолько
мал по сравнению с зарядами, встречающимися в повседневной жизни, что пришлось
бы постоянно работать с очень большими числами, что не удобно.

Важная производная величина~--- напряженность электрического поля $\vec{E}$. По
определению напряженность~--- это электростатическая сила, действующая на
пробный заряд и отнесенная к величине этого заряда: $\vec{E} = \vec{F}/q$.
Пробным считается заряд такой величины, что его присутствие не меняет
пространственного расположения других зарядов в системе. Несмотря на то, что в
чисто механическом смысле напряженность~--- производная от силы характеристика,
в электричестве эта величина применяется более широко, чем сила.

Кулоновская сила потенциальна, как следствие, можно определить потенциальную
энергию и, что более важно, потенциал (потенциальную энергию, отнесенную к
величине заряда): $\varphi = \Pi/q$. Потенциал определяется с точностью до
некоторой константы, которая зависит от системы отсчета, поэтому физически
измеряемая величина~--- разность потенциалов между двумя точками. Когда говорят
о потенциале отдельно взятой точки пространства, подразумевают, что потенциал
бесконечно удаленной точки равен нулю. Другими словами, \important{потенциал
точки пространства~--- это разность потенциалов между этой точкой и бесконечно
удаленной точкой}. Но такое определение не всегда имеет смысл. К примеру, при
расчете потенциала бесконечной плоскости напряженность поля на бесконечности не
будет равна нулю.

При переходе к электродинамике важную роль начинают играть скорости движения
зарядов. Для характеристики этих скоростей вводят понятие плотности тока~---
количества заряда, проходящего через площадку $\sigma$ за единицу времени:
\begin{equation}
	j = \frac{dq}{\sigma dt}.
\end{equation}
\todo[inline,author=Popov]{Нужно ли тут определять понятие плотности тока?}
Для проводников конечного размера говорят также о токе~--- заряде, протекшем
через сечение проводника за единицу времени:
\begin{equation}
	\eqmark{current}
	I = \frac{dq}{dt}.
\end{equation}

Опытным путем установлено, что движущиеся заряды, или токи, взаимодействуют
между собой (такое взаимодействие называют магнитным). Сила, действующая на
движущийся заряд (сила Лоренца), в общем виде выражается следующим образом:
\begin{equation}
	\vec{F} = q \left( \vec{E} + \vec{v} \times \vec{B} \right).
\end{equation}
\todo[inline,author=Popov]{Это \emph{не общее} выражение для силы Лоренца.
В СГС $F = q (E+v/c\times B)$}

Таким образом вводится величина $\vec{B}$, которая называется индукцией
магнитного поля. Данное выражение можно считать определением вектора индукции.
Величину этого вектора можно найти по закону Био--Савара--Лапласа:
\begin{equation}
	\vec{B} = k_{\mu}\int_{V}{\frac{\vec{j} \times \vec{r} dV}{r^3}},
\end{equation}
где $k_{\mu}$~--- размерная константа, зависящая от системы единиц.
\todo[inline,author=Popov]{Чтобы не вводить отдельно плотность тока,
которая смотрится здесь несколько неуместно, лучше вместо закона
Био-Савара-Лапласа базироваться на законе Ампера. Как это собственно
и делается в системе СИ при задании эталона ``ампера''}

\labsection{Системы единиц}

В электродинамике почти всегда используется Международная система единиц (СИ),
поэтому ток измеряется в амперах, а разность потенциалов~--- в вольтах.
\todo[inline,author=Popov]{А что такое ``ампер'' и что такое ``вольт''?
Стоит их сначала определить.}
Причины
выбора СИ довольно очевидны: ток в 1 ампер довольно характерен для радиотехники,
в то время как единицы СГС на много порядков меньше. В электростатике, напротив,
для расчетов наиболее удобна система СГС.
\todo[inline,author=Popov]{Это не соответствует действительности. СГС
прекрасно используется в электродинамике (например, в плазме). Разница скорее в том, что
СИ используется в технике повсеместно, а сегодня СГС используют только теоретики}
Более подробное описание различных
система единиц и историческая справка об их создании приведена в приложении
\ref{sec:app_units}.

\labsubsection{СИ}
Система уравнений Максвелла в международной системе единиц СИ выглядит следующим
образом (слева интегральная, справа дифференциальная формы):
\begin{equation}
	\eqmark{maxwell-si}
	\begin{aligned}
		\oint\limits_S\vec{D}\cdot d\vec{S} &= \int\limits_V\rho\,dV,
								& \Div\vec{E} 				&= \rho,
								\\
		\oint\limits_S\vec{B}\cdot d\vec{S} &= 0,
								& \Div\vec{B}				&= 0,
								\\
		\oint\limits_L\vec{E}\cdot d\vec{l} &=
		-\int\limits_S\frac{\partial{\vec{B}}} {\partial t}\cdot d\vec{S}, 							& \Rot\vec{E} 	   			&=
-\frac{\partial\vec B}{\partial t}, \\
		\oint\limits_L\vec{H}\cdot d\vec{l} &=
		\int\limits_S\vec{j}\cdot d\vec{S} +
\int\limits_S\frac{\partial{\vec{E}}}{\partial t}\cdot d\vec{S}, 	& \Rot\vec{H}
&= \vec{j} + \frac{\partial\vec E}{\partial t}.
	\end{aligned}
\end{equation}
\todo[inline,color=green,author=Popov]{Вставка ->}
В системе СИ индукция электрического поля
связана с напряженностью этого поля соотношением
\begin{equation}
	\vec{D} = \varepsilon_0 \vec{E},
\end{equation}
где $\varepsilon_0 \approx 8,85\cdot10^{-12}\;\text{Ф/м}$ ---
\term{электрическая постоянная}. Индукция и напряжённость магнитного поля
связаны
\begin{equation}
	\vec{B} = \mu_0 \vec{H},
\end{equation}
где $\mu_0 \equiv 4\pi\cdot10^{-7}\;\text{Гн/м} \approx
1,26\cdot10^{-6}\;\text{Гн/м}$ --- \term{магнитная постоянная}
При этом соблюдается соотношение
\begin{equation}
	\varepsilon_0 \mu_0 = \frac{1}{c^2},
\end{equation}
где~$c\equiv 299\,792\,458$~м/с~--- \term{электродинамическая постоянная},
совпадающая со \term{скоростью света в вакууме}.

Скорость света в современной системе СИ считается \emph{фундаментальной} константой,
заданной \emph{точно}. Величины $\varepsilon_0$ и $\mu_0$ являются
\emph{вспомогательными} переводными константами между механическими и электродинамическими
единицами и также считаются заданными \emph{точно}.
\todo[inline,color=green]{<-}

\labsubsection{СГС}
Система уравнений Максвелла в системе единиц СГС выглядит следующим образом:
\begin{equation}
	\eqmark{maxwell-sgs}
	    \begin{aligned}
        \oint\limits_S\vec{D}\cdot d\vec{S} &= 4\pi \int\limits_V\rho\,dV,
        & \Div\vec{E} &= 4\pi \rho, \\
        \oint\limits_S\vec{B}\cdot d\vec{S} &= 0,
        & \Div\vec{B} &= 0, \\
        \oint\limits_L\vec{E}\cdot d\vec{l} &=
        - \frac{1}{c} \int\limits_S\frac{\partial{\vec{B}}} {\partial t}\cdot d\vec{S},
        & \Rot\vec{E} &= -\frac{1}{c}\frac{\partial\vec B}{\partial t}, \\
        \oint\limits_L\vec{H}\cdot d\vec{l} &=
        \frac{4\pi}{c}\int\limits_S\vec{j}\cdot d\vec{S} +
        \frac{1}{c}\int\limits_S\frac{\partial{\vec{E}}}{\partial t}\cdot d\vec{S},
        & \Rot\vec{H} &= \frac{4\pi}{c}\vec{j} + \frac{1}{c}\frac{\partial\vec E}{\partial t}.
    \end{aligned}
\end{equation}
\todo[inline,author=Popov]{Сюда стоило бы добавить как минимум выражение
для силы Лоренца в СГС, и может быть какие-то комментарии... Хотя бы,
что размерности E,B,H,D одинаковы}
