\lab{Магнитометр}

\begin{lab:aim}
    определить горизонтальную составляющую магнитного поля Земли и установить
количественное соотношение между единицами электрического тока в системах СИ и
СГС.
\end{lab:aim}

\begin{lab:equipment}
    магнитометр, осветитель со шкалой, источник питания, вольтметр,
электромагнитный переключатель, конденсатор,намагниченный стержень, прибор для
определения периода крутильных колебаний, секундомер, рулетка, штангенциркуль.
\end{lab:equipment}

Магнитометром называют прибор для магнитных измерений, например компас,
теодолит, веберметр и пр. С помощью
магнитометров измеряют намагниченность ферромагнетиков, напряжённость магнитных
полей, исследуют магнитные аномалии.
Разработаны магнитометры различных конструкций: магнитостатические,
электромагнитные, магнитодинамические, индукционные,
резонансные. Эталонные магнитометры позволяют измерять горизонтальную и
вертикальную составляющие напряжённости
магнитного поля Земли с точностью~$10^{-6}$~Э ($1~\text{Э}=79,6~$~А/м).

%\rpic{5.0cm}{1_1_1}{Схема магнитометра}{fig:magnitometr}
\begin{wrapfigure}{r}{0.4\textwidth}
	\pic{0.38\textwidth}{Chapter_1/1_1_1}
	\caption{Схема магнитометра}
	\figmark{magnitometr}
\end{wrapfigure}

В нашей установке с помощью электромагнитного магнитометра измеряется
горизонтальная составляющая земного магнитного
поля и абсолютным образом определяется сила тока по его магнитному действию.

\experiment

Магнитометр (рис.~\figref{magnitometr}) состоит из нескольких
последовательно соединённых круговых витков~$K$, расположенных
вертикально. В центре кольца~$K$ радиусом $R$
на тонкой неупругой вертикальной нити подвешена
короткая магнитная стрелка~С. Жёстко
связанная со стрелкой крыльчатка погружена в масло и служит для демпфирования
колебаний.

В отсутствие других магнитных полей стрелка располагается по направлению
горизонтальной составляющей земного магнитного
поля $\vec{B}_0$, т.е. лежит в плоскости магнитного меридиана.

Прибор настраивают с помощью световых зайчиков, отражённых от двух зеркал:
$З_1$, прикреплённого к стрелке (подвижный
зайчик), и $З_2$, расположенного в плоскости кольца~К и жёстко связанного с ним
(неподвижный зайчик). Оба зеркала
освещаются одним и тем же осветителем~О. Вращением кольца вокруг вертикальной
оси можно совместить оба зайчика. При этом
плоскость витков совпадает с плоскостью магнитного меридиана.

При появлении дополнительного горизонтального магнитного поля $\vec{B}_{\perp}$
стрелка C установится по равнодействующей
обоих полей $\vec{B}_{\Sigma}$ (рис.~\figref{magnitometr-measure}). В нашей
установке дополнительное поле может быть создано либо малым ферромагнитным
стержнем, расположенным на кольце на его горизонтальном диаметре ($\vec{B}_1$),
либо током, проходящим по кольцу~($\vec{B}_2$).
В~обоих случаях дополнительное поле можно считать однородным,
т.\,к. размеры стрелки много меньше радиуса кольца.

\begin{figure}
\centering
	\pic{0.5\textwidth}{Chapter_1/1_1_2}
	\caption{Схема измерения угла отклонения магнитной стрелки}
	\figmark{magnitometr-measure}
\end{figure}

Поле намагниченного стержня вдали от него может быть приближённо
вычислено как поле точечного диполя:
\[
 \vec B(\vec{r}) = \frac{\mu_0}{4\pi}
 \left(3\frac{(\vec{\mathfrak{m}} \cdot \vec{r})\vec{r}}{r^5}
 - \frac{\vec{\mathfrak{m}}}{r^3}\right),
\]
где $\vec{\mathfrak{m}}$~--- магнитный момент стержня,
$\vec{r}$ --- радиус-вектор, проведённый из центра диполя в точку наблюдения.
На оси, перпендикулярной стержню, имеем
\begin{equation}
	\eqmark{1}
    B_1=\frac{\mu_0}{4\pi}\frac{\mathfrak{m}}{R^3},
\end{equation}
где $R$~--- радиус кольца.

Магнитное поле в центре кольца с током $I$ по закону Био и Савара равно
\begin{equation}
	\eqmark{2}
    B_2=\frac{\mu_0 I}{2R}N.
\end{equation}
Здесь $N$~--- число витков в кольце, $I$~--- сила тока в~единицах СИ (амперах).

Измерив угол отклонения стрелки $\varphi$, можно связать поля~$B_0$
и~$B_{\perp}$ ($B_1$ или $B_2$):
\begin{equation}
	\eqmark{3}
    B_{\perp}=B_0\cdot \tg{\varphi}.
\end{equation}


\labsection{Определение горизонтальной составляющей магнитного поля Земли}

Для определения горизонтальной составляющей земного магнитного поля~$B_0$ тонкий
короткий намагниченный стержень
устанавливается в отверстие Р на горизонтальном диаметре кольца
(рис.~\figref{magnitometr}). Измерив тангенс угла отклонения стрелки
\begin{equation}
	\eqmark{4}
    \tg\varphi_1=\frac{x_1}{2L},
\end{equation}
можно с помощью уравнений \eqref{1}, \eqref{3} и \eqref{4} рассчитать
поле~$B_0$, если исключить величину~$\mathfrak{m}$~--- магнитный
момент стержня.

Для исключения магнитного момента предлагается измерить
период крутильных колебаний стержня в поле Земли. Подвешенный горизонтально за
середину на тонкой длинной нити стержень в положении равновесия установится по
полю Земли (упругость нити пренебрежимо
мала). Если ось стержня отклонить в горизонтальной плоскости от
направления~$B_0$ на малый угол~$\alpha$, то под
действием возвращающего механического момента
\begin{equation*}
    M_\text{мех}=|\vec{\mathfrak{m}}\times \vec{B}|
    = \mathfrak{m}B_0\sin\alpha\approx \mathfrak{m}B_0\alpha
\end{equation*}
стержень с моментом инерции $J$ в~соответствии с уравнением
\begin{equation*}
    J\ddot{\alpha}+\mathfrak{m}\;B_0\;\alpha=0
\end{equation*}
будет совершать крутильные колебания c периодом
\begin{equation}
	\eqmark{5}
    T=2\pi\sqrt{\frac{J}{\mathfrak{m} B_0}}.
\end{equation}

Момент инерции цилиндрического стержня относительно оси вращения
\begin{equation}
	\eqmark{6}
J=m\left(\frac{l^2}{12}+\frac{r^2}{4}\right)=\frac{ml^2}{12}\left[1+3\left(\frac
r l\right)^2\right],
\end{equation}
где $m$~--- масса стержня, $l$~--- длина, а $r$~--- его радиус.

Таким образом, рассчитав момент инерции $J$ и измерив тангенс угла отклонения
стрелки $\varphi_1$ и период малых крутильных
колебаний стержня~$T$, можно с помощью формул \eqref{1}, \eqref{3}, \eqref{4} и
\eqref{5} определить горизонтальную составляющую
магнитного поля Земли:
\begin{equation}
	\eqmark{7}
    B_0=\frac{2\pi}{TR}\sqrt{\frac{\mu_0JL}{2\pi R\,x_1}}\quad \text{[ед. СИ]}.
\end{equation}

% Для устранения случайных помех стержень подвешивается в стеклянном сосуде.
Поскольку магнитометр установлен в железобетонном здании, магнитное поле в нём
может не только сильно отличаться от поля
Земли, но и заметно меняться от места к месту, поэтому период колебаний следует
измерять непосредственно \emph{вблизи магнитометра}.
Кроме того, для обеспечения максимальной однородности магнитного поля
в области измерений следует устранить (удалить на максимальное расстояние)
возможные источники сильного магнитного поля:
источники питания, токонесущие провода, сотовые телефоны, металлические
предметы и т.\,п.


\labsection{Определение электродинамической постоянной}

Ток в цепи кольца можно измерить двумя независимыми способами:
по магнитному действию тока на стрелку магнитометра и по заряду,
протекающему через цепь в единицу времени. Первый способ измерения
соответствует тому, как эталон тока определён в системе СИ,
а второй --- в гауссовой системе (СГС). По отношению этих измерений
можно определить электродинамическую постоянную~$c$.

Пропуская некоторый ток через витки магнитометра,
измерим тангенс угла отклонения стрелки ($\tg\varphi_2=x_2/2L$) и по формулам
\eqref{2} и \eqref{3} рассчитаем силу тока
\begin{equation}
    \eqmark{9}
    I=\frac{2B_0R}{\mu_0 N}\tg\varphi_2\quad \text{[ед. СИ]}.
\end{equation}
Величина $A=2B_0R/(\mu_0N)$ является постоянной прибора в данном месте земной поверхности
(точнее, в данном месте комнаты --- с учётом многочисленных сторонних источников
магнитного поля).
% Заметим, что при фиксированной постоянной~$A$ магнитометр
% может в приципе служить для изготовления эталонов и градуировки
% амперметров в системе СИ.

% Заметим, что если $B_0$ известно, то измерение силы тока не требует сравнения
% с какими-либо эталонами тока и является абсолютным,
% т.\,е. непосредственно связывает ток с основными
% единицами системы~СИ. При этом магнитометр может служить для изготовления
% эталонов и градуировки амперметров в системе СИ.




%\rpic{50mm}{1_1_3}{\cct Схема питания катушки
% магнитометра}{fig:magnitometer-power}
\begin{figure}
\centering
    \pic{0.4\textwidth}{Chapter_1/1_1_3}
    \caption{Схема питания катушки магнитометра}
    \figmark{magnitometer-power}
\end{figure}

Тот же ток можно измерить абсолютным образом по прошедшему
в единицу времени заряду, что соответствует определению
эталона тока в гауссовй системе (СГС). Если разрядить конденсатор известной ёмкости~$C$,
заряженный до напряжения~$U$, через витки, то через них протечёт заряд~$q=CU$
(рис.~\figref{magnitometer-power}).
Если $\nu$~раз в секунду последовательно заряжать конденсатор от источника и
разряжать через витки, то через них за секунду протечёт заряд~$CU\nu$. Средний
ток, прошедший через витки, равен при этом
\begin{equation}
    \eqmark{10}
    I=CU\nu \quad \text{[абс. ед.]}
\end{equation}

Таким образом, абсолютное измерение тока сводится к
нахождению величин~$C$ и~$U$, которые тоже могут быть определены
абсолютным образом. Так, ёмкость плоского конденсатора
можно вычислить из его размеров, то есть опираясь только на единицу длины.
Разность потенциалов также может быть определена абсолютным образом, например, через
силу, действующую на пластину заряженного
конденсатора, как это делается в абсолютном вольтметре
(см.~работу \ref{lab:absvolt}). Мы, однако, не будем полностью проводить эту
программу, а ограничимся только указанием на возможность её выполнения.

Итак, для вычисления абсолютного значения тока по \eqref{10}
необходимо измерить напряжение~$U$ на конденсаторе известной ёмкости~$C$.
Напряжение необходимо выразить его в единицах СГС
(измерительные приборы, как правило, проградуированы в единицах СИ:
$1~\text{В} \approx \frac{1}{300}\;\text{ед. СГС}$). Ёмкость конденсатора
$C\;\text[см]$ должна быть выражена в сантиметрах
($1\;\text{Ф} \approx 9\cdot 10^{11}\;\text{см}$).


% Для определения электродинамической постоянной $c$ необходимо провести
% независимые измерения одного и того же тока в
% разных системах: в СИ~--- $I_\textnormal{СИ}$ и в СГС~--- $I_\textnormal{СГС}$:
% \begin{equation}
%     \eqmark{8}
%     c=10\frac{\{I\}_\textnormal{СГС}}{\{I\}_\textnormal{СИ}}.
% \end{equation}
По отношению численных значений одного и того же тока, выраженных в единицах
СИ и СГС (гауссовой) по формулам
\eqref{9} и \eqref{10} соответственно,
можно определить значение электродинамической постоянной:
\begin{equation}
    \eqmark{8}
    c\;\left[\tfrac{\text{м}}{\text{с}}\right] = \frac{1}{10} \frac{I_{\text[СГС]}}{I_{\text[СИ]}}
\end{equation}
Предлагаем читателю самостоятельно получить соотношение \eqref{8},
исходя из определения ампера и абсолютной (гауссовой) единицы тока.

\begin{lab:task}

\taskpreamble{~}
\vspace*{-8ex} % workaround

\tasksection{А. Измерение горизонтальной составляющей магнитного поля Земли}

\taskpreamble{В этом упражнении предлагается измерить угол отклонения магнитной
стрелки в поле намагниченного стержня и период колебаний этого стержня в поле
Земли. По результатам измерений рассчитывается горизонтальная составляющая
магнитного поля Земли.}

        \item Включите осветитель и получите на горизонтальной шкале два чётких
световых \textquote{зайчика}. Плавным поворотом кольца К (рис.~\figref{magnitometr})
        вокруг вертикальной оси добейтесь совмещения зайчиков.
        Их чёткость можно подрегулировать перемещением линзы Л вдоль оси
        осветителя.

        \item В отверстие Р на горизонтальном диаметре кольца
(рис.~\figref{magnitometr}) вставьте намагниченный стержень и измерьте смещение
подвижного зайчика~$x_1$ (рис.~\figref{magnitometr-measure}).
% Оно должно составлять несколько сантиметров.
Поменяв ориентацию стержня в гнезде, измерьте отклонение зайчика
в другую сторону. При незначительном расхождении усредните результаты,
при значительном ($>5$\%) следует устранить причины расхождения.

        \item Измерьте расстояние $L$ от шкалы до зеркала.

        \item Измерьте период малых колебаний стержня $T$ в магнитном поле Земли.

        Для этого поставьте стеклянный сосуд \emph{вблизи} магнитометра
        и опустите на дно привязанный за середину намагниченный стержень.
        Плавным поворотом спицы, на которой закреплена нить, чуть
        приподнимите стержень и приближённо определите период малых
        крутильных колебаний.

        Оцените погрешность измерения времени и рассчитайте количество
        колебаний, необходимое для измерения периода с погрешностью не более 1\%.
        Учтите, что при визуальных наблюдениях основной вклад
        в погрешность вносит точность фиксации крайних положений стержня.

        \item С помощью штангенциркуля измерьте линейные размеры стержня;
запишите массу стержня и параметры магнитометра.

        \item Рассчитайте величину горизонтальной составляющей магнитного поля
        Земли $B_0$ и оцените погрешность результата.
        Сравните результат со справочными данными.

\tasksection{Б. Измерение электродинамической постоянной}

\taskpreamble{В этом упражнении предлагается по углу отклонения магнитной стрелки
в поле кругового тока и известному полю Земли рассчитать ток в системе СИ,
а по известным напряжению и параметрам вибратора рассчитать ток в гауссовой
системе (СГС); по результатам измерений определить электродинамическую постоянную
(скорость света в вакууме).}

        \item Уберите намагниченный стержень из гнезда магнитометра и соберите
электрическую схему, изображённую на рис.~\figref{magnitometer-power}.

        \item Убедитесь, что зайчики совмещены в отсутствие тока через витки.

        \item Включите в сеть источник питания и установите рабочее напряжение
$U\approx 90$--100~В.

        \item Замкнув ключ, подключите к цепи витки магнитометра.

        \item Включив кнопкой~$K$ электровибратор, измерьте напряжение~$U$ на
конденсаторе и отклонение $x_2$ зайчика на шкале.
        Поменяв полярность с помощью ключа, повторите измерения.

        \item Запишите характеристики приборов и параметры~$N$, $C$ и $\nu$,
указанные на установке.

        \item Рассчитайте токи по формулам \eqref{9} и \eqref{10}.
        Вычислите электродинамическую постоянную $c$ и оцените погрешность
        результата.
\end{lab:task}


\begin{lab:questions}
    \item Получите формулу \eqref{2} для магнитного поля в центре
    кругового витка с током.

    \item Как изменится поле \eqref{1}, создаваемое ферромагнитным стержнем
    в центре кольца,
    если ориентировать его в плоскости кольца в направлении центра?

    \item Каким должно быть внутреннее сопротивление источника напряжения, чтобы
ёмкость успевала разряжаться между замыканиями вибратора?

    \item В работе измеряется не поле Земли, а поле внутри здания. Влияет ли это на
точность определения электродинамической постоянной?

    \item Пользуясь определениями эталонов тока в системах СИ и СГС,
    докажите соотношение \eqref{8}.

    \item Получите теоретически соотношения между единицами
    СИ и СГС для 1)~напряжённости магнитного поля,
    2)~индукции магнитного поля, 3)~магнитного потока. Как называются
    соответствующие единицы?
\end{lab:questions}


\begin{lab:literature}
    \item \SivuhinIII~--- \S\S~50--55.

    \item \Kalashnikov~--- \S\S~83, 89, 125.
\end{lab:literature}
