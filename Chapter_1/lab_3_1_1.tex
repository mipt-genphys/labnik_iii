\lab{Магнитометр}

\begin{lab:aim}
    определить горизонтальную составляющую магнитного поля Земли и установить количественное соотношение между единицами электрического тока в системах СИ и СГС.
\end{lab:aim}

\begin{lab:equipment}
    магнитометр, осветитель со шкалой, источник питания, вольтметр, электромагнитный переключатель, конденсатор,намагниченный стержень, прибор для определения периода крутильных колебаний, секундомер, рулетка, штангенциркуль.
\end{lab:equipment}


Магнитометром называют прибор для магнитных измерений, например компас, теодолит, веберметр и пр. С помощью
магнитометров измеряют намагниченность ферромагнетиков, напряжённость магнитных полей, исследуют магнитные аномалии.
Разработаны магнитометры различных конструкций: магнитостатические, электромагнитные, магнитодинамические, индукционные,
резонансные. Эталонные магнитометры позволяют измерять горизонтальную и вертикальную составляющие напряжённости
магнитного поля Земли с точностью~$10^{-6}$~Э ($1~Э=79,6~$~А/м).

%\rpic{5.0cm}{1_1_1}{Схема магнитометра}{fig:magnitometr}
\begin{wrapfigure}{r}{0.4\textwidth}
	\pic{0.38\textwidth}{1_1_1}
	\caption{Схема магнитометра}
	\figmark{magnitometr}
\end{wrapfigure}

В нашей установке с помощью электромагнитного магнитометра измеряется горизонтальная составляющая земного магнитного
поля и абсолютным образом определяется сила тока по его магнитному действию.

\experiment Магнитометр (\figref{magnitometr}) состоит из нескольких последовательно соединённых круговых витков~К, расположенных
вертикально. В центре кольца~К на тонкой неупругой вертикальной нити подвешена короткая магнитная стрелка~С. Жёстко
связанная со стрелкой крыльчатка погружена в масло и служит для демпфирования колебаний.

В отсутствие других магнитных полей стрелка располагается по направлению горизонтальной составляющей земного магнитного
поля $\vec{B}_0$, т.е. лежит в плоскости магнитного меридиана.

Прибор настраивают с помощью световых зайчиков, отражённых от двух зеркал: $З_1$, прикреплённого к стрелке (подвижный
зайчик), и $З_2$, расположенного в плоскости кольца~К и жёстко связанного с ним (неподвижный зайчик). Оба зеркала
освещаются одним и тем же осветителем~О. Вращением кольца вокруг вертикальной оси можно совместить оба зайчика. При этом
плоскость витков совпадает с плоскостью магнитного меридиана.

\todo[author=Nozik]{Заменить wrapfigure на обычный рисунок. Можно сделать два рядом}
%\rpic{43mm}{1_1_2}{Схема измерения угла отклонения магнитной стрелки}{fig:magnitometr-measure}
\begin{wrapfigure}{r}{0.4\textwidth}
	\pic{0.38\textwidth}{1_1_2}
	\caption{Схема измерения угла отклонения магнитной стрелки}
	\figmark{magnitometr-measure}
\end{wrapfigure}

При появлении дополнительного горизонтального магнитного поля $\vec{B}_{\perp}$ стрелка C установится по равнодействующей
обоих полей $\vec{B}_{\Sigma}$ (\figref{magnitometr-measure}). В нашей установке дополнительное поле может быть создано либо ферромагнитным
стержнем, расположенным на кольце на его горизонтальном диаметре ($\vec{B}_1$), либо током, проходящим по кольцу
($\vec{B}_2$). В обоих случаях дополнительное поле можно считать однородным, т.к. размеры стрелки много меньше радиуса
кольца.

Поле намагниченного стержня (точечного диполя) на перпендикуляре к нему:

\begin{equation}
	\eqmark{1}
    B_1=\frac{\mu_0}{4\pi}\frac{\Mgot}{R^3},
\end{equation}

поле в центре кольца с током по закону Био и Савара:
\begin{equation}
	\eqmark{2}
    B_2=\frac{\mu_0 I}{2R}N.
\end{equation}

Здесь $\Mgot$~--- магнитный момент ферромагнитного стержня, $R$~--- радиус кольца, $N$~--- число витков в кольце,
$I$~--- сила тока в~единицах СИ (амперах).

Измерив угол отклонения стрелки $\phi$, можно связать поля~$B_0$ и~$B_{\perp}$ ($B_1$ или $B_2$):
\begin{equation}
	\eqmark{3}
    B_{\perp}=B_0\cdot \tan{\varphi}.
\end{equation}

\subsection*{Определение горизонтальной составляющей магнитного поля Земли}

Для определения горизонтальной составляющей земного магнитного поля~$B_0$ тонкий короткий намагниченный стержень
устанавливается в отверстие Р на горизонтальном диаметре кольца (\figref{magnitometr}). Измерив тангенс угла отклонения стрелки
\begin{equation}
	\eqmark{4}
    \tan\phi_1=\frac{x_1}{2L},
\end{equation}
можно с помощью уравнений \eqref{1}, \eqref{3} и \eqref{4} рассчитать поле~$B_0$, если исключить величину~$\Mgot$~--- магнитный
момент стержня.

Для исключения магнитного момента измерим период крутильных колебаний стержня в поле Земли. Подвешенный горизонтально за
середину на тонкой длинной нити стержень в положении равновесия установится по полю Земли (упругость нити пренебрежимо
мала). Если ось стержня отклонить в горизонтальной плоскости от направления~$B_0$ на малый угол~$\alpha$, то под
действием возвращающего механического момента
\begin{equation*}
    M_{мех}=\Mgot\,B_0\,\sin\alpha\approx \Mgot\,B_0\,\alpha
\end{equation*}
стержень с моментом инерции $J$ в~соответствии с уравнением
\begin{equation*}
    J\ddot{\alpha}+\Mgot\;B_0\;\alpha=0
\end{equation*}
будет совершать крутильные колебания c периодом
\begin{equation}
	\eqmark{5}
    T=2\pi\sqrt{\frac{J}{\Mgot B_0}}.
\end{equation}
Момент инерции цилиндрического стержня относительно оси вращения
\begin{equation}
	\eqmark{6}
    J=m\left(\frac{l^2}{12}+\frac{r^2}{4}\right)=\frac{ml^2}{12}\left[1+3\left(\frac r l\right)^2\right],
\end{equation}
где $m$~--- масса стержня, $l$~--- длина, а $r$~--- его радиус.

Таким образом, рассчитав момент инерции $J$ и измерив тангенс угла отклонения стрелки $\phi_1$ и период малых крутильных
колебаний стержня~$T$, можно с помощью формул \eqref{1}, \eqref{3}, \eqref{4} и \eqref{5} определить горизонтальную составляющую
магнитного поля Земли:
\begin{equation}
	\eqmark{7}
    B_0=\frac{2\pi}{TR}\sqrt{\frac{\mu_0JL}{2\pi R\,x_1}}.
\end{equation}

Поскольку магнитометр установлен в железобетонном здании, магнитное поле в нём может не только сильно отличаться от поля
Земли, но и заметно меняться от места к месту, поэтому период колебаний следует определять вблизи магнитометра. Для
устранения случайных помех стержень подвешивается в специальном стеклянном сосуде.

\begin{lab:task}

    В этом упражнении предлагается измерить угол отклонения магнитной стрелки в поле намагниченного стержня и период
    колебаний этого стержня в поле Земли. По результатам измерений рассчитывается горизонтальная составляющая магнитного
    поля Земли.

    \begin{enumerate}
        \item Включите осветитель и получите на горизонтальной шкале два чётких световых зайчика. Плавным поворотом кольца К (\figref{magnitometr})
        вокруг вертикальной оси добейтесь совмещения зайчиков. Их чёткость можно подрегулировать перемещением линзы Л вдоль оси
        осветителя.

        \item В отверстие Р на горизонтальном диаметре кольца (\figref{magnitometr}) вставьте намагниченный стержень и измерьте смещение подвижного
        зайчика~$x_1$ (\figref{magnitometr-measure}). Оно должно составлять несколько сантиметров. Поменяв ориентацию стержня в гнезде, измерьте
        отклонение зайчика в другую сторону. При незначительном расхождении усредните результаты, при значительном ($>5$\%)
        следует устранить причины расхождения.

        \item Измерьте расстояние $L$ от шкалы до зеркала.

        \item Для измерения периода малых колебаний поставьте стеклянный сосуд вблизи магнитометра и опустите на дно привязанный за
        середину намагниченный стержень. Плавным поворотом спицы, на которой закреплена нить, чуть приподнимите стержень и
        приближённо определите период малых крутильных колебаний. Оцените, сколько колебаний надо взять для расчёта периода,
        чтобы погрешность расчёта была меньше одного процента. Точность, с которой можно на глаз зафиксировать начало и конец
        колебаний, порядка одной секунды.

        Округлив результат, измерьте время нескольких десятков колебаний.

        \item С помощью штангенциркуля измерьте линейные размеры стержня; запишите массу стержня и параметры магнитометра.

        \item Рассчитайте величину $B_0$ и оцените погрешность.
        % Сравните результат с табличным.
    \end{enumerate}
\end{lab:task}


\labsection{Определение электродинамической постоянной}

Для определения электродинамической постоянной $c$ необходимо провести независимые измерения одного и того же тока в
разных системах: в СИ~--- $I_\textnormal{СИ}$ и в СГС~--- $I_\textnormal{СГС}$:
\begin{equation}
	\eqmark{8}
    c=10\frac{\{I\}_\textnormal{СГС}}{\{I\}_\textnormal{СИ}}.
\end{equation}

Пропуская ток через витки магнитометра, измеряют тангенс угла отклонения стрелки ($\tan\varphi_2=x_2/2L$) и по формулам
(\eqref{2}) и (\eqref{3}) рассчитывают величину
\begin{equation}
	\eqmark{9}
    I_{СИ}=\frac{2B_0R}{\mu_0 N}\tan\varphi_2=A\tan\varphi_2.
\end{equation}
Величина $A$ является постоянной прибора в данном месте земной поверхности.

Заметим, что если $B_0$ известно, то определение силы тока не требует сравнения с какими-либо эталонами тока или
напряжения и является абсолютным, т.е. непосредственно связывает ток с основными единицами системы СИ. При этом
магнитометр может служить для изготовления эталонов и градуировки амперметров в системе СИ.

%\rpic{50mm}{1_1_3}{\cct Схема питания катушки магнитометра}{fig:magnitometer-power}
\begin{wrapfigure}{r}{0.4\textwidth}
	\pic{0.38\textwidth}{1_1_3}
	\caption{Схема питания катушки магнитометра}
	\figmark{magnitometer-power}
\end{wrapfigure}

Одновременно тот же ток измеряется в системе СГС (\figref{magnitometer-power}). Если разрядить конденсатор ёмкости~$C$, заряженный до напряжения $U$, через витки, то через них протечёт заряд~$q=CU$. Если $n$~раз в секунду последовательно заряжать конденсатор от источника и разряжать через витки, то через них за секунду протечёт заряд~$CUn$. Средний ток, прошедший через витки,
равен при этом
\begin{equation}
	\eqmark{10}
    I_{СГС}=CUn.
\end{equation}

Таким образом, измерение тока в системе СГС сводится к нахождению величин~$C$ и~$U$, которые тоже могут быть определены
абсолютным образом. Так, ёмкость плоского конденсатора можно вычислить, опираясь только на единицу длины. Разность
потенциалов также может быть определена абсолютным образом, например, через силу, действующую на пластину заряженного
конденсатора, как это делается в абсолютном вольтметре (см.~работу~№~3.2.1). Мы, однако, не будем проводить эту
программу, а ограничимся только указанием на возможность её выполнения.

Вместо этого возьмём конденсатор, ёмкость которого выражена в сантиметрах (единица абсолютной гауссовой системы), и
измерим напряжение~$U$ на нём вольтметром~$V$, прокалиброванном в вольтах ($300~В = 1$~ед. СГС). Значения~$C$ и~$U$
в~единицах системы СГС подставим в формулу~\eqref{10}.

\begin{lab:task}

    В этом пункте предлагается по углу отклонения магнитной стрелки в поле кругового тока и известному полю Земли рассчитать
    ток в системе СИ, а по известным напряжению и параметрам вибратора рассчитать ток в гауссовой системе; по результатам
    измерений определить электродинамическую постоянную.

    \begin{enumerate}
        \item Уберите намагниченный стержень из гнезда магнитометра и соберите электрическую схему, изображённую на рис.~\figref{magnitometer-power}.

        \item Убедитесь, что зайчики совмещены в отсутствие тока через витки.

        \item Включите в сеть источник питания и установите рабочее напряжение $U\approx 90$--100~В (любое целое, близкое к
        максимальному).

        \item Замкнув ключ, подключите к цепи витки магнитометра.

        \item Включив кнопкой К электровибратор, измерьте напряжение~$U$ на конденсаторе и отклонение $x_2$ зайчика на шкале.
        Поменяв полярность с помощью ключа, повторите измерения.

        \item Запишите характеристики приборов и параметры~$N$, $C$ и $n$, указанные на установке.

        \item Рассчитайте токи. Вычислите электродинамическую постоянную и оцените погрешность.
    \end{enumerate}


\end{lab:task}


\begin{lab:questions}
    \item Приведите формулу для поля точечного магнитного диполя.

    \item Получите формулу для магнитного поля в центре кругового витка с током.

    \item Каким должно быть внутреннее сопротивление источника напряжения, чтобы ёмкость успевала разряжаться между замыканиями
    вибратора?

    \item Мы измеряем не поле Земли, а поле внутри здания. Влияет ли это на точность определения электродинамической постоянной?

    \item Установите соотношения между эрстедом и ампером на метр, гауссом и теслой, максвеллом и вебером.
\end{lab:questions}

\begin{lab:literature}
    \item~\emph{Сивухин~Д.В.} Курс общей физики.~Т.III.--- М.:~Наука, 1983, \S\S~50--55.

    \item~\emph{Калашников~С.Г.} Электричество.--- М.: Наука, 1970, \S\S~83, 89, 125.
\end{lab:literature}
