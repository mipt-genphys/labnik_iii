% !TeX encoding = UTF-8
\labsection{Точность измерения}

Одна из основных задач при проведении физических измерений~--- определение
точности этих измерений. Неточности, или погрешности, измерений могут быть двух
типов:

\begin{enumerate}
	\item Систематические смещения, в результате которых значение во всех
измерениях отличается от реального на фиксированную величину. Типичный пример
такой систематической погрешности~--- ошибка установки нуля. Такую же ошибку
можно получить при округлении значений на стрелочном (до ближайшей риски) или
цифровом (до определенного знака) приборе. В некоторых случаях систематические
погрешности имеют односторонний характер (реальное значение всегда больше или,
наоборот, меньше измеренного), а в некоторых случаях характер отклонения
неизвестен. В лабораторных работах, как правило, для простоты ошибки считают
симметричными, несмотря на то что это в некоторых случаях приводит к уменьшению
точности результата.

	\item Случайные смещения, в которых отклонение измеренного значения от
истинного отличается в двух одинаковых измерениях. Типичный пример таких
погрешностей~--- шумы аппаратуры, особенно актуальные при изучении электричества
и магнетизма.
\end{enumerate}

При наблюдении показаний измерительных приборов погрешность измеряемой величины
может носить как систематический, так и статистический характер (хотя в
большинстве случаев имеют место и те и другие). Отличить систематическое
смещение от статистического достаточно легко: при длительных измерениях в
результате шумов показания будут постоянно изменяться. Разброс этих изменений
можно считать статистической ошибкой. Систематическая погрешность прибора всегда
указана в его паспорте.

Как правило приборы конструируют с таким расчетом, что их точность (максимальная
систематическая ошибка) соответствует минимально различимой разнице показаний.
То есть ошибка равна половине цены деления для стрелочных приборов и изменению
последней значащей цифры на единицу для цифровых. Но это правило не является
строгим и сильно зависит от типа прибора, диапазона измерений и других факторов.
Для самодельных приборов оно вообще не соблюдается.

Важно понимать, что реальные погрешности во многих случаях определяются не
столько прибором, сколько схемой эксперимента. Реальные статистические
погрешности, как правило, оцениваются путем многократных измерений или
длительных наблюдений при непрерывных измерениях. Систематические погрешности
определяются на основании физической модели процесса.
