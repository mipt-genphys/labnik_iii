\lab{Спектральный анализ электрических сигналов (компьютерный вариант)}

\begin{lab:aim}
	изучить спектральный состав периодических электрических сигналов.
\end{lab:aim}

\begin{lab:equipment}
персональный компьютер, USB-осциллограф АКИП-4107, функциональный генератор WaveStation 2012, соединительные кабели.
\end{lab:equipment}

В работе изучаются спектры периодических электрических сигналов различной формы (последовательности прямоугольных импульсов и цугов, а также амплитудно- и фазо-модулированных гармонических колебаний). Спектры этих сигналов наблюдаются с помощью спектроанализатора, входящего в состав USB-осциллографа и сравниваются с рассчитанными теоретически.

\begin{figure}
	\label{fig:631}
\end{figure}

\experiment
Функциональный генератор WaveStation 2012 позволяет сформировать два различных электрических сигнала, которые выводятся на два независимых канала~--- CH1 и CH2. Сигнал с канала CH1 подается на вход А, а сигнал с канала CH2~--- на вход В USB-осциллографа. Затем эти сигналы подаются на вход компьютера через USB-соединение. При работе USB-осциллографа в режиме осциллографа, на экране компьютера можно наблюдать каждый из сигналов в отдельности, а также их произведение. В режиме спектроанализатора можно наблюдать спектры этих сигналов.

\begin{lab:task}
\subsection*{Исследование спектра периодической последовательности
прямоугольных импульсов}

В этом упражнении исследуется зависимость ширины спектра периодической последовательности прямоугольных импульсов от длительности отдельного импульса.

\begin{enumerate}
	\item Проверьте соединение блоков экспериментальной установки согласно рис.~\ref{fig:631} (канал CH1 соединен с разъемом А, а канал CH2 – с разъемом В). Включите компьютер и функциональный генератор.
	\item Ознакомьтесь с элементами управления передней панели функционального генератора и форматом вывода информации на его экране по техническому описанию.
	\item Подготовьте установку к измерениям спектра периодической последовательности прямоугольных импульсов, следуя техническому описанию.
	\item Установите на генераторе следующие параметры импульсов: частота повторения $f_{\text{повт}}=1$~кГц, длительность импульса $\uptau=100$~мкс.

	Проанализируйте, как меняется спектр ($\Delta\nu$ и $\delta\nu$ на рис.???:)
	\todo{Ссылка на правильный рисунок из введения}
	%~\oref{v6_r3}): 
	\begin{itemize}
		\item при увеличении $\uptau$ вдвое при неизменном $f_{\text{повт}}=1$~кГц;
		\item  при увеличении $f_{\text{повт}}$ вдвое при неизменном $\uptau=100$~мкс.
	\end{itemize}
	
	Опишите результаты или зарисуйте в~тетрадь качественную картину.
	
%	\warning{При изменении на генераторе $f_{повт}$, автоматически изменяется $\uptau$, поэтому после изменения частоты повторения импульсов, надо установить прежнее значение длительности импульса.}
%
	\item Проведите измерения зависимости ширины спектра от длительности импульса~$\Delta \nu(\uptau)$ при увеличении $\uptau$ от 40 до 200~мкс (6--8 значений при $f_{\text{повт}}=1$~кГц).
	\item\label{task:rect_spectrum} Для частоты повторения $f_{\text{повт}} = 1$~кГц и длительности импульса 50, а затем 100~мкс измерьте частоты и амплитуды спектральных составляющих сигнала и запишите результаты в таблицу следующего вида: \emph{номер гармоники, частота, амплитуда}. По полученным данным постройте картины спектров.
	\item Постройте график $\Delta \nu(1/\uptau)$ и по его наклону убедитесь в~справедливости соотношения неопределённостей.
\end{enumerate}

\subsection*{Исследование спектра периодической последовательности
цугов гармонических колебаний}

В этом упражнении исследуется зависимость расстояния между ближайшими спектральными компонентами от частоты повторения цугов.

\begin{enumerate}
	\item Подготовьте установку к измерениям спектра периодической последовательности цугов гармонических колебаний, следуя техническому описанию. 
	\item Установите несущую частоту $\nu_0 = 25$~кГц, частоту повторения прямоугольных импульсов $f_{\text{повт}} = 1$~кГц, длительность импульса $\uptau = 100$~мкс.
	\item Проанализируйте, как изменяется вид спектра при увеличении длительности импульса $\uptau$ вдвое от 100 до 200 мкс.
	\item Установите длительность импульса $\uptau= 100$~мкс. Проследите, как меняется картина спектра при изменении несущей частоты $\nu_0$ ($\nu_0$ = 10, 25 и 40~кГц). Опишите результаты или зарисуйте в тетради качественную картину.
	\item Установите частоту несущей $\nu_0 = 30$~кГц. Установите длительность импульса $\uptau = 100$~мкс. Определите расстояние  между соседними спектральными компонентами для разных частот повторения импульсов $f_{\text{повт}}$. Проведите измерения для $f_{\text{повт}} =$ 0,5, 1, 2, 4 и 5~кГц.
	\item\label{task:zug_spectrum} Установите частоту повторения $f_{\text{повт}} = 1$~кГц и $\uptau = 100$~мкс, измерьте частоты и амплитуды спектральных составляющих сигнала и запишите результаты в таблицу следующего вида: \emph{номер гармоники, частота, амплитуда}. Проведите аналогичные измерения для $f_{\text{повт}} = 2$~кГц и $\uptau = 100$~мкс. По полученным данным постройте картины спектров.
	\item Постройте график $\delta \nu(f_{\text{повт}})$. Найдите угловой коэффициент полученной зависимости и сравните с теоретическим значением.
	\item Сравните построенные спектры:
%	(пп.~\ref{task:rect_spectrum} и \ref{task:zug_spectrum}):
	\begin{itemize}
		\item прямоугольных импульсов при одинаковых периодах и разных длительностях импульса $\uptau$;
		\item цугов при одинаковых $\uptau$ и разных $f_{\text{повт}}$;
		\item цугов и прямоугольных импульсов при одинаковых значениях $\uptau$ и  $f_{\text{повт}}$.
	\end{itemize}
\end{enumerate}

\subsection*{Исследование спектра гармонических сигналов,
модулированных по амплитуде}

В этом упражнении исследуется зависимость отношения амплитуд спектральных линий синусоидального сигнала, модулированного низкочастотными гармоническими колебаниями, от коэффициента модуляции.
%, который определяется с помощью осциллографа.

\begin{enumerate}
	\item Подготовьте установку к измерениям спектра гармонических сигналов, модулированных по амплитуде, следуя техническому описанию.
	\item Установите частоту несущей $\nu_0 = 25$~кГц, а частоту модуляции $f_{\text{мод}} = 1$~кГц.
	\item Меняя амплитуду модулирующего сигнала (возьмите 5--6 значений), измеряйте для каждого значения максимальную $A_{max}$ и минимальную $A_{min}$ амплитуды сигналов модулированного колебания и амплитуды спектральных компонент.  Рассчитайте соответствующие значения глубины модуляции $m$ по формуле (???).
	\todo{Правильная ссылка}
	\item При 100\% глубине модуляции ($A_{min} = 0$) посмотрите, как меняется спектр при увеличении частоты модуляции $f_{\text{мод}}$.
	\item Постройте график отношения $A_{\text{бок}}/A_{\text{осн}}$ в зависимости от $m$. Определите угловой коэффициент наклона графика и сравните с рассчитанным с помощью формулы (???).
	\todo{правильная ссылка}
\end{enumerate}

%	\labsection{Исследование спектра периодической последовательности прямоугольных импульсов}
%	\begin{enumerate}
%		\item На генераторе кнопкой CH1/2 выберите вкладку для канала CH1 и нажмите кнопку Pulse (импульсный сигнал). Кнопками 4 экранного меню (рис. 3) установите: а) Ampl : 1 Vpp (разность максимального и минимального значений сигнала 1 В); б) Offset : 0.5 Vdc (смещение сигнала на 0,5 В); в) Freq : 1 kHz (частота повторения импульсов fповт = 1 кГц); г) PulWidth : 100 s (длительность импульса  = 100 мкс).
%		\item В окне программы нажмите кнопку  – режим Спектр, затем кнопку  – Параметры спектра, и в появившемся окне установите параметры: а) Масштаб: линейный; б) Элементы разрешения спектра: 2048. В верхней части экрана установите удобный масштаб (≈   2 В) по вертикальной оси (справа от кнопки ), а по горизонтальной оси – “48,83 кГц”. Отдельные области спектра на экране можно увеличивать (уменьшать) с помощью кнопок  и .
%		\item Проанализируйте, как меняется спектр  (∆ и  на рис. 6.3 Ввеления): а) при увеличении  вдвое при неизменной частоте fповт = 1 кГц; б) при увеличении fповт вдвое при неизменном  = 100 мкс. Опишите результаты или зарисуйте в тетрадь качественную картину.
%		\item Проведите измерения зависимости ширины спектра ∆ от длительности импульса  при увеличении  от 40 до 200 мкс (6 – 8 значений при fповт = 1 кГц).
%		\item 1.	Для fповт = 1 кГц установите  = 50 мкс. В окне программы нажмите кнопку  – Выбор. Если левой кнопкой мышки щелкнуть на вершине выбранной гармоники, то в отдельном окошке появляются значения ее амплитуды и частоты. Измерьте частоты и амплитуды спектральных составляющих сигнала и запишите результаты в таблицу: № гармоники, частота, амплитуда. Проведите аналогичные измерения для импульса с  = 100 мкс. По полученным данным постройте картины спектров.
%		\item Постройте график ∆(1/) и по его наклону убедитесь в справедливости соотношения неопределенностей.
%	\end{enumerate}
\end{lab:task}

%\begin{lab:questions}
%	
%\end{lab:questions}
%
%\begin{lab:literature}
%	
%\end{lab:literature}