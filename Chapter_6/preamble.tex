\section{Физический смысл спектрального разложения}

Латинское слово <<spectrum>> является синонимом слова <<изображение>>. Ньютон пользовался этим словом для выражения <<цветное изображение>>. Вот цитата из его знаменитого трактата <<Оптика>>: <<Я поместил в очень тёмной комнате у круглого отверстия, около трети дюйма шириной, в ставне окна стеклянную призму, благодаря чему пучок солнечного света, входящего в это отверстие, мог преломляться вверх к противоположной стене комнаты и образовать там цветное изображение (спектр) солнца>>. Так начинается доказательство его знаменитого утверждения: <<Солнечный свет состоит из лучей различной преломляемости>>.

Значительно позднее слово <<спектр>> приобрело в науке ещё и другой смысл.

Рассмотрим функцию вида
\begin{equation*}
	f(t)=A_1\cos(\omega_1t-\alpha_1)+A_2\cos(\omega_2t-\alpha_2)+\dots +A_N\cos(\omega_Nt-\alpha_N),
\end{equation*}

или в более короткой записи

\begin{equation*}
	f(t)=\sum_{n=1}^N A_n\cos(\omega_nt-\alpha_n),
\end{equation*}
%
где $A_n$, $\omega_n$, $\alpha_n$~--- постоянные величины. Множество пар $(\omega_1,\,A_1)$, $(\omega_2,\,A_2)$, ..., $(\omega_n,\,A_n)$ называется спектром функции $f(t)$. $N$ может быть конечным или бесконечным.

<<Спектр функции>>~--- понятие математическое. Между математическим и физическим понятием спектра существует тесная связь: характер спектра как реально существующей цветной картины (спектра в физическом смысле) определяется характером спектра (в математическом смысле)~--- функции, описывающей световую волну, падающую на призму. Установление этой связи составляет содержание одного из важнейших утверждений универсального учения о колебаниях и волнах самой различной физической природы (оптических, акустических, электрических и пр.).

В чём, например, физический смысл открытия Ньютона? Действительно ли солнечный свет состоит из лучей различной преломляемости?

На этот вопрос можно услышать такой ответ: <<С помощью своих опытов с призмой Ньютон доказал, что солнечный свет состоит из монохроматических (синусоидальных) волн различного цвета>>.

Абсурдность этого ответа очевидна. Нелепо думать, что в солнечном свете в самом деле есть монохроматические волны различного цвета. Солнечный свет~--- это хаотический процесс, в котором изменение электромагнитного поля происходит беспорядочным образом. Суть проблемы разъяснил Л.И.~Мандельштам.

Рассмотрим для примера амплитудно-модулированное колебание
\begin{equation*}
	f(t)=(a+2b\cos\Omega t)\cos\omega t,
\end{equation*}
%
здесь $\Omega$~--- частота модуляции, $\omega$~--- <<несущая>> частота, $a$ и $b$~--- постоянные величины. Можно видеть,
что

\begin{equation*}
	(a+2b\cos\Omega t)\cos\omega t=b\cos(\omega-\Omega)t+a\cos\omega t+b\cos(\omega+\Omega)t.
\end{equation*}

Что реально существует? Левая или правая часть этого тождества?

Если мы принимаем этот сигнал с помощью радиоприёмника, мы не сможем сказать, что реально на самом деле: издает ли в радиостудии скрипач звук на частоте $\Omega$ или работают три генератора на частотах $\omega-\Omega$, $\omega$, $\omega+\Omega$. Однако, если нас интересует, как действует амплитудно-модулированное колебание на набор остро настроенных колебательных контуров, наиболее целесообразным является представление, даваемое правой частью тождества. Здесь целесообразно говорить, что наше колебание состоит из трёх синусоидальных колебаний.

Так в чём же истинное содержание опыта Ньютона? В том, что призма есть спектральный прибор, что она физически выделяет синусоидальные составляющие, физически осуществляет спектральное разложение света.

Опыты Ньютона показывают, что солнечный свет действительно несинусоидален, и позволяют узнать, каков именно спектр солнечного света. Из опытов Ньютона мы узнаем, что он является весьма широким сплошным спектром, в котором содержатся интенсивные слагаемые всех видимых цветов, цветов радуги.

\section{Спектральный анализ линейных систем}
%\vn Спектральный анализ линейных систем
\subsection{Периодические сигналы}
%\vvn Периодические сигналы

%Многие физические процессы можно моделировать с помощью линейных дифференциальных уравнений. К решениям таких уравнений
%примен\'им принцип суперпозиции: разнообразные сложные явления удобно представлять в виде суммы простых решений линейных
%уравнений. Для линейных уравнений такими простыми решениями являются гармонические функции. Математическая теория
%представления сложных функций в~виде сумм гармонических составляющих получила название \emph{теории рядов и интегралов
%Фурье} (см. соответствующий учебник).

%При теоретическом анализе отклика линейных радиотехнических устройств
%на сложные внешние воздействия (негармонически меняющиеся токи и
%напряжения, импульсные процессы) широко используется принцип наложения
%(суперпозиции): отклик линейной системы на несколько внешних
%5воздействий (ток, напряжение) равен сумме
%откликов на каждый из них в~отдельности (система является линейной,
%если она описывается линейными алгебраическими или дифференциальными
%уравнениями).
%Поэтому при анализе сложные сигналы удобно разлагать
%на сумму простых сигналов, для которых отклик системы легко находится.

В физике широко используется разложение сложных сигналов на гармонические колебания различных частот $\omega$.
%Функция $F(\omega)$, описывающая зависимость амплитуды гармоник от их частоты, называется \emph{амплитудной спектральной
%характеристикой~--- спектром} исходного сигнала.
Представление периодического сигнала в~виде суммы гармонических сигналов в~математике называется \emph{разложением в~ряд Фурье}. Непериодические сигналы представляются в виде~\emph{интеграла Фурье}.

\begin{figure}
	\centering
	\pic{0.5\textwidth}{v6_1}
	\caption{График периодической функции с~периодом повторения~$T$}
	\figmark[spectrum-pre-1]
\end{figure}
%\cpic{v6_1}{График периодической функции с~периодом повторения~$T$}{1}

Пусть заданная функция $f(t)$ периодически повторяется с~частотой $\Omega_1=2\pi/T$, где $T$~--- период повторения
(\figref{spectrum-pre-1}). Её разложение в~ряд Фурье имеет вид

\begin{equation}
	f(t)=\frac{a_0}{2}+\sum_{n=1}^{\infty}[a_n\cos(n\Omega_1 t)+b_n\sin(n\Omega_1 t)],
	\eqmark[eq:61]
\end{equation}
%\be1
%f(t)=\frac{a_0}{2}+\sum_{n=1}^{\infty}[a_n\cos(n\Omega_1 t)+b_n\sin(n\Omega_1 t)],
%\ee
или

\begin{equation}
	f(t)=\frac{a_0}{2}+ \sum_{n=1}^{\infty}A_n\cos(n\Omega_1 t-\psi_n).
	\eqmark[eq:62]
\end{equation}
%\be2
%f(t)=\frac{a_0}{2}+ \sum_{n=1}^{\infty}A_n\cos(n\Omega_1 t-\psi_n).
%\ee

Здесь $a_0/2$~--- постоянная составляющая (среднее значение) функции~$f(t)$; $a_n$ и $b_n$~--- коэффициенты косинусных и синусных членов разложения. Они определяются выражениями \eqref{61}

\begin{equation}
	a_n=\frac{2}{T}\int\limits_{t_1}^{t_1+T}\!\!f(t)\cos(n\Omega_1t)\,dt;
\end{equation}

\begin{equation}
	b_n =\frac{2}{T}\int\limits_{t_1}^{t_1+T}\!f(t)\sin(n\Omega_1t)\,dt.
\end{equation}
%\be3
%a_n=\frac{2}{T}\int\limits_{t_1}^{t_1+T}\!\!f(t)\cos(n\Omega_1t)\,dt;
%\ee
%\be4
%b_n =\frac{2}{T}\int\limits_{t_1}^{t_1+T}\!f(t)\sin(n\Omega_1t)\,dt.
%\ee

Точку начала интегрирования $t_1$ можно выбрать произвольно.

В тех случаях, когда сигнал чётен относительно $t=0$, так что $f(t)=f(-t)$, в тригонометрической записи остаются только косинусные члены, т.к. все коэффициенты~$b_n$ обращаются в~нуль. Для нечётной относительно $t=0$ функции, наоборот, в~нуль обращаются коэффициенты~$a_n$, и ряд состоит только из синусных членов.

Амплитуда $A_n$ и фаза $\psi_n\; n$-й гармоники выражаются через коэффициенты~$a_n$ и $b_n$ следующим образом:

\begin{equation}
	A_n=\sqrt{a_n^2+b_n^2};\quad \psi_n=\arctg\frac{b_n}{a_n}.
\end{equation}

%\be5
%A_n=\sqrt{a_n^2+b_n^2};\quad \psi_n=\arctg\frac{b_n}{a_n}.
%\ee

Представим выражение~(\r2) в~комплексной форме. Для этого заменим косинусы экспонентами в~соответствии с~формулой
\begin{equation}
	\cos \alpha=\frac{e^{i\alpha}+e^{-i\alpha}}{2}.
\end{equation}
%\cos \alpha=\frac{e^{i\alpha}+e^{-i\alpha}}{2}.
%\]

Подстановка даёт
%\[
\begin{equation}
f(t)=\frac{1}{2}\left(a_0+ \sum_{n=1}^{\infty}A_n\,e^{-i\psi_n}\,e^{in\Omega_1 t} +\sum_{n=1}^{\infty}A_n\,e^{i\psi_n}\,e^{-in\Omega_1 t}\right).	
\end{equation}
%\]
Введём комплексные амплитуды $\hat A_n$ и $\hat A_{-n}$:

\begin{equation}
	\hat{A}_n=A_n\, e^{-i\psi_n};\quad \hat{A}_{-n}=A_n\,e^{i\psi_n}; \quad \hat{A}_0=a_0.
\end{equation}
%\be6
%\hat{A}_n=A_n\, e^{-i\psi_n};\quad \hat{A}_{-n}=A_n\,e^{i\psi_n}; \quad \hat{A}_0=a_0.
%\ee

Разложение $f(t)$ приобретает вид
\begin{equation}
	f(t)=\frac{1}{2}\sum_{n=-\infty}^{\infty} \hat A_n\,e^{in\Omega_1t}.
\end{equation}
%\be7
%f(t)=\frac{1}{2}\sum_{n=-\infty}^{\infty} \hat A_n\,e^{in\Omega_1t}.
%\ee

Таким образом, введение отрицательных частот (типа $-n\Omega_1$) позволяет записать разложение Фурье особенно простым
образом. Формулы~(\r6) обеспечивают действительность суммы~(\r7): каждой частоте~$k\Omega_1$ соответствуют в~(\r2)~--- один член ($n=k$), а в~(\r7)~--- два члена ($n=k$ и $n=-k$). Формулы~(\r6) позволяют переходить от действительного разложения~(\r2) к~комплексному~(\r7) и обратно.

Для расчёта комплексных амплитуд $A_n$ не обязательно пользоваться формулами~(\r6). Умножим левую и правую части~(\r7) на $e^{-ik\Omega_1 t}$ и проинтегрируем полученное равенство по времени на отрезке, равном одному периоду, например, от $t_1=0$ до $t_2=2\pi/\Omega_1$. В~правой части обратятся в нуль все члены, кроме одного, соответствующего $n=k$. Этот член даёт $A_k T/2$. Имеем поэтому

\begin{equation}
	\hat{A}_k=\frac{2}{T}\int\limits_{0}^{T}\!f(t)\, e^{-ik\Omega_1 t}\,dt.
\end{equation}
%\be8
%\hat{A}_k=\frac{2}{T}\int\limits_{0}^{T}\!f(t)\, e^{-ik\Omega_1 t}\,dt.
%\ee

Как мы видим, спектр любой периодической функции состоит из набора гармонических колебаний с дискретными частотами:
$\Omega_1$, $2\Omega_1$, $3\Omega_1$,~$\dots$ и постоянной составляющей, которую можно рассматривать как колебание
с~нулевой частотой ($0\cdot \Omega_1$). Такой спектр называют \term{линейчатым} или \term{дискретным}.

\subsection{Непериодический сигнал}
%\vvn Непериодический сигнал

Пусть непериодический сигнал $f(t)$ действует в конечном временном интервале $t_1<t<t_2$. Превратим  функцию $f(t)$ в периодическую путём повторения её с~произвольным периодом~$T>(t_1-t_2)$. Для этой новой функции применимо разложение в ряд Фурье. В соответствии с~формулами (\r3)~---~(\r4) абсолютная величина коэффициентов $a_n$ и $b_n$ обратно пропорциональна $T$, поэтому устремляя $T$ к бесконечности, в пределе получим бесконечно малые амплитуды гармонических составляющих. Количество составляющих, входящих в ряд Фурье, будет при этом бесконечно большим, так как при $T\to\infty$ частота~$\W_1=\frac{2\pi}{T}\to 0$. Другими словами, расстояние между спектральными линиями, равное частоте~$\W_1$, становится бесконечно малым, и спектр из дискретного переходит в~сплошной.

Выразим это теперь на языке математики. Воспользуемся комплексной формулой ряда Фурье (\r7) и подставим вместо $A_n$ выражение (\r8).
\ml
\l f(t)=\sum_{n\to-\infty}^{+\infty}\frac1T \left[\int_{t_1}^{t_2} f(t) e^{-in\W_1t}dt\right]e^{in\W_1 t}=\\
\r= \frac{1}{2\pi}\sum_{n\to-\infty}^{+\infty}\frac1T \left[\int_{t_1}^{t_2} f(t) e^{-in\W_1t}dt\right]e^{in\W_1
t}\cdot\W_1.


При записи второго выражения использована связь $T={2\pi}/{\W_1}$.

При $T\to\infty$ частота $\W_1$ превращается в $d\W,\; n\W_1$~--- в~текущую частоту~$\W$, а операция суммирования~--- в операцию интегрирования. В результате получаем двойной интеграл Фурье:

\begin{equation}
	f(t)=\frac{1}{2\pi}\int_{-\infty}^{\infty}\left[\int_{t_1}^{t_2} f(t) e^{-i\W t}dt\right]e^{i\W t}d\W.
\end{equation}
%\[
%f(t)=\frac{1}{2\pi}\int_{-\infty}^{\infty}\left[\int_{t_1}^{t_2} f(t) e^{-i\W t}dt\right]e^{i\W t}d\W.
%\]

Внутренний интеграл обозначим
\begin{equation}
	\hat{F}(\W)=\int_{t_1}^{t_2} f(t) e^{-i\W t}dt.
\end{equation}
%\[
%\hat{F}(\W)=\int_{t_1}^{t_2} f(t) e^{-i\W t}dt.
%\]
$\hat{F}(\W)$ называется \emph{спектральной плотностью} или \emph{спектральной характеристикой} функции $f(t)$.

Сравнивая полученное выражение с (\r8) для комплексной амплитуды соответствующей гармоники $(\W=\W_n)$ той же самой
функции, но уже периодической, получим
\begin{equation}
	2\hat{F}(\W_n)=T\cdot \hat{A}_n=2\pi\frac{\hat{A}_n}{\W_1}.
\end{equation}
%\[
%2\hat{F}(\W_n)=T\cdot \hat{A}_n=2\pi\frac{\hat{A}_n}{\W_1}.
%\]

Поскольку $\W_1$~--- это полоса частот, отделяющая соседние спектральные линии дискретного спектра, то $\hat{F}(\W)$ имеет смысл \emph{плотности амплитуд}.

Из вышеприведённого соотношения следует важный вывод: \emph{огибающая сплошного спектра} (модуль спектральной плотности) \emph{непериодической функции и огибающая линейчатого спектра той же периодической функции совпадают по форме и отличаются только масштабом}.

\subsection{Примеры спектров периодических функций}
%\vvn Примеры спектров периодических функций

Рассмотрим периодические функции, которые исследуются в~нашей работе.

{\bf А. Периодическая последовательность прямоугольных импульсов} (\figref{spectrum-pre-2}) с~амплитудой~$V_0$, длительностью~$\uptau$, частотой повторения $\Omega_1=2\pi/T$, где $T$~--- период повторения импульсов.

Найдём среднее значение (постоянную составляющую). Согласно формуле~(\r3)
\begin{equation}
	\sr{V}=\frac{a_0}{2}=\frac{A_0}{2}= \frac1T\int\limits_{-\vartau/2}^{\vartau/2}\!V_0\,dt= V_0\frac{\vartau}{T}.
\end{equation}
%\[
%\sr{V}=\frac{a_0}{2}=\frac{A_0}{2}= \frac1T\int\limits_{-\vartau/2}^{\vartau/2}\!V_0\,dt= V_0\frac{\vartau}{T}.
%\]

Коэффициенты при косинусных составляющих равны
\begin{equation}
	a_n=\frac2T\int\limits_{-\vartau/2}^{\vartau/2}\!\!\!V_0 \cos(n\Omega_1t)\,dt=2V_0\frac{\vartau}{T}
\frac{\sin(n\Omega_1\vartau/2)}{n\Omega_1\vartau/2} \sim\frac{\sin x}{x}.
\end{equation}
%\be9
%a_n=\frac2T\int\limits_{-\vartau/2}^{\vartau/2}\!\!\!V_0 \cos(n\Omega_1t)\,dt=2V_0\frac{\vartau}{T}
%\frac{\sin(n\Omega_1\vartau/2)}{n\Omega_1\vartau/2} \sim\frac{\sin x}{x}.
%\ee

%\piccapt{0.48\textwidth}{Pic/v6_2}

%\begin{figure}
%	\pic[0.9\textwidth]{Pic/v6_2.pic}
%	\figmark{spectrum-pre-2}
%\end{figure}

\begin{figure}
\left
\begin{minipage}{0.45\textwidth}
	\left
	\pic{0.5\linewidth}{v6_2}
	\caption{Периодическая последовательность прямоугольных~импульсов}
\end{minipage}
\right
\begin{minipage}{0.45\textwidth}
	\left
	\pic{0.5\linewidth}{v6_3}
	\caption{Спектр периодической последовательности прямоугольных~импульсов}
\end{minipage}
%	\figmark[spectrum-pre-3]
\end{figure}
%\cpicskip
%\noindent\piccapt{0.48\textwidth}{v6_2}{\cct Периодическая последовательность прямоугольных~импульсов}{2}
%\hfill
%\piccapt{0.47\textwidth}{v6_3}{\cct Спектр периодической последовательности прямоугольных~импульсов}{3}
%\cpicskip

Поскольку наша функция чётная, все коэффициенты синусоидальных гармоник $b_n=0$. Спектр~$a_n$ последовательности прямоугольных импульсов представлен на \figref{spectrum-pre-3}. Амплитуды гармоник~$A_n$ ($A_n=|a_n|$) меняются по закону $|\sin x/x|$.

На \figref{spectrum-pre-3} изображён случай, когда $T$ кратно $\vartau$. Назовём \emph{шириной спектра}~$\Delta \omega$ (или $\Delta\nu=\D\w/2\pi$) расстояние от главного максимума ($\w=0$) до первого нуля огибающей, возникающего, как нетрудно убедиться, при $n=2\pi/\vartau\Omega_1$.
%Это соотношение обычно записывают в виде
При этом
\begin{equation}
	\Delta\omega \vartau\simeq 2\pi\quad \mbox{или}\quad \Delta\nu\Delta t\simeq 1.
\end{equation}
%\be10
%\Delta\omega \vartau\simeq 2\pi\quad \mbox{или}\quad \Delta\nu\Delta t\simeq 1.
%\ee
Полученное соотношение взаимной связи интервалов~$\Delta\nu$ и $\Delta t$ является частным случаем \emph{соотношения неопределённости} в~квантовой механике. Несовместимость острой локализации волнового процесса во времени с~узким спектром частот~--- явление широко известное в~радиотехнике. Ширина селективной настройки $\Delta\nu$ радиоприёмника ограничивает приём радиосигналов длительностью $t<1/\Delta \nu$.

{\bf Б. Периодическая последовательность цугов} гармонического колебания~$V_0\cos(\omega_0 t)$ с~длительностью
цуга~$\vartau$ и периодом повторения~$T$ (\figref{spectrum-pre-4}).

%\etp[-1]

Функция $f(t)$ снова является чётной относительно $t = 0$. Коэффициент при $n$-й гармонике согласно формуле~(\r3) равен

\begin{equation}
	a_n=\frac{2}{T}\int\limits_{-\vartau/2}^{\vartau/2}\!\!V_0 \cos(\omega_0 t)\cdot \cos(n\Omega_1\,t)\,dt=
\end{equation}

\begin{equation}
	=V_0\frac{\vartau}{T}\left( \frac{\sin[(\omega_0-n\Omega_1)\frac{\vartau}{2}]} {(\omega_0-n\Omega_1)\frac{\vartau}{2}}+
\frac{\sin[(\omega_0+n\Omega_1)\frac{\vartau}{2}]} {(\omega_0+n\Omega_1)\frac{\vartau}{2}} \right).
\end{equation}

%\[
%a_n=\frac{2}{T}\int\limits_{-\vartau/2}^{\vartau/2}\!\!V_0 \cos(\omega_0 t)\cdot \cos(n\Omega_1\,t)\,dt=
%\]
%\be11
%=V_0\frac{\vartau}{T}\left( \frac{\sin[(\omega_0-n\Omega_1)\frac{\vartau}{2}]} {(\omega_0-n\Omega_1)\frac{\vartau}{2}}+
%\frac{\sin[(\omega_0+n\Omega_1)\frac{\vartau}{2}]} {(\omega_0+n\Omega_1)\frac{\vartau}{2}} \right).
%\ee

Зависимость (\r{11}) для случая, когда $T/\vartau$ равно целому числу, представлена на \figref{spectrum-pre-5}. Сравнивая спектр последовательности прямоугольных импульсов и спектр цугов (см.~\figref{spectrum-pre-3} и~\r{r5}), мы видим, что они аналогичны, но их максимумы сдвинуты по частоте на величину~$\omega_0$.

\begin{figure}
	\figmark[spectrum-pre-4]
\end{figure}
\begin{figure}
	\figmark[spectrum-pre-5]
\end{figure}
%\cpicskip
%\noindent
%\hfill\piccapt{0.45\textwidth}{v6_4}{\cct Периодическая последовательность~цугов}{4}
%\hfill\piccapt{0.45\textwidth}{v6_5}{\cct Спектр периодической последовательности~цугов}{5}
%\cpicskip

{\bf В. Амплитудно-модулированные колебания}. Рассмотрим гармонические колебания высокой частоты~$\omega_0$, амплитуда которых медленно меняется по гармоническому закону с~частотой~$\Omega$ ($\Omega\ll\omega_0$) (\figref{spectrum-pre-6}):

\begin{equation}
	f(t)=A_0 [1+m \cos\Omega t] \cos\omega_0 t.
\end{equation}
%
%\be12
%f(t)=A_0 [1+m \cos\Omega t] \cos\omega_0 t.
%\ee

Коэффициент $m$ называют \term{глубиной модуляции}. При $m<1$ амплитуда колебаний меняется от минимальной
$A_{\min}=A_0(1-m)$ до максимальной $A_{\max}=A_0(1+m)$. Глубина модуляции может быть представлена в~виде

\begin{equation}
	m=\frac{A_{\max}-A_{\min}}{A_{\max}+A_{\min}}.
\end{equation}
%\be13
%m=\frac{A_{\max}-A_{\min}}{A_{\max}+A_{\min}}.
%\ee

Простым тригонометрическим преобразованием уравнения~(\r{12}) можно найти спектр амплитудно-модулированных колебаний:
\begin{equation}
	f(t) =A_0\cos(\omega_0 t)+A_0 m\cos(\Omega t)\cos(\omega_0 t) =
\end{equation}

\begin{equation}
	=A_0\cos(\omega_0 t)+ \frac{A_0 m}{2}\cos(\omega_0+\Omega)t+ \frac{A_0 m}{2}\cos(\omega_0-\Omega)t.
\end{equation}
%\[
%f(t) =A_0\cos(\omega_0 t)+A_0 m\cos(\Omega t)\cos(\omega_0 t) =
%\]
%\be14
%=A_0\cos(\omega_0 t)+ \frac{A_0 m}{2}\cos(\omega_0+\Omega)t+ \frac{A_0 m}{2}\cos(\omega_0-\Omega)t.
%\ee

\begin{figure}
	\figmark{spectrum-pre-6}
\end{figure}

\begin{figure}
	\figmark{spectrum-pre-7}
\end{figure}

%\mbox{}\par\nobreak
%\vskip-13mm
%%\cpicskip
%\noindent
%\hfill\piccapt{0.45\textwidth}{v6_6}{\cct Гармонические колебания, модулированные по амплитуде}{6}
%\hfill\piccapt{0.45\textwidth}{v6_7}{\cct Спектр синусоидальных колебаний, модулированных по~амплитуде}{7}
%\hfill
%\cpicskip

Спектр таких колебаний содержит три составляющих~--- основную компоненту и две боковых (\figref{spectrum-pre-7}). Первое слагаемое в~правой части (\r{14}) представляет собой исходное немодулированное колебание с~\emph{основной (несущей) частотой} $\omega_0$ и амплитудой $a_{осн}=A_0$. Второе и третье слагаемые соответствуют новым гармоническим колебаниям с~частотами ($\omega_0+\Omega$) и ($\omega_0-\Omega$). Амплитуды этих двух колебаний одинаковы и составляют $m/2$ от амплитуды немодулированного колебания: $a_{бок}={A_0m}/{2}$. Начальные фазы всех трёх колебаний одинаковы.

