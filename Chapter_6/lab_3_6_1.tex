\lab{Спектральный анализ электрических сигналов}
\setcounter{figure}{0}
\setcounter{equation}{0}
\todo{тут сброс нумерации}

\begin{lab:aim}
	изучить спектральный состав периодических электрических сигналов
\end{lab:aim}

\begin{lab:equipment}
	анализатор спектра, генератор прямоугольных импульсов, генератор сигналов специальной формы, осциллограф.
\end{lab:equipment}

\begin{figure}[h]
	\hfill
	\pic{0.8\textwidth}{6_1_1}
	\hspace*{\fill}
	\caption{Структурная схема анализатора спектра}
	\label{fig:611}
%	\figmark[611]
\end{figure}

В работе изучается спектральный состав периодических электрических сигналов различной формы: последовательности прямоугольных импульсов, последовательности цугов и амплитудно-модулированных гармонических колебаний. Спектры этих сигналов наблюдаются с~помощью промышленного анализатора спектра и сравниваются c рассчитанными теоретически.

%\bfno{Принцип работы спектроанализатора} Для исследования спектров в~работе используется гетеродинный анализатор спектра
%типа СК4-56. Упрощённая структурная схема, поясняющая последовательный супергетеродинный метод спектрального анализа
%внешнего сигнала, изображена на \p{1}.
%
%\cpic{6_1_1}{Структурная схема анализатора спектра}{1}

Восстановление спектрального состава входного сигнала $f(t)$ происходит периодически с некоторым заданным периодом. Это время является периодом повторения пилообразного напряжения, которое вырабатывается генератором развёртки. Линейно нарастающее во времени напряжение с генератора развёртки подаётся на гетеродин, который генерирует переменное напряжение с частотой пропорциональной этому напряжению, но с постоянной амплитудой. При изменении пилообразного напряжения от нуля до некоторого максимального значения частота сигналов, вырабатываемых гетеродином, изменяется в пределах от 128 до 188~кГц. Исследуемый сигнал $f(t)$ и переменное напряжение с гетеродина одновременно поступают на смеситель. При нелинейном сложении этих колебаний на выходе смесителя возникают сигналы суммарной и разностной частоты. Для~анализа используется только разностный сигнал. Смешение частот исследуемого сигнала и частоты гетеродина лежит в основе большинства современных радиоприёмных устройств~--- супергетеродинов.

Со смесителя сигнал поступает на фильтр, который настроен на частоту 128~кГц. Таким образом мы <<извлекаем>> из спектра входного  сигнала $f(t)$ переменное напряжение с частотой равной разности частот гетеродина и фильтра. За время, равное периоду повторения пилообразного напряжения, фильтр пропускает колебания с частотами от нуля до 60~кГц. Затем эти колебания детектируются, усиливаются и подаются на вертикальный вход электронно-лучевой трубки (ЭЛТ). Одновременно сигнал с генератора развёртки поступает на горизонтальный вход ЭЛТ. На экране анализатора возникает, таким образом, график, изображающий зависимость амплитуды гармоник от частоты, т.е. \emph{фурье-спектр} исследуемого сигнала.

\subsection*{Исследование спектра периодической последовательности прямоугольных импульсов}

%\eo Схема для исследования спектра периодической последовательности прямоугольных импульсов представлена на рис.~\r{r2}.

%\fcpic[1]{6_1_2}{\cct Схема для исследования спектра периодической последовательности прямоугольных импульсов}{2}
\begin{figure}[h]
	\hfill
	\pic{0.8\textwidth}{6_1_2}
	\hspace*{\fill}
	\label{fig:612}
%	\figmark[612]
	\caption{Схема для исследования спектра периодической последовательности прямоугольных импульсов}
\end{figure}

\experiment
Схема для исследования спектра периодической последовательности прямоугольных импульсов представлена на рис.~\ref{fig:612}.

Сигнал с~выхода генератора прямоугольных импульсов Г5-54 подаётся на вход анализатора спектра и одновременно~--- на вход~$Y$ осциллографа. С~генератора импульсов на осциллограф подаётся также сигнал синхронизации, запускающий ждущую развёртку осциллографа. При этом на экране осциллографа можно наблюдать саму последовательность прямоугольных импульсов, а на экране ЭЛТ анализатора спектра~--- распределение амплитуд спектральных составляющих этой последовательности.

В наблюдаемом спектре отсутствует информация об амплитуде нулевой гармоники, т.е. о~величине постоянной составляющей; её местоположение (начало отсчёта шкалы частот) отмечено небольшим вертикальным выбросом.

\begin{lab:task}

В этом упражнении исследуется зависимость ширины спектра периодической последовательности прямоугольных импульсов от длительности отдельного импульса.

\begin{enumerate}
	\item Соберите схему согласно рис.~\ref{fig:612} и подготовьте приборы к работе, следуя техническому описанию, расположенному на установке.
	\item Установите на анализаторе спектра режим работы с~однократной развёрткой и получите на экране спектр импульсов с~параметрами $f_{повт}=10^3$~Гц; $\uptau=25$~мкс; частотный масштаб $m_x=5$~кГц/дел.

	Проанализируйте, как меняется спектр ($\Delta\nu$ и $\delta\nu$ на рис.???:
	\todo{Ссылка на правильный рисунок из введения}
	%~\oref{v6_r3}): 
	а)~при увеличении $\uptau$ вдвое при неизменном $f_{повт}=1$~кГц; 
	б) при увеличении $f_{повт}$ вдвое при неизменном $\uptau=25$~мкс.
	
	Опишите результаты или зарисуйте в~тетрадь качественную картину.
	\item Проведите измерения зависимости ширины спектра от длительности импульса~$\Delta \nu(\uptau)$ при увеличении $\uptau$ от 25 до 200~мкс при $f_{повт}=1$~кГц.
	\item Скопируйте на кальку  огибающие спектров с параметрами: $f_{повт}=1$~кГц, $m_x=5$~кГц/дел, а)~$\uptau=50$~мкс, б)~$\uptau=100$~мкс. Запишите на кальках эти параметры и приложите кальки к~отчёту.
	\item Постройте график $\Delta \nu(1/\uptau)$ и по его наклону убедитесь в~справедливости соотношения неопределённостей.
\end{enumerate}
\end{lab:task}

\subsection*{Исследование спектра периодической~последовательности цугов гармонических~колебаний}

%\eo Исследование спектра периодически чередующихся цугов гармонических колебаний проводится по схеме, изображённой на \p{3}. Генератор \mbox{Г6-34} вырабатывает синусоидальные колебания высокой частоты. На вход АМ (амплитудная модуляция) этого генератора подаются прямоугольные импульсы с генератора Г5-54, а на выходе мы получаем высокочастотные модулированные  колебания в виде отдельных кусков синусоиды~--- \emph{цугов}.  Эти цуги с выхода генератора Г6-34 поступают на вход спектроанализатора и одновременно на вход $Y$ осциллографа. Сигнал синхронизации подаётся на вход $X$ осциллографа с~генератора импульсов.

%\fcpic[1]{6_1_3}{\cct Схема для исследования спектра периодической последовательности цугов высокочастотных колебаний}{3}
\begin{figure}
	\hfill
	\pic{0.8\textwidth}{6_1_3}
	\hspace*{\fill}
	\label{fig:613}
%	\figmark[613]
	\caption{Схема для исследования спектра периодической последовательности цугов высокочастотных колебаний}
\end{figure}

\begin{lab:task}

В этом упражнении исследуется зависимость расстояния между ближайшими спектральными компонентами от частоты повторения цугов.

\begin{enumerate}
	\item Соберите схему, изображённую на рис.~\ref{fig:613}, и подготовьте приборы к работе, руководствуясь техническим описанием.
	\item Установите частоту несущей $\nu_0=25$~кГц и проанализируйте, как изменяется вид спектра: а)~при увеличении длительности импульса вдвое ($\uptau=50,\;100$~мкс для $f_{повт}=1$~кГц); б)~при изменении частоты несущей: $\nu_0=25$, 10 или 40~кГц.
	
	Опишите результаты эксперимента или зарисуйте качественную картину в~тетради.
	\item При фиксированной длительности импульсов $\uptau=50$~мкс исследуйте зависимость расстояния~$\delta \nu$ между соседними спектральными компонентами от периода~$T$ (частоты повторения импульсов $f_{повт}$ в~диапазоне 1--8~кГц).
	\item Скопируйте на кальку спектры цугов с~параметрами: $\uptau=100$~мкс, $m_x=5$~кГц/дел; а)~$f_{повт}=1$~кГц; б)~$f_{повт}=2$~кГц.
	
	Запишите на кальках эти параметры и приложите кальки к~отчёту.
	
	\item Постройте график $\delta \nu(f_{повт})$ и по его наклону убедитесь в~справедливости соотношения неопределённости.
	\item Сравните зарисованные на кальку спектры:
	\begin{itemize}
		\item прямоугольных импульсов при одинаковых периодах и разных длительностях импульса $\uptau$;
		\item цугов при одинаковых $\uptau$ и разных периодах;
		\item цугов и прямоугольных импульсов при одинаковых значениях~$\uptau$ и~$T$.
	\end{itemize}
	
\end{enumerate}

\end{lab:task}

\subsection*{Исследование спектра гармонических сигналов, модулированных по амплитуде}

%\eo Схема для исследования амплитудно-модулированного сигнала представлена на \p{4}. Модуляционный генератор встроен в~левую часть генератора сигналов Г6-34 . Синусоидальный сигнал с~частотой модуляции $f_{мод}=1$~кГц подаётся с~модуляционного генератора на вход АМ (амплитудная модуляция) генератора, вырабатывающего синусоидальный сигнал высокой частоты (частота несущей~$\nu_0=25$~кГц). Амплитудно-модулированный сигнал с~основного выхода генератора поступает на осциллограф и на анализатор спектра.
%
%\fcpic[0.95]{6_1_4}{\cct Схема для исследования спектра высокочастотного гармонического сигнала, промодулированного по~амплитуде низкочастотным гармоническим~сигналом}{4}

\begin{figure}
	\hfill
	\pic{0.8\textwidth}{6_1_4}
	\hspace*{\fill}
	\label{fig:614}
%	\figmark[614]
	\caption{Схема для исследования спектра высокочастотного гармонического сигнала, промодулированного по~амплитуде низкочастотным гармоническим~сигналом}
\end{figure}

\begin{lab:task}
	В этом упражнении исследуется зависимость отношения амплитуд спектральных линий синусоидального сигнала, модулированного низкочастотными гармоническими колебаниями, от коэффициента модуляции, который определяется с~помощью осциллографа.
	\begin{enumerate}
		\item Соберите схему, изображённую на рис.~\ref{fig:614} и подготовьте приборы к работе, следуя техническому описанию.
		\item Изменяя глубину модуляции, исследуйте зависимость отношения амплитуды боковой линии спектра к~амплитуде основной линии ($a_{бок}/a_{осн}$) от глубины модуляции~$m$; для расчёта глубины модуляции~$m$ по формуле (???)
		\todo{правильная ссылка на правильную формулу из введения}
%		\item (\oref{v6_13}) 
		измеряйте максимальную~$2A_{\max}$ и минимальную~$2A_{\min}$ амплитуды сигнала на экране осциллографа (см. рис. ??? и ???)
		\todo{ссылки на рисунки из введения}
%		~\oref{v6_r6} и \oref{v6_r7}).
		\item При 100\% глубине модуляции ($A_{\min}=0$) посмотрите, как меняется спектр при увеличении частоты модулирующего сигнала.
		\item Постройте график отношения $a_{бок}/a_{осн}$ в~зависимости от~$m$. Определите угол наклона графика и сравните с~рассчитанным с помощью формулы~(???).
		\todo{Правильная ссылка на формулу из введения}
%		\oref{v6_14}).
	\end{enumerate}
\end{lab:task}

\begin{lab:questions}
	\item Нарисуйте спектры $F(\omega)$:
	\begin{itemize}
		\item бесконечно длинной синусоиды;
		\item синусоиды конечной длины;
		\item периодической последовательности цугов;
		\item периодической последовательности прямоугольных импульсов;
		\item одного цуга;
		\item одного прямоугольного импульса.
	\end{itemize}
	
	\item Как изменится спектр периодической последовательности прямоугольных импульсов, если убрать каждый второй импульс? Как выглядит спектр, если повторять эту процедуру, пока не останется один импульс?
	\item Найдите спектр синусоидальных колебаний, модулированных по фазе:
$f(t)=A_0\cos(\omega t + m \cos \Omega t),$
считая $m\ll 1$.

Сравните со спектром синусоиды, модулированной по амплитуде.

\end{lab:questions}


%\n Как изменится спектр периодической последовательности цугов, если время общей продолжительности цугов уменьшится
%вдвое? Как изменяется вид спектра, если повторять эту процедуру, пока не останется один цуг?

%\lit

%\n \emph{Сивухин Д.В.} Общий курс физики. Т.~III. Электричество~--- М.: Наука, 1983. \S~128.

%\n \emph{Крауфорд Ф.} Берклеевский курс физики. Т.~III. Волны.~--- М.: Наука, 1976. \S~6.4.

