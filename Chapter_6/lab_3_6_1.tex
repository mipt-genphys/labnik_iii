\lab{Спектральный анализ электрических сигналов}
\aim{изучить спектральный состав периодических электрических сигналов.}

\equip{анализатор спектра, генератор прямоугольных импульсов, генератор сигналов
специальной формы, осциллограф.}

Перед выполнением работы необходимо ознакомиться с теоретическим введением
к разделу.

В работе изучается спектральный состав периодических электрических сигналов
различной формы: последовательности прямоугольных импульсов, последовательности
цугов и амплитудно-модулированных гармонических колебаний. Спектры этих сигналов
наблюдаются с помощью анализатора спектра и сравниваются c рассчитанными
теоретически.

Периодическая функция может быть представлена в виде бесконечного ряда
гармонических функций --- ряда Фурье (см. п.~\ref{sec:spectrum-periodic}
Введения):
\[
f(t) = \sum_{n=-\infty}^{\infty} c_n e^{in\omega_0 t}\qquad\text{или}\qquad
f=\sum_{n=0}^{\infty} a_n \cos (n\omega_0 t + \varphi_n).
\]
Здесь $\omega_0 = 2\pi/T$, где $T$ --- период функции $f(t)$.
Коэффициенты $\{c_n\}$ могут быть найдены по формуле
\chaptereqref{coef-Fourier-periodic osc}:
\[
    c_n=\frac{1}{T}\int\limits_{0}^{T} f(t)e^{-in\omega_0 t}\,dt.
\]
Наборы коэффициентов разложения в комплексной $\{c_n\}$ и действительной
$\{a_n,\varphi_n\}$ формах связаны соотношением \chaptereqref{Fourier-coefficient}.


В качестве простейшего спектрального анализатора можно использовать
высокодобротный колебательный контур с подстраиваемой ёмкостью или
индуктивностью, рис.~\figref{simple-RLC}
(см. также п.~\ref{sec:spectrum-meaning} Введения).
Такой контур усиливает те гармоники входного сигнала $f(t)$, частота которых близка к резонансной
$\Omega = 1/\sqrt{LC}$, и практически не реагирует на частоты,
далёкие от $\Omega$. С точки зрения преобразования гармоник колебательный контур
по сути является узкополосным \important{фильтром} с шириной полосы
пропускания порядка $\Delta \Omega \sim \Omega / Q$, где $Q =
\frac{1}{R}\sqrt{\frac{L}{C}} \gg 1$~--- его добротность. Амплитуда колебаний
в контуре пропорциональна амплитуде~$|c_n|$ гармоники $\omega_n=\Omega$
в фурье-спектре функции $f(t)$. Таким образом, меняя резонансную частоту контура
можно <<просканировать>> весь спектр.

У описанной выше схемы есть существенный недостаток: при изменении~$L$ или~$C$
меняется также и добротность, а значит и ширина полосы пропускания.
Кроме того, изготовление настраиваемых контуров с высокой добротностью
оказалось непрактично. В~связи с этим, как правило, для фильтрации сигнала
применяется другая схема.

Исследуемый сигнал $f(t)$ и сигнал от вспомогательного генератора гармонических
колебаний частоте~$\omega_{гет}$ (генератор в таких схемах называют
\important{гетеродином}) подаются на вход \important{смесителя}. Смеситель
представляет собой нелинейный элемент, который из колебаний с частотами
$\omega_1$ и $\omega_2$ на выходе генерирует сигналы на \emph{комбинированных}
частотах: $\omega_1 + \omega_2$ и $\omega_1 - \omega_2$.
<<Разностный>> сигнал смесителя поступает на фильтр ---
высокодобротный колебательный контур, настроенный на некоторую \emph{фиксированную}
частоту $\omega_0$. Таким образом, если $f(t)$ содержит гармонику
$\omega_n=\omega_{гет}-\omega_0$, она будет усилена, а отклик будет
пропорционален её амплитуде.

Отметим, что смешение частот исследуемого сигнала и частоты гетеродина лежит в
основе большинства современных радиоприёмных устройств~---
\emph{супергетеродинов}.

\begin{figure}[h!]
\hfil
\pic{0.9\textwidth}{Chapter_6/6_1_1}
\caption{Структурная схема анализатора спектра}
\figmark{Spectrum analyzer}
\end{figure}

В спектральном анализаторе частота гетеродина пропорциональна напряжению,
подаваемому на развертку по оси~$X$ встроенного в анализатор осциллографа.
Выходной сигнал подаётся на канал~$Y$. На экране анализатора возникает, таким
образом, график, изображающий зависимость амплитуды гармоник исходного сигнала
от частоты, т.\,е. его спектр (заметим, что информация о фазах гармоник при этом
теряется).

\experiment

\labsection{Исследование спектра периодической последовательности прямоугольных
импульсов}

\begin{figure}[h!]
\centering
\pic{0.9\textwidth}{6_1_2}
\caption{Схема для исследования спектра периодической последовательности
прямоугольных импульсов}
\figmark{Scheme for square pulses}
\end{figure}

Схема для исследования спектра периодической последовательности прямоугольных
импульсов представлена на рис.~\figref{Scheme for square pulses}.

Сигнал с~выхода генератора прямоугольных импульсов Г5-54 подаётся на вход
анализатора спектра и одновременно~--- на вход~$Y$ осциллографа. С~генератора
импульсов на осциллограф подаётся также сигнал синхронизации, запускающий ждущую
развёртку осциллографа. При этом на экране осциллографа можно наблюдать саму
последовательность прямоугольных импульсов, а на экране ЭЛТ анализатора
спектра~--- распределение амплитуд спектральных составляющих этой
последовательности.

В наблюдаемом спектре отсутствует информация об амплитуде нулевой гармоники,
т.е. о~величине постоянной составляющей; её местоположение (начало отсчёта шкалы
частот) отмечено небольшим вертикальным выбросом.

\begin{lab:task}

В этом упражнении исследуется зависимость ширины спектра периодической
последовательности прямоугольных импульсов от длительности отдельного импульса.

\begin{enumerate}
\item Соберите схему согласно рис.~\figref{Scheme for square pulses} и
подготовьте приборы к работе, следуя техническому описанию, расположенному на
установке.

\item Установите на анализаторе спектра режим работы с~однократной
развёрткой и получите на экране спектр импульсов с параметрами
$f_\text{повт}==10^3$~Гц; $\tau=25$~мкс; частотный масштаб $m_x=5$~кГц/дел.
\todo[inline]{Что было между знаками равно?}

Проанализируйте, как меняется спектр ($\Delta\nu$ и $\delta\nu$ на
рис.???):
\todo[inline,author = Andrew]{уточнить номер рисунка из введения}
%~\oref{v6_r3}):
а) при увеличении $\tau$ вдвое при неизменном $f_\text{повт}=1$~кГц;
б) при увеличении $f_\text{повт}$ вдвое при неизменном $\tau=25$~мкс.

Опишите результаты или зарисуйте в~тетрадь качественную картину.

\item Проведите измерения зависимости ширины спектра от длительности
импульса~$\Delta \nu(\tau)$ при увеличении $\tau$ от 25 до 200~мкс при
$f_\text{повт}=1$~кГц.

\item Скопируйте на кальку  огибающие спектров с параметрами:
$f_\text{повт}=1$~кГц, $m_x=5$~кГц/дел, а)~$\tau=50$~мкс, б)~$\tau=100$~мкс.
Запишите на кальках эти параметры и приложите кальки к~отчёту.

\item Постройте график $\Delta \nu(1/\tau)$ и по его наклону убедитесь
в~справедливости соотношения неопределённостей.
\end{enumerate}
\end{lab:task}

\labsection{Исследование спектра периодической~последовательности цугов
гармонических~колебаний}

%\eo Исследование спектра периодически чередующихся цугов гармонических
% колебаний проводится по схеме, изображённой на \p{3}. Генератор \mbox{Г6-34}
% вырабатывает синусоидальные колебания высокой частоты. На вход АМ (амплитудная
% модуляция) этого генератора подаются прямоугольные импульсы с генератора Г5-54,
% а на выходе мы получаем высокочастотные модулированные  колебания в виде
% отдельных кусков синусоиды~--- \emph{цугов}.  Эти цуги с выхода генератора Г6-34
% поступают на вход спектроанализатора и одновременно на вход $Y$ осциллографа.
% Сигнал синхронизации подаётся на вход $X$ осциллографа с~генератора импульсов.

\begin{figure}[h!]
\centering
\pic{0.9\textwidth}{6_1_3}
\caption{Схема для исследования спектра периодической последовательности
цугов высокочастотных колебаний}
\figmark{Scheme for high-frequency oscillations}
\end{figure}

\begin{lab:task}

В этом упражнении исследуется зависимость расстояния между ближайшими
спектральными компонентами от частоты повторения цугов.

\begin{enumerate}
\item Соберите схему, изображённую на рис.~\figref{Scheme for
high-frequency oscillations}, и подготовьте приборы к работе, руководствуясь
техническим описанием.

\item Установите частоту несущей $\nu_0=25$~кГц и проанализируйте, как
изменяется вид спектра: а)~при увеличении длительности импульса вдвое
($\tau=50,\;100$~мкс для $f_\text{повт}=1$~кГц); б)~при изменении частоты
несущей: $\nu_0=25$, 10 или 40~кГц.

Опишите результаты эксперимента или зарисуйте качественную картину
в~тетради.

\item При фиксированной длительности импульсов $\tau=50$~мкс исследуйте
зависимость расстояния~$\delta \nu$ между соседними спектральными компонентами
от периода~$T$ (частоты повторения импульсов $f_\text{повт}$ в~диапазоне
1--8~кГц).

\item Скопируйте на кальку спектры цугов с~параметрами: $\tau=100$~мкс,
$m_x=5$~кГц/дел; а)~$f_\text{повт}=1$~кГц; б)~$f_\text{повт}=2$~кГц.

Запишите на кальках эти параметры и приложите кальки к~отчёту.

\item Постройте график $\delta \nu(f_\text{повт})$ и по его наклону
убедитесь в~справедливости соотношения неопределённости.

\item Сравните зарисованные на кальку спектры:
\begin{itemize}
	\item прямоугольных импульсов при одинаковых периодах и разных
длительностях импульса $\tau$;
	\item цугов при одинаковых $\tau$ и разных периодах;
	\item цугов и прямоугольных импульсов при одинаковых значениях~$\tau$
и~$T$.
\end{itemize}
\end{enumerate}

\end{lab:task}

\labsection{Исследование спектра гармонических сигналов, модулированных по
амплитуде}

%\eo Схема для исследования амплитудно-модулированного сигнала представлена на
% \p{4}.
% Модуляционный генератор встроен в~левую часть генератора сигналов Г6-34 .
% Синусоидальный сигнал с~частотой модуляции $f_{мод}=1$~кГц подаётся
% с~модуляционного генератора на вход АМ (амплитудная модуляция) генератора,
% вырабатывающего синусоидальный сигнал высокой частоты (частота
% несущей~$\nu_0=25$~кГц). Амплитудно-модулированный сигнал с~основного выхода
% генератора поступает на осциллограф и на анализатор спектра.

\begin{figure}[h!]
\centering
\pic{0.9\textwidth}{6_1_4}
\caption{Схема для исследования спектра высокочастотного гармонического
сигнала, промодулированного по~амплитуде низкочастотным гармоническим~сигналом}
\figmark{Scheme for amplitude modulation}
\end{figure}

\begin{lab:task}
В этом упражнении исследуется зависимость отношения амплитуд
спектральных линий синусоидального сигнала, модулированного низкочастотными
гармоническими колебаниями, от коэффициента модуляции, который определяется
с~помощью осциллографа.
\begin{enumerate}
\item Соберите схему, изображённую на рис.~\figref{Scheme for
amplitude modulation} и подготовьте приборы к работе, следуя техническому
описанию.

\item Изменяя глубину модуляции, исследуйте зависимость
отношения амплитуды боковой линии спектра к~амплитуде основной линии
($a_\text{бок}/a_\text{осн}$) от глубины модуляции~$m$; для расчёта глубины
модуляции~$m$ по формуле (???)
\todo[inline,author = Andrew]{уточнить номер формулы из введения}
измеряйте максимальную~$2A_{max}$ и минимальную~$2A_{min}$
амплитуды сигнала на экране осциллографа (см. рис. ??? и ???)
\todo[inline,author = Andrew]{уточнить номера рисунков из введения}
%		~\oref{v6_r6} и \oref{v6_r7}).

\item При 100\% глубине модуляции ($A_{min}=0$) посмотрите, как
меняется спектр при увеличении частоты модулирующего сигнала.

\item Постройте график отношения $a_\text{бок}/a_\text{осн}$
в~зависимости от~$m$. Определите угол наклона графика и сравните с~рассчитанным
с помощью формулы~(???).
\todo[inline,author = Andrew]{уточнить номер формулы из введения}
%		\oref{v6_14}).
\end{enumerate}
\end{lab:task}

\begin{lab:questions}
\item Нарисуйте спектры $F(\omega)$:
\begin{itemize}
	\item бесконечно длинной синусоиды;
	\item синусоиды конечной длины;
	\item периодической последовательности цугов;
	\item периодической последовательности прямоугольных импульсов;
	\item одного цуга;
	\item одного прямоугольного импульса.
\end{itemize}

\item Как изменится спектр периодической последовательности
прямоугольных импульсов, если убрать каждый второй импульс? Как выглядит спектр,
если повторять эту процедуру, пока не останется один импульс?

\item Найдите спектр синусоидальных колебаний, модулированных по фазе:
$f(t)=A_0\cos(\omega t + m \cos \Omega t),$
считая $m\ll 1$.

Сравните со спектром синусоиды, модулированной по амплитуде.

\end{lab:questions}


%\n Как изменится спектр периодической последовательности цугов, если время
% общей продолжительности цугов уменьшится
%вдвое? Как изменяется вид спектра, если повторять эту процедуру, пока не
% останется один цуг?

%\lit

%\n \emph{Сивухин Д.В.} Общий курс физики. Т.~III. Электричество~--- М.: Наука,
% 1983. \S~128.

%\n \emph{Крауфорд Ф.} Берклеевский курс физики. Т.~III. Волны.~--- М.: Наука,
% 1976. \S~6.4.
