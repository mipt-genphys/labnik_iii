\lab{Синтез гармонических сигналов}

\aim{изучить возможности синтезирования периодических электрических сигналов при ограниченном наборе спектральных компонент.}

\equip{генератор гармонических сигналов Г6-1, источник питания, осциллограф.}

Сколь угодно сложный электрический сигнал $V(t)$ может быть разложен на более простые сигналы. В~радиотехнике широко используется разложение сигнала $V(t)$ на совокупность гармонических сигналов различных частот $\omega$. Функция $F(\omega)$, описывающая зависимость амплитуд отдельных гармоник от частоты, называется \important{амплитудной спектральной характеристикой} сигнала $V(t)$. Представление сложного периодического сигнала в виде суммы дискретных гармонических сигналов в~математике называется \important{разложением в ряд Фурье (прямое преобразование Фурье).}

Зная спектральный состав $F(\omega)$ периодической последовательности некоторого импульса $V(t)$, мы можем осуществить \important{обратное преобразование Фурье}: сложив отдельные гармоники со своими амплитудами и фазами, получить необходимую последовательность импульсов. Степень совпадения полученного сигнала с~$V(t)$ определяется количеством синтезированных гармоник: чем их больше, тем лучше совпадение.

Рассмотрим конкретные примеры периодических функций, которые будут предметом исследования в нашей работе.

\labsection{Периодическая последовательность прямоугольных импульсов (рис.~\figref{Square pulses})}
%{\bf I. Периодическая последовательность прямоугольных импульсов} (\p{1}). 
Амплитуда импульсов равна~$V_0$, длительность отдельного импульса~--- $\uptau$, частота повторения $f_{повт}=1/T,$ где $T$~--- период повторения. Отношение $\frac{T}{\uptau}=7$.

Применяя формулы (???),
\todo[author = Andrew]{уточнить номера формул из введения или переписать их здесь}
%(\oref{v6_1})~---~(\oref{v6_4}), 
найдём среднее значение (постоянную составляющую):
\begin{equation}
	\langle V \rangle = \frac{a_0}{2} = \frac{A_0}{2} = \frac{1}{T}\int\limits_{-\uptau/2}^{\uptau/2} \!V_0\,dt = V_0 \frac{\uptau}{T}.
	\eqmark{6.2.1}
\end{equation}
%\[
%\sr{V}=\frac{a_0}{2}=\frac{A_0}{2}= \frac{1}{T}\int_{-\uptau/2}^{\uptau/2} \!V_0\,dt=V_0\frac{\uptau}{T}.
%\]

Амплитуды косинусных составляющих равны:
\begin{equation}
	a_n = \frac{2}{T} \int\limits_{-\uptau/2}^{\uptau/2} \!V_0\cos(n\Omega_1t)\,dt = 2V_0 \frac{\uptau}{T} \frac{\sin(n\Omega_1\uptau/2)}{n\Omega_1\uptau/2} \sim \frac{\sin x}{x}.
	\eqmark{6.2.2}
\end{equation}
%\be1
%a_n=\frac{2}{T}\int\limits_{-\uptau/2}^{\uptau/2} \!V_0\cos(n\Omega_1t)\,dt=2V_0\frac{\uptau}{T}\frac{\sin
%(n\Omega_1\uptau/2)}{n\Omega_1\uptau/2} \sim \frac{\sin x}{x}.
%\ee

%\begin{figure}[t]%\cpicskip
%\noindent
%\hfill\piccapt{0.4\textwidth}{6_2_1}{\cct Периодическая последовательность прямоугольных~импульсов}{1}
%\hfill\piccapt{0.5\textwidth}{6_2_2}{\cct Спектр периодической последовательности прямоугольных~импульсов}{2}
%\hfill
%\uvcs
%\end{figure}%\cpicskip

\begin{figure}[h!]
\begin{minipage}{0.45\textwidth}
	\pic{0.45\textwidth}{6_2_1}
	\caption{Периодическая последовательность прямоугольных импульсов}
	\figmark{Square pulses}
\end{minipage}
\hfill
\begin{minipage}{0.45\textwidth}
	\pic{0.45\textwidth}{6_2_2}
	\caption{Спектр периодической последовательности прямоугольных~импульсов}
	\figmark{Spectrum of square pulses}
\end{minipage}
\end{figure}

Поскольку наша функция чётная, все амплитуды синусоидальных гармоник $b_n=0$. В этом случае $A_n=a_n$, а начальная фаза
колебаний $\psi_n=0$ в области частот

\begin{equation}
	\frac{4\pi n}{\uptau}<\Omega_n<\frac{2\pi (2n+1)}{\uptau}
	\eqmark{6.2.3}
\end{equation}
%\be2
%\frac{4\pi n}{\uptau}<\Omega_n<\frac{2\pi (2n+1)}{\uptau}
%\ee
или $\psi_n=\pi$ в области

\begin{equation}
	\frac{2\pi(2n+1)}{\uptau} < \Omega_n < \frac{2\pi (2n+2)}{\uptau},
	\eqmark{6.2.4}
\end{equation}
%\be2'
%\frac{2\pi(2n+1)}{\uptau}<\Omega_n<\frac{2\pi (2n+2)}{\uptau},
%\ee
где $n=0$, 1, 2, $\dots$ Спектральное распределение амплитуд дискретных гармоник для бесконечной последовательности прямоугольных импульсов при отношении $T/\uptau=7$ представлено на рис.~\figref{Spectrum of square pulses}.
<<Отрицательные>> амплитуды на рисунке соответствуют тем гармоникам, фаза которых $\psi_n=\pi$.

%{\bf II. Периодическая последовательность треугольных импульсов} с отношением $T/\uptau=3{,}5$ (рис.~\r{r3}).
\labsection{Периодическая последовательность треугольных импульсов с отношением $T/\uptau=3{,}5$ (рис.~\figref{Triangular pulses})}

Амплитуды в спектре последовательности треугольных импульсов меняются по закону
\begin{equation}
	|A_n|=\frac{V_0\uptau}{2T} \left(\frac{\sin[\pi n\uptau/(2T)]} {\pi n\uptau/(2T)}\right)^2.
	\eqmark{6.2.5}
\end{equation}
%\be3
%|A_n|=\frac{V_0\uptau}{2T} \left(\frac{\sin[\pi n\uptau/(2T)]} {\pi n\uptau/(2T)}\right)^2.
%\ee

Фаза $n$-й гармоники $\psi_n=0$ в области частот
\begin{equation}
	\frac{8\pi n}{\uptau}<\Omega_n<\frac{4\pi (2n+1)}{\uptau}
	\eqmark{6.2.6}
\end{equation}
%\be4
%\frac{8\pi n}{\uptau}<\Omega_n<\frac{4\pi (2n+1)}{\uptau}
%\ee
или $\psi_n=\pi$ в области
\begin{equation}
	\frac{4\pi(2n+1)}{\uptau}<\Omega_n<\frac{4\pi(2n+2)}{\uptau}.
	\eqmark{6.2.7}
\end{equation}
%\be4'
%\frac{4\pi(2n+1)}{\uptau}<\Omega_n<\frac{4\pi(2n+2)}{\uptau}.
%\ee

Модуль спектральной плотности $|a_n(\omega)|=|A_n|$ для такой функции представлен на рис.~\figref{Spectr of triangular pulses}.

\begin{figure}[t]
\begin{minipage}{0.45\textwidth}
	\pic{0.45\textwidth}{6_2_3}
	\caption{Периодическая последовательность треугольных~импульсов}
	\figmark{Triangular pulses}
\end{minipage}
\hfill
\begin{minipage}{0.45\textwidth}
	\pic{0.45\textwidth}{6_2_4}
	\caption{Спектр периодической последовательности треугольных~импульсов ($T/\uptau=3,5$)}
	\figmark{Spectr of triangular pulses}
\end{minipage}
\end{figure}

%\begin{figure}[!t]%\cpicskip \noindent
%\noindent
%\hfill\piccapt{0.45\textwidth}{6_2_3}{\cct Периодическая последовательность треугольных~импульсов}{3}
%\hfill\piccapt{0.5\textwidth}{6_2_4}
%{\cct Спектр периодической последовательности треугольных~импульсов ($T/\uptau=3,5$)}{4}
%\hfill
%\uvcs
%\end{figure}%\cpicskip

%\eo 
\experiment
Основным элементом экспериментальной установки является генератор гармонических сигналов Г6-1, который генерирует
одновременно основной сигнал (на выбранной частоте) и пять гармоник, кратных основному сигналу и синхронных с ним.
Например, если частота основного сигнала (1-я гармоника) составляет 1~кГц, то частоты остальных пяти гармоник~--- 2~кГц,
3~кГц, 4~кГц, 5~кГц и 6~кГц. Все 6 гармоник могут складываться при помощи электронного сумматора, на выходе которого
образуется сигнал сложной формы. Этот сигнал с выхода генератора подаётся на вход $Y$ осциллографа, на экране которого
можно наблюдать (в режиме непрерывной развёртки) периодическую последовательность синтезированных сигналов. Технические
данные генератора и порядок работы с ним изложены в отдельном техническом описании, расположенном на установке.

%\zad
\begin{lab:task}
	
В работе предлагается подобрать амплитуды синусоидальных колебаний с кратными частотами, сумма которых даёт
периодическую последовательность прямоугольных или треугольных импульсов.

\tasksection{Синтез последовательности прямоугольных импульсов}

\begin{enumerate}
	\item Прочтите техническое описание (ТО) генератора Г6-1.
	\item\label{task} Включите в сеть блок питания генератора. За время прогрева генератора рассчитайте относительные значения амплитуд первых шести гармоник в спектре периодической последовательности прямоугольных импульсов с отношением $T/\uptau=7$: нулевая гармоника (постоянная составляющая) не используется; первая гармоника соответствует основному сигналу генератора; приняв амплитудное значение первой гармоники за единицу, относительные амплитудные значения ($a_n/a_1$) остальных пяти гармоник рассчитайте по формуле \eqref{6.2.2}. Амплитуда седьмой гармоники в наших условиях (при $T/\uptau=7$) равна нулю. Значения синусов, необходимые для вычислений, приведены в таблице:
	
\begin{center}
\begin{tabular}{|c|c|c|c|c|c|c|} \hline
$n$ & 1 & 2 & 3 & 4 & 5 & 6 \\ \hline $\alpha_n$&$\pi /7$&$2\pi /7$&$3\pi /7$&$4\pi /7$&$5\pi /7$& $6\pi /7$ \\ \hline
$\sin \alpha_n$ & $0{,}434$ & $0{,}782$ & $0{,}975$ & $0{,}975$ &$0{,}782$ & $0{,}434$ \\ \hline $a_n\sim\frac{\sin
\alpha_n}{n}$& & & & & & \\ \hline $\frac{a_n}{a_1}$&1& & & & & \\ \hline
\end{tabular}
\end{center}
	
	\item Установите частоту первой гармоники~--- 10 кГц (см.~ТО,~I) и откалибруйте (уравняйте) напряжения гармоник.
	\item Включите в сеть осциллограф и проведите регулировку фазы гармоник. С помощью осциллографа установите рассчитанные Вами относительные амплитуды гармоник.
	\item Последовательно увеличивая число гармоник, копируйте на кальку сигнал, возникающий на экране осциллографа. По результирующей осциллограмме, соответствующей сумме всех шести гармоник, определите отношение $T/\uptau$ и сравните его с теоретическим значением.
\end{enumerate}

%\begin{center}
%\begin{tabular}{|c|c|c|c|c|c|c|} \hline
%$n$ & 1 & 2 & 3 & 4 & 5 & 6 \\ \hline $\alpha_n$&$\pi /7$&$2\pi /7$&$3\pi /7$&$4\pi /7$&$5\pi /7$& $6\pi /7$ \\ \hline
%$\sin \alpha_n$ & $0{,}434$ & $0{,}782$ & $0{,}975$ & $0{,}975$ &$0{,}782$ & $0{,}434$ \\ \hline $a_n\sim\frac{\sin
%\alpha_n}{n}$& & & & & & \\ \hline $\frac{a_n}{a_1}$&1& & & & & \\ \hline
%\end{tabular}
%\end{center}

\tasksection{Синтез последовательности треугольных импульсов}
\begin{enumerate}
	\item Рассчитайте с помощью формулы \eqref{6.2.5} относительные амплитуды гармоник в спектре периодической последовательности треугольных импульсов с отношением $T/\uptau=3,5$. Для этого возведите в~квадрат относительные амплитудные значения гармоник для спектра прямоугольных импульсов (см. п.~\ref{task}).
	\item Установите относительные амплитуды гармоник.
	\item Получите осциллограмму от всех шести гармоник и скопируйте её на кальку. Определите отношение $T/\uptau$ и сравните его с теоретическим.
	\item Закончив упражнение с реальным генератором, переходите к компьютерному варианту работы (см. дополнительное описание).
%	\item ~\r{p3}).
\end{enumerate}

%\n Прочтите техническое описание (ТО) генератора Г6-1.

%\n [p3] Включите в сеть блок питания генератора. За время прогрева генератора рассчитайте относительные значения
%амплитуд первых шести гармоник в спектре периодической последовательности прямоугольных импульсов с отношением
%$T/\uptau=7$: нулевая гармоника (постоянная составляющая) не используется; первая гармоника соответствует основному
%сигналу генератора; приняв амплитудное значение первой гармоники за единицу, относительные амплитудные значения
%($a_n/a_1$) остальных пяти гармоник рассчитайте по формуле (\r1). Амплитуда седьмой гармоники в наших условиях (при
%$T/\uptau=7$) равна нулю. Значения синусов, необходимые для вычислений, приведены в таблице:

%\newpage
%\begin{center}
%\begin{tabular}{|c|c|c|c|c|c|c|} \hline
%$n$ & 1 & 2 & 3 & 4 & 5 & 6 \\ \hline $\alpha_n$&$\pi /7$&$2\pi /7$&$3\pi /7$&$4\pi /7$&$5\pi /7$& $6\pi /7$ \\ \hline
%$\sin \alpha_n$ & $0{,}434$ & $0{,}782$ & $0{,}975$ & $0{,}975$ &$0{,}782$ & $0{,}434$ \\ \hline $a_n\sim\frac{\sin
%\alpha_n}{n}$& & & & & & \\ \hline $\frac{a_n}{a_1}$&1& & & & & \\ \hline
%\end{tabular}
%\end{center}

%\n Установите частоту первой гармоники~--- 10 кГц (см.~ТО,~I) и откалибруйте (урaвняйте) напряжения гармоник.

%\n Включите в сеть осциллограф и проведите регулировку фазы гармоник. С помощью осциллографа установите рассчитанные
%Вами относительные амплитуды гармоник.

%\n Последовательно увеличивая число гармоник, копируйте на кальку сигнал, возникающий на экране осциллографа. По
%результирующей осциллограмме, соответствующей сумме всех шести гармоник, определите отношение $T/\uptau$ и сравните его
%с теоретическим значением.

%\zn Синтез последовательности треугольных импульсов

%\n Рассчитайте с помощью формулы (\r3) относительные амплитуды гармоник в спектре периодической последовательности
%треугольных импульсов с отношением $T/\uptau=3,5$. Для этого возведите в~квадрат относительные амплитудные значения
%гармоник для спектра прямоугольных импульсов (см. п.~\r{p3}).

%\n Установите относительные амплитуды гармоник.

%\n Получите осциллограмму от всех шести гармоник и скопируйте её на кальку. Определите отношение $T/\uptau$ и сравните
%его с теоретическим.

%\n Закончив упражнение с реальным генератором, переходите к компьютерному варианту работы (см. дополнительное описание).
\end{lab:task}

\begin{lab:questions}

\item Нарисуйте спектры $F(\omega)$:
\begin{itemize}
	\item бесконечно длинной синусоиды;
	\item синусоиды конечной длины;
	\item периодической последовательности цугов;
	\item периодической последовательности прямоугольных импульсов;
	\item одного цуга;
	\item одного прямоугольного импульса.
\end{itemize}
\item Как изменится спектр периодической последовательности прямоугольных импульсов, если убрать каждый второй импульс? Как
выглядит спектр, если повторять эту процедуру, пока не останется один импульс?

%\n Как изменится спектр периодической последовательности цугов, если время общей продолжительности цугов уменьшится
%вдвое? Как изменяется вид спектра, если повторять эту процедуру, пока не останется один цуг?

\item Найдите спектр синусоидальных колебаний, модулированных по фазе:
\begin{equation*}
	f(t)=A_0\cos(\omega t + m \cos \Omega t),
\end{equation*}
%
считая $m\ll 1$. Сравните со спектром синусоиды, модулированной по амплитуде.

\end{lab:questions}

%\lit

%\n \emph{Сивухин Д.В.} Общий курс физики. Т.~III. Электричество~--- М.: Наука, 1983. \S~128.

%\n \emph{Крауфорд Ф.} Берклеевский курс физики. Т.~III. Волны.~--- М.: Наука, 1976. \S~6.4.

