\lab{Синтез гармонических сигналов}

\aim{изучить возможности синтезирования периодических электрических сигналов при
ограниченном наборе спектральных компонент.}

\equip{генератор гармонических сигналов, источник питания, осциллограф.}

Перед выполнение работы необходимо ознакомиться с теоретическим введением
к разделу, в частности с главой \ref{sec:synth}~<<Синтез сигналов>>.

% В работе изучается спектральный состав периодических электрических сигналов
% различной формы: последовательности прямоугольных импульсов, последовательности
% цугов и амплитудно-модулированных гармонических колебаний. Спектры этих сигналов
% наблюдаются с помощью анализатора спектра и сравниваются c рассчитанными
% теоретически.

Периодическая функция может быть представлена в виде бесконечного ряда
гармонических функций --- ряда Фурье (см. п.~\ref{sec:spectrum-periodic}
Введения):
\begin{equation*}
f(t) = \sum_{n=-\infty}^{\infty} c_n e^{in\omega_0 t}\qquad\text{или}\qquad
f=\sum_{n=0}^{\infty} a_n \cos (n\omega_0 t + \varphi_n).
\end{equation*}
Здесь $\omega_0 = 2\pi/T$, где $T$ --- период функции $f(t)$.
Коэффициенты $\{c_n\}$ могут быть найдены по формуле
\chaptereqref{fourier-coeff}:
\begin{equation*}
    c_n=\frac{1}{T}\int\limits_{0}^{T} f(t)e^{-in\omega_0 t}\,dt.
\end{equation*}
Наборы коэффициентов разложения в комплексной $\{c_n\}$ и действительной
$\{a_n,\varphi_n\}$ формах связаны соотношением \chaptereqref{Fourier-coefficient}:
\begin{equation*}
a_n = 2|c_n|,\qquad \varphi_n = \arg c_n.
\end{equation*}

Первые несколько слагаемых ряда Фурье (частичная сумма) можно использовать
для приближённого восстановления исходного сигнала.
В работе предлагается синтезировать периодические сигналы специальной формы
с помощью нескольких генераторов гармонических сигналов.
Рассмотрим конкретные примеры функций, которые будут предметом
исследования в нашей работе.

\paragraph{Периодическая последовательность прямоугольных импульсов.}
Рассмотрим импульсы единичной амплитуды c частотой повторения
$\nu_{0}=1/T$ ($T$~--- период) и длительностью каждого
импульса $\tau$ (рис.~\figref{Square pulses}).

Амплитуды гармоник с частотами $\nu_n = n \nu_0$ для этой функции были найдены
в разделе~\ref{sec:spectrum-periodic}. Для действительного представления
(ряд косинусов) из \chaptereqref{impulses-spectrum} имеем:
\begin{equation}
\eqmark{imp-a}
a_n = \frac{2\tau}{T}  \frac{\left|\sin \pi n \tau/T\right|}{\pi n \tau/T}
\propto \frac{|\sin \xi|}{\xi},
\end{equation}
где $\xi = \pi n \tau /T$. Для нулевой, т.\,е. постоянной, составляющей имеем
$a_0 = \frac{\tau}{T}$. При этом начальные фазы принимают значения
\begin{equation}
\eqmark{imp-p}
\varphi_n = \begin{cases}
    0, & \text{при~} \sin \frac{\pi n \tau}{T} \ge 0,\\
    \pi, & \text{при~} \sin \frac{\pi n \tau}{T} < 0.\\
\end{cases}
\end{equation}
Соответствующий спектр представлено на рис.~\figref{Spectrum of square pulses}
(<<отрицательные>> амплитуды на рисунке соответствуют гармоникам,
фаза которых $\varphi_n=\pi$).

\begin{figure}[h!]
\begin{minipage}{0.45\textwidth}
	\pic{\linewidth}{Chapter_6/6_2_1}
	\caption{Периодическая последовательность прямоугольных импульсов}
	\figmark{Square pulses}
\end{minipage}
\hfill
\begin{minipage}{0.45\textwidth}
    \pic{\linewidth}{Chapter_6/6_2_2}
	\caption{Спектр периодической последовательности прямоугольных~импульсов
    (при $T/\tau=7$)}
	\figmark{Spectrum of square pulses}
\end{minipage}
\end{figure}


\paragraph{Периодическая последовательность треугольных импульсов}

Рассмотрим последовательность треугольных импульсов
(см. рис.~\figref{Triangular pulses}). Непосредственно с помощью формулы
\chaptereqref{fourier-coeff} нетрудно получить выражения для амплитуд спектральных компонент
(вычисления проведите самостоятельно):
\begin{equation}
\eqmark{triangle-a}
a_n = \frac{\tau}{T}\frac{\sin^2 \xi/2}{(\xi/2)^2},
\end{equation}
где $\xi = \pi n \tau /T$. При этом все начальные фазы равны нулю $\varphi_n=0$.
Нулевая составляющая $a_0 = \tau/2T$. Спектр рассматриваемой функции представлен
на рис.~\figref{Spectr of triangular pulses}.

\begin{figure}[t]
\begin{minipage}{0.45\textwidth}
	\pic{\linewidth}{Chapter_6/6_2_3}
	\caption{Периодическая последовательность треугольных~импульсов}
	\figmark{Triangular pulses}
\end{minipage}
\hfill
\begin{minipage}{0.45\textwidth}
	\pic{\linewidth}{Chapter_6/6_2_4}
	\caption{Спектр последовательности треугольных~импульсов
(при $T/\tau=7/2$)}
	\figmark{Spectr of triangular pulses}
\end{minipage}
\end{figure}

\experiment
Основным элементом экспериментальной установки является генератор гармонических
сигналов, который генерирует одновременно основной сигнал на выбранной частоте
$\nu_0$ и пять гармоник $\{2\nu_0,\ldots 6\nu_0\}$, кратных основному сигналу.
Генератор позволяет устанавливать амплитуды и фазовые сдвиги гармоник.
Все 6 гармоник могут складываться при помощи электронного сумматора. Этот сигнал
с выхода генератора подаётся на вход $Y$ осциллографа, на экране которого
можно наблюдать (в режиме непрерывной развёртки) периодическую
последовательность синтезированных сигналов. Технические
данные генератора и порядок работы с ним изложены в отдельном техническом
описании, расположенном на установке.

\begin{lab:task}

В работе предлагается подобрать амплитуды синусоидальных колебаний с кратными
частотами, сумма которых даёт периодическую последовательность прямоугольных или
треугольных импульсов.

\tasksection{Синтез последовательности прямоугольных импульсов}

\begin{enumerate}
	\item Ознакомьтесь с техническим описанием генератора. Приведите
приборы в рабочее состояние.

\item Предварительно рассчитайте относительные значения амплитуд первых шести
гармоник в спектре периодической последовательности прямоугольных импульсов с
отношением $T/\tau=7$: нулевая гармоника (постоянная составляющая) не
используется; первая гармоника соответствует основному сигналу генератора;
приняв амплитудное значение первой гармоники за единицу, относительные
амплитудные значения ($a_n/a_1$) остальных пяти гармоник рассчитайте по формуле
\eqref{imp-a}. Амплитуда седьмой гармоники в наших условиях (при
$T/\tau=7$) равна нулю. Значения синусов, необходимые для вычислений, приведены
в таблице:
\begin{center}
\begin{tabular}{|c|c|c|c|c|c|c|} \hline
$n$ & 1 & 2 & 3 & 4 & 5 & 6 \\ \hline $\alpha_n$&$\pi /7$&$2\pi /7$&$3\pi
/7$&$4\pi /7$&$5\pi /7$& $6\pi /7$ \\ \hline
$\sin \alpha_n$ & $0{,}434$ & $0{,}782$ & $0{,}975$ & $0{,}975$ &$0{,}782$ &
$0{,}434$ \\ \hline $a_n\sim\frac{\sin
\alpha_n}{n}$& & & & & & \\ \hline $\frac{a_n}{a_1}$&1& & & & & \\ \hline
\end{tabular}
\end{center}

	\item Установите частоту первой гармоники~--- 10 кГц и
откалибруйте (уравняйте) напряжения гармоник.
	\item С помощью осциллографа установите рассчитанные относительные амплитуды
и фазы гармоник \eqref{imp-p}.
	\item Последовательно увеличивая число используемых гармоник, фиксируйте
(фотографируйте/копируйте на кальку) сигнал на экране осциллографа. По
результирующей осциллограмме, соответствующей сумме всех шести гармоник,
определите отношение $T/\tau$ и сравните его с теоретическим значением.

\item Изучите влияние фазовых соотношений на восстановление сигнала по
его спектру. Изменяя фазы гармоник, наблюдайте за искажением сигнала.
Сделайте вывод по результатам наблюдений.

\item Повторите опыт для прямоугольного сигнала с $\tau=T/2$ (меандр).

\end{enumerate}

\tasksection{Синтез последовательности треугольных импульсов}
\begin{enumerate}
	\item Рассчитайте с помощью формулы \eqref{triangle-a} относительные
амплитуды гармоник в спектре периодической последовательности треугольных
импульсов с отношением $T/\tau=3,5$ (используйте таблицу из предыдущего
упражнения).
	\item Установите необходимые относительные амплитуды гармоник и их фазы.
	\item Получите осциллограмму от всех шести гармоник и
зафиксируйте её (сфотографируйте/скопируйте на кальку).
Определите отношение $T/\tau$ и сравните его с теоретическим.
        \item Повторите опыт для треугольных импульсов с $\tau=T/2$. В каком
случае аппроксимация получилась лучше? Почему?
	\item Закончив упражнение с реальным генератором, переходите к компьютерному
варианту работы (см. дополнительное описание).

\todo[inline]{Где дополнительное описание? Почему оно не встроено сюда?}

\todo[inline]{Слишком простое задание. В работе ничего не измеряется. Почему
    бы не синтезировать фазовую модуляцию? Требуется разбирательство на месте.}
\end{enumerate}

\end{lab:task}


%\lit

%\n \textit{Сивухин Д.В.} Общий курс физики. Т.~III. Электричество~--- М.: Наука,
% 1983. \S~128.

%\n \textit{Крауфорд Ф.} Берклеевский курс физики. Т.~III. Волны.~--- М.: Наука,
% 1976. \S~6.4.

