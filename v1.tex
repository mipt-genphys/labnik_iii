\vn О системах единиц в классической электродинамике

При измерении физической величины $x$ её числовое значение \fs{x} свидетельствует о том, сколько раз в $x$ содержится
некоторая единица измерения \ks{x}. Это означает, что
\be1
\fs{x}=\frac{x}{\ks{x}}.
\ee
Если, например, сила тока $I=10$~А, то $\fs{I}=10$, $\ks{I}=1$~А. Соотношение (\r1) можно записать в виде
\be2
x=\fs{x}\ks{x}.
\ee
При уменьшении единицы измерения в $\alpha$ раз
\[
\ks{x}\rightarrow\ks{X}=\frac{1}{\alpha}\ks{x},\qquad\fs{x}\rightarrow\fs{X}=\alpha\fs{x}.
\]
Сама физическая величина при этом не изменяется, поскольку
\be3
x=\fs{x}\ks{x}=\fs{X}\ks{X}.
\ee

Для каждой физической величины можно в принципе установить свою единицу, никак не связанную с единицами других величин.
Это приводит, однако, к тому, что в уравнениях, выражающих физические законы, появляется множество численных
коэффициентов. Уравнения становятся необозримыми, формулы~--- слишком сложными. Чтобы избежать этого, в физике уже давно
отказались от независимого выбора единиц всех физических величин и стали применять системы единиц, построенные по
определённому принципу, который состоит в следующем. Некоторые величины принимаются за базисные, т.е. такие, для которых
единицы устанавливаются произвольно. Так, например, в механике применяется система ($L$, $M$, $T$), в которой за базисные
величины принимаются длина $L$, масса $M$ и время $T$. Выбор базисных величин и их число произвольны. Это вопрос
соглашения. В международной системе СИ в качестве базисных величин приняты девять величин: длина, масса, время, сила
электрического тока, температура, сила света, количество вещества, плоский угол, телесный угол. Величины, не являющиеся
базисными, называются производными. Для производных величин единицы устанавливаются на основе формул, служащих их
определением.

Здесь возникает понятие размерности. Если, например, число базисных величин равно трём и за них приняты длина $L$, масса
$M$ и время $T$, то для размерности производной величины $y$ имеем
\be4
\dim y=L^p\cdot M^q\cdot T^r,
\ee
где $p$, $q$, $r$~--- постоянные числа. Формула (\r4) показывает, что если единицы длины, массы и времени уменьшить в
$\alpha$, $\beta$ и $\gamma$ раз, то единица производной величины $y$ уменьшится в $\alpha^p\beta^q\gamma^r$ раз, и,
следовательно, её числовое значение увеличится в такое же число раз. В этом и состоит смысл понятия размерности.
Заметим, что для безразмерной величины~$z$
\[
\dim z=1.
\]
На практике величины $p$, $q$, $r$ оказываются рациональными числами. Это обусловлено соответствующими определениями
физических величин.

Часто размерность физической величины отождествляют с её единицей в соответствующей системе единиц. Так, например,
говорят, что скорость имеет размерность $м/с$, а давление $Н/м^2$. В этом нет большой беды, хотя, строго говоря, это
неверно: размерность скорости~--- $LT^{-1}$, а давления~--- $ML^{-1}T^{-2}$.

Рассмотрим вопрос о системах единиц в электродинамике. Законы макроскопической электродинамики определяются её
фундаментальными аксиомами~--- уравнениями Максвелла, которые являются концентрированным обобщением экспериментальных
фактов из области электричества и магнетизма. Запишем уравнения Максвелла для вакуума в произвольной системе единиц:

\def\urtab#1#2#3{%
\vskip-1ex
\be#3
\begin{array}{ll}
\hbox to 0.5\textwidth{$\ds #1$\hfil}&\hbox to 0.25\textwidth{$\ds #2$\hfil}
\end{array}
\ee
\vskip-1ex
}

\urtab{\oint_S \vec{E}d\vec{S}=\alpha\int_V\rho\,dV,}{\divv\vec{E}=\alpha\rho,}{5}

\urtab{\oint_S\vec{B}d\vec{S}=0,}{\divv\vec{B}=0,}{6}

\urtab{\oint_L\vec{E}d\vec{l}=-\beta\int_S\frac{\d{\vec{B}}}{\d t}\,d\vec{S},}{\rot\vec{E}=-\beta\frac{\d\vec B}{\d t},}{7}

\urtab{\oint_L\vec{B}d\vec{l}=\gamma\int_S\vec{j}\,d\vec{S}+\delta\int_S\frac{\d{\vec{E}}}{\d t}\,d\vec{S},}
{\rot\vec{B}=\gamma\vec{j}+\delta\frac{\d\vec E}{\d t};}{8}

\be9
\vec{F}=\xi q\vec{E}+\eta q\vec{v}\times\vec{B},
\ee
\be10
d\vec{F}=\xi dq\vec{E}+\eta Id\vec{l}\times\vec{B}.
\ee
Здесь приняты стандартные обозначения. Уравнение (\r9) или (\r{10}) служит для определения силовых векторов $\vec{E}$ и
$\vec{B}$. Множество коэффициентов ($\alpha$, $\beta$, $\gamma$, $\delta$, $\xi$, $\eta$) свидетельствует о том, что для
каждой физической величины, входящей в систему уравнений (\r5)~--~(\r{10}), принята собственная единица измерения,
независимая от единиц других величин.

Напомним физический смысл уравнений Максвелла. Уравнение (\r5) показывает, что источником электрического поля $\vec{E}$
является электрический заряд. Из него можно получить закон Кулона:
\be11
\vec{F}_{12}=\alpha\frac{q_1q_2}{4\pi r^3_{12}}\vec{r}_{12}.
\ee
Уравнение (\r6) говорит о том, что в природе отсутствуют, насколько известно в настоящее время, магнитные заряды.
Уравнение (\r7)~--- это математическая формулировка закона электромагнитной индукции. Оно свидетельствует о том, что
изменяющееся магнитное поле порождает вихревое электрическое поле. Уравнение (\r8) показывает, что магнитное поле
$\vec{B}$~--- всегда вихревое (силовые линии замкнуты), и его источником являются не только движущиеся заряды, но и
переменное электрическое поле. Для постоянного магнитного поля с помощью (\r8) можно получить закон Био--Савара (см.
Приложение):
\be12
d\vec{B}=\frac{\gamma}{4\pi}\frac{I\,d\vec{l}\times\vec{r}}{r^3}.
\ee

С помощью уравнения
\[
\oint\vec{B}\cdot d\vec{l}=\gamma\int\vec{j}\cdot\vec{S}
%\rot\vec{B}=\gamma\vec{j}
\]
можно найти отнесённую к единице длины силу взаимодействия между двумя токами $I_1$ и $I_2$, текущими по двум бесконечно
длинным параллельным проводам:
\be13
\frac{dF}{dl}=\gamma\eta\frac{I_1I_2}{2\pi r}.
\ee

Рассмотрим электромагнитное поле в области, где нет источников, т.е. $\rho=0$ и $\vec{j}=0$. В силу (\r7) и (\r8) имеем
\[
\rot\,\rot\vec{E}=-\beta\frac{\d}{\d t}\rot\vec{B}=-\beta\delta\frac{\d^2\vec{E}}{\d t^2}
\]
или
\[
\grad\,\divv\vec{E}-\nabla^2\vec{E}=-\beta\delta\frac{\d^2\vec{E}}{\d t^2},
\]
т.е.
\be14
\nabla^2\vec{E}=\beta\delta\frac{\d^2\vec{E}}{\d t^2}.
\ee
Волновое уравнение (\r{14}) описывает распространение электромагнитных волн в вакууме. Скорость распространения волн
равна ${1}/{\sqrt{\beta\delta}}$. Измерения дают ${1}/\sqrt{\beta\delta}=c$, где $c$~--- скорость света в вакууме. Таким
образом, из опыта следует, что $\beta\delta=1/c^2$, где $c$~--- универсальная фундаментальная постоянная.

Запишем уравнения (\r5)~--~(\r9) в безразмерном виде. Для каждой физической величины $f$, входящей в эту систему, введём
следующие обозначения:
\be15
\fs{f}\equiv f',\quad\ks{f}\equiv f_0,\quad\text{т.е.}\quad f'=\frac{f}{f_0}\quad или\quad f=f'\.f_0.
\ee
Для единиц длины $l$ и времени $\tau$ имеем
\be16
\vec{r}=\vec{r}\,'\cdot l,\qquad t=t'\cdot\tau.
\ee

Подставляя (\r{15}) и (\r{16}) в систему (\r5)~--~(\r9) и опуская штрихи, находим

\def\urtab#1#2#3{%
\vskip-1ex
\be#3
\begin{array}{ll}
\hbox to 0.5\textwidth{$\ds #1$\hfil}&\hbox to 0.3\textwidth{$\ds #2$\hfil}
\end{array}
\ee
\vskip-1ex
}

\urtab{E_0l^2\oint_S \vec{E}d\vec{S}=\alpha\rho_0l^3\int_V\rho\,dV,}
{\frac{E_0}{l}\divv \vec{E}=\alpha\rho_0\rho,}{17}

\urtab{\oint_S\vec{B}d\vec{S}=0,}{\divv\vec{B}=0,}{18}

\urtab{E_0l\oint_L\vec{E}d\vec{l}=-\beta\frac{B_0}{\tau}l^2\int_S\frac{\d{\vec{B}}}{\d t}\,d\vec{S},}
{\frac{E_0}{l}\rot\vec{E}=-\frac{\beta B_0}{\tau}\frac{\d\vec B}{\d t},}{19}

%B_0l\oint_L\vec{B}d\vec{l}=
%\gamma j_0l^2\int_S\vec{j}\,d\vec{S}+\delta\frac{E_0l^2}{\tau}\int_S\frac{\d{\vec{E}}}{\d t}\,d\vec{S}&
%\frac{B_0}{l}\rot\vec{B}=\gamma j_0\vec{j}+\frac{\delta E_0}{\tau}\frac{\d\vec E}{\d t},&\num{20}\\


\be20
\begin{array}{ll}
\hbox to 0.4\textwidth{$\ds B_0l\oint_L\vec{B}d\vec{l}=
\gamma j_0l^2\int_S\vec{j}\,d\vec{S}+\delta\frac{E_0l^2}{\tau}\int_S\frac{\d{\vec{E}}}{\d t}\,d\vec{S},$\hss}&\\ [4ex]
&\ds\frac{B_0}{l}\rot\vec{B}=\gamma j_0\vec{j}+\frac{\delta E_0}{\tau}\frac{\d\vec E}{\d t},
\end{array}
\ee

\iffalse
\be17
E_0l^2\oint_S \vec{E}d\vec{S}=\alpha\rho_0l^3\int_V\rho\,dV
\qquad\frac{E_0}{l}\divv \vec{E}=\alpha\rho_0\rho,
\ee
\be18
\oint_S \vec{B}d\vec{S}=0
\qquad\divv \vec{B}=0,
\ee
\be19
E_0l\oint_L\vec{E}d\vec{l}=-\beta\frac{B_0}{\tau}l^2\int_S\frac{\d{\vec{B}}}{\d t}\,d\vec{S}
\qquad\frac{E_0}{l}\rot\vec{E}=-\frac{\beta B_0}{\tau}\frac{\d\vec B}{\d t},
\ee
\be20
B_0l\oint_L\vec{B}d\vec{l}=\gamma j_0l^2\int_S\vec{j}\,d\vec{S}+\delta\frac{E_0l^2}{\tau}\frac{\d{\vec{E}}}{\d t}\,d\vec{S}
\qquad\frac{B_0}{l}\rot\vec{B}=\gamma j_0\vec{j}+\frac{\delta E_0}{\tau}\frac{\d\vec E}{\d t},
\ee

\fi

\be21
F_0\vec{F}=\xi q_0qE_0\vec{E}+\eta q_0qv_0B_0\vec{v}\times\vec{B}.
\ee
Здесь $\ds v_0=\frac{l}{\tau}$, $j_0=\rho_0v_0$.

Из (\r{17})~--~(\r{21}) следует, что
\[
\dim\alpha=\dim\frac{E_0}{\rho_0l},
\]
\[
\dim\beta=\dim\frac{1}{v_0}\frac{E_0}{B_0},
\]
\[
\dim\gamma=\dim\frac{B_0}{j_0l},
\]
\[
\dim\frac{\delta}{\gamma}=\dim\frac{j_0\tau}{E_0}=\dim\frac{\rho_0v_0\tau}{E_0}=\dim\frac{\rho_0l}{E_0},
\]
\[
\dim\delta=\dim\frac{1}{v_0}\frac{B_0}{E_0},
\]
\[
\dim\frac{\xi}{\eta}=\dim v_0\frac{B_0}{E_0}.
\]
Отсюда можно видеть, что
\[
\dim\frac{\alpha\delta}{\gamma}=1,\qquad \dim\frac{\xi\beta}{\eta}=1,
\]
\[
\dim\delta\beta=\dim\frac{1}{v_0^2}.
\]
Последнее соответствует тому, что
\[
\delta\beta=\frac{1}{c^2}.
\]
При выборе базисных единиц естественно предположить, что
\be22
\frac{\alpha\delta}{\gamma}=1,\quad\frac{\xi\beta}{\eta}=1,\quad \delta\beta=\frac{1}{c^2}.
\ee
В таблице \r{tab1} показано, как в различных системах пользуются произволом, который дают соотношения (\r{22}). В
настоящее время принято считать, что $c=299\,792\,458$~м/с (точно). Это означает, что базисные единицы <<привязаны>> к
этой величине. Это, конечно, соглашение. Мы полагаем в лаборатории $c\cong3\.10^8$~м/с.

В общей физике в настоящее время используются в основном две системы единиц: гауссова система СГС (далее~--- система СГС)
и международная система СИ (далее~--- система СИ). Система СГС, в которой в качестве базисных величин приняты длина,
масса и время, разработана на основе законов механики Ньютона. Электрические и магнитные величины вводятся в ней как
производные механических. Построенные по такому принципу системы единиц называются абсолютными. В системе СГС
электрические величины измеряются в единицах СГСЭ, а магнитные~--- в единицах СГСМ.

В системе СИ к трём базисным механическим величинам~--- длине, времени и массе~--- в электродинамике добавлена
независимая чисто электрическая величина, имеющая собственную размерность. В качестве таковой выбрана сила
электрического тока, а её единицей выбран ампер. Единицей заряда при этом является ампер-секунда, называемая кулоном.

\begin{table}[!t]
\tab{1}{Некоторые системы единиц, используемые при изучении макроскопической электродинамики}{%
\def\vph{\vphantom{$\ds\frac{A}{B}$}}\def\pp#1{\hbox to 5mm{\hfil #1\hfil}}%
\bt{|l|c|c|c|c|c|c|c|c|c|}\hline
\vph&\pp{$\alpha$}&\pp{$\beta$}&\pp{$\gamma$}&\pp{$\delta$}&\pp{$\xi$}&\pp{$\eta$}&\pp{$\frac{\alpha\delta}{\gamma}$}&
\pp{$\frac{\xi\beta}{\eta}$}&\pp{$\delta\beta$}\\ \hline
\vph СГСЭ&$4\pi$&1&$\frac{4\pi}{c^2}$&$\frac{1}{c^2}$&1&1&1&1&$\frac{1}{c^2}$\\ \hline
\vph СГСМ&$4\pi c^2$&1&$4\pi$&$\frac{1}{c^2}$&1&1&1&1&$\frac{1}{c^2}$\\ \hline
\vph СГС&$4\pi$&$\frac{1}{c}$&$\frac{4\pi}{c}$&$\frac{1}{c}$&1&$\frac{1}{c}$&1&1&$\frac{1}{c^2}$\\ \hline
\vph СИ&$\frac{1}{\e_0}$&1&$\frac{1}{\e_0c^2}$&$\frac{1}{c^2}$&1&1&1&1&$\frac{1}{c^2}$\\ \hline
\vph МКС&1&1&$\frac{1}{c^2}$&$\frac{1}{c^2}$&1&1&1&1&$\frac{1}{c^2}$\\ \hline
\multicolumn{10}{|l|}{\vph \hskip 1cm $c=299\,792\,458$~м/с (точно);}\\
\multicolumn{10}{|l|}{\vph\hskip 1cm $\e_0=8,854\.10^{-12}$~Ф/м;~~~$\mu_0=\frac{1}{\e_0c^2}=4\pi\.10^{-7}$~Гн/м.}\\ \hline
\et
}
\vskip-2em
\end{table}

Эталон силы электрического тока устанавливается на основе формулы (\r{13}). В системе СИ $\gamma=\frac{1}{\e_0c^2}$,
$\eta=1$, поэтому
\be23
\D F=\frac{1}{\e_0c^2}\frac{I_1I_2}{2\pi r}\D l.
\ee
На основании международного соглашения принято по определению, что ампер~--- это единица силы тока, который, проходя по
двум параллельным прямолинейным проводникам бесконечной длины и исчезающе малого кругового сечения, расположенным на
расстоянии 1~м друг от друга в вакууме, вызывал бы между проводниками силу, равную $2\.10^{-7}$~Н на каждый метр длины.
Реализовать эту единицу можно несколькими способами, например, измеряя силу взаимодействия двух катушек с постоянным
током.

Полагая в (\r{23}) $I_1=I_2=1$~А, имеем
\[
2\.10^{-7}~Н=\frac{1}{\e_0c^2}\frac{1}{2\pi}~Н,
\]
т.е.
\[
\frac{1}{\e_0c^2}=4\pi\.10^{-7}~\text{ед. СИ}
\]
или
\[
\e_0=\frac{10^{7}}{4\pi c^2}\approx 8,85\.10^{-12}~\text{ед. СИ.}
\]

В системе СГС единиц формула (\r{13}) имеет вид
\be24
\D F=\frac{4\pi}{c^2}\frac{I_1I_2}{2\pi r}\D l.
\ee

Установим соотношение между единицами силы электрического тока в системе СГС и системе СИ. Полагая в (\r{23})
$I_1=I_2=1$~А, $r=\D l=1$~м, находим
\be25
\D F=2\.10^{-7}~Н.
\ee
Воспользуемся теперь для вычисления той же силы формулой (\r{24}). Полагая в этой формуле $I_1=I_2=I$, $r=\D l=100$~см,
находим
\be26
\D F=\frac{4\pi}{c^2}\frac{I^2}{2\pi}=\frac{2I^2}{c^2}~дин=\frac{2I^2}{c^2}10^{-5}~Н.
\ee
Приравнивая (\r{25}) и (\r{26}), имеем
\[
2\.10^{-7}=\frac{2I^2}{c^2}10^{-5},\qquad т.е.\qquad I=3\.10^9~\text{ед. СГС}.
\]
Это означает, что
\be27
10\ks{I}_{СИ}=c\ks{I}_{СГС},
\ee
где $c=3\.10^{10}$~см/с, $\ks{I}_{СИ}=1$~А. Соотношение (\r{27}) можно представить в виде
\[
\ks{I}_{СИ}=3\.10^9\ks{I}_{СГС}
\]
или
\be28
c=10\frac{\fs{I}_{СГС}}{\fs{I}_{СИ}}\left(\frac{см}{с}\right),
\ee
что может быть проверено экспериментально (см. работу \textnumero~3.1.1).

В силу (\r{27}) для электрического заряда имеем
\[
\ks{q}_{СИ}=3\.10^9\ks{q}_{СГС}.
\]
Установим соотношение между единицами разности потенциалов в системе СГС и системе СИ. Воспользуемся для этого формулой
для отсчитываемого от бесконечно удалённой точки потенциала точечного заряда:
\[
\phi=\frac{q}{r}\quad (СГС),\qquad\qquad
\phi=\frac{1}{4\pi\e_0}\frac{q}{r}\quad (СИ).
\]
Пусть $q=1$~ед. СГС, а $r=1$~см, тогда $\phi=1~ед.~СГС\equiv\ks{\phi}_{СГС}$. Вычислим этот же потенциал в системе СИ:
\[
\phi=\frac{9\.10^9}{3\.10^9\cdot10^{-2}}=300~В.
\]
Это означает, что для единиц разности потенциалов имеем
\be29
\ks{U}_{СГС}=300\ks{U}_{СИ}.
\ee
Соотношение (\r{29}) может быть также проверено экспериментально, например, с помощью абсолютного вольтметра (см. работу
\textnumero~3.1.2).

Подобным образом устанавливаются соотношения между единицами других физических систем, величин (см. таблицу~\r{tab2}).
Основные формулы в системах СИ и СГС представлены в таблице~\r{tab3}.

\newpage

\tab{2}{Перевод числовых значений физических величин из~системы СИ в~систему СГС}{%
%\def\vs{\vphantom{$\ds\frac{A}{B}$}}%
\let\vs=\tabstrut
\def\bbx{\raggedright\baselineskip=9pt}%
\def\pbl#1#2{\parbox{#1}{\bbx #2}}%
\def\vr{\vbox to 3pt{}}%
\def\pb#1{$\vcenter{\vr\hbox{\pbl{30mm}{#1\hfill}}\vr}$}%
\bt{|l|c|c|c|}\hline
\vs Наименование&Обозн.&СИ&СГС\\ \hline
\vs Длина&$l$&1 м (метр)&$10^2$~см\\ \hline
\vs Масса&$m$&1 кг (килограмм)&$10^3$ г\\ \hline
\vs Время&$t$&1 с (секунда)&1 с\\ \hline
\vs Работа, энергия&$A$, $W$&1 Дж (джоуль)&$10^7$ эрг\\ \hline
\vs Мощность&$N$&1 Вт (ватт)&$10^7~\frac{эрг}{с\mathstrut}$\\ \hline
\vs Давление&$P$&1 Па (паскаль)&$10~\frac{дин}{см^2\mathstrut}$\\ \hline
\vs \pb{Сила электри-\\ческого тока}&$I$&1 А (ампер)&$3\.10^9$\\ \hline
\vs Электр. заряд&$q$&1 Кл (кулон)&$3\.10^9$\\ \hline
\vs Поляризация&$\vec{P}$&$1~\frac{Кл}{м^2}$&$3\.10^5$\\ \hline
\vs \pb{Электрическая индукция}&$\vec{D}$&$1~\frac{Кл}{м^2}$&$12\pi\.10^5$\\ \hline
\vs Электр. ёмкость&$C$&1 Ф (фарад)&$9\.10^{11}$ см\\ \hline
\vs \pb{Электрическое сопротивление}&$R$&1 Ом (ом)&$\frac{1}{9\.10^{11}}~\frac{с}{см\mathstrut}$\\ \hline
\vs \pb{Удельное сопротивление}&$\rho$&1 Ом\.м&$\frac{1}{9\.10^{9}\mathstrut}~с$\\ \hline
\vs \pb{Электрическая проводимость}&$\Lambda=\frac{1}{R}$&1 См (сименс)&$9\.10^{11}~\frac{см}{с\mathstrut}$\\ \hline
\vs \pb{Удельная проводимость}&$\sigma$&$1~\frac{См}{м}$&$9\.10^9~с^{-1}$\\ \hline
\vs Магнитный поток&$\Phi$&1 Вб (вебер)&$10^8$ Мкс\\ \hline
\vs \pb{Магнитная индукция}&$\vec{B}$&1 Тл (тесла)&$10^4$ Гс\\ \hline
\vs \pb{Напряжённость магнитного поля}&$\vec{H}$&$1~\frac{А}{м}$&$4\pi\cdot10^{-3}$ Э\\ \hline
\vs Намагниченность&$\vec{M}$&$1~\frac{А}{м}$&$\frac{1}{4\pi}\cdot 10^4$ Гс\\ \hline
\vs Индуктивность&$L$&1 Гн (генри)&$10^9$ см\\ \hline
\vs \pb{Электрический потенциал}&$\phi$&$1~В$ (вольт)&$\frac{1}{300}$\\ \hline
\vs \pb{Напряжённость электр. поля}&$\vec{E}$&$1~\frac{В}{м}$&$\frac{1}{3}\cdot 10^{-4}$\\ \hline
\et
}

\newpage

%\ftab{
\tab{3}{Основные формулы в СИ и СГС}{\small%
\let\vs=\tabstrut
\def\bbx{\raggedright\baselineskip=10pt}%
\def\pbl#1#2{\parbox{#1}{\bbx #2}}%
\def\vr{\vbox to 2pt{}}%
\def\pb#1{\vs$\vcenter{\vr\hbox{\pbl{38mm}{#1\hfill}}\vr}$}%
\bt{l|c|c}\hline
\vs Наименование&СИ&СГС\\ \hline
\pb{Уравнения Максвелла}&$\divv\vec{D}=\rho$&$\divv\vec{D}=4\pi\rho$\\
\pb{в дифференциальной}&$\divv\vec{B}=0$&$\divv\vec{B}=0$\\
\pb{форме}&$\rot\vec{E}=-\frac{\d\vec{B}}{\d t}$&$\rot\vec{E}=-\frac{1}{c}\frac{\d\vec{B}}{\d t}$\\
\pb{}&$\rot\vec{H}=\vec{j}+\frac{\d\vec{D}}{\d t}$&$\rot\vec{H}=\frac{4\pi}{c}\vec{j}+\frac{1}{c}\frac{\d\vec{D}}{\d t}$\\
\pb{Электрическая индукция}&$\vec{D}=\e_0\vec{E}+\vec{P}$&$\vec{D}=\vec{E}+4\pi\vec{P}$\\
\pb{Напряжённость магнитного поля}&$\vec{H}=\frac{1}{\mu_0}\vec{B}-\vec{M}$&$\vec{H}=\vec{B}-4\pi\vec{M}$\\
\\
\pb{Материальные}&$\vec{P}=\alpha\e_0\vec{E}$&$\vec{P}=\alpha\vec{E}$\\
\pb{уравнения}&$\vec{D}=\e\e_0\vec{E}$&$\vec{D}=\e\vec{E}$\\
%\pb{Связь между $\vec{D}$ и $\vec{E}$ в~вакууме}&$\vec{D}=\e_0\vec{E}$&$\vec{D}=\vec{E}$\\
&$\vec{M}=\kappa\vec{H}$&$\vec{M}=\kappa\vec{H}$\\
&$\vec{B}=\mu\mu_0\vec{H}$&$\vec{B}=\mu\vec{H}$\\
\pb{}&$\vec{j}=\sigma\vec{E}$&$\vec{j}=\sigma\vec{E}$\\
\\
%\pb{Связь между $\vec{B}$ и $\vec{H}$ в~вакууме}&$\vec{B}=\mu_0\vec{H}$&$\vec{B}=\vec{H}$\\
\pb{Уравнения Максвелла}&$\oint\limits_{S}\v{D}\,d\v{S}=\int_{V}\rho\,dV$&
$\oint\limits_{S}\v{D}\,d\v{S}=4\pi\int_{V}\rho\,dV$\\
\pb{в интегральной форме}&$\oint\limits_{S}\v{B}\,d\v{S}=0$&$\oint\limits_{S}\v{B}\,d\v{S}=0$\\
\pb{}&$\oint\limits_{L}\v{E}\,d\v{l}=-\int_{S}\frac{\d\vec{B}}{\d t}\,d\vec{S}$&
$\oint\limits_{L}\v{E}\,d\v{l}=-\frac{1}{c}\int_{S}\frac{\d\vec{B}}{\d t}\,d\vec{S}$\\
\pb{}&$\oint\limits_{L}\v{H}\,d\v{l}=$~~~~~~~~~~&$\oint\limits_{L}\v{H}\,d\v{l}=$~~~~~~~~~~\\
\pb{}&$=\int_{S}\vec{j}\,d\vec{S}+\int_{S}\frac{\d\vec{D}}{\d t}\,d\vec{S}$&
$=\frac{4\pi}{c}\int_{S}\vec{j}\,d\vec{S}+\frac{1}{c}\int_{S}\frac{\d\vec{D}}{\d t}\,d\vec{S}$\\
\pb{Сила Лоренца}&$\vec{F}=q\v{E}+\vp{\vec{v}}{\vec{B}}$&
$\ds\vec{F}=q\v{E}+\frac{q\mathstrut}{c\mathstrut}\vp{\vec{v}}{\vec{B}}$\\
\pb{Закон Кулона}&$\ds \v{F}=\frac{1}{4\pi\e_0}\frac{q_1q_2}{\e r^3}\v{r}$&$\ds\v{F}=\frac{q_1q_2}{\e r^3}\v{r}$\\[2ex]
\pb{Закон Био--Савара}&$\ds d\vec{H}=\frac{I}{4\pi}\frac{\vp{d\vec{l}}{\vec{r}}}{r^3}$&
$\ds d\vec{H}=\frac{I}{c}\frac{\vp{d\vec{l}}{\vec{r}}}{r^3}$\\[2ex]
\pb{Закон Ампера}&$d\vec{F}=I\vp{d\vec{l}}{\vec{B}}$&$d\vec{F}=\frac{I}{c}\vp{d\vec{l}}{\vec{B}}$\\
\pb{Плотность энергии электромагнитного поля}&$w=\frac12\bigl(\vec{E}\vec{D}+\vec{B}\vec{H}\bigr)$&
$w=\frac{1}{8\pi}\bigl(\vec{E}\vec{D}+\vec{B}\vec{H}\bigr)$\\
\pb{Вектор Пойнтинга}&$\vec{\Pi}=\vp{\vec{E}}{\vec{H}}$&$\ds\vec{\Pi}=\frac{c}{4\pi}\vp{\vec{E}}{\vec{H}}$\\[1ex]
\pb{Энергия магнитного поля тока}&$\ds W=\frac{LI^2}{2}$&$\ds W=\frac{1}{c^2}\frac{LI^2}{2}$\\[1ex]
\pb{Плотность импульса электромагнитного поля}&$\ds\vec{g}=\frac{1}{c^2}\vp{\vec{E}}{\vec{H}}$&
$\ds\vec{g}=\frac{1}{4\pi c}\vp{\vec{E}}{\vec{H}}$\\
\hline
\et
}
%}

\newpage

\conttab{\small%
\let\vs=\tabstrut
\def\bbx{\raggedright\baselineskip=10pt}%
\def\pbl#1#2{\parbox{#1}{\bbx #2}}%
\def\vr{\vbox to 3pt{}}%
\def\pb#1{\vs$\vcenter{\vr\hbox{\pbl{35mm}{#1\hfill}}\vr}$}%
\bt{l|c|c}\hline
\vs Наименование&СИ&СГС\\ \hline
\pb{Индуктивность (определение)}&$\Phi=LI$&$\ds\Phi=\frac{1}{c}LI$\\
\pb{Индуктивность длинного соленоида}&$\ds L=\frac{\mu\mu_0N^2S}{l}$&$\ds L=\frac{4\pi\mu N^2S}{l}$\\
\pb{Магнитный момент витка с током}&$\vec{\Mgot}=I\vec{S}$&$\ds\vec{\Mgot}=\frac{1}{c}I\vec{S}$\\
\pb{Момент сил, действующий на виток с~током}&$\vec{M}=\vp{\vec{\Mgot}}{\vec{B}}$&$\vec{M}=\vp{\vec{\Mgot}}{\vec{B}}$\\
\pb{Поле точечного магнитного диполя}&
$\vec{B}=\frac{\mu_0}{4\pi}\left(\frac{3(\vec{\Mgot}\vec{r})}{r^5}\vec{r}-\frac{\vec{\Mgot}}{r^3}\right)$
&$\vec{B}=\frac{3(\vec{\Mgot}\vec{r})}{r^5}\vec{r}-\frac{\vec{\Mgot}}{r^3}$\\
\pb{Сила, действующая на магнитный диполь в неоднородном поле}&$\vec{F}=(\vec{\Mgot}\vec{\nabla})\vec{B}$&
$\vec{F}=(\vec{\Mgot}\vec{\nabla})\vec{B}$\\
\pb{Поле точечного электрического диполя}&
$\vec{E}=\frac{1}{4\pi\e_0}\left(\frac{3(\vec{p}\vec{r})}{r^5}\vec{r}-\frac{\vec{p}}{r^3}\right)$&
$\vec{E}=\frac{3(\vec{p}\vec{r})}{r^5}\vec{r}-\frac{\vec{p}}{r^3}$\\
\pb{Ёмкость плоского конденсатора}&$\ds C=\frac{\e\e_0S}{d}$&$\ds C=\frac{\e S}{4\pi d}$\\
\pb{Энергия заряженного конденсатора}&$W=\frac{q^2}{2C}=\frac{qU}{2}=\frac{CU^2}{2}$&
$W=\frac{q^2}{2C}=\frac{qU}{2}=\frac{CU^2}{2}$\\ \hline
\et
}

Международная система СИ хорошо приспособлена для практических инженерных измерений. Она мало отличается от
электротехнической системы, предложенной Джорджи в начале ХХ~века (см. Приложение). В то время уравнения Максвелла мало
использовались в электротехнике, преобладали механические воззрения на природу электромагнитного поля. С точки зрения
рассмотренной выше структуры безразмерных параметров уравнений Максвелла выбор единиц системы СИ представляется
совершенно случайным, хотя вполне допустимым. С~теоретической точки зрения предпочтительной является система МКС, для
которой все коэффициенты равны единице, кроме коэффициентов $\gamma$ и  $\delta$, каждый из которых равен $1/c^2$ (см.
таблицу~\r{tab1}).
%Это обусловлено тем, что единственной электродинамической постоянной вакуума является скорость света $c$.

{\small

\lit

\n \emph{Камке Д., Кремер К.} Физические основы единиц измерения.~--- М.: Мир, 1980.

\n \emph{Власов А.Д., Мурин Б.П.}~Единицы физических величин в науке и технике. Справочник.~--- М.: Энергоатомиздат,
1990.

}
