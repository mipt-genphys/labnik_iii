\lab{Скин-эффект в полом цилиндре}

\aim{исследование проникновения переменного магнитного поля в медный полый цилиндр.}

\equip{генератор звуковой частоты, соленоид, намотанный на полый цилиндрический каркас из диэлектрика, медный экран в
виде трубки, измерительная катушка, амперметр, вольтметр, осциллограф.}


В случае цилиндрического сплошного или полого проводника следует решать уравнения \eqref{12} и \eqref{13}, записанные в
цилиндрической системе координат. Решение таких уравнений выражается уже не экспоненциальной функцией, а через функции
Бесселя первого и нулевого порядков. Нас будет интересовать случай тонкостенного цилиндра, когда толщина стенки $h\ll a$,
где $a$~--- радиус цилиндра. В этом случае заменим проводящий цилиндрический слой проводящей плоскопараллельной пластиной
толщиной $h$. Пусть эта пластина находится в продольном поле $H_x(t)=H_0e^{i\omega t}$ (рис.~2). Для нашего одномерного случая
уравнение \eqref{13} запишется в виде
\begin{equation} \eqmark{21}
\frac{\partial^2H_x}{\partial z^2}=\mu_0\mu\lambda\ddt{H_x}.
\end{equation}

%TODO: figure

%\rfr{35mm}{2}{}{2}{
%\psfrag{x}{$x$}
%\psfrag{y}{$z$}
%\psfrag{a}[bl]{$a$}
%\psfrag{b}[tl]{$a-h$}
%\psfrag{h}[bc]{$H_0e^{i\omega t}$}
%\psfrag{o}[ct]{$0$}
%}

Решение этого уравнения ищем в виде
\begin{equation} \eqmark{22}
H(z,\,t)=H(z)e^{i\omega t}.
\end{equation}
После подстановки \eqref{22} в \eqref{21} получим уравнение для функции $H(z)$:
\begin{equation} \eqmark{23}
\frac{\partial^2H(z)}{\d z^2}=i\mu_0\mu\lambda\omega H(z).
\end{equation}
Решение этого уравнения будем искать в виде
\begin{equation} \eqmark{24}
H(z)=Ae^{\frac{\sqrt{2i}}{\delta}z}+Be^{-\frac{\sqrt{2i}}{\delta}z},
\end{equation}
где $A$ и $B$~--- константы, а $\delta$~--- глубина проникновения поля \eqref{20}. Ранее при решении аналогичного
уравнения \eqref{16} мы использовали только один корень уравнения \eqref{18}: со знаком плюс, а в этом случае мы имеем дело
уже не с полупространством, а с конечной областью в виде плоского слоя, поэтому решение \eqref{24} должно содержать оба
корня уравнения \eqref{18}. Решение \eqref{24} мы ищем для $a-h\le z\le a$, а при $z\ge a$ поле $H(z)=H_0$, где $H_0$~---
величина поля в соленоиде без экрана; при $z\le a-h$ поле $H(z)=H_{0с}$, здесь $H_{0с}$~--- величина поля в соленоиде с
экраном. Граничные условия для решения \eqref{24} запишутся в следующем виде:
\[
H(a)=H_0,\qquad H(a-h)=H_{0с}.
\]
Введя обозначение $q=\sqrt{2i}/\delta$, запишем граничные условия в виде
\[
H_0=Ae^{qa}+Be^{-qa},\qquad H_{0с}=Ae^{q(a-h)}+Be^{-q(a-h)}.
\]
Из совместного решения этой системы найдём $A$ и $B$:
\[
A=\frac{H_0e^{-q(a-h)}-H_{0с}e^{-qa}}{e^{qh}-e^{-qh}},\qquad B=\frac{H_{0с}e^{qa}-H_0e^{q(a-h)}}{e^{qh}-e^{-qh}}.
\]
Как видно из полученных выражений для $A$ и $B$, обе константы зависят от неизвестной величины $H_{0с}$, поэтому
следующий этап решения~--- это нахождение $H_{0с}$. Мы не будем проводить дальнейшего подробного решения (в силу
громоздких математических выкладок), а ограничимся схематическим изложением.

Снова переходим к цилиндрическому проводящему слою радиуса $R=a$ и толщиной $h$. Используя уравнение Максвелла \eqref{2},
находим распределение электрического поля $E(r)$ внутри проводящего слоя. Дифференциальный закон Ома \eqref{7} позволяет
найти распределение плотности тока проводимости $j(r)$. Проинтегрировав $j(r)$ по слою, найдём полный ток $I$ в слое. На
последнем этапе мы находим поле $H_{0с}$ как суперпозицию внешнего поля $H_0$ и поля, создаваемого суммарным током $I$.
Полученное уравнение будет содержать $H_{0с}$ в качестве неизвестного. В приближении $\frac{h}{a}\ll 1$
\begin{equation} \eqmark{25}
H_{0с}=\frac{2H_0}{ak\sh(kh)+2\ch(kh)},
\end{equation}
где $\displaystyle k=\frac{1+i}{\delta}$, а $\displaystyle\delta=\sqrt{\frac{2}{\mu_0\mu\lambda\omega}}$.

Соотношение \eqref{25} выражает связь между амплитудой магнитного поля $H_0$ вне проводящего цилиндра радиуса $a$ и
толщиной $h$ и амплитудой поля $H_{0с}$ внутри цилиндра. При малых частотах, когда $\frac{h}{\delta}\ll 1$, формула
\eqref{25} запишется в виде
\begin{equation} \eqmark{26}
H_{0с}\approx\frac{H_0}{1+i\frac{ah}{\delta^2}}.
\end{equation}
В этом случае отношение абсолютных величин амплитуд
\begin{equation} \eqmark{27}
\frac{|H_{0с}|}{|H_0|}\approx 1-\frac{(\mu_0\lambda ah)^2\omega^2}{8}.
\end{equation}
При больших частотах, когда $\frac{h}{\delta}\gg 1$, $\sh(kh)\approx\ch(kh)\approx\frac12e^{kh}$, выражение \eqref{25}
можно записать в виде
\begin{equation} \eqmark{28}
H_{0с}=(1-i)\frac{\delta}{2a}e^{-kh}\,H_0=\sqrt{2}e^{-\frac{h}{\delta}}\,e^{-i(\frac{\pi}{4}+\frac{h}{\delta})}.
\end{equation}
Как видно из формулы \eqref{28}, поле внутри цилиндра запаздывает по фазе:
\begin{equation} \eqmark{29}
\Delta\psi=\frac{\pi}{4}+\frac{h\sqrt{\mu_0\lambda\omega}}{\sqrt{2}}.
\end{equation}

%TODO:figure

%{
%\psfrag{a}[cb]{I}
%\psfrag{b}[cb]{II}
%\psfrag{A}[cc]{$A$}
%\psfrag{V}[cc]{$V$}
%\psfrag{R}[cb]{$R$}
%\psfrag{c}[cb]{Осциллограф}
%\psfrag{d}[cb]{Генератор}
%\psfrag{1}[tr]{1}
%\psfrag{2}[br]{2}
%\psfrag{3}[cr]{3}
%\fcris{4}{Схема экспериментальной установки для исследования скин-эффекта в полом цилиндре}{3}
%}

\experiment
Схема экспериментальной установки для исследования проникновения переменного магнитного поля в медный полый цилиндр
изображена на рис.~3. Переменное магнитное поле создаётся с помощью соленоида, намотанного на полый цилиндрический каркас 1
из полихлорвинила, который подключается к генератору звуковой частоты. Внутри соленоида расположен медный цилиндрический
экран 2. Для измерения магнитного поля внутри экрана используется измерительная катушка 3. Необходимые параметры
соленоида, экрана и измерительной катушки указаны на установке. Действующее значение переменного тока в цепи соленоида
измеряется цифровым амперметром <<$A$>>, а действующее значение напряжения на измерительной катушке измеряет цифровой
вольтметр <<$V$>>. Для измерения сдвига фаз между током в цепи соленоида и напряжением на измерительной катушке
используется двухканальный осциллограф. На вход одного канала подаётся напряжение с резистора $R$, которое
пропорционально току, а на вход второго канала~--- напряжение с измерительной катушки.

\labsection{Измерение отношения абсолютных величин амплитуд магнитного поля внутри и вне экрана}

С помощью вольтметра <<$V$>> мы измеряем действующее значение ЭДС индукции, которая возникает в измерительной катушке,
находящейся в переменном магнитном поле:
\[
H_{0с}(t)=H_{0с}e^{i\omega t}.
\]
ЭДС индукции в измерительной катушке равна
\[
\mathcal{E}_i=-SN\frac{dB_{0с}(t)}{dt}=-\mu_0SN\frac{dH_{0с}(t)}{dt}=-i\mu_0SN\omega H_{0с}e^{i\omega t},
\]
где $SN$~--- произведение площади витка на число витков измерительной катушки. Обозначим показание вольтметра <<$V$>>
через $U_к$, тогда
\[
U_к=\frac{\mu_0SN\omega}{\sqrt{2}}|H_{0с}|.
\]
Из этого соотношения следует, что абсолютная величина амплитуды магнитного поля внутри экрана
\[
|H_{0с}|\sim\frac{U_к}{\omega}\sim\frac{U_к}{f},
\]
где $f$~--- частота генератора в герцах ($\omega=2\pi f$). Но поле внутри экрана пропорционально полю вне экрана $H_0$, а
$H_0\sim I_A$, где $I_A$~--- показание амперметра <<$A$>> в цепи соленоида. Следовательно, амплитуда поля внутри экрана,
приведённая к единичному току через соленоид,
\[
|H_{0с}|\sim\frac{U_к}{fI_A}.
\]
Обозначим величину, пропорциональную $|H_{0с}|$, через $\xi_{0с}$:
\begin{equation} \eqmark{30}
\xi_{0с}=\frac{U_к}{fI_A}.
\end{equation}
Нам теперь необходимо найти амплитуду поля вне экрана при том же единичном токе через соленоид. Для этого воспользуемся
соотношением \eqref{27}. Проведя измерения $\xi_{0с}$ в диапазоне самых малых частот, мы строим график зависимости
$\xi_{0с}$ от $f^2$. Согласно \eqref{27} эта зависимость имеет вид прямой. Экстраполируя её к $f\to 0$, мы и получим
величину $\xi_{0с}(0)=\xi_0$, которая пропорциональна амплитуде поля вне экрана при единичном токе через соленоид.
Отношение амплитуд магнитного поля внутри экрана и вне при фиксированной частоте $f$ будет равно
\begin{equation} \eqmark{31}
\frac{|H_{0с}|}{|H_0|}=\frac{\xi_{0с}(f)}{\xi_0}=\frac{U_к}{fI_A\xi_0}.
\end{equation}
Такой способ измерения коэффициента ослабления магнитного поля проводящим экраном не требует поддерживать постоянный ток
через соленоид при измерении частотной зависимости этого коэффициента.

\labsection{Определение проводимости материала экрана}

В нашем случае в качестве экрана используется медная труба промышленного производства. Технология изготовления труб
оказывает заметное влияние на электропроводимость. Электропроводимость меди нашей трубы отличается от табличного
значения в область заниженного значения. Для определения электропроводимости меди нашего экрана предлагается
использовать частотную зависимость фазового сдвига между магнитными полями внутри экрана и вне в области больших частот
\eqref{29}. Как видно из выражения \eqref{29}, в области больших частот зависимость $\Delta\psi(\sqrt{f})$ аппроксимируется
прямой проходящей через точку $\Delta\psi(0)=\pi/4$. По наклону этой прямой можно вычислить проводимость материала экрана.

Несколько слов об измерении сдвига фаз $\Delta\psi$. На схеме, изображённой на рис.~3, видно, что на II входной канал
осциллографа подаётся сигнал с измерительной катушки, который пропорционален не полю внутри экрана, а его производной по
времени, а это означает, что появляется дополнительный сдвиг по фазе на $\pi/2$. Поэтому измеренный по экрану
осциллографа сдвиг по фазе между двумя синусоидами будет на $\pi/2$ больше фазового сдвига между магнитными полями вне и
внутри экрана.

\begin{lab:task}
\item В области низких частот 20--100~Гц (через $\Delta f=10$~Гц) снимите зависимость <<амплитуды>> $\xi_{0с}$ магнитного поля
внутри экрана от частоты.

\item Одновременно исследуйте зависимость $\xi_{0с}$ и фазового сдвига $\Delta\psi$ от частоты $f$ в диапазоне 100~Гц--30~кГц
(100, 500, 1~кГц, 2\dots10~кГц, 12, 14\dots 30~кГц).

\tasksection{Обработка результатов}

\item Результаты измерений <<амплитуды>> $\xi_{0с}$ в области низких частот постройте на графике в координатах $\xi_{0с}$
от $f^2$. Через отложенные точки проведите прямую и, экстраполируя её к $f=0$, определите <<амплитуду>> внешнего поля
$\xi_0$.

\item Частотную зависимость фазового сдвига $\Delta\psi$ в диапазоне частот 100~Гц--30~кГц изобразите на графике в координатах
$\Delta\psi$ и $\sqrt{f}$. Через точку $\Delta\psi=\pi/4$ при $f=0$ проведите прямую, которая будет касаться экспериментальной
кривой при больших частотах. По наклону этой прямой вычислите значение проводимости $\lambda$ материала экрана. Сравните с
табличным значением $\lambda$ для меди.

\item Используя ранее найденное значение <<амплитуды>> внешнего поля $\xi_0$ и результаты измерений $\xi_{0с}$ в диапазоне
100~Гц--30~кГц, вычислите коэффициент ослабления магнитного поля $\frac{|H_{0с}|}{|H_0|}$ по формуле \eqref{31}. Изобразите
эти результаты на графике $\frac{|H_{0с}|}{|H_0|}$ в зависимости от $\sqrt{f}$. Рассчитайте аналогичную теоретическую
зависимость \eqref{25} и для сравнения нанесите её на экспериментальный график.

\item Воспользовавшись найденным значением $\lambda$, вычислите глубину проникновения поля $\delta$ при двух частотах: 50~Гц и
$10^5$~Гц.
\end{lab:task}


\begin{lab:questions}
\item Какие два экспериментальных закона лежат в основе современной электродинамики? Запишите два уравнения Максвелла,
которые отражают эти два закона?

\item Какое уравнение описывает поведение электромагнитного возмущения в проводящей среде?

\item Какой физический смысл имеет параметр $\delta$, называемый глубиной проникновения поля, и от чего он зависит?

\item Для увеличения чувствительности измерительной катушки имеет смысл увеличивать её число витков. Какой физический
эффект сдерживает неограниченный рост числа витков измерительной катушки? Как нужно наматывать измерительную катушку,
чтобы уменьшить влияние этого эффекта?
\end{lab:questions}


\begin{lab:literature}
\item {\em Сивухин Д.В.} Общий курс физики. Т.~III. Электричество.~--- М.: Наука, 1983. \S~144.

\item {\em Кингсеп А.С., Локшин Г.Р., Ольхов О.А.} Основы физики. Т.~I. М.: ФИЗМАТЛИТ, 2001. \S~8.4.
\end{lab:literature}
