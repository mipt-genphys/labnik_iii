\lab{Скин-эффект в полом цилиндре}

\aim{исследование проникновения переменного магнитного поля в медный полый цилиндр.}

\equip{генератор звуковой частоты, соленоид, намотанный на полый цилиндрический каркас из диэлектрика, медный экран в
виде трубки, измерительная катушка, амперметр, вольтметр, осциллограф.}

В работе изучается скин-эффект в длинном тонкостенном медном цилиндре,
помещённом внутрь соленоида.

Теоретически такая задача сложнее,
чем рассмотренный в п.~\ref{sec:skin} скин-эффект в полубесконечном пространстве:
здесь требуется совместное решение уравнений
скин-эффекта (уравнения диффузии поля) \chaptereqref{skin-H},
\chaptereqref{skin-E} в стенке цилиндра и квазистационарных
уравнений поля в его полости.

\begin{wrapfigure}{o}{0.4\textwidth}
    \pic{\linewidth}{Chapter_7/skin-cyl}
    \caption{Электрическое и магнитное в тонкостенном цилиндре}
    \figmark{skin-cyl}
\end{wrapfigure}
Пусть цилиндр достаточно длинный, так что в нём можно пренебречь
краевыми эффектами.
В этом приближении магнитное поле~$\vec{H}$ всюду направлено
по оси системы (ось $z$), а вихревое электрическое поле $\vec{E}$
будет всюду перпендикулярно радиусу, то есть линии поля образуют соосные окружности
(рис.~\figref{skin-cyl}).
Все величины будем считать колеблющимися по гармоническому закону
с некоторой частотой $\omega$, задаваемой частотой колебания тока
в соленоиде. Тогда для ненулевых компонент поля можно записать
\[
H_z = H(r) e^{i\omega t},\qquad E_{\varphi} = E(r) e^{i\omega t},
\]
где~$H(r)$ и $E(r)$~--- комплексные амплитуды колебаний соответствующих полей,
зависящие только от расстояния $r$ до оси системы.
Заметим, что на границе цилиндра должны быть непрерывны касательные
к поверхности компоненты как $\vec{E}$, так и $\vec{B}$,
поэтому функции $E(r)$ и $H(r)$ непрерывны во всей исследуемой области.

Пусть длинный полый цилиндр имеет радиус $a$ и толщину стенки $h \ll a$.
Последнее условие позволяет для описания поля внутри стенки
ограничиться \emph{одномерным} приближением. При этом для полного
решения задачи необходимо вычислить и распределение поля \emph{внутри} цилиндра.
%Для не ферромагнитного цилиндра можно положить $\mu \approx 1$.

Поскольку внутри цилиндра ток отсутствует, магнитное поле там является
однородным (аналогично полю внутри пустого соленоида):
$H_z(r,\,t)=H_1 e^{i\omega t}$, где $H_1=\const$~--- амплитуда поля на внутренней
поверхности цилиндра. Для нахождения вихревого электрического поля
воспользуемся законом электромагнитной индукции \chaptereqref{3} в интегральной
форме:
\[
E_{\varphi}\cdot 2\pi r = -\mu_0 \pi r^2 \cdot \frac{dH_z}{dt}\qquad
\to
\qquad
E(r) = -\frac12 \mu_0 r \cdot i\omega H_1.
\]
Отсюда получим связь амплитуд колебаний электрического и магнитного полей
на внутренней ($r=a$) границе цилиндра:
\begin{equation}\eqmark{borderE}
E_1 = -\frac12  i\omega a\mu_0 H_1.
\end{equation}
Соотношение \eqref{borderE} используем далее как дополнительное граничное
условие для задачи о распределении поля внутри стенки.

\begin{wrapfigure}{o}{0.27\textwidth}
    \pic{\linewidth}{Chapter_7/skin-flat}
    \caption{Поле в стенке цилиндра}
    \figmark{skin-flat}
\end{wrapfigure}
Поле внутри тонкой стенки цилиндра (<<экрана>>) описывается уравнением 
скин-эффекта~\chaptereqref{14} (уравнением диффузии поля) в плоской геометрии
(рис.~\figref{skin-flat}).
Поместим начало отсчёта на внешнюю поверхность цилиндра и направим ось~$x$
к оси системы, и аналогично \chaptereqref{16}
запишем дифференциальное уравнение для комплексной амплитуды магнитного поля:
\begin{equation}\eqmark{d2Hdx2}
\frac{d^2H}{dx^2} = i\omega \sigma \mu_0 H
\end{equation}
(для медного цилиндра можно положить $\mu\approx 1$).

Граничные условия для \eqref{d2Hdx2} зададим в виде
\begin{equation}\eqmark{borderH}
H(0) = H_0,\qquad H(h) = H_1.
\end{equation}
Здесь $H_0$ --- амплитуда колебаний магнитного поля на внешней
границе цилиндра. Её значение определяется только током
в обмотке соленоида, и совпадает с полем внутри соленоида
в отсутствие цилиндра. Величина~$H_1$ также поддаётся непосредственному
измерению --- это амплитуда колебаний однородного поля внутри цилиндра.
Поля~$H_0$ и~$H_1$ не являются независимыми --- они связаны через
решение уравнений поля вне проводника, т.\,е. внутри <<экрана>>. Эта связь
выражена соотношением~\eqref{borderE}.


Решение \eqref{d2Hdx2} ищем в виде
\begin{equation} \eqmark{24}
H(x)=A e^{\alpha x}+ B e^{-\alpha x},
\end{equation}
где $A$, $B$~--- определяемые из граничных условий константы,
\begin{equation}\eqmark{alpha}
\alpha=\sqrt{i\omega \sigma \mu_0}=\frac{1+i}{\delta} =
\frac{\sqrt{2}}{\delta} e^{i\pi/4}
\end{equation} --- один из корней уравнения \chaptereqref{18},
$\delta$ --- глубина скин-слоя \chaptereqref{20}.
Заметим, что это решение немного отличается от \chaptereqref{19}: ранее мы использовали только
один корень уравнения \chaptereqref{18}, однако здесь мы имеем дело уже не с
полупространством, а с \emph{конечной} областью в виде плоского слоя $h$,
поэтому решение должно содержать оба корня.

Первое условие \eqref{borderH} даёт $A+B = H_0$, что
позволяет исключить $A$ из~\eqref{24}:
\[
H(x) = H_0 e^{-\alpha x} + 2B \sh \alpha x.
\]
Выразим электрическое поле из закона Ампера \chaptereqref{ampere-dif}.
В~одномерном случае
\[
E(x) = \frac{1}{\sigma} \frac{dH}{dx} =
\frac{\alpha}{\sigma}\left(-H_0 e^{-\alpha x}+ 2B \ch \alpha x\right).
\]
Далее положим $x=h$, воспользуемся условием \eqref{borderE},
и, исключив константу $B$, получим после преобразований связь
между~$H_0$ и~$H_1$:
\begin{equation}\eqmark{H0H1}
H_1 = \frac{H_0}{\ch \alpha h + \frac12 \alpha a \sh (\alpha h)}.
\end{equation}
%Пусть заданы граничные значения электрического поля $E_0 = E(0)$ ---
%амплитуда поля на внешней границе цилиндра, и $E_1 = E(h)$ --- на внутренней.
%Тогда
%\[
%E(0)=A+B=E_0,\qquad E(h)= Ae^{-\alpha h} + Be^{\alpha h}=E_1.
%\]
%Отсюда найдём константы $A$ и $B$:
%\begin{equation}\eqmark{AB}
%A=\frac{e^{\alpha h}E_0-E_1}{e^{\alpha h}-e^{-\alpha h}},\qquad
%B=\frac{ E_1 - e^{-\alpha h}E_0}{e^{\alpha h}-e^{-\alpha h}}.
%\end{equation}
%Решение \eqref{24} мы ищем для $0\le z\le h$,
%отсчитывая $z$ от поверхности вглубь цилиндра.
%При $z \le 0$: $H(z)=H_0$, где $H_0$~--- магнитное поле
%в пустом соленоиде (без цилиндра);
%при $z \ge h$: $H(z)=H_1$, здесь $H_1$~--- поле внутри цилиндра,
%помещённого в соленоид.
%Граничные условия для решения \eqref{24} запишутся в следующем виде:
%\[
%H(0)=A+B=H_0,\qquad H(h)= Ae^{\alpha h} + Be^{-\alpha h}=H_1.
%\]
%Отсюда найдём $A$ и $B$, для краткости обозначив $q = e^{\alpha h}$:
%\[
%A=\frac{H_0-qH_1}{1-q^2},\qquad
%B=\frac{q H_1 - q^2 H_0}{1-q^2}.
%\]

%Непосредственному измерению поддаётся магнитное поле $H_0$ на внешней границе
%цилиндра (оно совпадает с полем внутри пустого соленоида),
%а также магнитное поле внутри цилиндра $H_1$.
%
%Электрическое и магнитное поля в цилиндре связаны
%уравнением~\chaptereqref{ampere-dif}. В одномерном приближении
%\[
%\frac{dH_z}{dx} = -\sigma E_{\varphi}.
%\]
%Проинтегрируем обе части по толщине стенки $0\le x \le h$
%и, пользуясь полученными выше выражениями \eqref{AB} для $A$ и $B$,
%после преобразований найдём:
%\[
%H_0 - H_1 = \sigma \left(
%-\frac{A}{\alpha} (e^{-\alpha h}-1) +
%\frac{B}{\alpha} (e^{\alpha h}-1) \right) =
%\sigma \frac{E_0 + E_1}{\alpha} \th \frac{\alpha h}{2},
%\]
%где $\th x = \frac{e^x-e^{-x}}{e^x+e^{-x}}$ --- гиперболический тангенс.
%
%\[
%\frac{dE}{dx} = -i\omega \mu_0 H
%\]
%\[
%H = - \sigma \int E dx
%\]

%Для упрощения вычислений учтём, что при $h\ll a$ поток магнитного
%поля в полости цилиндра (через площадь $\pi a^2$) значительно больше потока
%в тонкой стенке (через площадь $2\pi a h$). Поэтому в силу закона
%электромагнитной индукции поля $E_0$ и $E_1$ различаются незначительно
%(),
%так что $E_0+E_1\approx 2E_1$. При этом поле $E_1$ определяется
%соотношением~\eqref{borderE}, где $dH_1/dt = i\omega H_1$.
%В~результате получаем уравнение, связывающее~$H_0$ и~$H_1$:
%\[
%H_0 - H_1 = -  H_1 \frac{i\omega \sigma \mu_0 a}{\alpha}\th \frac{\alpha h}{2}.
%\]
%С учётом выражения для $\alpha$ \chaptereqref{18},
%окончательно найдём связь между комплексными амплитудами
%поля снаружи и внутри цилиндра:
%\begin{equation}\eqmark{H1}
%H_1 = \frac{H_0}{1- \alpha a \th \frac{\alpha h}{2}}.
%\end{equation}

Рассмотрим предельные случаи \eqref{H0H1}.

1. При \emph{малых частотах} толщина скин-слоя превосходит толщину цилиндра
$\delta \gg h$. Тогда $|\alpha h| \ll 1$,
поэтому $\ch \alpha h \approx 1$, $\sh \alpha h\approx \alpha h$ и
\begin{equation}\eqmark{H1}
H_1 \approx \frac{H_0}{1 + i \frac{a h}{\delta ^2}}.
\end{equation}
Заметим, что величина $ah/\delta^2$ в общем случае не мала, поскольку
при $h\ll a$ возможна ситуация $h\ll \delta \ll a$.
Отношение модулей амплитуд здесь будет равно
\begin{equation} \eqmark{27}
\frac{|H_1|}{|H_0|} = \frac{1}{\sqrt{1+\left(\dfrac{ah}{\delta^2}\right)^2}} =
\frac{1}{\sqrt{1+\frac14 (ah \sigma \mu_0 \omega)^2}}.
\end{equation}
При этом колебания $H_1$ отстают по фазе от $H_0$ на угол $\psi$,
определяемый равенством $\tg \psi = \frac{ah}{\delta^2}$.

2. При достаточно \emph{больших частотах} толщина скин-слоя станет меньше толщины стенки:
$\delta \ll h$. Тогда $|\alpha h| \gg 1$ и $|\alpha a| \gg 1$, а
также $\sh(\alpha h)\approx\ch(\alpha h)\approx\frac12e^{\alpha h}$.
Выражение \eqref{H0H1} с учётом \eqref{alpha} переходит в
\begin{equation} \eqmark{28}
\frac{H_1}{H_0}= \frac{4}{\alpha a} e^{-\alpha h}
= \frac{2\sqrt{2}\delta}{a}e^{-\tfrac{h}{\delta}}\,e^{-i\left(\tfrac{\pi}{4}+\tfrac{h}{\delta}\right)}.
\end{equation}
Как видно из формулы \eqref{28}, в этом пределе поле внутри цилиндра по модулю в
$\frac{2\sqrt{2}\delta}{a}e^{-h/\delta}$ раз меньше, чем снаружи, и,
кроме того, запаздывает по фазе на
\begin{equation} \eqmark{29}
\psi=\frac{\pi}{4}+\frac{h}{\delta}=
\frac{\pi}{4}+h\sqrt{\frac{\omega\sigma\mu_0}{2}}.
\end{equation}

На рис.~\figref{skin2} схематично изображено распределение магнитного 
поля от координаты в двух рассмотренных предельных случаях.

\begin{figure}
    \centering
    \pic{\textwidth}{Chapter_7/skin2}\par
    \caption{Распределение амплитуды колебаний магнитного поля (пунктир)
        и его мгновенного значения при некотром $t$ (сплошная) в зависимости 
        от расстояния до внешней стенки цилиндра. 
        Слева случай низких частот ($\delta \gg h$), справа --- скин-эффект
        при высоких частотах ($\delta \ll h$)}
    \figmark{skin2}
\end{figure}

%\rfr{35mm}{2}{}{2}{
%\psfrag{x}{$x$}
%\psfrag{y}{$z$}
%\psfrag{a}[bl]{$a$}
%\psfrag{b}[tl]{$a-h$}
%\psfrag{h}[bc]{$H_0e^{i\omega t}$}
%\psfrag{o}[ct]{$0$}
%}


%TODO:figure

%{
%\psfrag{a}[cb]{I}
%\psfrag{b}[cb]{II}
%\psfrag{A}[cc]{$A$}
%\psfrag{V}[cc]{$V$}
%\psfrag{R}[cb]{$R$}
%\psfrag{c}[cb]{Осциллограф}
%\psfrag{d}[cb]{Генератор}
%\psfrag{1}[tr]{1}
%\psfrag{2}[br]{2}
%\psfrag{3}[cr]{3}
%\fcris{4}{Схема экспериментальной установки для исследования скин-эффекта в полом цилиндре}{3}
%}

\experiment
Схема экспериментальной установки для исследования проникновения переменного магнитного поля в медный полый цилиндр
изображена на рис.~3. Переменное магнитное поле создаётся с помощью соленоида, намотанного на полый цилиндрический каркас 1
из поливинилхлорида, который подключается к генератору звуковой частоты. Внутри соленоида расположен медный цилиндрический
экран 2. Для измерения магнитного поля внутри экрана используется измерительная катушка 3. Необходимые параметры
соленоида, экрана и измерительной катушки указаны на установке. Действующее значение переменного тока в цепи соленоида
измеряется амперметром~$A$, а действующее значение напряжения
на измерительной катушке измеряет вольтметр~$V$. Для измерения сдвига фаз между током в цепи соленоида и напряжением на измерительной катушке
используется двухканальный осциллограф. На вход одного канала подаётся напряжение с резистора $R$, которое
пропорционально току, а на вход второго канала~--- напряжение с измерительной катушки.

\begin{figure}[h!]
    \centering
    \pic{0.8\textwidth}{Chapter_7/skin4}
    \caption{Экспериментальная установка для изучения скин-эффекта}
\end{figure}
\paragraph{Измерение отношения амплитуд магнитного поля внутри и вне экрана}

С помощью вольтметра~$V$ измеряется действующее значение ЭДС индукции,
которая возникает в измерительной катушке, находящейся в переменном магнитном поле
$H_1e^{i\omega t}$.
ЭДС индукции в измерительной катушке равна
\[
\mathcal{E}=-SN\frac{dB_{1}(t)}{dt}=-i\omega \mu_0 S N H_1 e^{i\omega t},
\]
где $SN$~--- произведение площади витка на число витков измерительной катушки.
Показания вольтметра, измеряющего это напряжение:
\[
V= \frac{SN\omega}{\sqrt{2}}\mu_0|H_1|.
\]
Видно, что модуль амплитуды магнитного поля внутри экрана $|H_1|$ пропорционален~$V$
и обратно пропорционален частоте сигнала $\nu = \omega / 2\pi$:
\[
|H_1|\propto \frac{V}{\nu}.
\]
При этом поле вне экрана $|H_0|$ пропорционально току $I$ в цепи соленоида,
измеряемому амперметром~$A$:
\[
|H_0| \propto I.
\]
Следовательно,
\begin{equation} \eqmark{H1H0exp}
\frac{|H_1|}{|H_0|} = \const \cdot \frac{V}{\nu I}.
\end{equation}

Таким образом, отношение амплитуд магнитных полей снаружи и вне экрана
(коэффициент ослабления) может быть измерено
по отношению $V/\nu I$ при разных частотах. Неизвестная константа
в соотношении \eqref{H1H0exp} может быть определена по измерениям
при малых частотах $\nu \to 0$, когда согласно \eqref{27} $|H_1|/|H_0| \to 1$.

%Такой способ измерения коэффициента ослабления магнитного поля проводящим
%экраном не требует поддерживать постоянный ток через соленоид при измерении
%частотной зависимости этого коэффициента.

\paragraph{Определение проводимости материала экрана}

В установке в качестве экрана используется медная труба промышленного
производства. Технология изготовления труб оказывает заметное влияние
на электропроводимость. Из-за наличия примесей проводимость меди нашей
трубы отличается от табличного значения (в меньшую сторону).
Для определения $\sigma$ нашего экрана предлагается использовать
частотную зависимость~\eqref{29} фазового сдвига между магнитными полями
внутри и вне экрана при высоких частотах. Как видно из выражения~\eqref{29},
в области больших частот $\omega \gg 1/(h^2\sigma\mu_0)$ зависимость
$\psi(\sqrt{\omega})$ аппроксимируется прямой, проходящей
через точку $\psi(0)=\pi/4$.
По наклону этой прямой можно вычислить проводимость материала экрана.

Процедура измерения разности фаз с помощью осциллографа
подробно описана в работе \ref{lab:phase}.

Заметим, что на схеме, изображённой на рис.~3, на входной канал~II
осциллографа подаётся сигнал с измерительной катушки, который пропорционален
не полю внутри экрана, а его \emph{производной} по времени, а это означает,
что появляется дополнительный сдвиг по фазе на $\pi/2$. Поэтому измеренный
по экрану осциллографа сдвиг по фазе между двумя синусоидами будет
на $\pi/2$ больше фазового сдвига между магнитными полями вне и
внутри экрана.



\begin{lab:task}
\item По известным параметрам установки, приняв проводимость меди
для оценки равной $\sigma \sim 5\cdot 10^{7}$~См/м, рассчитайте частоту
$\nu_h$~[Гц], соответствующую равенству $h=\delta$ толщины стенок экрана
скиновой длине.

\item В области низких частот --- от  $\sim 0,01\nu_h$ до $0,1\nu_{h}$ --- получите зависимость отношения $\xi = V/\nu I$ от
частоты $\nu$ (всего не менее 10 точек). Согласно~\eqref{H1H0exp} величина $\xi$ прямо
пропорциональна коэффициенту ослабления магнитного поля внутри
экрана относительно поля снаружи:
\[\xi=\xi_0 |H_1|/|H_0|.\]

\item Исследуйте зависимость величины $\xi$
и фазового сдвига $\psi$ от частоты $\nu$ при высоких частотах
в диапазоне от $0,2\nu_h$ до $\sim 20\nu_h$
(всего не менее 20 точек).

\tasksection{Обработка результатов}

\item По результатам измерений п.~2 (в области низких частот)
постройте график в координатах $1/\xi^2=f(\nu^2)$.
Убедитесь в том, что зависимость линейна.

Экстраполируя зависимость к точке $\nu=0$, соответствующей
$|H_1|/|H_0|=1$, определите величину $\xi_0$ --- коэффициент
пропорциональности между $\xi=V/\nu I$ и коэффициентом ослабления магнитного поля
$|H_1|/|H_0|$.

По угловому коэффициенту зависимости рассчитайте проводимость меди $\sigma$,
используя \eqref{H0H1}.

\item Частотную зависимость фазового сдвига,
измеренную в п.~3, изобразите на графике в координатах
$\psi(\sqrt{\nu})$ (не забудьте учесть дополнительный сдвиг фаз $\pi/2$!).
Через точку ($\psi=\pi/4$, $\nu=0$) проведите прямую,
которая будет касаться экспериментальной кривой при больших частотах.
По наклону этой прямой вычислите значение проводимости $\sigma$ материала экрана
(см. \eqref{29}).
Сравните с результатом предыдущего пункта и с табличным значением~$\sigma$ для меди.

\item Используя ранее вычисленные значения коэффициента $\xi_0$,
рассчитайте экспериментальные значения коэффициентов
ослабления поля $|H_1|/|H_0|$ для всех измерений п.~2 и~3.
Пользуясь полученным в п.~5 коэффициентом проводимости $\sigma$,
рассчитайте теоретическую зависимость по общей формуле \eqref{H0H1}
(учтите соотношения для гиперболических функций комплексного аргумента:
\[\ch(x+iy) = \ch x \cos y + i \sh x \sin y,
\]
\[\sh(x+iy) = \sh x \cos y + i \ch x \cos y).
\]

Изобразите на графике теоретические и экспериментальные
результаты для зависимости $\frac{|H_1|}{|H_0|}$ от $\nu$
в логарифмическом масштабе по оси абсцисс.
Проанализируйте совпадение эксперимента и теории.

% h = 1,3 mm, a+р = 21 mm
% delta = 70 mm / sqrt{f}
% f_h =

\end{lab:task}


\begin{lab:questions}
\item Воспользовавшись экспериментальным (или табличным) значением
проводимости меди, вычислите глубину проникновения поля~$\delta$
при 50~Гц и 50~кГц. Как изменится ответ для материалов с меньшей
проводимостью?

\item Получите уравнение, описывающее динамику низкочастотного электромагнитного
поля в проводящей среде.

\item Какого рода уравнениями описывается скин-эффект? Какие еще физические
процессы подчиняются аналогичному уравнению?

\item Используя параметры установки, оцените диапазон частот,
при которых использованная теория скин-эффекта применима.

\item Хорошо проводящий медный цилиндр радиусом $r=10$~см помещают в постоянное внешнее магнитное поле
(параллельно силовым линиям). Оцените время, за которое магнитное поле полностью проникнет в образец.

\item Как наличие медной тонкостенной трубки внутри соленоида влияет
на его коэффициент самоиндукции?
Проанализируйте случаи малых и больших частот.

\item Рассчитайте мощность джоулевых потерь в медной трубке из-за токов Фуко.
Рассмотрите случаи малых и больших частот.
%\item Какие два экспериментальных закона лежат в основе современной электродинамики?
%Запишите два уравнения Максвелла, которые отражают эти два закона?
%

%
%\item Какой физический смысл имеет параметр $\delta$,
%называемый глубиной проникновения поля, и от чего он зависит?
%
%\item Для увеличения чувствительности измерительной катушки
%имеет смысл увеличивать её число витков. Какой физический
%эффект сдерживает неограниченный рост числа витков
%измерительной катушки?
%Как нужно наматывать измерительную катушку,
%чтобы уменьшить влияние этого эффекта?

\end{lab:questions}


\begin{lab:literature}
\item \Kirichenko~--- Гл.~15.

\item \SivuhinIII~--- \S~144.

\item \KingLokOlh~--- Ч.~II. \S~8.4.
\end{lab:literature}
