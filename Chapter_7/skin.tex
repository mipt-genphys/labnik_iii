\lab{140}{Скин-эффект в полом цилиндре}

\cel{исследование проникновения переменного магнитного поля в медный полый цилиндр.}

\prin{генератор звуковой частоты, соленоид, намотанный на полый цилиндрический каркас из диэлектрика, медный экран в
виде трубки, измерительная катушка, амперметр, вольтметр, осциллограф.}

Начало современной, классической, макроскопической электродинамики сплошных сред было положено открытием двух
экспериментальных законов: в 1820 году Ампер установил закон взаимодействия электрических токов. Одной из формулировок
этого закона является теорема о циркуляции вектора напряжённости магнитного поля $\vH$:
\[
\oint_{\Gamma}\vH\,d\v{l}=I,
\]
где $I$~--- полный ток через замкнутый контур $\Gamma$, а в 1831~году Фарадей экспериментально открыл и сформулировал
явление электромагнитной индукции:
\[
\oint_{\Gamma}\vE\,d\v{l}=-\frac{d}{dt}\int_S\vB\, d\v{S}.
\]
Максвелл постулировал, что эти законы имеют силу совершенно независимо от присутствия в пространстве проводящих пробных
контуров, т.~е. магнитное поле и вихревое электрическое поле являются объективной реальностью. Этот постулат лежит в
основе теории электромагнитного поля, а вытекающие из него уравнения Максвелла~--- фундамент современной электродинамики
сплошных сред. В системе СИ эти уравнения имеют следующий вид:

\def\mmlskip{\vskip 0.5ex plus 0.2ex minus 0.1ex}
\def\mmlI{\kern10mm}
\def\mmlII{\kern50mm}
\def\mml#1#2#3{\par
\mmlskip
\noindent\mmlI\hbox to 0pt{$\ds #1$\hss}\mmlII\hbox{$\ds #2$}\hfill\hbox{$#3$}%
\mmlskip
}
\mmlskip
\mml{\text{дифференциальная форма}}{\text{интегральная форма}}{} \mml{\divv\vD=\rho,}{\oint\vD\,d\v{S}=q,}{\num{1}}
\mml{\rot\vE=-\ddt{\vB},}{\oint\vE\,d\v{l}=-\int\ddt{\vB}\,d\v{S},}{\num{2}}
\mml{\divv\vB=0,}{\oint\vB\,d\v{S}=0,}{\num{3}}
\mml{\rot\vH=\v{j}+\ddt{\vD},}{\oint\vH\,d\v{l}=\int\v{j}\,d\v{S}+\int\ddt{\vD}\,d\v{S}.}{\num{4}}
\mmlskip

\noindent Уравнение (\r1)~--- это одна из форм записи закона Кулона. Уравнение (\r2)~--- формулировка закона
электромагнитной индукции Фарадея: изменяющееся во времени магнитное поле порождает вихревое электрическое поле.
Уравнение (\r3) утверждает факт отсутствия магнитных зарядов, и, наконец, уравнение (\r4) показывает, что магнитное поле
порождается не только движущимися зарядами (первый член в правой части уравнения), но и изменяющимся во времени
электрическим полем. Этот член был введён Максвеллом. Величина $\ddt{\vD}$ по аналогии с $\v{j}$ называется плотностью
токов смещения.

Фигурирующие в уравнениях (\r1)~--~(\r4) векторы $\vD$ (электрическая индукция) и $\vE$ (напряжённость электрического
поля), $\vB$ (магнитная индукция) и $\vH$ (напряжённость магнитного поля), $\v{j}$ и $\vE$ попарно связаны между собой.
Эта связь определяется средой, в которой происходит электромагнитный процесс. Если можно пренебречь электрическими и
магнитными потерями в среде, то уравнения, описывающие связь этих векторов (материальные уравнения), будут иметь вид
\be5
\vD=\e_0\e\vE,
\ee
\be6
\vB=\mu_0\mu\vH,
\ee
\be7
\v{j}=\s\vE,
\ee
где $\e_0$ и $\mu_0$~--- электрическая и магнитная постоянные вакуума, а $\e$ и $\mu$~--- электрическая и магнитная
проницаемости среды. Характерная для электродинамики величина
\[
c=\frac{1}{\sqrt{\e_0\mu_0}}
\]
имеет размерность скорости и называется электродинамической постоянной, а численно она равна скорости света в вакууме.

\bfno{Электромагнитные волны в проводящей среде} В проводящей среде из-за высокой проводимости можно считать, что токи
смещения равны нулю, а основную роль играет ток проводимости. Будем полагать, что объёмные свободные заряды отсутствуют
по всему объёму проводника. Поскольку током смещения мы пренебрегаем, то $\ddt{\vD}=0$, а $\v{j}=\s\vE$, то уравнение
(\r4) приобретает вид
\[
\rot\vH=\s\vE.
\]
Продифференцируем это уравнение по времени:
\be8
\rot\ddt{\vH}=\s\ddt{\vE}.
\ee
Уравнение (\r2) с учётом материального уравнения (\r6) запишется в виде
\[
\rot\vE=-\mu_0\mu\ddt{\vH}.
\]
Возьмём от обеих частей этого уравнения $\rot$:
\be9
\rot\rot\vE=-\mu_0\mu\rot\ddt{\vH}.
\ee
Из векторного анализа известно, что
\be10
\rot\rot\vE=\grad\divv\vE-\D\vE.
\ee
Подставляя (\r{10}) в (\r9) и учитывая, что $\divv\vE=0$, получим
\be11
\D\vE=\mu_0\mu\rot\ddt{\vH}.
\ee
Исключая из уравнений (\r8) и (\r{11}) $\rot\ddt{\vH}$, получим
\be12
\D\vE=\mu_0\mu\s\ddt{\vE}.
\ee
Совершенно аналогично взяв $\rot$ от обеих частей уравнения (\r4), а затем сравнив полученное уравнение с уравнением
(\r2), найдём, что
\be13
\D\vH=\mu_0\mu\s\ddt{\vH}.
\ee

\rfr[11]{30mm}{1}{}{1}{
\psfrag{x}[tr]{$x$}
\psfrag{z}{$z$}
\psfrag{a}[cl]{$E_0e^{i\w t}$}
\psfrag{b}[cl]{$E_x(z,\,t)$}
}

Рассмотрим одномерный случай, когда вектор $\vE$ направлен, например, вдоль оси $x$ (\p1), т.~е. $E_y=E_z=0$. В этом
случае уравнение для электрического поля в проводящей среде будет иметь вид
\be14
\frac{\d^2E_x}{\d x^2}=\mu_0\mu\s\ddt{E_x}.
\ee
Пусть полупространство $z>0$ заполнено проводящей средой с проводимостью $\s$, а полупространство $z<0$ является
свободным пространством, в котором электрическое поле изменяется по гармоническому закону: $E_x=E_0e^{i\w t}$. Будем
искать решение уравнения (\r{14}) в виде
\be15
E_x(z,\,t)=E(z)e^{i\w t}.
\ee
После подстановки этого решения в (\r{14}) получим уравнение для функции $E(z)$:
\be16
\frac{\d^2E(z)}{\d z^2}=i\mu_0\mu\s\w E(z).
\ee
Решение этого уравнения будем искать в виде
\be17
E(z)=Ae^{\alpha z},
\ee
где $A$~--- константа. Подставляя (\r{17}) в (\r{16}), получим уравнение для $\alpha$:
\be18
\alpha^2=i\mu_0\mu\s\w.
\ee
Отсюда
\[
\alpha=-\sqrt{i}\,\sqrt{\mu_0\mu\s\w}=-\sqrt{\mu_0\mu\s\w}\,e^{i\frac{\pi}{4}}=-\sqrt{\frac{\mu_0\mu\s\w}{2}}(1+i).
\]
Второй корень уравнения (\r{18}) со знаком плюс соответствует росту амплитуды поля, что противоречит физическому смыслу.
Окончательное решение уравнения (\r{14}) для нашего случая будет иметь вид
\be19
E_x(z,\,t)=E_0e^{-\rho z}\,e^{i(\w t-\rho z)},
\ee
где
\[
\rho=\sqrt{\frac{\mu_0\mu\s\w}{2}}.
\]
Из полученного решения (\r{19}) видно, что по мере проникновения переменного электрического поля с частотой $\w$ вглубь
проводника фаза колебаний поля растёт линейно, а амплитуда убывает по экспоненциальному закону. Такой закон спадания
характеризуется расстоянием, на котором амплитуда поля уменьшается в $e$ раз. Это расстояние называется глубиной
проникновения поля:
\be20
\delta=\frac{1}{\rho}=\sqrt{\frac{2}{\mu_0\mu\s\w}}.
\ee
Как видно из этого выражения, с ростом частоты $\w$ электрическое поле всё более <<вытесняется>> к поверхности
проводника. Это явление называется скин-эффектом. Поскольку уравнение для магнитного поля (\r{13}) совершенно аналогично
уравнению для напряжённости электрического поля (\r{12}), то очевидно, что магнитное поле убывает вглубь проводника
точно по такому же закону, как и $E_x(z)$.

В случае цилиндрического сплошного или полого проводника следует решать уравнения (\r{12}) и (\r{13}), записанные в
цилиндрической системе координат. Решение таких уравнений выражается уже не экспоненциальной функцией, а через функции
Бесселя первого и нулевого порядков. Нас будет интересовать случай тонкостенного цилиндра, когда толщина стенки $h\ll a$,
где $a$~--- радиус цилиндра. В этом случае заменим проводящий цилиндрический слой проводящей плоскопараллельной пластиной
толщиной $h$. Пусть эта пластина находится в продольном поле $H_x(t)=H_0e^{i\w t}$ (\p2). Для нашего одномерного случая
уравнение (\r{13}) запишется в виде
\be21
\frac{\d^2H_x}{\d z^2}=\mu_0\mu\s\ddt{H_x}.
\ee

\rfr{35mm}{2}{}{2}{
\psfrag{x}{$x$}
\psfrag{y}{$z$}
\psfrag{a}[bl]{$a$}
\psfrag{b}[tl]{$a-h$}
\psfrag{h}[bc]{$H_0e^{i\w t}$}
\psfrag{o}[ct]{$0$}
}

\noindent Решение этого уравнения ищем в виде
\be22
H(z,\,t)=H(z)e^{i\w t}.
\ee
После подстановки (\r{22}) в (\r{21}) получим уравнение для функции $H(z)$:
\be23
\frac{\d^2H(z)}{\d z^2}=i\mu_0\mu\s\w H(z).
\ee
Решение этого уравнения будем искать в виде
\be24
H(z)=Ae^{\frac{\sqrt{2i}}{\delta}z}+Be^{-\frac{\sqrt{2i}}{\delta}z},
\ee
где $A$ и $B$~--- константы, а $\delta$~--- глубина проникновения поля (\r{20}). Ранее при решении аналогичного
уравнения (\r{16}) мы использовали только один корень уравнения (\r{18}): со знаком плюс, а в этом случае мы имеем дело
уже не с полупространством, а с конечной областью в виде плоского слоя, поэтому решение (\r{24}) должно содержать оба
корня уравнения (\r{18}). Решение (\r{24}) мы ищем для $a-h\le z\le a$, а при $z\ge a$ поле $H(z)=H_0$, где $H_0$~---
величина поля в соленоиде без экрана; при $z\le a-h$ поле $H(z)=H_{0с}$, здесь $H_{0с}$~--- величина поля в соленоиде с
экраном. Граничные условия для решения (\r{24}) запишутся в следующем виде:
\[
H(a)=H_0,\qquad H(a-h)=H_{0с}.
\]
Введя обозначение $q=\sqrt{2i}/\delta$, запишем граничные условия в виде
\[
H_0=Ae^{qa}+Be^{-qa},\qquad H_{0с}=Ae^{q(a-h)}+Be^{-q(a-h)}.
\]
Из совместного решения этой системы найдём $A$ и $B$:
\[
A=\frac{H_0e^{-q(a-h)}-H_{0с}e^{-qa}}{e^{qh}-e^{-qh}},\qquad B=\frac{H_{0с}e^{qa}-H_0e^{q(a-h)}}{e^{qh}-e^{-qh}}.
\]
Как видно из полученных выражений для $A$ и $B$, обе константы зависят от неизвестной величины $H_{0с}$, поэтому
следующий этап решения~--- это нахождение $H_{0с}$. Мы не будем проводить дальнейшего подробного решения (в силу
громоздких математических выкладок), а ограничимся схематическим изложением.

Снова переходим к цилиндрическому проводящему слою радиуса $R=a$ и толщиной $h$. Используя уравнение Максвелла (\r2),
находим распределение электрического поля $E(r)$ внутри проводящего слоя. Дифференциальный закон Ома (\r7) позволяет
найти распределение плотности тока проводимости $j(r)$. Проинтегрировав $j(r)$ по слою, найдём полный ток $I$ в слое. На
последнем этапе мы находим поле $H_{0с}$ как суперпозицию внешнего поля $H_0$ и поля, создаваемого суммарным током $I$.
Полученное уравнение будет содержать $H_{0с}$ в качестве неизвестного. В приближении $\frac{h}{a}\ll 1$
\be25
H_{0с}=\frac{2H_0}{ak\sh(kh)+2\ch(kh)},
\ee
где $\ds k=\frac{1+i}{\delta}$, а $\ds\delta=\sqrt{\frac{2}{\mu_0\mu\s\w}}$.

Соотношение (\r{25}) выражает связь между амплитудой магнитного поля $H_0$ вне проводящего цилиндра радиуса $a$ и
толщиной $h$ и амплитудой поля $H_{0с}$ внутри цилиндра. При малых частотах, когда $\frac{h}{\delta}\ll 1$, формула
(\r{25}) запишется в виде
\be26
H_{0с}\approx\frac{H_0}{1+i\frac{ah}{\delta^2}}.
\ee
В этом случае отношение абсолютных величин амплитуд
\be27
\frac{|H_{0с}|}{|H_0|}\approx 1-\frac{(\mu_0\s ah)^2\w^2}{8}.
\ee
При больших частотах, когда $\frac{h}{\delta}\gg 1$, $\sh(kh)\approx\ch(kh)\approx\frac12e^{kh}$, выражение (\r{25})
можно записать в виде
\be28
H_{0с}=(1-i)\frac{\delta}{2a}e^{-kh}\,H_0=\sqrt{2}e^{-\frac{h}{\delta}}\,e^{-i(\frac{\pi}{4}+\frac{h}{\delta})}.
\ee
Как видно из формулы (\r{28}), поле внутри цилиндра запаздывает по фазе:
\be29
\D\psi=\frac{\pi}{4}+\frac{h\sqrt{\mu_0\s\w}}{\sqrt{2}}.
\ee

{
\psfrag{a}[cb]{I}
\psfrag{b}[cb]{II}
\psfrag{A}[cc]{$A$}
\psfrag{V}[cc]{$V$}
\psfrag{R}[cb]{$R$}
\psfrag{c}[cb]{Осциллограф}
\psfrag{d}[cb]{Генератор}
\psfrag{1}[tr]{1}
\psfrag{2}[br]{2}
\psfrag{3}[cr]{3}
\fcris{4}{Схема экспериментальной установки для исследования скин-эффекта в полом цилиндре}{3}
}

\eo Схема экспериментальной установки для исследования проникновения переменного магнитного поля в медный полый цилиндр
изображена на \p3. Переменное магнитное поле создаётся с помощью соленоида, намотанного на полый цилиндрический каркас 1
из полихлорвинила, который подключается к генератору звуковой частоты. Внутри соленоида расположен медный цилиндрический
экран 2. Для измерения магнитного поля внутри экрана используется измерительная катушка 3. Необходимые параметры
соленоида, экрана и измерительной катушки указаны на установке. Действующее значение переменного тока в цепи соленоида
измеряется цифровым амперметром <<$A$>>, а действующее значение напряжения на измерительной катушке измеряет цифровой
вольтметр <<$V$>>. Для измерения сдвига фаз между током в цепи соленоида и напряжением на измерительной катушке
используется двухканальный осциллограф. На вход одного канала подаётся напряжение с резистора $R$, которое
пропорционально току, а на вход второго канала~--- напряжение с измерительной катушки.

\vn Измерение отношения абсолютных величин амплитуд магнитного поля внутри и вне экрана

С помощью вольтметра <<$V$>> мы измеряем действующее значение \eds индукции, которая возникает в измерительной катушке,
находящейся в переменном магнитном поле:
\[
H_{0с}(t)=H_{0с}e^{i\w t}.
\]
\eds индукции в измерительной катушке равна
\[
\E_i=-SN\frac{dB_{0с}(t)}{dt}=-\mu_0SN\frac{dH_{0с}(t)}{dt}=-i\mu_0SN\w H_{0с}e^{i\w t},
\]
где $SN$~--- произведение площади витка на число витков измерительной катушки. Обозначим показание вольтметра <<$V$>>
через $U_к$, тогда
\[
U_к=\frac{\mu_0SN\w}{\sqrt{2}}|H_{0с}|.
\]
Из этого соотношения следует, что абсолютная величина амплитуды магнитного поля внутри экрана
\[
|H_{0с}|\sim\frac{U_к}{\w}\sim\frac{U_к}{f},
\]
где $f$~--- частота генератора в герцах ($\w=2\pi f$). Но поле внутри экрана пропорционально полю вне экрана $H_0$, а
$H_0\sim I_A$, где $I_A$~--- показание амперметра <<$A$>> в цепи соленоида. Следовательно, амплитуда поля внутри экрана,
приведённая к единичному току через соленоид,
\[
|H_{0с}|\sim\frac{U_к}{fI_A}.
\]
Обозначим величину, пропорциональную $|H_{0с}|$, через $\xi_{0с}$:
\be30
\xi_{0с}=\frac{U_к}{fI_A}.
\ee
Нам теперь необходимо найти амплитуду поля вне экрана при том же единичном токе через соленоид. Для этого воспользуемся
соотношением (\r{27}). Проведя измерения $\xi_{0с}$ в диапазоне самых малых частот, мы строим график зависимости
$\xi_{0с}$ от $f^2$. Согласно (\r{27}) эта зависимость имеет вид прямой. Экстраполируя её к $f\to 0$, мы и получим
величину $\xi_{0с}(0)=\xi_0$, которая пропорциональна амплитуде поля вне экрана при единичном токе через соленоид.
Отношение амплитуд магнитного поля внутри экрана и вне при фиксированной частоте $f$ будет равно
\be31
\frac{|H_{0с}|}{|H_0|}=\frac{\xi_{0с}(f)}{\xi_0}=\frac{U_к}{fI_A\xi_0}.
\ee
Такой способ измерения коэффициента ослабления магнитного поля проводящим экраном не требует поддерживать постоянный ток
через соленоид при измерении частотной зависимости этого коэффициента.

\vn Определение проводимости материала экрана

В нашем случае в качестве экрана используется медная труба промышленного производства. Технология изготовления труб
оказывает заметное влияние на электропроводимость. Электропроводимость меди нашей трубы отличается от табличного
значения в область заниженного значения. Для определения электропроводимости меди нашего экрана предлагается
использовать частотную зависимость фазового сдвига между магнитными полями внутри экрана и вне в области больших частот
(\r{29}). Как видно из выражения (\r{29}), в области больших частот зависимость $\D\psi(\sqrt{f})$ аппроксимируется
прямой проходящей через точку $\D\psi(0)=\pi/4$. По наклону этой прямой можно вычислить проводимость материала экрана.

Несколько слов об измерении сдвига фаз $\D\psi$. На схеме, изображённой на \p3, видно, что на II входной канал
осциллографа подаётся сигнал с измерительной катушки, который пропорционален не полю внутри экрана, а его производной по
времени, а это означает, что появляется дополнительный сдвиг по фазе на $\pi/2$. Поэтому измеренный по экрану
осциллографа сдвиг по фазе между двумя синусоидами будет на $\pi/2$ больше фазового сдвига между магнитными полями вне и
внутри экрана.

\zad

\n В области низких частот 20--100~Гц (через $\D f=10$~Гц) снимите зависимость <<амплитуды>> $\xi_{0с}$ магнитного поля
внутри экрана от частоты.

\n Одновременно исследуйте зависимость $\xi_{0с}$ и фазового сдвига $\D\psi$ от частоты $f$ в диапазоне 100~Гц--30~кГц
(100, 500, 1~кГц, 2\dots10~кГц, 12, 14\dots 30~кГц).

\csf{Обработка результатов}\resetnum

\n Результаты измерений <<амплитуды>> $\xi_{0с}$ в области низких частот постройте на графике в координатах $\xi_{0с}$
от $f^2$. Через отложенные точки проведите прямую и, экстраполируя её к $f=0$, определите <<амплитуду>> внешнего поля
$\xi_0$.

\n Частотную зависимость фазового сдвига $\D\psi$ в диапазоне частот 100~Гц--30~кГц изобразите на графике в координатах
$\D\psi$ и $\sqrt{f}$. Через точку $\D\psi=\pi/4$ при $f=0$ проведите прямую, которая будет касаться экспериментальной
кривой при больших частотах. По наклону этой прямой вычислите значение проводимости $\s$ материала экрана. Сравните с
табличным значением $\s$ для меди.

\n Используя ранее найденное значение <<амплитуды>> внешнего поля $\xi_0$ и результаты измерений $\xi_{0с}$ в диапазоне
100~Гц--30~кГц, вычислите коэффициент ослабления магнитного поля $\frac{|H_{0с}|}{|H_0|}$ по формуле (\r{31}). Изобразите
эти результаты на графике $\frac{|H_{0с}|}{|H_0|}$ в зависимости от $\sqrt{f}$. Рассчитайте аналогичную теоретическую
зависимость (\r{25}) и для сравнения нанесите её на экспериментальный график.

\n Воспользовавшись найденным значением $\s$, вычислите глубину проникновения поля $\delta$ при двух частотах: 50~Гц и
$10^5$~Гц.

{\small

\kv

\n Какие два экспериментальных закона лежат в основе современной электродинамики? Запишите два уравнения Максвелла,
которые отражают эти два закона?

\n Какое уравнение описывает поведение электромагнитного возмущения в проводящей среде?

\n Какой физический смысл имеет параметр $\delta$, называемый глубиной проникновения поля, и от чего он зависит?

\n Для увеличения чувствительности измерительной катушки имеет смысл увеличивать её число витков. Какой физический
эффект сдерживает неограниченный рост числа витков измерительной катушки? Как нужно наматывать измерительную катушку,
чтобы уменьшить влияние этого эффекта?

\bv

\lit

\n {\em Сивухин Д.В.} Общий курс физики. Т.~III. Электричество.~--- М.: Наука, 1983. \S~144.

\n {\em Кингсеп А.С., Локшин Г.Р., Ольхов О.А.} Основы физики. Т.~I. М.: ФИЗМАТЛИТ, 2001. \S~8.4.

}
