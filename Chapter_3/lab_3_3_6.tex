\newpage
\lab{Влияние магнитного поля на проводимость полупроводников}

\aim{измерение магнетосопротивления полупроводниковых образцов различной формы.}

\equip{электромагнит, милливеберметр или цифровой магнитометр на основе датчика Холла, цифровой вольтметр, амперметр, миллиамперметр, реостат, образцы монокристаллического антимонида индия (InSb) $n$-типа.}

Элементарная теория свободных носителей заряда в металлах и полупроводниках изложена во введении.

\experiment Схема установки для исследования магнетосопротивления полупроводников и геометрического резистивного эффекта представлена на рис.~\figref{fig3.6.1}.
\begin{figure}[h!]
	\pic{0.9\textwidth}{3_6_1}
	\caption{Схема установки для~исследования влияния магнитного~поля на~проводимость полупроводников}
	\figmark{fig3.6.1}
\end{figure}

В зазоре электромагнита (рис.~\figref{fig3.6.1}а) создаётся постоянное магнитное поле, величину которого можно менять с помощью источника питания электромагнита.

Магнитная индукция в зазоре измеряется при помощи милливеберметра ( его описание и правила работы с ним приведены на с.~\pageref{MWB}) или цифрового магнитометра на основе датчика Холла.

Образец в форме кольца (диск Корбино) или пластинки, смонтированный в специальном держателе, подключается к источнику постоянного напряжения 5 В. При замыкании ключа $K_2$ сквозь образец течёт ток, величина которого измеряется миллиамперметром $A_2$ и регулируется реостатом $R_2$. Балластное сопротивление $R_0$ ограничивает ток через образец. Измеряемое напряжение подаётся на вход цифрового вольтметра.

\begin{lab:task}

В работе предлагается при постоянном токе через образец исследовать зависимость напряжения на образце от величины магнитного поля и от ориентации образца в магнитном поле; по результатам измерений рассчитать подвижность электронов, удельное сопротивление материала образца и концентрацию электронов.

\begin{enumerate}

\item{ Подготовьте приборы к работе.}

\item{ Концы от точек 3 и 4 разъёма подсоедините к клеммам вольтметра.}

\item{ Присоедините диск Корбино через разъём к цепи питания. Определите диапазон изменения силы тока через образец.}

\item{ Определите диапазон изменения силы тока через электромагнит и подберите подходящий предел измерений амперметра $A_1$.}

\item{ Прокалибруйте электромагнит~--- с помощью цифрового магнитометра, либо с помощью милливеберметра измерьте зависимость индукции $B$ магнитного поля в зазоре от тока $I_M$ через обмотки магнита. В случае использования милливеберметра для расчёта индукции измерьте поток $\Phi$ вектора магнитной индукции, который пронизывает пробную катушку, находящуюся в зазоре ($\Phi=BSN$). Значение $SN$ (площадь сечения контура катушки на число витков в ней) указано на держателе катушки.}

\item{ Проведите измерения магнитного потока для 6 -- 8 значений тока $I_M$ через электромагнит.}

\item{ Исследуйте магнетосопротивление образцов. Для этого вставьте диск в зазор выключенного электромагнита и установите ток через образец $I_0=25$ мА. Измерьте падение напряжения $U_0$ на образце.}

\item{ Включите электромагнит и снимите зависимость напряжения $U$ на образце от тока $I_M$ через обмотки магнита при фиксированном токе $I_0=25$ мА через образец.}

\item{ Проверьте, что результат измерения не зависит от направления магнитного поля.}

\item{ Вместо диска Корбино подключите к измерительной цепи образец, имеющий форму пластинки. Поместите образец в зазор выключенного электромагнита и измерьте падение напряжения $U_0$ на образце при токе через образец $10$ мА.}

\item{ Включите электромагнит и снимите зависимость напряжения $U$ на образце от тока через магнит при постоянном токе $I=10$ мА. При измерениях длинная сторона образца должна быть направлена поперёк поля, а средняя (ширина) в~одной серии опытов располагается вдоль, а в другой~--- поперёк поля.}

\item{ Запишите размеры диска (толщину $h$, внутренний $r_1$  и внешний $r_2$ радиусы) и характеристики приборов.}
\end{enumerate}

\tasksection{Обработка результатов}
\begin{enumerate}

\item { Рассчитайте индукцию магнитного поля и постройте график зависимости $B=f(I_{M})$.}

\item { На одном листе постройте графики для всех трёх серий, отложив по оси $X$ величину $B^2$, а по оси $Y$~--- $(U-U_0)/U_0$.}

\item { По наклону прямолинейного участка графика для диска Корбино рассчитайте с помощью формулы (\ref{MagnetoSoprot}) подвижность носителей.}

\item { Вычислив сопротивление диска в отсутствие магнитного поля и зная геометрические размеры образца, рассчитайте удельное сопротивление материала образца $\rho_0$ на основе формулы (\ref{KorbinoSoprot}).}

\item {С помощью формулы (\ref{UdSoprot}) найдите концентрацию носителей тока.}

\item { Оцените погрешности и сравните результаты с табличными.}
\end{enumerate}
\end{lab:task}

\begin{lab:questions}

\item{ Исследуйте уравнения движения электронов в прямоугольной пластинке. Зависит ли сопротивление пластинки от индукции магнитного поля?}

\item{ Поясните качественно (без формул), почему сопротивление образца зависит от магнитного поля.}
\end{lab:questions}

\begin{lab:literature}

\item{ \emph{Сивухин Д.В.} Общий курс физики. Т.~III. Электричество~--- М.: Физматлит, 2015. \S\S~98, 100.}

\item{ \emph{Парселл Э.} Электричество и магнетизм.~--- М.: Наука, 1983. Гл.~4, \S\S~4--6; Гл.~6, \S~9 (Берклеевский курс физики. Т.~II).}

\end{lab:literature}

\newpage

\labsection{МИЛЛИВЕБЕРМЕТР}\label{MWB}

\labsection{Устройство и принцип действия}

Милливеберметр (флюксметр) служит для измерения постоянного во времени магнитного потока. Это прибор
магнитоэлектрической системы, работающий в баллистическом режиме: рамка с током вращается в поле постоянного магнита; отклонение рамки пропорционально заряду, если через неё пропускается короткий импульс тока. От обычных гальванометров постоянного тока милливеберметр отличается тем, что на его рамку не действуют никакие упругие силы, поэтому его подвижная часть находится в безразличном равновесии.

\begin{figure}[h!]
	\pic{0.75\textwidth}{mwb-01}
	\caption{Рамка в магнитном поле}
	\figmark{figMWB.1}
\end{figure}

В цепь рамки прибора включается наружная измерительная (пробная) катушка. При изменении магнитного потока,
пронизывающего эту катушку, в ней возникает ЭДС индукции, и по цепи рамки течёт индукционный ток. При этом отклонение рамки, независимо от её начального положения, пропорционально изменению магнитного потока $\Delta\Phi$ и может служить для его измерения.

Рассмотрим работу милливеберметра. Уравнение вращательного движения рамки имеет вид
\begin{equation}
	J\ddot{\varphi}=M,
	\eqmark{MWB.1}
\end{equation}
где $J$~--- момент инерции рамки милливеберметра, $\varphi$~--- угол её поворота (рис.~\figref{figMWB.1}). Момент сил $M$ определяется путём умножения силы $F=IlNB_0$, действующей на каждую из продольных сторон рамки (направленных вдоль оси вращения), на удвоенное плечо, т.е. на поперечный размер рамки $a$, здесь $I$~--- сила тока в рамке, $l$~--- длина продольной стороны,$N$~--- число витков намотанного на рамку провода, $B_0$~--- индукция поля постоянного магнита милливеберметра. Поле магнита радиально, это обеспечивает равномерность шкалы прибора. Таким образом,
\begin{equation*}
	J\ddot{\varphi}=ISNB_0,
\end{equation*}
где $S=la$~--- площадь рамки. Введя обозначение $K=SNB_0$, получим
\begin{equation}
	J\ddot{\varphi}=KI.
	\eqmark{MWB.2}
\end{equation}

Вычислим теперь ток $I$. Этот ток генерируется под действием как внешней ЭДС индукции $\mathscr{E}_K$, возникающей в измерительной катушке, так и внутренней $\mathscr{E}_P$, возникающей в рамке при её движении в магнитном поле:
\begin{equation}
	RI=\mathscr{E}_K+\mathscr{E}_P,
	\eqmark{MWB.3}
\end{equation}
где $R$~--- полное сопротивление цепи рамки.

Внешняя ЭДС $\mathscr{E}_K$ наводится в измерительной катушке при изменении проходящего сквозь неё магнитного потока $\Phi$:
\begin{equation}
	\mathscr{E}_K=-\frac{d\Phi}{dt},
	\eqmark{MWB.4}
\end{equation}
а $\mathscr{E}_P$ возникает в продольных сторонах рамки при их движении в поле постоянного магнита со скоростью $v=\dot{\varphi}a/2$:

\begin{equation}
	\mathscr{E}_P=-SNB_0\dot\varphi=-K\dot\varphi.
	\eqmark{MWB.5}
\end{equation}

Подставим \eqref{MWB.3} -- \eqref{MWB.5} в \eqref{MWB.2}, и уравнение движения рамки принимает вид
\begin{equation}
	\frac{JR}{K^2}\ddot{\varphi}+\dot{\varphi}=-\frac{1}{K}\dot{\Phi}.
	\eqmark{MWB.6}
\end{equation}

Проинтегрируем это уравнение по времени:
\begin{equation}
	\frac{JR}{K^2}(\dot{\varphi_2}-\dot{\varphi_1})+(\varphi_2-\varphi_1)=-\frac{1}{K}(\Phi_2-\Phi_1).
	\eqmark{MWB.7}
\end{equation}

Для измерения магнитного потока с помощью милливеберметра можно:

а) вынести измерительную катушку из области измеряемого в область нулевого поля;

б) оставив катушку в поле неподвижной, отключить измеряемое поле.

В любом  из этих вариантов скорость изменения потока $\dot{\Phi}$ в начале и в конце опыта равна нулю. В начале опыта рамка милливеберметра не движется, так что $\dot{\varphi}_1=0$. Покажем, что и $\dot{\varphi}_2=0$.

В~самом деле, при $\dot{\Phi}=0$ в уравнении \eqref{MWB.6} пропадает правая часть. В отсутствие внешних сил рамка рано или поздно должна остановиться вследствие действия сил электромагнитного торможения. Можно найти закон движения при торможении, решив дифференциальное уравнение \eqref{MWB.6}:
\begin{equation}
	\dot{\varphi}=\dot{\varphi}(0)\,\exp\left(-\frac{K^2}{JR}t\right),
	\eqmark{MWB.8}
\end{equation}
где $\dot{\varphi}(0)$~--- начальная угловая скорость рамки. При больших $t$ угловая скорость $\dot{\varphi}$ оказывается экспоненциально мала, т.~е. $\dot{\varphi}_2\rightarrow 0$.

Подставляя $\dot{\varphi}_1=0$ и $\dot{\varphi}_2=0$ в \eqref{MWB.7}, найдём
\begin{equation}
	\varphi_2-\varphi_1=-\frac{1}{K}(\Phi_2-\Phi_1).
	\eqmark{MWB.9}
\end{equation}

Таким образом, угол отклонения рамки милливеберметра пропорционален изменению магнитного потока, пронизывающего измерительную катушку. Коэффициент пропорциональности выбирается так, что шкала прибора градуируется в милливеберах. Разделив поток на площадь и число витков измерительной (пробной) катушки, мы определим индукцию $B$ внешнего магнитного поля.

Обратим внимание на структуру формулы \eqref{MWB.8}. Время $t$, в течение которого затухает движение рамки, должно быть небольшим, т.к. рамка находится в~безразличном равновесии и склонна дрейфовать. Самопроизвольное перемещение стрелки искажает результаты измерений. Из \eqref{MWB.8} видно, что время успокоения прибора падает с уменьшением $R$, поэтому милливеберметр работает правильно лишь при замыкании его рамки на достаточно малое сопротивление. Допустимая величина сопротивления измерительной катушки указана на приборе.

Принципиальная схема милливеберметра изображена на рис.~\figref{figMWB.2}. Так как прибор не имеет противодействующего механического момента, стрелка его после измерения не возвращается к начальному положению. Для установки стрелки на нужную отметку служит электромагнитный корректор~--- вторая магнитная система, состоящая из постоянного магнита и сердечника с обмоткой. Когда ручка переключателя находится в положении <<Корректор>>, обмотка корректора замкнута на рамку прибора, в которой в момент поворота ручки корректора (вследствие пересечения силовых линий магнита корректора) возникает ток.
Изменяя направление и угол поворота ручки корректора, можно установить стрелку прибора на любом делении шкалы.\
\begin{figure}[h!]
	\pic{0.9\textwidth}{mwb-02}
	\caption{Схема прибора}
	\figmark{figMWB.2}
\end{figure}
При положении ручки переключателя на отметке <<Арретир>> рамка прибора замкнута накоротко, и подвижная система прибора находится в сильно успокоенном режиме.

В положении <<Измерение>> прибор готов к работе.

\labsection{ПРАВИЛА РАБОТЫ}

\labsection{Общие указания}

\begin{enumerate}

\item{Для измерения магнитного потока подключённая к прибору измерительная катушка помещается в магнитное поле перпендикулярно ему.}

\item{Для исключения погрешности от паралакса отсчёт показаний следует проводить так, чтобы изображение стрелки в зеркале шкалы совпадало с самой стрелкой.}
\end{enumerate}

\labsection{Измерение магнитного потока}

\begin{enumerate}

\item{Поставьте переключатель в положение <<Корректор>> и поворотом рукоятки корректора установите начальное положение стрелки, удобное для измерений.

Если ручка корректора дошла до упора, а стрелка сместилась недостаточно, поверните рукоятку корректора в обратную сторону до упора, а затем снова поворачивайте её, пока стрелка не встанет на нужное деление.}

\item{Поставьте переключатель в положение <<Измерение>>. Заметьте начальное положение стрелки (вся шкала --- 10 дел. --- 10~mWb).

Измените магнитный поток сквозь катушку до нуля и заметьте новое положение стрелки. Разность показаний определяет магнитный поток.

Изменять магнитный поток рекомендуется одним из способов:

а) быстро удаляя пробную катушку из области действия магнитного поля на расстояние, где магнитный поток практически равен нулю (рекомендуется);

б) выключая магнитное поле, если катушка закреплена жёстко.

Не рекомендуется переполюсовывать магнит для измерений поля, т.~к. при этом часто ломаются переключатели.

Величина $SN$, необходимая для расчёта индукции поля, указана на пробной катушке.}

\item{По окончании работы следует заарретировать прибор~--- поставить переключатель в положение <<Арретир>>.}

\item{Не реже одного раза в месяц рекомендуется проверять состояние приборов по образцовому прибору.}

%Один раз в два года, а также после каждого ремонта, приборы должны проверяться в местном отделении Комитета %стандартов, мер и измерительных приборов.

\end{enumerate}
