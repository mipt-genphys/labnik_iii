\lab{Влияние магнитного поля на проводимость полупроводников}

\aim{измерение магнетосопротивления полупроводниковых образцов различной формы.}

\equip{электромагнит, милливеберметр или цифровой магнитометр на основе датчика
Холла, цифровой вольтметр, амперметр, миллиамперметр, реостат, образцы
монокристаллического антимонида индия (InSb) $n$-типа.}

Элементарная теория свободных носителей заряда в металлах и полупроводниках
изложена во введении.

\todo[inline,author=Popov]{Ну хоть пару слов о работе!}

\experiment
Схема установки для исследования магнетосопротивления
полупроводников и геометрического резистивного эффекта представлена на
рис.~\figref{Scheme}.
\begin{figure}[h!]
	\pic{0.9\textwidth}{3_6_1}
	\caption{Схема установки для~исследования влияния магнитного~поля
на~проводимость полупроводников}
	\figmark{Scheme}
\end{figure}

В зазоре электромагнита (рис.~\figref{Scheme}а) создаётся постоянное магнитное
поле, величину которого можно менять с помощью источника питания электромагнита.

Магнитная индукция в зазоре измеряется при помощи милливеберметра (его описание
и правила работы с ним приведены на с.~\pageref{MWB}) или цифрового магнитометра
на основе датчика Холла.

Образец в форме кольца (диск Корбино) или пластинки, смонтированный в
специальном держателе, подключается к источнику постоянного напряжения 5~В. При
замыкании ключа~$K_2$ сквозь образец течёт ток, величина которого измеряется
миллиамперметром~$A_2$ и регулируется реостатом~$R_2$. Балластное сопротивление~$R_0$
ограничивает ток через образец. Измеряемое напряжение подаётся на вход
цифрового вольтметра.

\begin{lab:task}

В работе предлагается при постоянном токе через образец исследовать зависимость
напряжения на образце от величины магнитного поля и от ориентации образца в
магнитном поле; по результатам измерений рассчитать подвижность электронов,
удельное сопротивление материала образца и концентрацию электронов.

\begin{enumerate}

\item{ Подготовьте приборы к работе.}

\item{ Концы от точек~3 и~4 разъёма подсоедините к клеммам вольтметра.}

\item{ Присоедините диск Корбино через разъём к цепи питания. Определите
диапазон изменения силы тока через образец.}

\item{ Определите диапазон изменения силы тока через электромагнит и подберите
подходящий предел измерений амперметра~$A_1$.}

\item{ Прокалибруйте электромагнит~--- с помощью цифрового магнитометра, либо с
помощью милливеберметра измерьте зависимость индукции~$B$ магнитного поля в
зазоре от тока~$I_M$ через обмотки магнита. В~случае использования
милливеберметра для расчёта индукции измерьте поток~$\Phi$ вектора магнитной
индукции, который пронизывает пробную катушку, находящуюся в зазоре
($\Phi=BSN$). Значение~$SN$ (площадь сечения контура катушки на число витков в
ней) указано на держателе катушки.}

\item{ Проведите измерения магнитного потока для 6--8 значений тока~$I_M$
через электромагнит.}

\item{ Исследуйте магнетосопротивление образцов. Для этого вставьте диск в зазор
выключенного электромагнита и установите ток через образец $I_0=25$~мА. Измерьте
падение напряжения~$U_0$ на образце.}

\item{ Включите электромагнит и снимите зависимость напряжения~$U$ на образце от
тока~$I_M$ через обмотки магнита при фиксированном токе $I_0=25$~мА через
образец.}

\item{ Проверьте, что результат измерения не зависит от направления магнитного
поля.}

\item{ Вместо диска Корбино подключите к измерительной цепи образец, имеющий
форму пластинки. Поместите образец в зазор выключенного электромагнита и
измерьте падение напряжения~$U_0$ на образце при токе через образец $10$~мА.}

\item{ Включите электромагнит и снимите зависимость напряжения~$U$ на образце от
тока через магнит при постоянном токе $I=10$~мА. При измерениях длинная сторона
образца должна быть направлена поперёк поля, а средняя (ширина) в~одной серии
опытов располагается вдоль, а в другой~--- поперёк поля.}

\item{ Запишите размеры диска (толщину~$h$, внутренний~$r_1$  и внешний~$r_2$
радиусы) и характеристики приборов.}
\end{enumerate}

\tasksection{Обработка результатов}
\begin{enumerate}

\item { Рассчитайте индукцию магнитного поля и постройте график зависимости
$B=f(I_{M})$.}

\item { На одном листе постройте графики для всех трёх серий, отложив по оси~$X$
величину~$B^2$, а по оси $Y$~--- $(U-U_0)/U_0$.}

\item { По наклону прямолинейного участка графика для диска Корбино рассчитайте
с помощью формулы \chaptereqref{MagnetoSoprot} подвижность носителей.}

\item { Вычислив сопротивление диска в отсутствие магнитного поля и зная
геометрические размеры образца, рассчитайте удельное сопротивление материала
образца~$\rho_0$ на основе формулы \chaptereqref{KorbinoSoprot}.}

\item {С помощью формулы \chaptereqref{UdSoprot} найдите концентрацию носителей
тока.}

\item { Оцените погрешности и сравните результаты с табличными.}
\end{enumerate}
\end{lab:task}

\begin{lab:questions}

\item{ Исследуйте уравнения движения электронов в прямоугольной пластинке.
Зависит ли сопротивление пластинки от индукции магнитного поля?}

\item{ Поясните качественно (без формул), почему сопротивление образца зависит
от магнитного поля.}
\end{lab:questions}

\begin{lab:literature}

\item{ \emph{Сивухин Д.В.} Общий курс физики. Т.~III. Электричество~--- М.:
Физматлит, 2015. \S\S~98, 100.}

\item{ \emph{Парселл Э.} Электричество и магнетизм.~--- М.: Наука, 1983. Гл.~4,
\S\S~4--6; Гл.~6, \S~9 (Берклеевский курс физики. Т.~II).}

\end{lab:literature}

