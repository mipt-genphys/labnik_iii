\lab[Влияние магнитного поля на проводимость\\ полупроводников]%
{Влияние магнитного поля на проводимость полупроводников}

\aim{измерение зависимости сопротивления 
    полупроводниковых образцов различной формы от индукции магнитного поля.}

\equip{электромагнит, милливеберметр или миллитесламетр (на основе датчика
Холла), вольтметр, амперметр, миллиамперметр, реостат, образцы
монокристаллического антимонида индия (InSb) $n$-типа.}

Перед выполнением работы необходимо ознакомиться с основами
элементарной теории движения носителей заряда в~металлах и полупроводниках
(п.~\ref{sec:halleffect} введения к разделу).

В работе изучается зависимость проводимости полупроводников от величины 
магнитного поля~$B$ в геометрии \emph{диска Корбино}.
По диску, помещенному в перпендикулярное ему магнитное поле, пропускается
ток в радиальном направлении. Магнитное поле искривляет линии 
тока, благодаря чему эффективное сопротивление $R$ образца возрастает
(эффект \emph{геометрического магнетосопротивления}). По результатам
измерений зависимости $R(B)$ могут быть вычислены подвижность и концентрация 
носителей тока в образце.

\experiment
Схема установки для исследования магнетосопротивления
полупроводников и геометрического резистивного эффекта представлена на
рис.~\figref{Scheme}.

\begin{figure}[h!]
    \centering
    \pic{0.9\textwidth}{Chapter_3/3_6_1}
    \caption{Схема установки для~исследования влияния магнитного~поля
        на~проводимость полупроводников}
    \figmark{Scheme}
\end{figure}

В зазоре электромагнита (рис.~\figref{Scheme}а) создаётся постоянное магнитное
поле, величину которого можно менять с помощью источника питания электромагнита.
Ток электромагнита измеряется амперметром А$_1$ (отдельным или встроенным в источник).
Магнитная индукция в зазоре измеряется при помощи милливеберметра (его описание
и правила работы с ним приведены на с.~\pageref{MWB}) или 
миллитесламетра на основе датчика Холла.

Образец в форме кольца (диск Корбино) или пластинки, смонтированный в
специальном держателе, подключается к источнику постоянного напряжения 5~В. При
замыкании ключа~К$_2$ сквозь образец течёт ток, величина которого измеряется
миллиамперметром~А$_2$ и регулируется реостатом~$R_2$. Балластное сопротивление~$R_0$
ограничивает ток через образец. Измеряемое напряжение подаётся на вход
вольтметра~V.

\begin{lab:task}

\taskpreamble{В работе предлагается при постоянном токе через образец исследовать зависимость
напряжения на образце от величины магнитного поля и от ориентации образца в
магнитном поле; по результатам измерений рассчитать подвижность электронов,
удельное сопротивление материала образца и концентрацию электронов.}

\item Подготовьте приборы к работе согласно описанию на установке.

\item Концы от точек~3 и~4 разъёма подсоедините к клеммам вольтметра.
Присоедините диск Корбино через разъём к цепи питания. Определите
диапазон изменения силы тока через образец.

\item Установите ручки регулировки источника питания электромагнита 
в минимальное положение и включите источник. 
Плавно меняя ток магнита~$I_{М}$, определите диапазон его изменения.

\item Измерьте калибровочную кривую электромагнита~---
зависимость между индукцией~$B$ магнитного поля в его зазоре и 
током~$I_{М}$ через обмотки магнита.
Магнитное поле измеряется милливеберметром или миллитесламетром
(датчиком Холла). Калибровочная кривая должна содержать не менее
15 точек во всём диапазоне изменения токов.

В~случае использования милливеберметра измерьте зависимость 
магнитного потока~$\Phi$, пронизывающего пробную катушку, 
находящуюся в зазоре, от тока~$I_{М}$ ($\Phi=BSN$). 
Значение~$SN$ (произведение площади сечения контура катушки на
число витков в ней) указано на держателе катушки.

\item \label{p1} Вставьте образец в зазор \emph{выключенного} электромагнита 
и измерьте падение напряжения~$U_0$ в образце при некотором токе $I_0$
через образец ($I_0\sim 25$~мА, см. рекомендации в описании). 

\item \label{p2} Включите электромагнит и измерьте зависимость напряжения~$U$ 
на образце от тока через обмотки магнита~$I_{М}$ при фиксированном токе
через образец~$I_0$.

\item Проверьте, что результат измерения не зависит от направления магнитного
поля.

%\item Повторите измерения пп. \ref{p1}, \ref{p2} 
%для 3--4 других значений тока $I_0$ через образец.

\item Вместо диска Корбино подключите к измерительной цепи образец, имеющий
форму прямоугольной пластинки. Поместите образец в зазор выключенного 
электромагнита и измерьте падение напряжения~$U_0$ на образце 
при токе через образец $I\sim 10~мА$ (см. рекомендации в описании).

\item Включите электромагнит и получите зависимость напряжения~$U$ на образце от
тока через магнит~$I_{М}$ при постоянном токе через образец $I=10$~мА. 
При измерениях длинная сторона образца должна быть направлена поперёк поля, 
а средняя (ширина) в~одной серии опытов располагается вдоль, 
а в другой~--- поперёк поля.

\item Запишите геометрические размеры образцов и характеристики приборов.

\tasksection{Обработка результатов}

\item Постройте калибровочный график зависимости $B(I_{M})$. 
    Используйте эту зависимость для дальнейшего пересчёта (интерполяции)
    значений поля по измеренным токам $I_{М}$.

\item Для всех серий измерений постройте графики, отложив по оси абсцисс
величину~$B^2$, а по оси ординат~--- величину $(U-U_0)/U_0$. Какие выводы
можно сделать на основании построенных зависимостей?

\item По наклону прямолинейного участка графика для диска Корбино рассчитайте
с помощью формулы \chaptereqref{MagnetoSoprot} подвижность носителей.

\item Определите сопротивление диска $R_0$ в отсутствие магнитного поля.
Рассчитайте концентрацию носителей тока~$n$, удельное сопротивление~$\rho_0$ 
и удельную проводимость $\sigma_0=1/\rho_0$ материала образца.

\item Оцените погрешности и сравните результаты с табличными.

\end{lab:task}


\begin{lab:questions}

\item Какие вещества называют диэлектриками, проводниками, полупроводниками?
Чем объясняется различие их электрических свойств? Как зависит от температуры
проводимость металлов и полупроводников?

\item В чем причина зависимости сопротивления образца от магнитного
поля в геометрии диска Корбино?

\item Зависит ли эффект магнетосопротивления от геометрии образца? 
В каких случаях эффект магнетосопротивления может проявиться
при протекании тока по плоской пластинке?

\item Как устроен милливеберметр? Зависят ли его показания от сопротивления
измерительной катушки? Каким должно быть это сопротивление по сравнению с
сопротивлением рамки прибора: большим или маленьким?

\item По результатам измерений оцените частоту столкновений,
длину пробега и коэффициент диффузии носителей тока в исследуемом материале.

\item Получите выражение для сопротивления пластинки, изготовленной
из материала с~двумя типами носителей. 
Для простоты можно ограничиться случаем $\mu_e \gg \mu_h$.

\end{lab:questions}


\begin{lab:literature}
\item \SivuhinIII~--- \S\S~98, 100.
\item \textit{Парселл Э.} Электричество и магнетизм.~--- М.\,: Наука, 1983. Гл.~4,
\S\S~4--6; Гл.~6, \S~9.
\end{lab:literature}

