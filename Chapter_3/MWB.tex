\newpage

\labsupplement

% TODO: сделать стиль для приложений
\section*{Принцип действия милливеберметра}\label{MWB}
\addcontentsline{toc}{subsection}{\small\textsf{Приложение.} 
Устройство и принцип действия милливеберметра}

\setcounter{figure}{0}
\small 

Милливеберметр (флюксметр) служит для измерения постоянного во времени
магнитного потока. Это прибор магнитоэлектрической системы, 
работающий в баллистическом режиме: рамка с током вращается в поле 
постоянного магнита; отклонение рамки пропорционально заряду,
если через неё пропускается короткий импульс тока. 
От обычных гальванометров постоянного тока милливеберметр 
отличается тем, что на его рамку не действуют никакие упругие силы, 
поэтому его подвижная часть находится в безразличном равновесии.

% TODO: рисунок!
\begin{figure}[h!]
%    \pic{0.75\textwidth}{mwb-01}
    \caption{Рамка в магнитном поле}
    \figmark{Frame in a magnetic field}
\end{figure}

В цепь рамки прибора включается наружная измерительная (пробная) катушка. 
При изменении магнитного потока, пронизывающего эту катушку, 
в ней возникает ЭДС индукции, и по цепи рамки течёт индукционный ток. 
При этом отклонение рамки, независимо от её начального
положения, пропорционально изменению магнитного потока $\Delta\Phi$ 
и может служить для его измерения.

Рассмотрим детально работу милливеберметра. 
Уравнение моментов для рамки имеет вид
\begin{equation}
    J\ddot{\varphi}=M,
    \eqmark{MWB.1}
\end{equation}
где $J$~--- момент инерции рамки, 
$\varphi$~--- угол её поворота (рис.~\figref{Frame in a magnetic field}),
$M$ --- момент сил Ампера, действующих на рамку.
Последний вычисляется как произведение силы Ампера $F=IlNB_0$, 
действующей на каждую из продольных сторон,
на удвоенное плечо, т.е. на поперечный размер
рамки $a$. Здесь $I$~--- сила тока в рамке, $l$~--- длина продольной
стороны, $N$~--- число витков намотанного на рамку провода, 
$B_0$~--- индукция поля постоянного магнита милливеберметра. 
Поле магнита практически радиально, и его величина не зависит от угла поворота
рамки, что обеспечивает равномерность шкалы прибора. 

Таким образом, вводя обозначение $K=alNB_0$, из \eqref{MWB.1} 
получим
\begin{equation}
    J\ddot{\varphi}=KI.
    \eqmark{MWB.2}
\end{equation}

Рассмотрим теперь уравнение цепи, в которую включена рамка.
Ток~$I$ в рамке генерируется под действием как внешней ЭДС
индукции $\mathcal{E}_{К}$, возникающей в измерительной катушке, так и внутренней
$\mathcal{E}_{Р}$, возникающей в рамке при её вращении в магнитном поле:
\begin{equation}
    RI=\mathcal{E}_{К}+\mathcal{E}_{Р},
    \eqmark{MWB.3}
\end{equation}
где $R$~--- полное сопротивление цепи рамки.

Внешняя ЭДС $\mathcal{E}_{К}$ наводится в измерительной катушке при изменении
проходящего сквозь неё магнитного потока~$\Phi$:
\begin{equation}
    \mathcal{E}_{К}=-\frac{d\Phi}{dt}.
    \eqmark{MWB.4}
\end{equation}

ЭДС в рамке $\mathcal{E}_{Р}$ возникает в её продольных сторонах 
при их движении в поле $B_0$ со скоростью $v=\dot{\varphi}a/2$:
\begin{equation}
    \mathcal{E}_{Р}=-2lNB_0\frac{a}{2}\dot\varphi=-K\dot\varphi.
    \eqmark{MWB.5}
\end{equation}

Подставим \eqref{MWB.3}--\eqref{MWB.5} в \eqref{MWB.2}, и 
получим уравнение движения рамки:
\begin{equation}
    \frac{JR}{K^2}\ddot{\varphi}+\dot{\varphi}=-\frac{1}{K}\dot{\Phi}.
    \eqmark{MWB.6}
\end{equation}

Проинтегрируем полученное уравнение по времени. Получим
\begin{equation}
\frac{JR}{K^2} \Delta \dot{\varphi}+
\Delta \varphi = -\frac{1}{K}\Delta \Phi,
    \eqmark{MWB.7}
\end{equation}
где $\Delta$ обозначает разность между конечным и начальным состояниями.

Покажем, что первое слагаемое \eqref{MWB.7} мало и им можно пренебречь.
В начальный момент рамка неподвижна. В начальный момент рамка неподвижна
($\dot{\varphi}_1=0$). Пусть в конце измерения поток через 
измерительную катушку постоянен, $\dot{\Phi}=0$. Тогда в уравнении
\eqref{MWB.6} правая часть равна нулю, и его решение есть 
экспоненциально затухающая функция:
\begin{equation*}
\dot{\varphi}=\dot{\varphi}_0\,\exp\left(-\frac{K^2}{JR}t\right).
\end{equation*}
Следовательно, по прошествии времени $\tau=JR/K^2$ рамка практически
остановится: $\dot{\varphi}(t)\to0$ при $t\gg\tau$. Подбирая достаточно малое 
сопротивление цепи~$R$, можно добиться того, чтобы
время торможения рамки~$\tau$ было достаточно малым в условиях опыта.

Таким образом, угол поворота рамки милливеберметра оказывается прямо пропорционален 
изменению магнитного потока через измерительную катушку:
\begin{equation}
\Delta \varphi=-\frac{1}{K} \Delta \Phi.
\eqmark{MWB.9}
\end{equation}

Для измерения магнитного потока с помощью милливеберметра можно:

а) вынести измерительную катушку из области измеряемого в область нулевого поля;

б) оставив катушку в поле неподвижной, отключить измеряемое поле.

Разделив поток $\Delta \Phi$ на площадь и число витков измерительной катушки, 
можно определить изменение индукции~$\Delta B$ внешнего магнитного поля,
пронизывающего катушку.
Отметим, что в обоих вариантах поток в конце опыта постоянен, как того
требует сделанное выше предположение.


%Обратим внимание на структуру формулы \eqref{MWB.8}. Время~$t$, в течение
%которого затухает движение рамки, должно быть небольшим, так как рамка находится
%в~безразличном равновесии и склонна дрейфовать. Самопроизвольное перемещение
%стрелки искажает результаты измерений. Из \eqref{MWB.8} видно, что время
%успокоения прибора падает с уменьшением~$R$, поэтому милливеберметр работает
%правильно лишь при замыкании его рамки на достаточно малое сопротивление.
%Допустимая величина сопротивления измерительной катушки указана на приборе.

%Принципиальная схема милливеберметра изображена на рис.~\figref{Device scheme}.
%Так как прибор не имеет противодействующего механического момента, стрелка его
%после измерения не возвращается к начальному положению. Для установки стрелки на
%нужную отметку служит электромагнитный корректор~--- вторая магнитная система,
%состоящая из постоянного магнита и сердечника с обмоткой. Когда ручка
%переключателя находится в положении <<Корректор>>, обмотка корректора замкнута
%на рамку прибора, в которой в момент поворота ручки корректора (вследствие
%пересечения силовых линий магнита корректора) возникает ток.
%Изменяя направление и угол поворота ручки корректора, можно установить стрелку
%прибора на любом делении шкалы.
%\begin{figure}[h!]
%%    \pic{0.9\textwidth}{mwb-02}
%    \caption{Схема прибора}
%    \figmark{Device scheme}
%\end{figure}
%При положении ручки переключателя на отметке <<Арретир>> рамка прибора замкнута
%накоротко, и подвижная система прибора находится в сильно успокоенном режиме.
%
%В положении <<Измерение>> прибор готов к работе.
%
%\labsection{ПРАВИЛА РАБОТЫ}
%
%\labsection{Общие указания}
%
%\begin{enumerate}
%
%\item{Для измерения магнитного потока подключённая к прибору измерительная
%катушка помещается в магнитное поле перпендикулярно ему.}
%
%\item{Для исключения погрешности от паралакса отсчёт показаний следует проводить
%так, чтобы изображение стрелки в зеркале шкалы совпадало с самой стрелкой.}
%\end{enumerate}
%
%\labsection{Измерение магнитного потока}
%
%\begin{enumerate}
%
%\item{Поставьте переключатель в положение <<Корректор>> и поворотом рукоятки
%корректора установите начальное положение стрелки, удобное для измерений.
%
%Если ручка корректора дошла до упора, а стрелка сместилась недостаточно,
%поверните рукоятку корректора в обратную сторону до упора, а затем снова
%поворачивайте её, пока стрелка не встанет на нужное деление.}
%
%\item{Поставьте переключатель в положение <<Измерение>>. Заметьте начальное
%положение стрелки (вся шкала --- 10 дел. --- 10~mWb).
%
%Измените магнитный поток сквозь катушку до нуля и заметьте новое положение
%стрелки. Разность показаний определяет магнитный поток.
%
%Изменять магнитный поток рекомендуется одним из способов:
%
%а) быстро удаляя пробную катушку из области действия магнитного поля на
%расстояние, где магнитный поток практически равен нулю (рекомендуется);
%
%б) выключая магнитное поле, если катушка закреплена жёстко.
%
%Не рекомендуется переполюсовывать магнит для измерений поля, т.~к. при этом
%часто ломаются переключатели.
%
%Величина $SN$, необходимая для расчёта индукции поля, указана на пробной
%катушке.}
%
%\item{По окончании работы следует заарретировать прибор~--- поставить
%переключатель в положение <<Арретир>>.}
%
%\item{Не реже одного раза в месяц рекомендуется проверять состояние приборов по
%образцовому прибору.}

%Один раз в два года, а также после каждого ремонта, приборы должны проверяться
%в местном отделении Комитета стандартов, мер и измерительных приборов.

%\end{enumerate}
