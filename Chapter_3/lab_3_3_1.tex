

{\large \bf 3.3.1 Измерение удельного заряда электрона методами магнитной~фокусировки и~магнетрона}

{\bf Цель работы:}определение отношения заряда электрона к его массе методом магнитной фокусировки и методом магнетрона.



{\LARGE  Метод магнитной фокусировки}

{\bf В работе используются:}{электронно-лучевая трубка и блок питания к ней; соленоид; источник постоянного тока; электростатический вольтметр; милливеберметр; ключи.}

Теоретическая часть работы изложена во введении к разделу в пункте [TO ADD].

Идея опыта заключается в следующем. Электронно-лучевая трубка, вынутая из осциллографа С1-1, помещается в длинный соленоид, создающий магнитное поле, направленное вдоль оси трубки. Электроны вылетают из электронной пушки трубки практически с одинаковыми продольными скоростями $v_{\parallel}$. Небольшое напряжение, подаваемое на отклоняющие пластины, изменяет только поперечную составляющую скорости. Это означает, что все электроны в магнитном поле будут двигаться по спиралям с одним и тем же шагом $L$ и, следовательно, электроны будут встречаться вновь, пересекая ось пучка на расстояниях $L$, $2L$ и т.~д. В~этих точках сечение пучка будет наименьшим, т.~е. в них электронный пучок будет фокусироваться. Следовательно, при изменении магнитного поля изображение пучка на экране будет периодически стягиваться в ярко светящее ся пятнышко. Если расстояние от пушки до экрана $l$, то пучок сфокусируется на экране при условии

$l=nL$, где $n=$1,\, 2,\, 3,\, \ldots,

или
$$
l=\frac{2\pi v_{\parallel}}{(e/m)B_F}n.
$$
Выразив в этой формуле скорость электронов через ускоряющее напряжение, получаем выражение для удельного заряда через измеряемые физические величины:

\begin{equation}
\frac{e}{m}=\frac{8\pi^2V}{l^2}\left(\frac{n^2}{B_F^2}\right).
\label{eq3.1.11}
\end{equation}

{\bf Экспериментальная установка.} Основной частью установки является электронный осциллограф С1-1, трубка которого вынута и установлена в длинном соленоиде, создающим магнитное поле. Напряжение на отклоняющие пластины и питание подводятся к~трубке многожильным кабелем.

\begin{figure}
%\rpic{40mm}
\caption{ Схема измерений по~методу магнитной~фокусировки}
\label{fig3.1.1}
\end{figure}

Пучок электронов, вылетающих из катода с разными скоростями (энергия электрона $\approx 0,1$~эВ), ускоряется анодным напряжением~$\approx 1$~кВ. После прохождения двух диафрагм из пучка выделяются электроны с~практически одинаковой
продольной скоростью $v_{\parallel}$. Небольшое переменное напряжение, поступающее с клеммы <<Контрольный сигнал>>
осциллографа на отклоняющие пластины, изменяет  только поперечную составляющую скорости. Угол~$\alpha$ отклонения пучка от оси трубки, таким образом, зависит  от времени, и электроны прочерчивают на экране трубки светящуюся линию. При увеличении магнитного поля линия на экране сокращается, постепенно стягиваясь в точку, а затем снова удлиняется. Второе прохождение через фокус происходит в том случае, когда электроны на пути от катода к экрану описывают два витка спирали, третье~--- при трёх витках.

Анодное напряжение, определяющее продольную скорость электронов, измеряется электростатическим киловольтметром.

Магнитное поле в соленоиде создаётся постоянным током (\ref{fig3.1.1}), сила которого задается источником питания постоянного тока и измеряется амперметром $A$ источника. Ключ К служит для изменения направления поля в соленоиде.

Величина магнитного поля определяется с помощью измерительной катушки, подключённой к милливеберметру. Этот прибор
измеряет изменение магнитного потока, пронизывающего измерительную катушку, которая намотана  на один каркас с
соленоидом. Описание милливеберметра и правила работы с ним приведены на с.~\pageref{MWB}.

На точность результатов может влиять внешнее магнитное поле, особенно продольное. Оно не вызывает размытия фокуса, но изменяет величину фокусирующего поля. Присутствие внешнего магнитного поля проще всего обнаружить с помощью
переполюсовки соленоида: при изменении направления поля показания милливеберметра будут отличаться, но их полусумма не зависит от наличия постоянного продольного поля.

Измерение магнитного поля с помощью милливеберметра обычно производится в предварительных опытах: при отключении ключа~К устанавливается связь между силой тока, протекавшего через соленоид, и индукцией магнитного поля в соленоиде. По измеренным значениям строится калибровочный график, который используется при обработке результатов основных измерений для пересчёта от тока к индукции магнитного поля.

{\Large \bf ЗАДАНИЕ}

В работе предлагается определить значения магнитных полей, при которых происходит фокусировка электронного пучка, и по результатам измерений рассчитать $e/m$.

\begin{enumerate}
\item{ Ознакомьтесь с назначением ручек управления источника питания по описанию на приборе.}
\item{Познакомьтесь с устройством миливебберметра. См раздел. [ссылка]}
\item{ Прокалибруйте электромагнит~--- определите связь между индукцией~$B$ магнитного поля и током $I$ через обмотки магнита. Для этого с помощью милливеберметра снимите зависимость магнитного потока $\Phi=BSN$, пронизывающего пробную катушку, находящуюся в~магнитном поле, от тока~$I$, измеряемого источником. Значение $SN$ (произведение площади сечения пробной катушки на число витков в ней) указано на установке.

Проведите измерения магнитного потока $\Phi$ во всем диапазоне изменения тока при двух направлениях тока через обмотку.}

\item{ При минимальном или нулевом токе через соленоид включите осциллограф  и подайте напряжение с клеммы <<Контрольный сигнал>> на вертикальный (или горизонтальный) вход усилителя. На экране появится светящаяся линия.}

\item{ Пстепенно увеличивая ток через соленоид, найдите значение тока $I_F$, при котором линия первый раз стягивается в точку (сила тока $I_F$ зависит, конечно, от ускоряющего напряжения $V$, а величина $V$ меняется с изменением яркости луча, поэтому не следует изменять яркость до конца измерений).

     Продолжая увеличивать ток, снимите зависимость $I_F$ от порядкового номера фокуса $n$.}
\item{ Повторите измерения $I_F=f(n)$ для другого направления магнитного поля.}
\item{ Запишите ускоряющее напряжение $V$, величины $L$ и $SN$, указанные на установке, и характеристики приборов. Погрешность миливебберметра зависит от сопротивления измерительной катушки. В нашей установке оно составляет примерно 5 Ом.}
\item{Установите регуляторы источника питания на минимум сигнала и отключите источник. Отключите осциллограф.}
\end{enumerate}

{\rm Обработка результатов}

\begin{enumerate}
\item{Постройте график $B_F=f(I)$.}
\item{По графику $B_F=f(I)$ определите усреднённые значения $B_F$ для каждого фокуса и постройте график зависимости $B_F=f(n)$. Используйте наклон графика для расчёта $e/m$ с помощью формулы~([ссылка]).}
\item{Оцените погрешности и сравните результат с табличным.}

\end{enumerate}


{\Large Измерение ${e/m}$ методом магнетрона}

{\bf В работе используются:}{электронная лампа с цилиндрическим анодом; источники питания лампы и соленоида; соленоид; миллиамперметр; амперметр.}

В настоящей работе отношение $e/m$ для электрона определяется с помощью метода, получившего название <<метод
магнетрона>>. Это название связано с тем, что применяемая в работе конфигурация электрического и магнитного полей
напоминает конфигурацию полей в магнетронах~--- генераторах электромагнитных колебаний сверхвысоких частот.

\begin{figure}
%\noindent\hfil\piccapt{38mm}{3_1_2}
\caption{Схема установки для измерения e/m методом магнетрона}
%\hfil
\label{fig3.1.2}
\end{figure}

\begin{figure}
%\piccapt{45mm}{3_1_3}
\caption{Траектории электронов, вылетающих из~катода, при~разных значениях индукции магнитного~поля}
\label{fig3.1.3}
\end{figure}

Движение электронов в этом случае происходит в кольцевом пространстве, заключённом между катодом и анодом
двухэлектродной электронной лампы (\ref{eq3.1.2}). Нить накала лампы (катод) располагается вдоль оси цилиндрического анода, так что электрическое поле между катодом и анодом имеет радиальное направление. Лампа помещается внутри соленоида, создающего магнитное поле, параллельное оси лампы. Движение электронов в такой лампе рассмотрено в приложении к работе.

Рассмотрим траектории электронов, вылетевших из катода, более подробно. Пусть потенциал анода равен $V_A$. В отсутствие магнитного поля (\ref{eq3.1.1}) электрон движется прямолинейно по радиусу. При слабом поле траектории несколько искривляются, но электроны всё же попадают на анод. При некотором критическом значении индукции магнитного поля $B_{KP}$ траектории искривляются настолько, что только касаются анода. Наконец, при $B>B_{KP}$ электроны вовсе не попадают на анод и возвращаются к катоду. Величину~$B_{KP}$ нетрудно найти по выведенной в приложении формуле (\ref{eq3.1p.18}), заметив, что в этом случае радиальная скорость электрона $\dot{r}$ при $r=r_A$ (при радиусе анода) обращается в нуль:

\begin{equation}
V_A=\frac{eB_{KP}^2r_A^2}{8m}.
\label{eq3.1.19}
\end{equation}

Преобразуя (\ref{eq3.1.19}), найдём

\begin{equation}
\frac{e}{m}=\frac{8V_A}{B_{KP}^2r_A^2}.
\label{eq3.1.20}
\end{equation}

Формула (\ref{eq3.1.20}) позволяет вычислять $e/m$, если при заданном $V_A$ найдено такое значение магнитного поля (или, наоборот, при заданном $B$ такое значение $V_A$), при котором электроны перестают попадать на анод.

\begin{figure}
%\rpic{42mm}{3_1_4}
\caption{Зависимость анодною тока от индукции магнитного поля в соленоиде}
\label{fig3.1.4}
\end{figure}

До сих пор мы рассматривали идеальный случай, когда при $B<B_{кр}$ все электроны без исключения попадают на анод, а при $B>B_{KP}$ все они возвращаются на катод, не достигнув анода. Анодный ток $I_A$ с увеличением магнитного поля изменялся бы при этом так, как это изображено на \ref{fig3.1.4} штриховой линией. В реальных условиях невозможно обеспечить полную коаксиальность анода и катода, вектор индукции магнитного поля всегда несколько наклонён по отношению к катоду, магнитное поле не вполне однородно и т.~д. Все эти причины приводят к сглаживанию кривой на рис. \ref{fig3.1.4} и она приобретает вид сплошной линии. В~хорошо собранной установке перелом функции $I_A=f(B)$ остаётся, однако, достаточно резким и с успехом может быть использован для измерения $e/m$.

{\bf Экспериментальная установка.} Схема установки изображена на рис. \ref{fig3.1.5}. Двухэлектродная лампа Л с цилиндрическим анодом специально изготовлена из немагнитных материалов. Анод лампы состоит из трёх металлических (нержавеющая сталь) цилиндров одинакового диаметра. Два крайних цилиндра электрически изолированы от среднего небольшими зазорами и используются для устранения краевых эффектов на торцах среднего цилиндра, ток с которого используется при измерениях. В качестве катода используется тонкая (диаметром 50~мкм) хорошо натянутая вольфрамовая проволока, расположенная по оси всех трёх цилиндров анодной системы.
Катод лампы разогревается переменным током, отбираемым от стабилизированного источника питания. С этого источника на анод лампы подаётся постоянное напряжение (0--120~В), регулируемое с помощью потенциометра и измеряемое вольтметром $V$.

\begin{figure}
%\fcpic[0.8]{3_1_5}
\caption{Схема измерительной установки}
\label{fig3.1.5}
\end{figure}

Лампа закреплена в соленоиде. Ток, проходящий через соленоид, подаётся с выпрямителя и измеряется амперметром $A$.
Индукция магнитного поля в соленоиде рассчитывается по току, протекающему через обмотку соленоида. Коэффициент
пропорциональности между ними указан на установке.

{\Large \bf ЗАДАНИЕ}

В работе предлагается исследовать зависимость анодного тока от тока, протекающего через соленоид, при различных
напряжениях на аноде лампы и по результатам измерений рассчитать удельный заряд электрона.

\begin{enumerate}
\item{ Установите на аноде лампы потенциал $V_A=60$~В. Снимите зависимость анодного тока $I_A$ от индукции магнитного поля в соленоиде (от тока $I_{M}$ через соленоид). В области резкого изменения тока точки должны лежать чаще (рис. \ref{fig3.1.4})}
\item{ Снимите аналогичные зависимости $I_A=f(I_M)$ для 5--6 фиксированных значений $V_A$ в диапазоне 60--120~В.}

\item{ Запишите параметры установки и характеристики приборов.}
\end{enumerate}

{\rm Обработка результатов}
\begin{enumerate}
\item{Используйте полученные результаты для построения семейства кривых $I_{A}(B)$. Для каждого значения $V_A$ определите по графику критическое значение индукции магнитного поля $B_{KP}$}.
\item Постройте  график зависимости $B_{KR}^2$ от $V_A$. По угловому коэффициенту полученной прямой определите удельный заряд электрона $e/m$. Сравните результат с табличным.
\end{enumerate}

{ \small
{\bf \Large Контрольные вопросы}

\begin{enumerate}
\item{ Нарисуйте и объясните схемы измерения удельного заряда электрона методом магнитной фокусировки и методом магнетрона.}
\item{Объясните принцип действия электронно-лучевой трубки осциллографа.}
\item{ Объясните принцип работы милливеберметра.}
\item{ Почему в методе магнетрона используется анод из трёх цилиндров, а не из одного?}
\end{enumerate}

{\bf \Large Список литераратуры}
\begin{enumerate}
\item{ {\em Сивухин Д.В.} Общий курс физики. Т. III. Электричество.~--- М.: Наука, 1983, \S\S~86, 89.}
\item{ {\em Калашников С.Г.} Электричество.~--- М.: Наука, 1977, \S\S~181--184.}
\end{enumerate}
}



{\LARGE Движение электрона в магнетроне}

Рассмотрим траекторию электронов, движущихся в лампе под действием электрического и магнитного полей. Для вычисленийвоспользуемся цилиндрической системой координат, т.~е. будем характеризовать положение точки расстоянием от оси цилиндра $r$, полярным углом $\phi$ и смещением вдоль оси $z$ (рис.~\ref{fig3.1p.1}). Рассмотрим сначала силы, действующие на электрон со стороны электрического поля. Напряжённость электрического поля в цилиндрическом конденсаторе имеет только радиальную компоненту $E_r=-E$. Поэтому сила, действующая на электрон в таком поле, направлена по радиусу, так что
\begin{equation}
F_r^{el}=eE,\qquad F_z^{el}=F_{\phi}^{el}=0.
\label{eq3.1p.9}
\end{equation}

Рассмотрим теперь силы, действующие на электрон со стороны магнитного поля. Поскольку магнитное поле в нашем случае
направлено по оси $z$, для проекции силы на ось $z$ имеем
\begin{equation}
F_z^{mag}=0.
\label{eq3.1p.10}
\end{equation}

Остальные две составляющие силы найдём с помощью формулы Лоренца. Как нетрудно убедиться,
\begin{equation}
F_{\phi}^{mag}=ev_rB,\qquad F_{r}^{mag}=-ev_{\phi}B.
\label{fig3.1p.11}
\end{equation}

Из простых кинематических соображений ясно, что
\begin{equation}
v_r=\dot{r}=\frac{dr}{dt},\qquad v_{\phi}=r\dot{\phi}=r\frac{d\phi}{dt}.
\label{eq3.1p.12}
\end{equation}

Как видно из формул (\ref{eq3.1p.9}) и (\ref{eq3.1p.10}), ни магнитные, ни электрические силы, действующие на электрон, не имеют составляющих по оси $z$. Движение вдоль оси $z$ является равномерным.

Движение в плоскости ($r$, $\phi$) удобно описывать с помощью уравнения моментов. Для проекции на ось $z$ имеем
\begin{equation}
\frac{dL_{z}}{dt}=M_z,
\label{eq3.1p.13}
\end{equation}

где $L_{z}$~--- момент импульса электрона относительно оси $z$, равный, как известно, $mr^2\dot{\phi}$. Величина $M_z$ равна $rF_{\phi}$. С помощью (\ref{eq3.1p.11}) и (\ref{eq3.1p.13}) найдём
\begin{equation}
M_z=erv_rB.
\label{eq3.1p.14}
\end{equation}

Подставляя (\ref{eq3.1p.12}) и (\ref{eq3.1p.14}) в (\ref{eq3.1p.13}), найдём
\begin{equation}
\frac{d}{dt}\left(mr^2\dot{\phi}\right)=eBr\frac{dr}{dt}=\frac12eB\frac{d(r^2)}{dt}.
\label{eq3.1p.15}
\end{equation}
Интегрируя уравнение (\ref{eq3.1p.15}), получаем

\begin{equation}
r^2\dot{\phi}+A=\frac{eBr^2}{2m},
\label{eq3.1p.16}
\end{equation}

где $A$~--- постоянная интегрирования, которую следует определить из начальных условий. В начале движения радиус $r$ равен радиусу катода, т.~е. очень мал. Правая часть (\ref{eq3.1p.16}) поэтому тоже очень мала. Электроны вылетают из катода с небольшой скоростью, так что $r^{2}\dot{\phi}$ в начальный момент также мало. С хорошей точностью можно поэтому полагать $A=0$. Наше уравнение приобретает при этом простой вид:
\begin{equation}
\dot{\phi}=\frac{eB}{2m}.
\label{eq3.1p.17}
\end{equation}

Рассмотрим теперь движение электрона по радиусу. Работа сил электрического поля, совершаемая при перемещении электрона от катода до точки с потенциалом $V$, равна $W=eV$. Магнитное поле никакой работы не производит. Найденная работа должнабыть поэтому равна кинетической энергии электрона (начальной скоростью электрона мы снова пренебрегаем):
$$
eV=\frac{mv^2}{2}=\frac{(v_r^2+v_\phi^2)}{2m}.
$$
С помощью (\ref{eq3.1p.12}) и (\ref{eq3.1p.17}) найдём
\begin{equation}
eV=\frac{m}{2}\left[\dot{r}^2+\left(\frac{reB}{2m}\right)^2\right].
\label{eq3.1p.18}
\end{equation}
Уравнение (\ref{eq3.1p.18}) полностью определяет радиальное движение электрона.



