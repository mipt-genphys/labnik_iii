\lab{Измерение удельного заряда электрона методами магнитной фокусировки и
магнетрона}

\aim{определение отношения заряда электрона к его массе методом магнитной
фокусировки и методом магнетрона.}

\labsection{А. Метод магнитной фокусировки}

\equip{электронно-лучевая трубка и блок питания к ней; соленоид; источник
постоянного тока; вольтметр;  магнитометр; ключи.}

Теоретическая часть работы изложена во введении к разделу в пункте \ref{2.1}.
%на  странице \pageref{magnetic focusing}.

Идея опыта заключается в следующем. Электронно-лучевая трубка, вынутая из
осциллографа, помещается в длинный соленоид, создающий магнитное поле,
направленное вдоль оси трубки. Электроны вылетают из электронной пушки трубки
практически с одинаковыми продольными скоростями~$v_{\parallel}$. Небольшое
напряжение, подаваемое на отклоняющие пластины, изменяет только поперечную
составляющую скорости. Это означает, что все электроны в магнитном поле будут
двигаться по спиралям с одним и тем же шагом~$L$, и, следовательно, электроны
будут встречаться вновь, пересекая ось пучка на расстояниях~$L$,~$2L$ и т.~д.
В~этих точках сечение пучка будет наименьшим, т.~е. в них электронный пучок будет
фокусироваться. Следовательно, при изменении магнитного поля изображение пучка
на экране будет периодически стягиваться в ярко светящееся пятнышко. Если
расстояние от пушки до экрана~$l$, то пучок сфокусируется на экране при условии
$l=nL$, где $n=1,\, 2,\, 3,\, \ldots,$ или
\begin{equation*}
	l=\frac{2\pi v_{\parallel}}{\frac{e}{m} B_F}n.
\end{equation*}

Выразив в этой формуле скорость электронов через ускоряющее напряжение, получаем
выражение для удельного заряда через измеряемые физические величины:
\begin{equation}
	\frac{e}{m}=\frac{8\pi^2V}{l^2}\left(\frac{n^2}{B_F^2}\right).
	\eqmark{3.1.1}
\end{equation}

\experiment Основной частью установки является электронный осциллограф, трубка
которого вынута и установлена в длинном соленоиде, создающим магнитное поле.
Напряжение на отклоняющие пластины и питание подводятся к трубке многожильным
кабелем.Пучок электронов, вылетающих из катода с разными скоростями (энергия
электрона~$\approx 0,1$~эВ), ускоряется анодным напряжением~$\approx 1$~кВ.
После прохождения двух диафрагм из пучка выделяются электроны с практически
одинаковой продольной скоростью~$v_{\parallel}$.
\begin{figure}[h!]
	\pic{0.9\textwidth}{Chapter_3/3_1_1}
	\caption{ Схема измерений по~методу магнитной~фокусировки}
	\figmark{Magnetic focusing scheme}
\end{figure}
Небольшое переменное напряжение, подаваемое на отклоняющие пластины, изменяет
только поперечную составляющую скорости. Угол~$\alpha$ отклонения пучка от оси
трубки, таким образом, зависит от времени, и электроны прочерчивают на экране
трубки светящуюся линию. При увеличении магнитного поля линия на экране
сокращается, постепенно стягиваясь в точку, а затем снова удлиняется. Второе
прохождение через фокус происходит в том случае, когда электроны на пути от
катода к экрану описывают два витка спирали, третье~--- при трёх витках.

Анодное напряжение, определяющее продольную скорость электронов, измеряется
вольтметром.

Магнитное поле в соленоиде создаётся постоянным током
(рис.~\figref{Magnetic focusing scheme}), сила которого задается источником
питания постоянного тока и измеряется амперметром~$A$ источника. Ключ~$K$
служит для изменения направления поля в соленоиде.

Величина магнитного поля определяется с помощью магнитометра, датчик которого
расположен внутри соленоида. В~качестве магнитометра  может использоваться
милливеберметр, у которого датчиком является измерительная катушка,намотанная на
один каркас с соленоизом. Последний измеряет изменение магнитного потока,
пронизывающего измерительную катушку. Описание милливеберметра и правила работы
с ним приведены на с.~\pageref{MWB}\todo[inline,author = nozik]{Заменить ссылкой на
рисунок}.

На точность результатов может влиять внешнее магнитное поле, особенно
продольное. Оно не вызывает размытия фокуса, но изменяет величину фокусирующего
поля. Присутствие внешнего магнитного поля проще всего обнаружить с помощью
переполюсовки соленоида: при изменении направления поля показания
милливеберметра будут отличаться, но их полусумма не зависит от наличия
постоянного продольного поля.

Измерение магнитного поля обычно производится в предварительных опытах: при
отключении ключа~$K$ устанавливается связь между силой тока, протекавшего через
соленоид, и индукцией магнитного поля в соленоиде. По измеренным значениям
строится калибровочный график, который используется при обработке результатов
основных измерений для пересчёта от тока к индукции магнитного поля.

\begin{lab:task}

В~работе предлагается определить значения магнитных полей, при которых
происходит фокусировка электронного пучка, и по результатам измерений рассчитать
$e/m$.

\begin{enumerate}
\item{Ознакомьтесь с назначением ручек управления источника питания по описанию
на приборе.}
\item{Ознакомьтесь с устройством используемого в работе магнитометра}
\item{Прокалибруйте электромагнит. Для этого снимите зависимость магнитного поля
$B$ от тока~$I$ через соленоид.  В~случае использования установки с
милливеберметром,~$B$ вычисляется через поток $\Phi=BSN$, пронизывающего пробную
катушку (значение параметра~$SN$ катушки указано на установке).

Проведите указанные измерения во всем диапазоне изменения тока при двух
направлениях тока через обмотку.}
\item{При минимальном или нулевом токе через соленоид включите осциллограф  и
подайте напряжение с внешнего или внутреннего генератора на вертикальный (или
горизонтальный) вход усилителя. На экране появится светящаяся линия.}
\item{Постепенно увеличивая ток через соленоид, найдите значение тока~$I_F$, при
котором линия в первый раз стягивается в точку (сила тока~$I_F$ зависит,
конечно, от ускоряющего напряжения~$V$, а величина~$V$ меняется с изменением
яркости луча, поэтому не следует изменять яркость до конца измерений).

Продолжая увеличивать ток, снимите зависимость~$I_F$ от порядкового номера
фокуса~$n$.}
\item{ Повторите измерения $I_F=f(n)$ для другого направления магнитного поля.}
\item{ Запишите измеренное значение ускоряющего напряжения~$V$ и длину трубки
$L$, указанную на установке, и характеристики приборов.}
\item{Установите регуляторы источника питания на минимум сигнала и отключите
источник. Отключите осциллограф.}
\end{enumerate}

\tasksection{Обработка результатов}

\begin{enumerate}
\item{Постройте график $B=f(I)$.}
\item{По графику $B=f(I)$ определите усреднённые значения~$B_F$ для каждого
фокуса и постройте график зависимости $B_F=f(n)$. Используйте наклон графика для
расчёта~$e/m$ с помощью формулы~\eqref{3.1.1}.}
\item{Оцените погрешности и сравните результат с табличным.}

\end{enumerate}
\end{lab:task}

\labsection{Б. Измерение ${e/m}$ методом магнетрона}

\equip{электронная лампа с цилиндрическим анодом; источники питания лампы и
соленоида; соленоид; миллиамперметр; амперметр.}

В~настоящей работе отношение~$e/m$ для электрона определяется с помощью метода,
получившего название <<метод
магнетрона>>. Это название связано с тем, что применяемая в работе конфигурация
электрического и магнитного полей
напоминает конфигурацию полей в магнетронах~--- генераторах электромагнитных
колебаний сверхвысоких частот.

\begin{figure}[h!]
	\begin{minipage}[b]{0.49\textwidth}
		\pic{0.9\textwidth}{Chapter_3/3_1_2}
		\caption{Схема устройства двухэлектродной лампы}
		\figmark{Two-electrode lamp}
	\end{minipage}
	\hfill
	\begin{minipage}[b]{0.49\textwidth}
		\pic{0.9\textwidth}{Chapter_3/3_1_3}
		\caption{Траектории электронов, вылетающих из~катода, при~разных
значениях индукции магнитного~поля}
		\figmark{Path of electrons}
	\end{minipage}
\end{figure}

Движение электронов в этом случае происходит в кольцевом пространстве,
заключённом между катодом и анодом
двухэлектродной электронной лампы (рис.~\figref{Two-electrode lamp}). Нить
накала лампы (катод) располагается вдоль оси цилиндрического анода, так что
электрическое поле между катодом и анодом имеет радиальное направление. Лампа
помещается внутри соленоида, создающего магнитное поле, параллельное оси лампы.
Движение электронов в такой лампе рассмотрено в приложении к работе.

Рассмотрим траектории электронов, вылетевших из катода, более подробно. Пусть
потенциал анода равен~$V_A$. В~отсутствие магнитного поля (рис.~\figref{Path of
electrons}) электрон движется прямолинейно по радиусу. При слабом поле
траектории несколько искривляются, но электроны всё же попадают на анод. При
некотором критическом значении индукции магнитного поля~$B_\text{кр}$ траектории
искривляются настолько, что только касаются анода. Наконец, при~$B>B_\text{кр}$
электроны вовсе не попадают на анод и возвращаются к катоду.
Величину~$B_\text{кр}$ нетрудно найти по выведенной в приложении
формуле~\eqref{3.1.13}, заметив, что в этом случае радиальная скорость электрона
$\dot{r}$ при~$r=r_A$ (при радиусе анода) обращается в нуль:

\begin{equation}
	V_A=\frac{eB_\text{кр}^2r_A^2}{8m}.
	\eqmark{3.1.2}
\end{equation}

Преобразуя \eqref{3.1.2}, найдём
\begin{equation}
	\frac{e}{m}=\frac{8V_A}{B_\text{кр}^2r_A^2}.
	\eqmark{3.1.3}
\end{equation}

Формула~\eqref{3.1.3} позволяет вычислять~$e/m$, если при заданном~$V_A$ найдено
такое значение магнитного поля (или, наоборот, при заданном~$B$ такое значение
$V_A$), при котором электроны перестают попадать на анод.

\begin{figure}[h!]
	\pic{0.9\textwidth}{Chapter_3/3_1_4}
	\caption{Зависимость анодного тока от индукции магнитного поля в соленоиде}
	\figmark{Anode current from induction}
\end{figure}

До сих пор мы рассматривали идеальный случай, когда при $B<B_\text{кр}$ все
электроны без исключения попадают на анод, а при $B>B_\text{кр}$ все они
возвращаются на катод, не достигнув анода. Анодный ток~$I_A$ с увеличением
магнитного поля изменялся бы при этом так, как это изображено на
рис.~\figref{Anode current from induction} штриховой линией. В~реальных условиях
невозможно обеспечить полную коаксиальность анода и катода, вектор индукции
магнитного поля всегда несколько наклонён по отношению к катоду, магнитное поле
не вполне однородно и т.~д. Все эти причины приводят к сглаживанию кривой на
рис.~\figref{Anode current from induction} и она приобретает вид сплошной линии.
В~хорошо собранной установке перелом функции~$I_A=f(B)$ остаётся, однако,
достаточно резким и с успехом может быть использован для измерения~$e/m$.

\experiment Схема установки изображена на рис.~\figref{Scheme}. Двухэлектродная
лампа~$\text{Л}$ с цилиндрическим анодом специально изготовлена из немагнитных
материалов. Анод лампы состоит из трёх металлических (нержавеющая сталь)
цилиндров одинакового диаметра.
\begin{figure}[h!]
	\pic{0.9\textwidth}{Chapter_3/3_1_5}
	\caption{Схема измерительной установки}
	\figmark{Scheme}
\end{figure}
Два крайних цилиндра электрически изолированы от среднего небольшими зазорами и
используются для устранения краевых эффектов на торцах среднего цилиндра, ток с
которого используется при измерениях. В~качестве катода используется тонкая
(диаметром 50~мкм) хорошо натянутая вольфрамовая проволока, расположенная по оси
всех трёх цилиндров анодной системы. Катод лампы разогревается переменным током,
отбираемым от стабилизированного источника питания. С~другого,  регулируемого,
источника на анод лампы подаётся постоянное напряжение, измеряемое вольтметром
$V$. Ток через среднюю секцию анода измеряется с помощью миллиамперметра~$mA$.

Лампа закреплена в соленоиде. Ток, проходящий через соленоид, подаётся с
третьего источника и измеряется амперметром~$A$. Индукция магнитного поля в
соленоиде рассчитывается по току, протекающему через обмотку соленоида.
Коэффициент пропорциональности между ними указан на установке.

\begin{lab:task}

В~работе предлагается исследовать зависимость анодного тока от тока,
протекающего через соленоид, при различных
напряжениях на аноде лампы и по результатам измерений рассчитать удельный заряд
электрона.

\begin{enumerate}
\item{ Установите на аноде лампы минимальный потенциал~$V_A$, рекомендуемый в
описании конкретной установки. Снимите зависимость анодного тока~$I_A$ от
индукции магнитного поля в соленоиде (от тока~$I_{M}$ через соленоид). В~области
резкого изменения тока точки должны лежать чаще (рис.~\figref{Anode current from
induction})}.
\item{ Снимите аналогичные зависимости $I_A=f(I_M)$ для 5 -- 6 фиксированных
значений~$V_A$ в диапазоне, указанном в описании установки.}

\item{ Запишите параметры установки и характеристики приборов.}
\end{enumerate}

\tasksection{Обработка результатов}
\begin{enumerate}
\item{Используйте полученные результаты для построения семейства кривых
$I_{A}(B)$. Для каждого значения~$V_A$ определите по графику критическое
значение индукции магнитного поля~$B_\text{кр}$}.
\item Постройте  график зависимости~$B_\text{кр}^2$ от~$V_A$. По угловому
коэффициенту полученной прямой определите удельный заряд электрона~$e/m$.
Сравните результат с табличным.
\end{enumerate}
\end{lab:task}

\begin{lab:questions}
\item{ Нарисуйте и объясните схемы измерения удельного заряда электрона методом
магнитной фокусировки и методом магнетрона.}
\item{Объясните принцип действия электронно-лучевой трубки осциллографа.}
\item{ Объясните принцип работы милливеберметра.}
\item{ Почему в методе магнетрона используется анод из трёх цилиндров, а не из
одного?}
\end{lab:questions}

\begin{lab:literature}
\item{ \emph {Сивухин~Д.В.} Общий курс физики. Т. III. Электричество.~--- М.:
Физматлит, 2015, \S\S~86, 89.}
\item{ \emph {Калашников~С.Г.} Электричество.~--- М.: Физматлит, 2003,
\S\S~181--184.}
\end{lab:literature}




\labsection{Движение электрона в магнетроне}

Рассмотрим траекторию электронов, движущихся в лампе под действием
электрического и магнитного полей. Для вычислений воспользуемся цилиндрической
системой координат, т.~е. будем характеризовать положение точки расстоянием от
оси цилиндра~$r$, полярным углом~$\varphi$ и смещением вдоль оси~$z$
(рис.~\figref{Two-electrode lamp}). Рассмотрим сначала силы, действующие на
электрон со стороны электрического поля. Напряжённость электрического поля в
цилиндрическом конденсаторе имеет только радиальную компоненту~$E_r=-E$. Поэтому
сила, действующая на электрон в таком поле, направлена по радиусу, так что
\begin{equation}
	F_r^{el}=eE,\qquad F_z^{el}=F_{\varphi}^{el}=0.
	\eqmark{3.1.4}
\end{equation}

Рассмотрим теперь силы, действующие на электрон со стороны магнитного поля.
Поскольку магнитное поле в нашем случае
направлено по оси~$z$, для проекции силы на ось~$z$ имеем
\begin{equation}
	F_z^{mag}=0.
	\eqmark{3.1.5}
\end{equation}

Остальные две составляющие силы найдём с помощью формулы Лоренца. Как нетрудно
убедиться,
\begin{equation}
	F_{\varphi}^{mag}=ev_rB,\qquad F_{r}^{mag}=-ev_{\varphi}B.
	\eqmark{3.1.6}
\end{equation}

Из простых кинематических соображений ясно, что
\begin{equation}
	v_r=\dot{r}=\frac{dr}{dt},\qquad
v_{\varphi}=r\dot{\varphi}=r\frac{d\varphi}{dt}.
	\eqmark{3.1.7}
\end{equation}

Как видно из формул \eqref{3.1.4} и \eqref{3.1.5} ни магнитные, ни электрические
силы, действующие на электрон, не имеют составляющих по оси~$z$. Движение вдоль
оси~$z$ является равномерным.

Движение в плоскости ($r$,~$\varphi$) удобно описывать с помощью уравнения
моментов. Для проекции на ось~$z$ имеем
\begin{equation}
	\frac{dL_{z}}{dt}=M_z,
	\eqmark{3.1.8}
\end{equation}
где~$L_{z}$~--- момент импульса электрона относительно оси~$z$, равный, как
известно, $mr^2\dot{\varphi}$. Величина~$M_z$ равна $rF_{\varphi}$. С~помощью
\eqref{3.1.4} и \eqref{3.1.6} найдём
\begin{equation}
	M_z=erv_rB.
	\eqmark{3.1.9}
\end{equation}

Подставляя \eqref{3.1.7} и \eqref{3.1.9} в \eqref{3.1.8}, найдём
\begin{equation}
\frac{d}{dt}\left(mr^2\dot{\varphi}\right)=eBr\frac{dr}{dt}=
\frac12eB\frac{dr^2}{dt}.
	\eqmark{3.1.10}
\end{equation}

Интегрируя уравнение \eqref{3.1.10}, получаем
\begin{equation}
	r^2\dot{\varphi}+A=\frac{eBr^2}{2m},
	\eqmark{3.1.11}
\end{equation}
где~$A$~--- постоянная интегрирования, которую следует определить из начальных
условий. В~начале движения радиус~$r$ равен радиусу катода, т.~е. очень мал.
Правая часть \eqref{3.1.11} поэтому тоже очень мала. Электроны вылетают из
катода с небольшой скоростью, так что $r^{2}\dot{\varphi}$ в начальный момент
также мало. С~хорошей точностью можно поэтому полагать $A=0$. Наше уравнение
приобретает при этом простой вид:
\begin{equation}
	\dot{\varphi}=\frac{eB}{2m}.
	\eqmark{3.1.12}
\end{equation}

Рассмотрим теперь движение электрона по радиусу. Работа сил электрического поля,
совершаемая при перемещении электрона от катода до точки с потенциалом~$V$,
равна~$W=eV$. Магнитное поле никакой работы не производит. Поэтому найденная
работа должна быть равна кинетической энергии электрона (начальной скоростью
электрона мы снова пренебрегаем):
\begin{equation*}
	eV=\frac{mv^2}{2}=\frac{v_r^2+v_\varphi^2}{2m}.
\end{equation*}

С~помощью \eqref{3.1.7} и \eqref{3.1.12} найдём
\begin{equation}
	eV=\frac{m}{2}\left[\dot{r}^2+\left(\frac{reB}{2m}\right)^2\right].
	\eqmark{3.1.13}
\end{equation}

Уравнение \eqref{3.1.13} полностью определяет радиальное движение электрона.



