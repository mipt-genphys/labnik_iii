\lab{Измерение удельного заряда электрона методами магнитной фокусировки и
магнетрона}

\aim{определение отношения заряда электрона к его массе методом магнитной
фокусировки и методом магнетрона.}

\equip{А)~электронно-лучевая трубка (с блоком питания), соленоид, 
    регулируемый источник постоянного тока, вольтметр, 
    магнитометр (миллитесламетр или милливеберметр);
Б)~электронная лампа с цилиндрическим анодом, 
регулируемый источник постоянного тока, соленоид, вольтметр, два амперметра.}

Перед выполнение работы необходимо ознакомиться с теоретическим введением
к Разделу (п. \ref{sec:freemotion}).

\labsection{А.~Метод магнитной фокусировки}


В постоянном однородном магнитном поле траектории заряженных частиц представляют
собой спирали, радиус которой определяется формулой \chaptereqref{3.4}.
За время $T_B= \frac{2\pi r_B}{v_{\perp}}$, которое можно назвать
\term{циклотронным периодом}, заряд сместится вдоль магнитного поля на 
расстояние $L$ (шаг спирали):
\begin{equation}
    \eqmark{3.6}
    L = v_{\parallel}T_B =\frac{2\pi v\cos\alpha}{(e/m)B},
\end{equation}
где $\alpha$ --- угол между вектором скорости $\vec{v}$ и направлением поля $\vec{B}$.
Если углы малы, $\alpha \ll 1$, то $\cos\alpha \approx 1$ и
\begin{equation}
    \eqmark{3.7}
    L \approx \frac{2\pi v}{(e/m)B}.
\end{equation}
Таким образом, при малых углах расстояние~$L$ не зависит от~$\alpha$, так
что все электроны, вышедшие из одной точки, после одного оборота вновь соберутся
в одной точке~--- \emph{сфокусируются}. Как следует из \eqref{3.7}, 
индукция поля~$B$, при которой точка фокусировки отстоит от точки вылета 
на расстоянии~$L$, определяется величиной~$e/m$~--- удельным зарядом частицы.

В работе исследуется пучок электронов, создаваемый электронно-лучевой трубкой, 
помещённой в магнитное поле соленоида.
Скорость движения электронов определяется разность потенциалов~$U$,
пройденную им до попадания в магнитное поле:
\begin{equation*}
  \frac{mv^2}{2}=eU,
\end{equation*}
откуда
\begin{equation}
  \eqmark{3.3}
  v=\sqrt{\frac{2eU}{m}}.
%   = 6\cdot10^5\sqrt{V}~\frac{m}{c}.
\end{equation}

Пусть $B_{ф}$ --- индукция поля, при которой наступает фокусировка.
Из \eqref{3.3} и \eqref{3.7} выразим удельный заряд электрона~$e/m$
через $B_{ф}$:
\begin{equation}
\eqmark{3.8}
\frac{e}{m}=\frac{8\pi^2 U}{L^2B_{ф}^2}.
\end{equation}
Эта формула и лежит в основе экспериментального измерения удельного заряда
электрона по \important{методу магнитной фокусировки}.

\experiment 

Основной частью установки является электронный осциллограф, трубка
которого вынута и установлена в длинном соленоиде, создающим магнитное поле,
 направленное вдоль оси трубки. 
Вылетая с катода, электроны имеют разные скорости (тепловая энергия $\sim 0,1\;эВ$).
Однако далее они ускоряются большой анодной разностью потенциалов ($U_А~\sim 1\;кВ$) и отсеиваются
на двух диафрагмах, благодаря чему получается пучок частиц с малой расходимостью
по углу ($\Delta \alpha \ll 1$) и практически одинаковыми продольными скоростями
$v_{\parallel}=\sqrt{2eU_А/m}$.

В магнитном поле соленоида все электроны будут двигаться по спиралям 
с практически одним и тем же шагом~$L$
(см. \eqref{3.6}), и, следовательно, будут встречаться вновь, 
пересекая ось пучка на расстояниях~$nL$, $n=1,\,2,\,3,\ldots$
В~этих точках сечение пучка будет наименьшим, и при изменении магнитного 
поля изображение пучка на экране будет периодически стягиваться 
в ярко светящуюся точку. 
Таким образом, удельный заряд может быть получен из соотношения
\begin{equation}
\frac{e}{m}=\frac{8\pi^2U}{L^2}\cdot\frac{n^2}{B_ф^2(n)}.
\eqmark{3.1.1}
\end{equation}

\begin{figure}[h]
    \centering
	\pic{6cm}{Chapter_3/3_1_1}
	\caption{Схема измерений по~методу магнитной фокусировки}
	\figmark{Magnetic focusing scheme}
\end{figure}

Анодное напряжение, определяющее продольную скорость электронов, измеряется
вольтметром. Магнитное поле в соленоиде создаётся постоянным током
(рис.~\figref{Magnetic focusing scheme}), сила которого задается источником
питания постоянного тока и измеряется амперметром~A источника. Ключ~К
служит для изменения направления поля в соленоиде.

Величина магнитного поля определяется с помощью магнитометра, датчик которого
расположен внутри соленоида. В~качестве магнитометра может использоваться
\emph{милливеберметр} (\emph{флюксметр}), у которого датчиком является 
измерительная катушка, намотанная на один каркас с соленоидом. 
Таким образом измеряется изменение магнитного потока,
пронизывающего измерительную катушку. Описание милливеберметра и правила работы
с ним приведены на с.~\pageref{MWB}. Альтернативно индукция поля может измеряться
миллитесламетром (датчиком Холла).

На точность результатов может влиять внешнее магнитное поле, особенно
продольное. Оно не вызывает размытия фокуса, но изменяет величину фокусирующего
поля. Присутствие внешнего магнитного поля проще всего обнаружить с помощью
переполюсовки соленоида: при изменении направления поля показания
милливеберметра будут отличаться, но их полусумма не зависит от наличия
постоянного продольного поля.

Измерение магнитного поля производится в предварительных опытах: 
при отключенном ключе~К устанавливается связь между силой тока, протекавшего через
соленоид, и индукцией магнитного поля в соленоиде. 
По измеренным значениям строится \emph{калибровочный график} $B(I)$, 
который используется при обработке результатов
основных измерений для пересчёта индукции магнитного поля по известному току.

\begin{lab:task}

\taskpreamble{Измерьте значения магнитных полей, при которых происходит фокусировка 
электронного пучка, и по результатам измерений рассчитайте удельный
заряд электрона $e/m$.}
    
\item Ознакомьтесь с назначением ручек источника питания и с устройством 
    используемого в работе магнитометра. 
    
\item Измерьте калибровочную кривую $B(I)$ зависимости магнитного поля
от тока через соленоид. В~случае использования установки с
милливеберметром,~$B$ вычисляется через поток $\Phi=BSN$, пронизывающий пробную
катушку (значение параметра~$SN$ катушки указано на установке).

Проведите измерения во всём доступном диапазоне изменения тока при двух
направлениях тока через обмотку.

\item При минимальном или нулевом токе через соленоид включите осциллограф и
подайте напряжение с внешнего или внутреннего генератора на вертикальный (или
горизонтальный) вход усилителя. На экране появится светящаяся линия.

\item Постепенно увеличивая ток через соленоид, найдите значение тока~$I_ф$, при
котором линия в первый раз стягивается в точку (сила тока~$I_ф$ зависит,
конечно, от ускоряющего напряжения~$U_А$, которое в свою очередь пропорционально 
яркости луча, поэтому не следует менять настройку яркости до конца измерений).

Продолжая увеличивать ток, получите зависимость~$I_ф(n)$ от порядкового номера
фокуса~$n$.

\item Повторите измерения $I_ф(n)$ для обратного направления магнитного поля
в соленоиде.

\item Запишите значение ускоряющего напряжения~$U_А$, длину трубки
$L$ и характеристики приборов.

\item Установите регуляторы источника питания на минимум и выключите его. 
Выключите осциллограф.

\tasksection{Обработка результатов}

\item Постройте калибровочный график $B(I)$.

\item Пользуясь графиком $B(I)$ определите усреднённые значения~$B_ф$ для каждого
фокуса и постройте график зависимости $B_ф(n)$ фокусирующего поля от номера $n$. 
Используйте наклон графика для расчёта~$e/m$ с помощью формулы~\eqref{3.1.1}.

\item Оцените погрешности и сравните результат с табличным.

\end{lab:task}


\labsection{Б. Измерение ${e/m}$ методом магнетрона}

В~так называемом {\important{методе магнетрона}} отношение~$e/m$ измеряется на
основе исследования движения электрона в перпендикулярных друг другу (скрещенных) 
электрическом и магнитном полях. Название метода связано с тем, что такая
конфигурация полей реализуется в магнетронах~--- 
генераторах электромагнитных колебаний сверхвысоких частот.

\begin{figure}[h!]
    \centering
    \pic{8cm}{Chapter_3/v3_3}
    \caption{Движение заряда в скрещенных полях (без начальной скорости)}
    \figmark{Crossed fields}
\end{figure}
%\todo[inline,author=Popov]{Рисунок странный. Что там по центру? Надо бы
%    нарисовать другой}

Для уяснения идеи метода магнетрона, рассмотрим вначале упрощённую
задачу о движение заряда в <<плоском магнетроне>>. 
Пусть имеется плоский конденсатор, в пространстве между пластинами которого создан
высокий вакуум (вакуумный диод). Поместим его в однородное магнитное поле (например,
внутрь соленоида) так, что $\vec{E}\perp\vec{B}$ (рис.~\figref{Crossed fields}). 
При этом отрицательная пластина конденсатора играет роль катода, 
положительная~--- анода. Если бы магнитного поля не было, то все электроны, 
вылетевшие без начальной скорости из катода, попадали бы на анод. 
При наличии же магнитного поля траектории электронов искривляются, 
вследствие чего при достаточно большом $B$ ни один электрон не достигнет анода.
Таким образом, при
заданном напряжения $V$ между пластинами существует некоторое критическое
значение магнитной индукции~$B_\text{кр}(V)$, при котором траектории касаются
поверхности анода. Если~$B<B_\text{кр}$, то все электроны достигают анода и ток
через магнетрон имеет то же значение, что и без магнитного поля. Если же
$B>B_\text{кр}$, то электроны не достигают анода и ток через диод равен нулю.

Рассчитаем это критическое магнитное поле для плоского конденсатора.
Движение электрона будет иметь характер электрического дрейфа. 
Если начальная скорость равна нулю 
(начальные условия $x(0)=y(0)=0$, $v_x(0)=v_y(0)=0$), 
то как нетрудно получить из уравнений \chaptereqref{3.9}, 
траекторией частицы будет \emph{циклоида}:
\begin{equation}
\eqmark{3.11}
x = Vt - R\sin \omega_B t,\qquad y = R(1-\cos\omega_B t),
\end{equation}
где $V=E/B$ --- дрейфовая скорость, $R=V/\omega_B=Em/(eB^2)$.
Касание анода происходит при $2R=h$ ($h$~--- расстояние между анодом и катодом).
Этому значению соответствует критическое поле
\begin{equation}
B_\text{кр}=\frac{\sqrt{2U}}{h\sqrt{e/m}},
\eqmark{3.12}
\end{equation}
где $U=Eh$ --- напряжение между пластинами.
Отсюда находим удельный заряд:
\begin{equation}
\eqmark{3.13}
\frac{e}{m}=\frac{2U}{h^2B_\text{кр}^2}.
\end{equation}

Формула \eqref{3.13} позволяет вычислить~$e/m$, если при заданном 
значении напряжения на диоде~$U$ найти такое значение
магнитного поля, при превышении которого ток в магнетроне отсутствует.

\experiment

В данной работе движение электронов случае происходит в кольцевом пространстве,
заключённом между катодом и анодом двухэлектродной электронной вакуумной лампы 
(рис.~\figref{Two-electrode lamp}).
Нить накала лампы (катод) располагается вдоль оси цилиндрического анода, так что
электрическое поле между катодом и анодом имеет \emph{радиальное} направление. 
Лампа помещается внутри соленоида, создающего магнитное поле, \emph{параллельное оси} лампы.

\begin{figure}[h!]
    \begin{minipage}[b]{0.4\textwidth}
        \centering
        \pic{4cm}{Chapter_3/3_1_2}
        \caption{Схема устройства двухэлектродной лампы}
        \figmark{Two-electrode lamp}
    \end{minipage}
    \hfill
    \begin{minipage}[b]{0.5\textwidth}
        \centering
        \pic{5cm}{Chapter_3/3_1_3}
        \caption{Траектории электронов, вылетающих из~катода, при~разных
            значениях индукции магнитного~поля}
        \figmark{Path of electrons}
    \end{minipage}
\end{figure}

Таким образом, реализуется геометрия скрещенных полей $\vec{E}$ и $\vec{B}$.
Поскольку поле $\vec{E}$ в данном случае не является однородным (оно зависит от расстояния
до оси), траектории частиц будут несколько отличаться от рассмотренного выше плоского
случая. Тем не менее, все качественные особенности траектории сохранятся:
выражение для критического поля будет отличаться от \eqref{3.12} только
численным коэффициентом порядка единицы.
Подробно задача о движении электронов в такой лампе рассмотрено в Приложении к работе.
В частности, там получена связь \eqref{cyl_B_crit} критического поля 
$B_{кр}$ и напряжения на лампе $U_{А}$.
Для удельного заряда имеет место следующее выражение:
\begin{equation}
	\frac{e}{m}=\frac{8U_{А}}{B_\text{кр}^2r_{А}^2},
	\eqmark{3.1.3}
\end{equation}
где $r_{А}$~--- радиус анода.

До сих пор мы рассматривали идеальный случай: при $B<B_\text{кр}$ все
электроны без исключения попадают на анод, а при $B>B_\text{кр}$ все они
возвращаются на катод, не достигнув анода. Анодный ток~$I_{А}$ с увеличением
магнитного поля изменялся бы при этом так, как это изображено штриховой линией 
на рис.~\figref{Anode current from induction}. В~реальных условиях
невозможно обеспечить полную коаксиальность анода и катода, вектор индукции
магнитного поля всегда несколько наклонён по отношению к катоду, магнитное поле
не вполне однородно и т.~д. Всё это приводит к сглаживанию кривой 
$I_{А}(B)$ (сплошная линия на рис.~\figref{Anode current from induction}).
Тем не менее, в~хорошо собранной установке перелом функции~$I_A(B)$ остаётся
достаточно резким и может быть использован для измерения~$e/m$.

\begin{figure}[h]
    \centering
    \pic{7cm}{Chapter_3/3_1_4}
    \caption{Зависимость анодного тока от индукции магнитного поля в соленоиде}
    \figmark{Anode current from induction}
\end{figure}

Схема установки изображена на рис.~\figref{Scheme}. Двухэлектродная
лампа имеет цилиндрический анод. Анод лампы состоит из трёх немагнитных металлических 
цилиндров одинакового диаметра.
Два крайних цилиндра электрически изолированы от среднего небольшими зазорами и
используются для устранения краевых эффектов на торцах среднего цилиндра, ток с
которого используется при измерениях. В~качестве катода используется тонкая
(диаметр $2r_{К}=50~\text{мкм}$) натянутая вольфрамовая проволока, расположенная по оси
всех трёх цилиндров анодной системы. Катод разогревается проходящим 
через него переменным током (\emph{лампа прямого накала}),
отбираемым от стабилизированного источника питания. 
На анод лампы подаётся постоянное напряжение с регулируемого источника, 
измеряемое вольтметром $U_{А}$. Ток $I_{А}$ через среднюю секцию анода  
измеряется с помощью миллиамперметра.

\begin{figure}[h]
    \centering
	\pic{6cm}{Chapter_3/3_1_5}
	\caption{Схема измерительной установки}
	\figmark{Scheme}
\end{figure}

Лампа закреплена в соленоиде. Ток $I_{С}$, проходящий через соленоид, подаётся с
независимого источника и измеряется амперметром. Индукция магнитного поля в
соленоиде рассчитывается по току, протекающему через обмотку соленоида.
Коэффициент пропорциональности между ними указан на установке.


\begin{lab:task}

\taskpreamble{В~работе предлагается исследовать зависимость анодного тока от магнитного
поля в соленоиде при различных напряжениях на аноде лампы.
По результатам измерений рассчитать удельный заряд электрона.}

    
\item Установите минимальный потенциал~$U_{А}$ на аноде лампы, 
рекомендуемый в описании установки. Измерьте зависимость анодного тока~$I_A$ 
от тока через соленоид $I_{С}$. В~области резкого изменения тока
экспериментальные точки должны лежать чаще 
(см. рис.~\figref{Anode current from induction}).

\item Измерьте аналогичные зависимости $I_{А}(I_{С})$ для 6--8 
значений анодного напряжения~$U_{А}$ в диапазоне, указанном в описании установки.

\item Запишите параметры установки и характеристики приборов. 

\item Установите регуляторы источников питания на минимум и выключите их.

\tasksection{Обработка результатов}

\item Постройте семейство зависимостей анодного тока от магнитного поля $I_{A}(B)$ 
для всех значений $U_{А}$. 

\item По участкам графика с максимальным наклоном для каждого значения~$U_{А}$ 
определите критическое значение индукции магнитного поля~$B_\text{кр}$.

\item Постройте  график зависимости~$B_\text{кр}^2$ от~$U_{А}$. 
Убедитесь, что зависимость имеет прямой характер. Используя
формулу \eqref{3.13}, по угловому коэффициенту полученной 
прямой определите удельный заряд электрона~$e/m$.

\item Оцените погрешности. Сравните результат с табличным.

\end{lab:task}


\begin{lab:questions}
\item Что такое циклотронная частота и ларморовский радиус?
\item Что такое дрейф в скрещенных полях? Чему равна и куда направлена дрейфовая скорость?
\item Что представляет собой траектория движения частицы без начальной скорости 
в однородных скрещенных электрическом и магнитном полях?
\item При каких условиях возможна фокусировка пучка электронов внешним 
магнитным полем?
\item Получите выражение для критического магнитного поля $B_{кр}$ для метода магнетрона
с плоскими электродами.
\item Объясните принцип работы электронно-лучевой трубки осциллографа.
\item Объясните принцип работы милливеберметра.
\item Почему в методе магнетрона используется анод из трёх разделённых цилиндров?
\item Найдите распределение электрического поля $E(r)$ и потенциала $\varphi(r)$ 
в зависимости от расстояния $r$ до оси в лампе, используемой в методе магнетрона.
\end{lab:questions}

\begin{lab:literature}
\item \SivuhinIII \S\S~86, 89.
\item \emph{Калашников~С.Г.} Электричество.~--- М.: Физматлит, 2003,
\S\S~181--184.
\end{lab:literature}


\labsupplement

\section*{Движение электрона в цилиндрическом магнетроне}

\begin{small}
Рассмотрим траекторию электронов, движущихся в лампе магнетрона.
Воспользуемся цилиндрической системой координат: будем характеризовать 
положение точки расстоянием от оси цилиндра~$r$, 
полярным углом~$\theta$ и смещением вдоль оси системы~$z$.

Электрическое поле в цилиндрическом конденсаторе 
имеет только радиальную компоненту. Магнитное поле $B$, 
созданное внешним соленоидом, однородно и направлено по оси $z$.
 
В такой геометрии все силы, действующие на заряды, лежат в плоскости,
перпендикулярной оси $z$.  Поэтому движение вдоль $z$ является равномерным:
($v_z=\mathrm{const}$).

Рассмотрим движение в плоскости, перпендикулярной оси $z$. 
Представим скорость частицы как сумму радиальной и угловой компонент:
\[
 v_r=\dot{r},\qquad v_{\theta} = r\dot{\theta}.
\]

Применим к движению электрона уравнение моментов в проекции на~$z$:
$dL_z/dt = M_z$.
Здесь $L_z = mr^2 \dot{\theta}$ --- момент импульса электрона.
Момент сил создаётся только магнитной составляющей силы Лоренца и равен
$M_z = e v_r B r =  e B r \dot{r}$. Таким образом, имеем
\begin{equation}
\eqmark{3.1.10}
\frac{d}{dt}\left(r^2\dot{\theta}\right) = eBr\frac{dr}{dt}.
\end{equation}
Интегрируя \eqref{3.1.10}, найдём
\begin{equation}
	r^2\dot{\theta}+C=\frac{eBr^2}{2m}.
	\eqmark{3.1.10x}
\end{equation}
Здесь $C$~--- постоянная интегрирования, которую следует определить из начальных
условий. В~начале движения радиус~$r$ равен радиусу катода, т.\,е. очень мал,
поэтому мала и правая часть \eqref{3.1.10x}. Электроны вылетают из
катода с небольшой скоростью, так что $r^{2}\dot{\theta}$ в начальный момент
также мало. Таким образом, можно с хорошей точность полагать $C=0$. 
Уравнение \eqref{3.1.10x} приобретает при этом простой вид:
\begin{equation}
	\dot{\theta}=\frac{eB}{2m} = \frac12 \omega_B.
	\eqmark{3.1.12}
\end{equation}

Радиальное движение электрона можно описать, используя закон сохранения
энергии. Поскольку магнитное поле работы не совершает, имеем
\begin{equation}	\eqmark{3.1.13}
eU(r)=m\frac{v_r^2+v_\theta^2}{2} = 
\frac{m}{2} \left[\left(\frac{dr}{dt}\right)^2 + \left(r \frac{eB}{2m}\right)^{2}\right]
\end{equation}
Уравнение \eqref{3.1.13} полностью определяет радиальное 
движение электрона. Его решение не выражается в аналитических функциях,
но может быть найдено численно.

Для определения максимального радиуса траектории достаточно положить $dr/dt=0$.
Тогда из \eqref{3.1.13} находим критическое магнитное поле, при котором
радиус траектории равен радиусу анода $r_{\rm max} = r_{А}$:
\begin{equation}
\eqmark{cyl_B_crit}
U_{А}=\frac18 \frac{e}{m} r_{А}^2 B_{кр}^2.
\end{equation}
Полученная формула отличается от плоского случая \eqref{3.13} на множитель $\frac14$.
\end{small}