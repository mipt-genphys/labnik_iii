\lab{Эффект Холла в полупроводниках}

\aim{измерение подвижности и концентрации носителей заряда в полупроводниках.}

\equip{электромагнит с регулируемым источником питания; 
вольтметр; амперметр; миллиамперметр; милливеберметр или миллитесламетр;
источник питания (1,5~В), образцы легированного германия.}

Перед выполнением работы необходимо ознакомиться с основами
элементарной теории движения носителей заряда в~металлах и полупроводниках
(п.~\ref{sec:halleffect} введения к разделу).

\etp{2}

В работе изучаются особенности проводимости полупроводников
в геометрии \emph{мостика Холла}.
Ток пропускается по плоской полупроводниковой пластинке, 
помещённой в перпендикулярное пластинке магнитное поле.
Измеряется разность потенциалов между краями пластинки в поперечном
к току направлении. По измерениям определяется \emph{константа Холла},
тип проводимости (\emph{электронный} или \emph{дырочный}) и на основе
соотношения \chaptereqref{HallConstant} вычисляется концентрация основных
носителей заряда.

\experiment

Электрическая схема установки для измерения ЭДС Холла представлена
на рис.~\figref{Scheme}. В зазоре электромагнита (рис.~\figref{Scheme}а) 
создаётся постоянное магнитное поле, величину которого можно менять с помощью 
регулятора источника питания электромагнита. Ток питания электромагнита 
измеряется амперметром А$_1$ (внешним или встроенным в источник).
Направление тока в обмотках электромагнита меняется переключением
разъёма~К$_1$.

Градуировка электромагнита (связь тока с индукцией поля) проводится 
при помощи милливеберметра (его описание и правила работы 
с ним приведены на с.~\pageref{MWB}) или миллитесламетра на основе
датчика Холла.

Прямоугольный образец из легированного германия, смонтированный в специальном держателе
(рис.~\figref{Scheme}б), подключается к источнику питания ($\simeq 1,5$~В). При
замыкании ключа~К$_2$ вдоль длинной стороны образца течёт ток, величина которого
регулируется реостатом~$R_2$ и измеряется миллиамперметром~А$_2$.

В образце, помещённом в зазор электромагнита, между 
контактами~3 и~4 возникает разность потенциалов~$U_{34}$, 
которая измеряется с помощью вольтметра~V.

\begin{figure}[h!]
    \centering
    \pic{0.9\textwidth}{Chapter_3/3_4_1}
    \caption{Схема установки для исследования эффекта Холла в~полупроводниках}
    \figmark{Scheme}
\end{figure}
\pagebreak

Контакты~3 и~4 вследствие неточности подпайки могут лежать не на одной
эквипотенциали. Тогда напряжение между ними связано не только с эффектом
Холла, но и с омическим падением напряжения вдоль пластинки. 
Исключить этот эффект можно, изменяя направление магнитного поля, 
пронизывающего образец. 
При обращении поля ЭДС Холла меняет знак, а омическое падение напряжения
остаётся неизменным. Поэтому ЭДС Холла~$U_{\perp}$ может быть определена
как половина алгебраической разности показаний вольтметра, полученных для двух
противоположных направлений магнитного поля в~зазоре:
$U_{\perp} = \frac12 (U_{34}^{(+)}-U_{34}^{(-)})$.

Альтернативно можно исключить влияние омического падения напряжения, 
если при каждом значении тока через образец измерять напряжение
между точками~3 и~4 в отсутствие магнитного поля. При фиксированном токе через
образец это дополнительное к ЭДС Холла напряжение~$U_0$ остаётся неизменным. 
От него следует (с учётом знака) отсчитывать величину ЭДС Холла:
\begin{equation}
	U_{\perp} = U_{34} - U_0.
	\eqmark{3.4.1}
\end{equation}
При таком способе измерения нет необходимости проводить повторные измерения
с противоположным направлением магнитного поля.

По знаку $U_{\perp}$ можно определить характер проводимости~--- 
электронный или дырочный. Для этого необходимо знать направление тока 
в образце и направление магнитного поля.

Измерив ток~$I$ в образце и напряжение~$U_{35}$ между контактами~3 и~5 в
отсутствие магнитного поля, можно, зная
параметры образца, рассчитать проводимость материала образца по формуле
\begin{equation}
	\rho_0=\frac{U_{35}ah}{Il},
	\eqmark{3.4.2}
\end{equation}
где $l$~--- расстояние между контактами~3 и~5, $a$~--- ширина образца, $h$~---
его толщина.

\begin{lab:task}

\taskpreamble{В работе предлагается исследовать зависимость ЭДС Холла от величины магнитного
поля при различных значениях тока через образец для определения константы Холла;
определить знак носителей заряда и проводимость материала образца.}

\item Подготовьте приборы к работе согласно описанию на установке.

\item Проверьте работу цепи питания образца. Ток через образец не должен
превышать 1~мА.

\item Установите ручки регулировки источника питания электромагнита 
в минимальное положение и включите источник. 
Плавно изменяя ток магнита, определите диапазон его изменения.

\item Измерьте калибровочную кривую электромагнита~---
зависимость между индукцией~$B$ магнитного поля в его зазоре и 
током~$I_{М}$ через обмотки магнита.
Магнитное поле измеряется милливеберметром или миллитесламетром
(датчиком Холла). Калибровочная кривая должна содержать не менее
15 точек во всём диапазоне изменения токов.

В~случае использования милливеберметра измерьте зависимость 
магнитного потока~$\Phi$, пронизывающего пробную катушку, 
находящуюся в зазоре, от тока~$I_{М}$ ($\Phi=BSN$). 
Значение~$SN$ (произведение площади сечения контура катушки на
число витков в ней) указано на держателе катушки.

\item \label{p334-1} Вставьте образец в зазор \emph{выключенного} электромагнита 
и определите напряжение $U_0$ между холловскими
контактами~3 и~4 при минимальном токе через образец ($I\simeq 0,2$~мА). Это
напряжение~$U_0$ вызвано несовершенством контактов~3, 4 и при фиксированном токе
через образец остаётся неизменным. Значение~$U_0$ с учётом знака следует принять
за начало отсчёта напряжения.

\item \label{p334-2} Проведите измерение ЭДС Холла: снимите зависимость напряжения~$U_{34}$ 
от тока электромагнита~$I_{М}$ (6--8 точек) при фиксированном токе~$I$ 
через образец.

\item Повторите измерения п.~\ref{p334-1} и \ref{p334-2} при 6--8 токах $I$ через образец
(рекомендованные токи указаны в описании установки).  
Учтите, что при каждом новом токе $I$ величина~$U_0$ будет иметь 
своё значение.

\item При максимальном токе через образец проведите измерения $U_{34}(I_{M})$ 
при обратном направлении магнитного поля.

\item Определите знак носителей заряда в образце. Для этого необходимо знать
направление тока через образец, направление магнитного поля и знак ЭДС Холла.

Направление тока в образце показано знаками~<<$+$>> и~<<$-$>> на
рис.~\figref{Scheme}. Направление тока в обмотках электромагнита при 
установке разъёма~К$_1$ в положение~1 показано стрелкой на торце магнита.
Знак напряжения $U_{34}$ показан на дисплее цифрового вольтметра.

Зарисуйте в тетради образец. Укажите на рисунке направления тока, магнитного
поля и отклонение носителей. Определите характер проводимости образца
(дырочный или электронный).

\item Измерьте удельную проводимость образца. Для этого: 
удалите держатель с образцом из зазора электромагнита;
установите максимальный ток через образец, используемый в предыдущих измерениях;
и с помощью вольтметра измерьте падение напряжения между проводниками, 
подключёнными к точкам~3 и~5 образца.

\item Запишите характеристики приборов и параметры образца, указанные на держателе:
длину~$l$ (расстояние между точками~3 и~5), ширину~$a$, толщину~$h$.


\tasksection{Обработка результатов}

\item Постройте калибровочный график зависимости $B(I_{M})$. 
Используйте эту зависимость для дальнейшего пересчёта (интерполяции)
значений поля по измеренным токам $I_{М}$.


\item Рассчитайте ЭДС Холла по формуле \eqref{3.4.1} и постройте на одном листе
семейство характеристик $U_{\perp}(B)$ при разных значениях тока~$I$ через
образец. Убедитесь в линейности зависимостей и определите угловые 
коэффициенты $k=dU_{\perp}/dB$ полученных прямых.

\item Постройте график $k(I)$. Рассчитайте угловой коэффициент прямой 
и по формуле \chaptereqref{3.19} определите величину постоянной Холла~$R_{\rm H}$.

\item Рассчитайте концентрацию $n$ носителей тока в образце,
удельное сопротивление $\rho_0$ и удельную проводимость $\sigma_0$ материала.

\item Используя найденные значения концентрации~$n$ и удельной проводимости
$\sigma_0=1/\rho_0$, вычислите 
подвижность~$\mu$ носителей тока. Ответ представьте в общепринятых для этой величины 
внесистемных единицах $[\mu]=$~см$^2/(В\cdotс)$
(размерность напряжённости $[E]=$~B/см, скорости $[v]=$~см/с,
поэтому $[\mu]=[v/E]=$~см$^2/(В\cdot с)$).

\item Оцените погрешности и сравните результаты с табличными.

\end{lab:task}


\begin{lab:questions}

\item Какие вещества называют диэлектриками, проводниками, полупроводниками?
Чем объясняется различие их электрических свойств? Как зависит от температуры
проводимость металлов и полупроводников?

\item Дайте определение константы Холла. Как зависит константа Холла от
температуры у металлов и полупроводников?

\item Зависит ли результат измерения константы Холла от геометрии образца?

\item Зависит ли сопротивление образца от магнитного поля 
в условиях опыта?

\item Как устроен милливеберметр? Зависят ли его показания от сопротивления
измерительной катушки? Каким должно быть это сопротивление по сравнению с
сопротивлением катушки прибора?

\item По результатам измерений оцените частоту столкновений, длину пробега
и коэффициент диффузии носителей тока в образце.

\item Получите выражение константы Холла для материалов с~двумя типами
носителей. \emph{Указание}: воспользуйтесь условием равенства нулю поперечного тока.

\end{lab:questions}


\begin{lab:literature}
\item \SivuhinIII~--- \S\S~98, 100.
\item \textit{Парселл Э.} Электричество и магнетизм.~--- М.: Наука, 1983. Гл.~4,
\S\S~4--6; Гл.~6, \S~9
\end{lab:literature}

