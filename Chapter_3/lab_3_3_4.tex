\lab{Эффект Холла в полупроводниках}

\aim{измерение подвижности и концентрации носителей заряда в полупроводниках.}

\equip{электромагнит с источником питания, амперметр, миллиамперметр,
милливеберметр или цифровой магнитометр, реостат, цифровой вольтметр,
источник питания (1,5~В), образцы легированного германия.}

Элементарная теория свободных носителей заряда в~металлах и полупроводниках
изложена во введении к разделу.

\todo[inline,author=Popov]{Ну хоть пару слов о работе!}

\experiment
Электрическая схема установки для измерения ЭДС Холла представлена
на рис.~\figref{Scheme}.

В зазоре электромагнита (рис.~\figref{Scheme}а) создаётся постоянное магнитное
поле, величину которого можно менять с помощью регулятора~$R_1$ источника
питания электромагнита. Ток питания электромагнита измеряется амперметром~$A_1$.
Разъём~$K_1$ позволяет менять направление тока в обмотках электромагнита.

Градуировка магнита проводится при помощи милливеберметра(его описание и правила
работы с ним приведены на с.~\pageref{MWB}) или цифрового магнитометра на основе
датчика Холла.
\begin{figure}[h!]
	\pic{0.9\textwidth}{Chapter_3/3_4_1}
	\caption{Схема установки для исследования эффекта Холла в~полупроводниках}
	\figmark{Scheme}
\end{figure}

Образец из легированного германия, смонтированный в специальном держателе
(рис.~\figref{Scheme}б), подключается к источнику питания ($\simeq 1,5$~В). При
замыкании ключа~$K_2$ вдоль длинной стороны образца течёт ток, величина которого
регулируется реостатом~$R_2$ и измеряется миллиамперметром~$A_2$.

В образце с током, помещённом в зазор электромагнита, между контактами~3 и~4
возникает разность потенциалов~$U_{34}$, которая измеряется с помощью цифрового
вольтметра.

Иногда контакты~3 и~4 вследствие неточности подпайки не лежат на одной
эквипотенциали, и тогда напряжение между ними связано не только с эффектом
Холла, но и с омическим падением напряжения, вызванным протеканием основного
тока  через образец. Измеряемая разность потенциалов при одном направлении
магнитного поля равна сумме ЭДС Холла и омического падения напряжения, а при
другом~--- их разности. В этом случае ЭДС Холла~$V_{xy}$ может быть определена
как половина алгебраической разности показаний вольтметра, полученных для двух
противоположных направлений магнитного поля в~зазоре. Знак измеряемого
напряжения высвечивается на цифровом табло вольтметра.

Можно исключить влияние омического падения напряжения иначе, если при каждом
токе через образец измерять напряжение
между точками~3 и~4 в отсутствие магнитного поля. При фиксированном токе через
образец это дополнительное к ЭДС Холла напряжение~$U_0$ остаётся неизменным. От
него следует (с учётом знака) отсчитывать величину ЭДС Холла:
\begin{equation}
	V_{xy} = U_{34}\pm U_0.
	\eqmark{3.4.1}
\end{equation}

При таком способе измерения нет необходимости проводить повторные измерения
с противоположным направлением магнитного поля.

По знаку $V_{xy}$ можно определить характер проводимости~--- электронный или
дырочный. Для этого необходимо знать направление тока в образце и направление
магнитного поля.

Измерив ток~$I$ в образце и напряжение~$U_{35}$ между контактами~3 и~5 в
отсутствие магнитного поля, можно, зная
параметры образца, рассчитать проводимость материала образца по формуле
\begin{equation}
	\rho_0=\frac{U_{35}wh}{Il},
	\eqmark{3.4.2}
\end{equation}
где $l$~--- расстояние между контактами~3 и~5, $w$~--- ширина образца, $h$~---
его толщина.

\begin{lab:task}

В работе предлагается исследовать зависимость ЭДС Холла от величины магнитного
поля при различных токах через образец для определения константы Холла;
определить знак носителей заряда и проводимость материала образца.

\begin{enumerate}
\item{ Подготовьте приборы к работе.}

\item{ Проверьте работу цепи питания образца. Ток через образец не должен
превышать 1~мА.}

\item{ Проверьте работу цепи магнита. Определите диапазон изменения тока через
магнит.}

\item{ Прокалибруйте электромагнит~--- определите связь между индукцией~$B$
магнитного поля в зазоре электромагнита и током~$I_M$ через обмотки магнита с
помощью цифрового магнитаметра, либо с помощью милливеберметра. В~случае
сипользования милливеберметра снимите зависимость магнитного потока~$\Phi$,
пронизывающего пробную катушку, находящуюся в зазоре, от тока~$I_M$
($\Phi=BSN$). Значение~$SN$ (произведение площади сечения контура катушки на
число витков в ней) указано на держателе катушки.}

\item{ Проведите измерение ЭДС Холла. Для этого вставьте образец в зазор
выключенного электромагнита и определите напряжение $U_0$ между холловскими
контактами~3 и~4 при минимальном токе через образец ($\simeq 0,2$~мА). Это
напряжение~$U_0$ вызвано несовершенством контактов~3, 4 и при фиксированном токе
через образец остаётся неизменным. Значение~$U_0$ с учётом знака следует принять
за нулевое.

Включите электромагнит и снимите зависимость напряжения~$U_{34}$ от тока~$I_M$
через обмотки магнита при фиксированном токе через образец.

Проведите измерения $U_{34}=f(I_{M})$ при постоянном токе через образец для 6 --
8 его значений в интервале, указанном в описании конкретной установки. При
каждом новом значении тока через образец величина~$U_0$ будет иметь своё
значение.

При максимальном токе через образец проведите измерения $U=f(I_{M})$ при другом
направлении магнитного поля.}

\item{ Определите знак носителей в образце. Для этого необходимо знать
направление тока через образец, направление магнитного поля и знак ЭДС Холла.

Направление тока в образце показано знаками~<<$+$>> и~<<$-$>> на
рис.~\figref{Scheme}. Направление тока в обмотках электромагнита при установке
разъёма $K_1$ в положение I показано стрелкой на торце магнита.

Зарисуйте в тетради образец. Укажите на рисунке направления тока, магнитного
поля и отклонение носителей. По знаку ($\pm$) на клеммах цифрового вольтметра
определите характер проводимости.}

\item{Для определение удельной проводимости удалите держатель с образцом из
зазора. Подключите к клеммам~<<$H_{x}$>> и~<<$L_{x}$>> вольтметра потенциальные
концы 3 и 5. Измерьте падение напряжения между ними при токе через образец
порядка максимального значения в предыдущих измерениях.}

\item{ Запишите характеристики приборов и параметры образца~$l$, $w$, $h$,
указанные на держателе.}
\end{enumerate}

\tasksection{Обработка результатов}
\begin{enumerate}

\item {Постройте график зависимости $B=f(I_{M})$.}

\item{ Рассчитайте ЭДС Холла по формуле \eqref{3.4.1} и постройте на одном листе
семейство характеристик $V_{xy}=f(B)$ при разных значениях тока~$I$ через
образец. Определите угловые коэффициенты $k(I)=\Delta{V_{xy}}/\Delta B$
полученных прямых.

Постройте график $k=f(I)$. Рассчитайте угловой коэффициент прямой и по формуле
\chaptereqref{3.19} из введения определите величину постоянной Холла~$R_{X}$.
Рассчитайте концентрацию $n$ носителей тока в образце по формуле
\chaptereqref{HallConstant}.

Оцените погрешность результата и сравните результат с табличным.}


\item{ Рассчитайте удельную проводимость~$\rho_0$ материала образца по формуле
\eqref{3.4.2}.

Используя найденные значения концентрации~$n$ и удельного сопротивления
$\rho_0$, с помощью формулы \chaptereqref{UdSoprot} вычислите подвижность~$b$
носителей тока в~общепринятых для этой величины внесистемных единицах:
размерность напряжённости электрического поля $[E]=[U/L]=$~B/см, размерность
скорости $[v]=$~см/с, поэтому размерность
подвижности $[b]=$~см$^2$/(В$\cdot$с).

Оцените погрешности и сравните результаты с~табличными.}
\end{enumerate}
\end{lab:task}

\begin{lab:questions}

\item{ Какие вещества называют диэлектриками, проводниками, полупроводниками?
Чем объясняется различие их электрических свойств? Как зависит от температуры
проводимость металлов и полупроводников?}

\item{ Дайте определение константы Холла. Как зависит константа Холла от
температуры у металлов и полупроводников?}

\item{ Зависит ли результат измерения константы Холла от геометрии образца?}

\item{ Как устроен милливеберметр? Зависят ли его показания от сопротивления
измерительной катушки? Каким должно быть это сопротивление по сравнению с
сопротивлением катушки прибора: большим или маленьким?}

\item{ Получите выражение константы Холла для материалов с~двумя типами
носителей. При выводе используйте условие равенства нулю поперечного тока.}

\end{lab:questions}

\begin{lab:literature}

\item{ \emph{Сивухин Д.В.} Общий курс физики. Т.~III. Электричество~--- М.:
Наука, 1983. \S\S~98, 100.}

\item{ \emph{Парселл Э.} Электричество и магнетизм.~--- М.: Наука, 1983. Гл.~4,
\S\S~4--6; Гл.~6, \S~9 (Берклеевский курс физики. Т.~II).}

\end{lab:literature}
