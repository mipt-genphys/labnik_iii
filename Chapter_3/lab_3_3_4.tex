\lab{Эффект Холла в полупроводниках}

\aim{измерение подвижности и концентрации носителей заряда в полупроводниках.}

\equip{электромагнит с источником питания, амперметр, миллиамперметр, милливеберметр, реостат, цифровой вольтметр,
источник питания (1,5~В), образцы легированного германия.}

Элементарная теория свободных носителей заряда в~металлах и полупроводниках изложена во введении к разделу.

{\bf Экспериме
нтальная установка.} Электрическая схема установки для измерения ЭДС~Холла представлена на рис.~\ref{fig3.4.1}.

В~зазоре электромагнита (рис.~\ref{fig3.4.1}а) создаётся постоянное магнитное поле, величину которого можно менять с~помощью регулятора~$R_1$ источника питания электромагнита. Ток питания электромагнита измеряется амперметром~А$_1$. Разъём~К$_1$ позволяет менять направление тока в~обмотках электромагнита.

Градуировка магнита проводится при помощи милливеберметра. Описание милливеберметра и правила работы с ним приведены на с.~\pageref{MWB}.

\begin{figure}
%\fcpic[0.9]{3_4_1}
\caption{Схема установки для исследования эффекта Холла в~полупроводниках}
\label{fig3.4.1}
\end{figure}

Образец из легированного германия, смонтированный в~специальном держателе (рис. \ref{fig3.4.1}б), подключается к~источнику питания ($\simeq 1,5$~В). При замыкании ключа~К$_2$ вдоль длинной стороны образца течёт ток, величина которого регулируется реостатом~$R_2$ и измеряется миллиамперметром~А$_2$.

В образце с током, помещённом в зазор электромагнита, между контактами 3 и 4 возникает разность потенциалов~$U_{34}$, которая измеряется с~помощью цифрового вольтметра.

Иногда контакты 3 и 4 вследствие неточности подпайки не лежат на одной эквипотенциали, и тогда напряжение между ними связано не только с~эффектом Холла, но и с~омическим падением напряжения, вызванным протеканием основного тока  через образец. Измеряемая разность потенциалов при одном направлении магнитного поля равна сумме ЭДС~Холла и омического падения напряжения, а при другом~--- их разности. В~этом случае ЭДС ~Холла~$\epsilon_x$ может быть определена как половина алгебраической разности показаний вольтметра, полученных для двух противоположных направлений магнитного поля в~зазоре. Знак измеряемого напряжения высвечивается на цифровом табло вольтметра.

Можно исключить влияние омического падения напряжения иначе, если при каждом токе через образец измерять напряжение
между точками 3 и 4 в~отсутствие магнитного поля. При фиксированном токе через образец это дополнительное к~ЭДС~Холла напряжение~$U_0$ остаётся неизменным. От него следует (с~учётом знака) отсчитывать величину ЭДС~Холла:
\begin{equation}
\epsilon_x=U_{34}\pm U_0.
\label{fig3.4.1}
\end{equation}
При таком способе измерения нет необходимости проводить повторные измерения с~противоположным направлением магнитного поля.

По знаку $\epsilon_x$ можно определить характер проводимости~--- электронный или дырочный. Для этого необходимо знать
направление тока в~образце и направление магнитного поля.

Измерив ток~$I$ в об
разце и напряжение~$U_{35}$ между контактами~3 и 5 в~отсутствие магнитного поля, можно, зная
параметры образца, рассчитать проводимость материала образца по формуле
\begin{equation}
\sigma=\frac{I\, L_{35}}{U_{35}\,a\,l},
\label{eq3.4.2}
\end{equation}
где $L_{35}$~--- расстояние между контактами 3 и 5, $a$~--- толщина образца, $l$~--- его ширина.

{\Large \bf ЗАДАНИЕ}

В работе предлагается исследовать зависимость ЭДС~Холла от величины магнитного поля при различных токах через образец для определения константы Холла; определить знак носителей заряда и проводимость материала образца.

\begin{enumerate}
\item{ Подготовьте приборы к работе.}

\item{ Проверьте работу цепи питания образца. Ток через образец не должен превышать 1~мА.}

\item{ Проверьте работу цепи магнита. Определите диапазон изменения тока через магнит.}

\item{ Прокалибруйте электромагнит~--- определите связь между индукцией~$B$ магнитного поля в зазоре электромагнита и током $I_M$ через обмотки магнита. Для этого с помощью милливеберметра снимите зависимость магнитного потока $\Phi$, пронизывающего пробную катушку, находящуюся в~зазоре, от тока~$I_M$ ($\Phi=BSN$). Значение $SN$ (произведение площади сечения контура катушки на число витков в ней) указано на держателе катушки.}

\item{ Проведите измерение ЭДС Холла. Для этого вставьте образец в~зазор выключенного электромагнита и определите напряжение~$U_0$ между холловскими контактами 3 и 4 при минимальном токе через образец ($\simeq 0,2$~мА). Это напряжение~$U_0$ вызвано несовершенством контактов~3, 4 и при фиксированном токе через образец остаётся неизменным. Значение~$U_0$ с~учётом знака следует принять за нулевое.

Включите электромагнит и снимите зависимость напряжения~$U_{34}$ от тока~$I_M$ через обмотки магнита при фиксированном токе через образец.

Проведите измерения $U_{34}=f(I_{M})$ при постоянном токе через образец для 6--8 его значений в~интервале 0,2--1~мА. При каждом новом значении тока через образец величина~$U_0$ будет иметь своё значение.

При максимальном токе через образец ($\simeq 1$~мА) проведите измерения $U=f(I_{M})$ при другом направлении магнитного поля.
}
\item{ Определите знак носителей в образце. Для этого необходимо знать направление тока через образец, направление магнитного поля и знак ЭДС Холла.

Направление тока в образце показано знаками~<<+>> и <<$-$>> на рис. \ref{fig3.4.1}. Направление тока в~обмотках электромагнита при установке разъёма~K$_1$ в~положение~I показано стрелкой на торце магнита.

Зарисуйте в тетради образец. Укажите на рисунке направления тока, магнитного поля и отклонение носителей. По знаку
($\pm$) на клеммах цифрового вольтметра определите характер проводимости.
}
\item{Для определение удельной проводимости удалите держатель с~образцом из зазора. Подключите к~клеммам~<<$H_{x}$>> и <<$L_{x}$>> вольтметра потенциальные концы~3 и 5. Измерьте падение напряжения между ними при токе через образец 1~мА.}

\item{ Запишите характеристики приборов и параметры образца $L_{35}$, $a$, $l$, указанные на держателе.}
\end{enumerate}

{\rm Обработка результатов}
\begin{enumerate}

\item {Постройте график зависимости $B=f(I_{M})$.}

\item{ Рассчитайте ЭДС~Холла по формуле~(\ref{eq3.4.1}) и постройте на одном листе семейство характеристик $\epsilon_x=f(B)$ при разных значениях тока~$I$ через образец. Определите угловые коэффициенты~$k(I)=\Delta{\epsilon_x}/\Delta B$ полученных прямых.

Постройте график $k=f(I)$. Рассчитайте угловой коэффициент прямой и по формуле~(TO ADD REF) Приложения определите величину постоянной Холла~$R_{X}$. Рассчитайте концентрацию~$n$ носителей тока в~образце по формуле ([TO ADD REF]).

Оцените погрешность результата и сравните результат с~табличным.}


\item{ Рассчитайте удельную проводимость~$\sigma$ материала образца по формуле~(\ref{eq3.4.2}).

Используя найденные значения концентрации~$n$ и проводимости~$\sigma$, с~помощью формулы~([TO ADD REF]) вычислите подвижность~$b$ носителей тока в~общепринятых для этой величины внесистемных единицах: размерность напряжённости электрического поля $[E]=[U/L]=$~B/см, размерность скорости $[v]=$~см/с, поэтому размерность
подвижности~$[b]=$см$^2$/(В$\cdot$с).

Оцените погрешности и сравните результаты с~табличными.}
\end{enumerate}

{\small

{\bf \Large Контрольные вопросы}
\begin{enumerate}



\item{ Какие вещества называют диэлектриками, проводниками, полупроводниками? Чем объясняется различие их электрических свойств? Как зависит от температуры проводимость металлов и полупроводников?}

\item{ Дайте определение константы Холла. Как зависит константа Холла от температуры у металлов и полупроводников?}

\item{ Зависит ли результат измерения константы Холла от геометрии образца?}

\item{ Как устроен милливеберметр? Зависят ли его показания от сопротивления измерительной катушки? Каким должно быть это сопротивление по сравнению с сопротивлением катушки прибора: большим или маленьким?}

\item{ Получите выражение константы Холла для материалов с~двумя типами носителей. При выводе используйте условие равенства нулю поперечного тока.}

\end{enumerate}

{\bf \Large Список литераратуры}

\begin{enumerate}


\item{ \emph{Сивухин Д.В.} Общий курс физики. Т.~III. Электричество~--- М.: Наука, 1983. \S\S~98, 100.}

\item{ \emph{Парселл Э.} Электричество и магнетизм.~--- М.: Наука, 1983. Гл.~4, \S\S~4--6; Гл.~6, \S~9 (Берклеевский курс физики. Т.~II).
    }

\end{enumerate}
}


