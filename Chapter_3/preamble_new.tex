Электрический ток представляет собой направленный перенос зарядов. Микрочастицы,
осуществляющие этот перенос, называют \term{носителями тока}. В конце
XIX века в опытах Дж.~Дж.~Томсона с <<катодными лучами>> были открыты
элементарные носители заряда~--- \term{электроны}. Вскоре были проведены
первые измерения величины элементарного заряда: Р.~Милликен
в опытах в 1908--1916 годов измерял заряд микроскопических капелек масла,
помещённых во внешнее электрическое поле. Идея этих опытов проста:
если элементарный заряд действительно существует, то величина заряда~$q$
любого тела может принимать только дискретную последовательность значений.
Сравнивая между собой заряды капель, можно убедиться в том, что все они
кратны одному и тому же числу~--- элементарному заряду~$e$:
\begin{equation*}
    q = 0,\,\pm e,\,\pm2e,\,\pm3e,\, \ldots
\end{equation*}
Величина заряда оказалась равной
\[
e\approx1,6\cdot 10^{-19}~Кл\qquad (e=4,8\cdot 10^{-10}~ед.~СГС).
\]
Сразу после этого по отклонению <<катодных лучей>> в магнитном поле была
измерена и масса электрона $m_e\approx 9,1\cdot 10^{-31}$~кг.

Свободные электроны могут являться носителями тока в свободном от вещества
пространстве (ток в вакуумном диоде, ионный пучок в масс-спектрометре, и
т.\,д.). Понятие носителя тока \important{в веществе} уже не является таким
наглядным. Хотя в металлах и полупроводниках перенос заряда происходит
вследствие перемещения всё тех же электронов, их движение уже не является
движением свободных частиц, как в вакууме. Электроны движутся в сильном
периодическом поле, образованном ионами кристаллической решётки, и
взаимодействуют между собой, причём это движение и это взаимодействие
подчиняются законам квантовой механики. По этим законам получается, что перенос
заряда можно по-прежнему интерпретировать как движение свободных заряженных
частиц (точнее, \important{квази}частиц), но масса этих частиц, называемая
\term{эффективной массой}, \important{не совпадает с массой свободного
электрона}: $m_{e}^{эфф}\ne m_e$.

Более того, полупроводники и некоторые металлы
ведут себя так, будто вместо электронов ток в них переносят некоторые
положительные частицы. Такие квазичастицы получили название \term{дырки}
(\emph{англ.} holes).
В электрических и магнитных полях они ведут себя как частицы с положительным
элементарным зарядом $q_h=+e$ и с некоторой эффективной массой
$m_h^{эфф}\ne m_e$. Дырка в некотором смысле подобна <<пузырьку>>, образующемуся
если из данной точки удалить электрон.

Таким образом, в физике металлов и полупроводников в качестве носителей тока
рассматривают \term{квазичастицы},
\important{не существующие отдельно от рассматриваемого вещества}.
Заряд этих носителей численно точно равен элементарному и может быть как
отрицательным, так и положительным. В~первом случае они по-прежнему называются
\term{электронами} (хотя их масса не равна $m_e$),
во втором~--- \term{дырками}. В~полупроводниках присутствуют оба типа этих
носителей. В большинстве металлов имеются только отрицательные носители, но
существуют металлы и с дырочным типом проводимости (Zn, Cd).


\introsection{Движение частиц во внешних полях}

Рассмотрим два простейших примера движения свободных зарядов в вакууме под
действием электрического и магнитного полей. Такие условия движения реализуются,
например, в электронных вакуумных приборах, таких, как электронно-лучевая
трубка или вакуумный диод.

\introsubsection{Движение в однородном магнитном поле}

Как известно, на заряд~$q$, движущийся со скоростью~$\vec{v}$ в магнитном поле
$\vec{B}$, действует \term{сила Лоренца}:
\begin{equation*}
    \eqmark{3.1}
    \vec{F}=q{\vec{v}}\times{\vec{B}}.
\end{equation*}

Пусть точечный заряд движется с некоторой скоростью~$\vec{v}$ в
однородном магнитном поле $\vec{B}=\const$, перпендикулярном $\vec{v}$.
На движущийся электрон действует сила $F=qvB$. Она перпендикулярна
скорости движения, и поэтому не изменяет её абсолютной величины. Траектория
движения заряда в этом случае является \important{окружностью}, причём
\[
m\frac{v^2}{r}=qvB.
\]
Отсюда находим радиус~$r$ траектории, называемый \term{ларморовским радиусом}:
\begin{equation}
    \eqmark{3.2}
    \boxed{r_B =\frac{m v}{qB}= \frac{v}{\omega_B}},
\end{equation}
где
\begin{equation}
    \eqmark{omega-B}
    \boxed{\omega_B=\frac{qB}{m}}
\end{equation}
--- \term{циклотронная частота} (или \term{гирочастота}).
Такое движение частицы называется \term{циклотронным вращением}.

Заметим, что циклотронная частота не зависит от энергии частицы, так
что в однородном магнитном поле все частицы одного сорта вращаются с
\important{одинаковой} частотой. Частицы с разными знаками заряда вращаются в
противоположные стороны.

\begin{wrapfigure}{o}{0.25\textwidth}
\centering
 \pic{\linewidth}{Chapter_3/spiral}
 \caption{Траектория в параллельных полях $\vec{B}$ и $\vec{E}$}
 \figmark{spiral}
\end{wrapfigure}

Пусть теперь заряд движется под некоторым углом~$\alpha$ к вектору
индукции~$\vec{B}$. Скорость заряда~$\vec{v}$ можно разложить
на две составляющие, перпендикулярную и параллельную магнитному полю:
\begin{equation*}
    v_{\perp}=v\sin\alpha,\qquad v_{\parallel}=v\cos\alpha.
\end{equation*}
Параллельная составляющая не вызывает появление силы Лоренца, поэтому
в проекции на плоскость, перпендикулярную~$\vec{B}$,
траектория по-прежнему будет окружностью с радиусом
\begin{equation}
    \eqmark{3.4}
    r_B = \frac{v_{\perp}}{\omega_B} = \frac{m v_{\perp}}{qB}.
\end{equation}
В~направлении поля~$\vec{B}$ сила Лоренца равна нулю, поэтому заряд движется
равномерно со скоростью $v_{\parallel}=\const$.
Таким образом, траектория заряда представляет собой \important{винтовую линию}.

Наконец, если включить электрическое поле, \important{параллельное}
магнитному $\vec{E}\parallel\vec{B}$, то частица будет двигаться с ускорением
вдоль оси $\vec{E}$. На её циклотронное движение в плоскости, перпендикулярной
$\vec{B}$, это никак не повлияет.

\begin{lab:note}
Формула \eqref{3.2} может быть обобщена на случай релятивистских скоростей
частицы $v\sim c$. Можно получить (получите самостоятельно),
что траектория будет окружностью с радиусом
\begin{equation}
\eqmark{rB-rel}
    r_B = \frac{p}{\omega_B} =
    \frac{\gamma m v}{qB},
\end{equation}
где $\gamma = (1-v^2/c^2)^{-1/2}$ --- релятивистский фактор.

Также отметим, что в общем случае заряд, движущийся ускоренно (в частности,
по ларморовской окружности), излучает электромагнитные волны
(\term{циклотронное излучение}). Формулы~\eqref{3.2} и~\eqref{rB-rel} получены в
предположении, что интенсивность этого излучения достаточно мала,
так что его влиянием на движение заряда можно пренебречь.
\end{lab:note}


\introsubsection{Дрейф в скрещенных полях}

Рассмотрим движение заряда $q$ во взаимно перпендикулярных однородных
электрическом и магнитном полях $\vec{E}\perp\vec{B}$.
% (рис.~\figref{Crossed fields}).

Уравнение движения заряда в таком случае имеет вид
\[
m\dot{\vec{v}} = q\vec{E} + q \vec{v}\times \vec{B}.
\]
Направим ось $z$ вдоль~$\vec{B}$, а ось $y$~--- вдоль~$\vec{E}$.
Тогда, разделив на $m$, получим
\begin{equation}
    \eqmark{3.9}
    \begin{aligned}
        \dot{v}_x&=\omega_B v_y,\\
        \dot{v}_y&=\tfrac{q}{m}E-\omega_B v_x,\\
        \dot{v}_z&=0,
\end{aligned}
\end{equation}
где $\omega_B = qB/m$~--- циклотронная частота.

Сделаем замену переменных $v_x' = v_x - V$,
где $V=qE/(m\omega_B) = E/B$. Тогда первые два уравнения системы \eqref{3.9}
сводится к $\dot{v}'_x =\omega_B v_y$, $\dot{v}_y=-\omega_B v_x'$.
Решение --- вектор, вращающийся с постоянной угловой скоростью $\omega_B$:
\begin{equation*}
 v_x' = v'_{x0} \cos(\omega_B t + \varphi_0),\quad
 v_y = v_{y0} \sin(\omega_B t + \varphi_0).
\end{equation*}
Следовательно, в системе отсчёта, движущейся по оси $x$ со скоростью
$V=E/B$, траектория частицы будет ларморовской окружностью.
Радиус этой окружности определяется начальной скоростью
\emph{в этой системе отсчёта}: $r=v_{\perp 0}'/\omega_B$.

\begin{figure}[h]
\centering
\pic{5cm}{Chapter_3/drift}
\caption{Дрейф центра ларморовской окружности в скрещенных полях}
\figmark{drift}
\end{figure}

Таким образом, движение в скрещенных электрическом и магнитном полях
$\vec{E}\perp \vec{B}$ представляет собой наложение
а)~ларморовского вращения в плоскости, перпендикулярной~$\vec{B}$,
и б)~смещения (\term{дрейфа}) центра ларморовской
окружности в направлении, в перпендикулярном~$\vec{E}$ и~$\vec{B}$,
с постоянной скоростью
\begin{equation}
\eqmark{Edrift}
    \boxed{\vec{V}_{др} = \frac{\vec{E}\times \vec{B}}{B^2}.}
\end{equation}
Это явление называют \term{электрическим дрейфом} (см. рис. \figref{drift}).

Дрейф можно интерпретировать как увеличение радиуса кривизны циклотронной
траектории при движении по полю $\vec{E}$ и его уменьшение при движении
против поля, что и приводит к смещению центра траектории поперёк $\vec{E}$.

Примечательно, что скорость и направления электрического дрейфа
\important{не зависят от свойств частицы}: ни от её массы,
ни от величины или знака заряда.


\paragraph{Альтернативный вывод скорости дрейфа.}
Тот же результат можно получить, воспользовавшись формулой для
преобразования полей при смене системы отсчёта. В нерелятивистском
случае:
\[
\vec{E}' = \vec{E} + \vec{V}\times \vec{B},\qquad \vec{B}'=\vec{B}.
\]
Отметим, что эти соотношения есть ни что иное как условие инвариантности силы
Лоренца при смене системы отсчёта. Подберём скорость~$\vec{V}$ так, чтобы
электрическое поле $\vec{E}'$ в системе, движущейся с этой скоростью, обратилось
в нуль:
\[
 0 = \vec{E} + \vec{V}\times \vec{B}.
\]
Поскольку $\vec{E}\perp \vec{B}$, приходим к соотношению \eqref{Edrift}.
% \begin{equation}
%     \vec{V}_{др} = \frac{\vec{E}\times \vec{B}}{B^2}.
% \end{equation}
При переходе в эту систему останется только магнитное поле, поэтому
траектория частицы будет \important{ларморовской} (\important{циклотронной})
\important{окружностью}. Движение же самой системы будет как раз соответствовать
\important{дрейфу} центра этой окружности поперёк~$\vec{E}$ и~$\vec{B}$.

\begin{lab:note}
Заметим, что наши результаты получены в \important{нерелятивистском}
приближении. Для их применимости необходимо выполнение условия $V_{др}\ll c$,
то есть электрическое поле должно быть мало по сравнению с магнитным: $E\ll cB$.
Если $E>cB$, то во-первых, корректное рассмотрение возможно только с учётом
релятивизма, а во-вторых, движение не будет иметь характер дрейфа.
\end{lab:note}


\introsection{Ток в вакуумном диоде}

Вакуумный диод --- это откачанный до высокого вакуума сосуд,
в котором разность потенциалов подана расположенные в нём электроды:
катод и анод. Электрический ток в диоде представляет собой упорядоченное
движение свободных электронов, испускаемых катодом.  При этом,
в отличие от обычного проводника, электроны не испытывают сопротивления
своему движению. Еще одной характерной особенностью такой системы
является наличие пространственного заряда. Как следствие, для вакуумного
диода не применим закон Ома.

Явление испускания электронов поверхностью твёрдого тела или жидкости называется
\term{электронной эмиссией}. Для удаления электрона из твёрдого вещества в
вакуум необходимо совершить работу, называемую \term{работой выхода} $A_{вых}$
(у чистых металлов $A_{вых}\sim 1\;эВ$).
Один из механизмов эмиссии --- испускание электронов с поверхности сильно
нагретых тел (\term{термоэлектронная эмиссия}). Работа выхода при этом
совершается за счёт кинетической энергии электронов, которой они обладают
внутри тела.

Если создать электрическое поле вне металла, оно будет увлекать вышедшие
электроны и через вакуум потечёт электрический ток.
С повышением температуры поверхности экспоненциально быстро растёт доля частиц,
способных преодолеть потенциальный барьер $A_{вых}$, и следовательно, растёт
интенсивность эмиссии электронов. Это приводит к тому, что в пространстве
диода~--- особенно вблизи катода~--- накапливается отрицательный
объёмный заряд, экранирующий внешнее поле. Из-за этого результирующий ток
в диоде может оказаться существено меньше тока, который может обеспечить эмиссия
с катода. Такой режим работы диода называют \term{режимом объёмного заряда}.

При достижении определённого напряжения дальнейшее нарастание тока практически
прекращается~--- ток достигает предельного значения $I_{нас}$, называемого
\term{током насыщения}. Это обусловлено ограниченностью эмиссионной способности
катода~--- величина тока насыщения определяется количеством электронов,
которое способно выйти из поверхности катода в единицу времени.

\paragraph{<<Закон 3/2>> для вакуумного диода.}
Рассмотрим режим объёмного заряда в простейшем случае \important{плоского}
диода. Его электроды представимы в виде двух параллельных плоскостей,
между которыми задано напряжение $V$. Расстояние~$d$ между электродами много
меньше их площади. Направим ось~$x$ перпендикулярно к поверхности катода
в сторону анода, совместив начало координат с катодом. Задача
становится одномерной: все величины являются функциями только координаты~$x$.

Запишем для электрического поля \important{теорему Гаусса} в дифференциальной
форме:
\[
\frac{dE}{dx} = \frac{\rho}{\varepsilon_0},
\]
где~$\rho(x)$~--- \important{плотность электрического заряда}. По определению
потенциала электростатического поля имеем
\[
E = -\frac{d\varphi}{dx}.
\]
Отсюда находим, что $\varphi(x)$ удовлетворяет уравнению
\begin{equation}
    \eqmark{3.14}
    \frac{d^2\varphi}{dx^2}=-\frac{\rho}{\varepsilon_0}.
\end{equation}
Это частный (одномерный) случай \term{уравнения Пуассона}.

\important{Плотность тока} в диоде равна $j=\rho v$, где $v$~---
средняя скорость потока электронов. В стационаре заряд нигде не накапливается,
поэтому плотность тока всюду одинакова: $j=\const$. Из закона сохранения
энергии имеем:
\begin{equation*}
    \frac{mv^2}{2}=e\varphi.
\end{equation*}
Здесь мы выбрали начало отсчёта потенциала на катоде $\varphi(0)=0$, а также
\important{пренебрегли начальными (тепловыми) скоростями} ($v_0=0$),
с которыми вылетают электроны с поверхности катода. Это возможно, если
приложенное напряжение достаточно велико: $eV\gg mv_0^2/2$.

Исключив из полученных соотношений плотность электронов~$\rho$ и скорость~$v$,
приходим к уравнению
\begin{equation}
    \eqmark{3.15}
    \frac{d^2\varphi}{dx^2}=\sqrt{\frac{m}{2e\varphi}} \frac{j}{\varepsilon_0}
\end{equation}
с граничными условиями
\begin{equation*}
 \varphi(0)=0,\qquad \varphi(d)=V.
\end{equation*}

Для однозначного решения этого дифференциального уравнения необходимо ещё одно
граничное условие. В общем случае это должна быть связь между плотностью
тока~$j$ и электрическим полем на поверхности катода
$E_0 = -\left.\frac{d\varphi}{dx}\right|_{x=0}$,
которую однако в общем виде теоретически установить затруднительно.
Учтём, что в режиме объёмного заряда количество электронов, способных покинуть
катод из-за его нагрева, значительно превосходит ток в диоде.
Следовательно, \important{эмиссионная способность катода практически не
ограничена}. Поэтому, чтобы плотность тока оставалась конечной,
напряжённость электрического поля внутри катода должна быть мала: $E_0\to 0$.
Таким образом, получаем дополнительное граничное условие в виде
\begin{equation*}
    \left.\frac{d\varphi}{dx}\right|_{x = 0}=0.
\end{equation*}

Применяя к \eqref{3.15} стандартные методы интегрирования дифференциальных
уравнений, можно получить, что удовлетворяющим выбранным граничным условиям
решением является функция
\begin{equation*}
    \varphi = A x^{4/3},
\end{equation*}
где $A = \left(\frac{9}{4}\sqrt{\frac{m}{2e}}\frac{j}{\varepsilon_0}\right)^{2/3}$.
Подставляя условие $\varphi(d)=V$, найдём связь между током и напряжением~---
\important{вольт-амперную характеристику} вакуумного диода:
\begin{equation}
    I = \mathrm{const} \cdot V^{3/2}.
\end{equation}
Это так называемый <<\term{закон 3/2}>> \term{Чайлда--Ленгмюра}.


В реальной системе, как видно из проведённого вывода, <<закон~3/2>>
нарушается как при слишком малых напряжениях, когда нельзя пренебрегать
начальными тепловыми скоростями электронов, так и при слишком больших
напряжениях, когда диод переходит в режим насыщения. В~промежуточной
области закон хорошо подтверждается на опыте, в том числе и для
электродов неплоской геометрии.

\begin{lab:note}
Ещё одна область применимости теории Чайлд--Ленгмюровского диода ---
эксперименты со сверхсильными токами ($\gtrsim 1$~МА),
проводимые в рамках программы создания управляемого
термоядерного синтеза. При таких токах катод фактически взрывается, поэтому
вопрос о его эмиссионной способности не возникает. При этом однако нельзя
не учитывать влияние \emph{собственного магнитного поля} пучка электронов
на их движение: если циклотронный радиус в собственном поле окажется
меньше расстояния между пластинами, траектории электронов не смогут
оставаться прямыми и дальнейшее нарастание тока будет невозможно.
\end{lab:note}


\introsection{Движение носителей заряда в металлах и~полупроводниках}

Проводимость большинства твёрдых тел связана с движением электронов. Электроны
входят в состав атомов всех тел, однако одни тела не проводят электрический ток
(диэлектрики), а другие являются хорошими его проводниками. Причина различия
заключается в особенностях энергетического состояния внешних электронов атомов в
этих веществах.


\introsubsection{Зонная модель}

При объединении атомов в твёрдое тело (кристалл) внешние электроны теряют связь
со <<своими>> атомами и принадлежат \important{всему} кристаллу.
Каждому уровню энергии электрона \important{одиночного} атома в кристалле
соответствует \important{группа} близких уровней в кристалле,
<<сливающихся>> в непрерывную \term{зону}.
Число доступных состояний в зоне при <<слиянии>> остаётся неизменным~--- оно
равно числу мест на соответствующем атомном уровне,
умноженному на число атомов в кристалле. Оно определяет максимальное число
электронов, которое может <<поместиться>> в зоне в силу принципа
\important{запрета Паули}. Между зонами разрешённых состояний нет~---
эти области называют \term{запрещёнными зонами}.

\begin{wrapfigure}{o}{0.5\textwidth}
\centering
\pic{\linewidth}{Chapter_3/zones}
\caption{Структура состояний а)~металла, б)~полупроводника, в)~диэлектрика}
\end{wrapfigure}

Если одна из зон до конца заполнена электронами, а следующая~---
пуста, то под действием слабого внешнего электрического поля
электроны не могут изменить своё состояние, а значит и не могут
прийти в движение. Тогда вещество является \term{диэлектриком}.
Верхняя из заполненных зон называется \term{валентной зоной}.

Положение меняется, если в кристалле имеется зона, \important{частично}
заполненная электронами. В~этом случае внешнее электрическое поле может изменить
распределение электронов по уровням энергии и вызвать их упорядоченное движение.
Частично заполненная зона называется \term{зоной проводимости}.
Такая зона имеется у всех твёрдых проводников электрического тока;
в том числе её имеют все металлы.

Если ширина запрещённой зоны относительно невелика, тепловое движение
перебрасывает часть электронов из валентной зоны в свободную зону над ней~---
зону проводимости. При этом в зоне проводимости появляются электроны,
а в валентной зоне~--- свободные места~--- \term{дырки}.
Электроны в зоне проводимости и дырки валентной зоны участвуют в переносе
заряда. Такие вещества называются \term{полупроводниками}.
Проводимость полупроводников экспоненциально растёт с~повышением
температуры (так как вероятность преодолеть запрещенную зону
определяется распределением Больцмана $\propto e^{-\Delta E/\kB T}$).
Обычно к полупроводникам относят материалы с шириной запрещённой зоны
$\Delta E \lesssim 2$~эВ.

Таким образом, электроны в твёрдом теле можно приближённо разделить на три
почти несмещивающиеся подсистемы: 1)~электроны в заполненных валентных
зонах, не участвующие в переносе заряда; 2)~электроны в зоне проводимости,
которые могут свободно распространяться по твёрдому телу
% и, если их концентрация
% достаточно мала, могут быть с удовлетворительной точностью описаны моделью
% \important{идеального электронного газа};
3)~электроны в верхней части валентной
зоны, где есть небольшое количество вакантных мест.
Последние можно наглядно представить как электронную жидкость, в которой имеется
небольшое количество пузырьков, т.\,е. <<дырок>>. Под действием внешних сил
пузырёк ведёт себя как частица, которой можно приписать отрицательную массу
(например, пузырьки в бутылке газировки всплывают вверх). Традиционно всё же
массу дырки принимают положительной, $m_h>0$, но приписывают ей
заряд обратного знака $q_h=+e$.
% При малых концентрациях дырок для них также применяется модель
% \important{идеального газа}.
Если дырка и электрон проводимости окажутся в одной точке пространства,
они могут нейтрализовать друг друга~---
\term{рекомбинировать}. При этом на самом деле произойдёт переход
одного электрона из зоны проводимости в валентную зону.

Большинство чистых металлов обладает \term{электронным типом проводимости}.
Однако в ряде металлов (цинк, кадмий, бериллий и некоторые сплавы) зонная структура
такова, что в зоне проводимости слишком мало вакантных мест, поэтому
в них имеет место \term{дырочный тип проводимости}.
Для чистых полупроводников характерно одновременное наличие двух типов носителей.
В легированных полупроводниках (то есть при наличии примесей) может доминировать один из
типов носителей~--- <<электроны>> (\term{полупроводники $n$-типа}) или <<дырки>>
(\term{полупроводники $p$-типа}).



\introsubsection{Закон Ома}

При наложении внешнего электрического поля~$\vec{E}$ носители заряда начинают
ускоряться. Однако после некоторого <<свободного пробега>> происходит
взаимодействие с решёткой, частица теряет набранный импульс, и процесс
ускорения начинается заново. В результате баланса ускоряющей силы и трения
о решётку частица приобретает некоторую среднюю установившуюся скорость,
пропорциональную приложенной силе~$q\vec{E}$:
\begin{equation}
    \eqmark{3.22}
    \vec{u}_{уст}= b \cdot q\vec{E}.
\end{equation}
Коэффициент~$b$ называется \term{подвижностью} носителя тока.
В физике твёрдого тела подвижностью чаще называют коэффициент
пропорциональности между установившейся скоростью и приложенным
\important{полем}: $\mu = bq$.
% Знак в формуле~\eqref{3.22} совпадает со знаком заряда.

Отметим, что действие кристаллической решётки в среднем эквивалентно
постоянной силе трения, пропорциональной скорости:
\begin{equation}
    \eqmark{3.23}
    \vec{F}_{тр}=-\frac{\vec{u}}{b}.
\end{equation}
Это соотношение также можно принять за определение коэффициента подвижности.

При концентрации носителей $n$ плотность тока равна
\begin{equation*}
    \vec{j} = qn\vec{u} = q^2 n b \vec{E}.
\end{equation*}
Коэффициент пропорциональности между~$\vec{j}$ и~$\vec{E}$ называют
\term{проводимостью}. Соответствующую связь
\begin{equation}
    \eqmark{3.25}
    \boxed{\vec{j}=\lambda \vec{E}}
\end{equation}
называют \term{законом Ома в дифференциальной форме}.
Видно, что проводимость связана с подвижностью как
\begin{equation}
    \eqmark{3.26}
    \lambda = q^2 n b.
\end{equation}

В простейшей модели (\term{модель Друде--Лоренца}) движение носителей между столкновениями
считается свободным, а при столкновении они теряют весь набранный импульс.
Пусть частота столкновений носителя тока с решёткой равна $\nu$. Тогда
средняя сила трения есть $F_{тр}=mv \cdot \nu $, то есть подвижность
равна $b = \frac{1}{m\nu}$, и, соответственно, проводимость
\begin{equation}
    \eqmark{drude}
    \lambda = \frac{q^2 n}{m\nu}.
\end{equation}

Если имеется несколько типов носителей заряда, например, электроны
и дырки в полупроводнике, проводимость равна сумме вкладов от каждого из них
(поскольку полный ток есть сумма токов всех носителей):
\begin{equation}
    \eqmark{3.27}
    \lambda= e^2 (n_e b_e+n_h b_h),
\end{equation}
где соответственно $n_e$ и~$n_h$~--- концентрации электронов и дырок,
а~$b_e$ и~$b_h$~--- их подвижности.


\introsection{Эффект Холла и магнетосопротивление}

Во внешнем магнитном поле~$\vec{B}$ на заряды действует сила Лоренца
\begin{equation}
\eqmark{F-lorentz}
 \vec{F} = q\vec{E}+q\vec{u}\times \vec{B}.
\end{equation}
Эта сила вызывает движение носителей, направление которого
в общем случае не совпадает с~$\vec{E}$.
Действительно, либо траектории частиц будут либо искривляться,
либо, если геометрия этого не позволяет, возникнет дополнительное электрическое
поле, компенсирующее магнитную составляющую силы Лоренца.
Возникновение в образце, помещённом во внешнее магнитное поле,
поперечных току компонент электрического поля называют \term{эффектом Холла}.

Связь между электрическим полем $\vec{E}$ и плотностью тока $\vec{j}$ в
условиях эффекта Холла уже не может быть описана скалярным коэффициентом проводимости
$\lambda$. Тем не менее, закон Ома можно по-прежнему записать
в форме
\begin{equation}
\vec{j} = \hat{\lambda} \vec{E},
\end{equation}
если под $\hat{\lambda}$
понимать \term{тензор проводимости}. В~заданном базисе он представляется
матрицей~$3\times 3$:
\begin{equation}
    \vec{j} =\hat{\lambda}\vec{E}\equiv \left(
    \begin{matrix}
     \lambda_{xx} & \lambda_{xy} & \lambda_{xz}\\
     \lambda_{yx} & \lambda_{yy} & \lambda_{yz}\\
     \lambda_{zx} & \lambda_{zy} & \lambda_{zz}
    \end{matrix}
\right) \vec{E}.
\end{equation}
или
\begin{equation*}
    j_{i} = \sum_{k} \lambda_{ik} E_k,\quad \text{где~}i,\,k=\{x,y,z\}.
\end{equation*}
Тензорная связь между полем и током имеет место в общем случае, когда
проводящая среда не является \important{изотропной}.
В условиях эффекта Холла тензор проводимости становится \important{недиагональным}.


\begin{wrapfigure}{o}{0.5\textwidth}
\centering
    \pic{0.4\textwidth}{Chapter_3/Hall_forces}
    \caption{Силы, действующие на носитель заряда в проводящей среде
    при наличии магнитного поля}
    \figmark{Hall forces}
\end{wrapfigure}

Пусть система содержит носители только \important{одного сорта}
(например, электроны, как в большинстве металлов).
Рассмотрим сперва простейший случай плоской геометрии:
пусть ток течёт по оси~$x$, а магнитное поле направлено по оси $z$
(см. рис.~\figref{Hall forces}).
Магнитное поле действует на движущиеся заряды по оси $y$ с силой $F_y=-qu_xB_z$.
По построению ток течёт вдоль~$x$, что возможно,
если заряды в среде перераспределятся таким образом, чтобы полностью
скомпенсировать магнитную силу, создав в $y$-направлении электрическое поле
\begin{equation*}
E_y=u_x B_z=\frac{j_x}{nq} B_z,
\end{equation*}
называемое \term{холловским} (здесь $n$ --- концентрация носителей).
По оси $x$ носители будут двигаться так, как если бы магнитного поля не было:
$j_x=\lambda_0 E_x$ ($j_y=j_z=0$), где $\lambda_0 = e^2nb$~---
удельная проводимость среды в отсутствие~$B$.

Выразим общую связь между~$\vec{E}$ и~$\vec{j}$ для случая носителей одного сорта.
Ось $z$ по-прежнему направим вдоль магнитного поля~$\vec{B}$, а о
направлении~$\vec{E}$ и~$\vec{j}$ никаких предположений делать не будем.
При течении носителей с постоянной средней скоростью сила Лоренца
\eqref{F-lorentz} будет уравновешена трением со стороны среды \eqref{3.23}:
\begin{equation*}
    q(\vec{E}+\vec{u}\times \vec{B}) - \frac{\vec{u}}{b} =0.
\end{equation*}
С учётом введённых выше обозначений этот баланс сил можно переписать как
\begin{equation}
    \eqmark{ohm-hall}
    \vec{E} = \frac{\vec{j}}{\lambda_0} -
    \frac{1}{nq} \vec{j}\times\vec{B},
\end{equation}
Соотношение \eqref{ohm-hall} можно назвать \term{обобщённым законом Ома} при
наличии внешнего магнитного поля.

Записывая равенство \eqref{ohm-hall} по-компонентно
\[
E_x = \frac{j_x}{\lambda_0}  -  \frac{j_y B}{nq} ,\qquad
    E_y = \frac{j_y}{\lambda_0} +  \frac{j_xB}{nq} ,\qquad
    E_z = \frac{j_z}{\lambda_0},
\]
получим в явном виде
\term{тензор удельного сопротивления}~$\hat{\rho}$ (обратный к тензору
проводимости):
\begin{equation}
    \vec{E}=\hat{\rho}\vec{j}= \left(
    \begin{matrix}
        1 & -\beta & {0} \\
        \beta & 1 & {0} \\
        {0} &{0}& 1
    \end{matrix}
    \right)
    \frac{\vec{j}}{\lambda_0}.
    \eqmark{3.17}
\end{equation}
Здесь введено обозначение $\beta = \lambda_0 B/nq = q b B$~---
величина, называемая \term{параметром замагниченности}. Физический
смысл $\beta$~--- отношение эффективной длины пробега частиц~$l=mub$
к ларморовскому радиусу кривизны их траектории $r_B=mu/qB$.

Обращением матрицы \eqref{3.17} получим тензор проводимости
в условиях эффекта Холла:
\begin{equation}
    \hat{\lambda} = \hat{\rho}^{-1}=
    \frac{\lambda_0}{1+\beta^2}\left(
    \begin{matrix}
        {1} & \beta & {0} \\
        -\beta & 1 & {0} \\
        {0} &{0}& 1 \\
    \end{matrix}
    \right).
    \eqmark{3.18}
\end{equation}


\begin{lab:note}
Приведённый вывод описывает классическую модель движения носителей
в среде. При низких температурах  в сильных полях ($T\lesssim 1$~К, $B\gtrsim1$~Тл)
явление преобретает существенно квантовый характер
(\term{квантовый эффект Холла}). Особый интерес представляет
\term{дробный} квантовый эффект Холла, возникающий в сверхсильных полях,
в котором квазичастицы среды ведут себя так, будто имеют
дробный элементарный заряд.
\end{lab:note}


\introsubsection{Измерение проводимости в магнитном поле}

Существуют две основных и принципиально различных геометрии для исследования
зависимости проводимости среды от магнитного поля: геометрия \term{мостика
Холла} и геометрия \term{диска Корбино} (см. рис.~\figref{Geometries}).
Рассмотрим их подробнее.

\begin{figure}[h!]
\centering
    \pic{0.9\linewidth}{Chapter_3/2schemes}
    \caption{Две геометрии для исследования влияния магнитного поля на
проводящие свойства: мостик Холла (слева) и диск Корбино (справа).}
    \figmark{Geometries}
\end{figure}

\paragraph{Мостик Холла.}
В~данной геометрии (см. рис.~\figref{Geometries}, слева) ток вынуждают течь по
оси~$x$ вдоль плоской пластинки (ширина пластинки~$a$, толщина~$h$,
длина~$l$).
Сила Лоренца, действующая со стороны перпендикулярного
пластинки магнитного поля, прибивает носители заряда к краям образца,
создавая тем самым холловское поле, компенсирующее эту силу.
Поперечное напряжение между краями пластинки
(\term{холловское напряжение}) равно~$V_{\perp}=E_ya$,
где, согласно уравнению~\eqref{3.17},
\[
E_y=\rho_{yx}\cdot j_x=\frac{j_x B}{nq}.
\]
Плотность тока, текущего через образец, равна $j_x=I/ah$, где $I$~---
полный ток, $ah$~--- поперечное сечение.
Таким образом, для холловского напряжения имеем:
\begin{equation}
    V_{\perp}=\frac{B}{nqh}\cdot I =R_{\rm H}\cdot \frac{B}{h}\cdot I .
    \eqmark{3.19}
\end{equation}
Здесь мы ввели константу
\begin{equation}
    R_{\rm H}= \frac{V_{\perp}h}{IB} = \frac{1}{nq},
    \eqmark{HallConstant}
\end{equation}
которую принято называть \term{постоянной Холла}. В~таблице в Приложении даны
постоянные Холла для различных металлов. Для полупроводников постоянные Холла
сильно зависят от наличия малых концентраций примеси и температуры.
Знак постоянной Холла определяется знаком заряда носителей.

Продольная напряжённость электрического поля равна
\[E_x = \rho_{xx}\cdot j_x = j_x/\lambda_0.\]
Поэтому падение напряжения \important{вдоль} пластинки $V_{\parallel}=E_x l$
определяется просто законом Ома:
\begin{equation}
    V_{\parallel}= I R_0,
    \eqmark{3.20}
\end{equation}
где $R_0 = \frac{1}{\lambda_0} \frac{l}{ah}$~--- омическое сопротивление
образца при протекании тока по $x$.

Интересно отметить, что несмотря на то, что тензор проводимости~\eqref{3.18}
явно зависит от~$B$, сопротивление образца в данной геометрии от магнитного поля
\important{не зависит}.

\paragraph{Диск Корбино.}
В~геометрии Корбино (см. рис.~\figref{Geometries}, справа) электрическое поле
направлено по радиусу системы. В~перпендикулярном диску магнитном поле ток
вынужден протекать под углом к электрическому полю, то есть линии тока
представляют собой \important{спирали}. Дополнительное (холловское) электрическое
поле при этом не возникает.

Ввиду симметрии системы вклад в полный ток даёт только \important{радиальная}
компонента плотности тока $j_r=\lambda_{r} E_r$. Полный ток равен
$I=j_r \cdot 2\pi r h$, где $r$ --- радиус диска, $h$ --- толщина.
Если система \important{однокомпонентная}, проводимость в радиальном
направлении~$\lambda_r$ соответствует компоненте~$\lambda_{xx}$ тензора
\eqref{3.18}:
\begin{equation}
\lambda_r = \frac{\lambda_0}{1+\beta^2},
\end{equation}
где $\beta = q b B$ --- введённый выше параметр замагниченности.
Напряжение между центром и краем диска равно
\begin{equation*}
V=\int\limits_{r_1}^{r_2}E_r dr=
\frac{1}{\lambda_r}\int\limits_{r_1}^{r_2} \frac{I}{2\pi r h}dr =
\frac{\lambda_r}{\lambda_0}R_0 I,
\end{equation*}
где $R_0 = \frac{1}{\lambda_0 2\pi r h} \ln \frac{r_2}{r_1}$ есть
сопротивление диска в отсутствие магнитного поля. Поэтому закон Ома
в геометрии Корбино можно записать как
\begin{equation}
    \eqmark{MagnetoSoprot}
    V=I R,\qquad \text{где~}R = R_0 (1+\beta^2).
\end{equation}

Таким образом, в данной геометрии появляется зависимость сопротивления
образца от магнитного поля. Причина~--- в геометрии
системы: магнитное поле искривляет линии тока, делая их длиннее.
Такой эффект называют \term{геометрическим магнетосопротивлением}.

\begin{lab:note}
Отметим одну экспериментальную особенность обсуждаемых систем. Измерения в
геометрии мостика Холла представляют собой \important{четырехконтактные}
измерения, то есть два контакта используются для задания тока через образец, а с
двух контактов снимается падение напряжения. Вольтметр обладает
большим сопротивлением (то есть ток через него практически не
течёт), поэтому измеряемое падение напряжения не зависит от свойств
контактов, а определяется только свойствами материала.

Измерения же на диске Корбино проводятся по \important{двухточечной}
схеме, то есть сопротивление образца в ней суммируется с сопротивлениями
контактов. Поэтому исключительно важно создать низкоомные контакты к образцу,
сопротивлением которых можно пренебречь. Для наблюдения этого
магнетосопротивления выбирают систему с большой подвижнотью носителей
(как правило, это полупроводник с низкой эффективной массой электронов,
например InSb).
\end{lab:note}


\introsubsection{Магнетосопротивление}
Зависимость сопротивления образца от величины магнитного поля называют
\term{магнетосопротивлением}.

На примере мостика Холла мы увидели, что для \important{изотропных} веществ с
\important{одним} типом носителя эффект магнетосопротивления
\important{отсутствует}. Для таких веществ зависимость $R(B)$ может проявляться
только в силу геометрических эффектов, как в примере с диском Корбино.

В~общем случае магнетосопротивление материалов может быть отлично от нуля
и в схеме Холла. Это имеет место, если
\important{диагональные} компоненты тензора сопротивления $\hat{\rho}$
зависят от магнитного поля. Этому могут служить следующие причины:
\begin{enumerate}
    \item Система может быть \important{анизотропной} и без магнитного поля, то есть в разных
направлениях ($x,y,z$) токопроводящие свойства различны.

\item Система может быть \important{многокомпонентной}. Например, в
полупроводниках часто одновременно существуют электроны и дырки, концентрации
($n_e$ и~$n_h$) и подвижности ($b_e$ и~$b_h$) которых в общем случае
различаются.
% Тогда полный тензор проводимости будет суммой тензоров проводимости двух
% компонент вида \eqref{3.18}. Обращением тензора проводимости в пределе малых
% магнитных полей можно показать, что холловское сопротивление двухкомпонентной
% системы в полупроводнике равно:
% \begin{equation}
%     R_{xy}\equiv \frac{V_{xy}}{I}=\frac{nb_n^2-pb_h^2}{eh(nb_n+pb_h)^2}B
%     \eqmark{3.21}
% \end{equation}

\item Существуют \important{квантовые эффекты} в проводимости, которые приводят
к тому, что \important{подвижность} зависит от магнитного поля. Например, если
проводящий материал является ферромагнетиком, то с ростом поля он
намагничивается, количество доменов уменьшается, а доменные стенки являются
причиной сильного рассеяния, то есть уменьшают подвижность.
\end{enumerate}

\begin{lab:example}
Рассмотрим простейший пример многокомпонентной системы, в которой возникает
эффект магнетосопротивления. Выкладки даже в двух компонентной
системе резко усложняются, поэтому мы приведём лишь схему вывода.

Пусть в среде имеется два типа носителей --- электроны и дырки. Система
уравнений, которым подчиняются носители, по-прежнему представляет собой баланс
сил Лоренца и трения (см. \eqref{ohm-hall}) для каждого компонента в
отдельности:
\begin{equation}
    \eqmark{two-component}
    \begin{aligned}
-e(\vec{E}+\vec{u}_e\times \vec{B}) - \frac{\vec{u}_e}{b_e} &=0,\\
e(\vec{E}+\vec{u}_h\times \vec{B}) - \frac{\vec{u}_h}{b_h} &=0.
\end{aligned}
\end{equation}
Повторяя вывод \eqref{3.18} для каждого сорта носителей отдельно, можно
получить связь напряжённостью поля и плотностями тока этих носителей:
\begin{equation}
    \vec{j}_e = -en_e \vec{u}_e = \hat{\lambda}_e \vec{E},\qquad
    \vec{j}_h = en_h \vec{u}_h = \hat{\lambda}_h \vec{E}.
\end{equation}
Тензоры проводимости будут определяться \eqref{3.18}, где нужно проставить
индексы, соответствующие сорту носителя.
Полная плотность тока есть
\[
\vec{j} = \vec{j}_e + \vec{j}_h = \hat{\lambda}_e \vec{E} +
\hat{\lambda}_h \vec{E},
\]
поэтому тензор проводимости среды равен
\[
\hat{\lambda} = \hat{\lambda}_e + \hat{\lambda}_h.
\]
Обращая $\hat{\lambda}$, можно получить тензор удельного сопротивления:
$\hat{\rho}=
\left(\hat{\rho}_e^{-1}+\hat{\rho}_i^{-1}\right)^{-1}$.

Для справки приведём ответы.
% Диагональная компонента тензора сопротивления,
% отвечающая за эффект \important{магнетосопротивления}, равна
% \[
% \rho_{xx} = \frac{\lambda_h (1+\beta_e^2) + \lambda_e (1+\beta_h^2)}%
% {(\beta_h\lambda_e-\beta_e\lambda_h)^2 + (\lambda_e+\lambda_h)^2}.
% \]
% <<Косая>> компонента тензора, отвечающая \important{эффекту Холла}:
% \[
% \rho_{yx} = \frac{\beta_h\lambda_h (1+\beta_e^2) -
% \beta_e\lambda_e (1+\beta_h^2)}%
% {(\beta_h\lambda_e-\beta_e\lambda_h)^2 + (\lambda_e+\lambda_h)^2}.
% \]
В опыте с мостиком Холла продольная компонента электрического поля равна
\[
E_x = \frac{(\beta_h^2+1) \lambda_e+ (\beta_e^2 +1)\lambda_h}%
{ (\beta_e \lambda_h-\beta_h \lambda_e)^2 + (\lambda_e+\lambda_h)^2}
j.
\]
Видно, что удельное сопротивление в этом направлении сложным образом
зависит от~$B$, то есть имеет место эффект \important{магнетосопротивления}.
Однако в слабых полях ($\beta \ll 1$) эффект довольно мал (поправка
квадратична по~$\beta$). При $B=0$ имеем просто
$E_x = j/(\lambda_e+\lambda_h)$.

Поперечная компонента (\important{холловское поле}) также имеет сложную
зависимость от $\beta$. Приведём ответ для константы Холла
в пределе малой замагниченности ($\beta \ll 1$):
% \[
% E_y = \frac{\beta_e\beta_h(\beta_e\lambda_h-\beta_h\lambda_e)+
%     \beta_h \lambda_h -\beta_e \lambda_e}%
% {(\beta_e \lambda_h - \beta_h \lambda_e)^2 + (\lambda_e+\lambda_h)^2} j.
% \]
\begin{equation}
    \eqmark{3.21}
    R_{\rm H} = \frac{b_h^2 n_h - b_e^2 n_e}%
{e(b_en_e+b_hn_h)^2}.
\end{equation}
\end{lab:example}


\begin{lab:literature}
    \item \SivuhinIII \S\S~86, 95, 98, 100.
    \item \KingLokOlh \S\S~8.1--8.3.
\end{lab:literature}

