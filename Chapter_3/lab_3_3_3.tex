\newpage
\lab{Опыт Милликена.}

\aim{измерение элементарного заряда методом масляных капель.}

\equip{плоский конденсатор в защитном кожухе, осветитель, измерительный микроскоп, электростатический вольтметр,
электронный секундомер, переключатель напряжения, пульверизатор с маслом.}


Идея опыта очень проста. Если элементарный заряд действительно существует, то заряд $q$ любого тела может принимать
только дискретную последовательность значений:
\begin{equation}
	q=0,\,\pm e,\,\pm 2e,\,\pm 3e,\,\ldots\pm ne,\,\ldots,
	\eqmark{3.3.1}
\end{equation}
где $e$~--- элементарный заряд. В предлагаемом опыте измеряется заряд небольших капелек масла, несущих всего несколько элементарных зарядов. Сравнивая между собой заряды капель, можно убедиться, что все они по модулю кратны одному и тому же числу, которое равно, очевидно, элементарному заряду.

Для измерения заряда капель будем исследовать их движение в вертикальном электрическом поле.

Движение заряженной капли в электрическом поле зависит как от электрических сил, так и от массы капли. Масса капли может быть определена по скорости её падения в отсутствие поля.

Рассмотрим свободное падение капли. Уравнение её движения при падении имеет вид
\begin{equation}
	m\frac{dv}{dt}=mg-F_\text{тр},
	\eqmark{3.3.2}
\end{equation}
где $m$~--- масса капли, $v$ --- её скорость, $g$~--- ускорение свободного падения, а $F_\text{тр}$~--- сила вязкого трения капли в воздухе, которая для сферической капли определяется формулой Стокса:
\begin{equation}
	F_\text{тр}=6\pi\eta rv=kv.
	\eqmark{3.3.3}
\end{equation}

Здесь $r$~--- радиус капли, $\eta$~--- коэффициент вязкости воздуха, $k=6\pi\eta r$. Подставляя \eqref{3.3.3} в \eqref{3.3.2}, получим
\begin{equation}
	m\frac{dv}{dt}=mg -kv.
	\eqmark{3.3.4}
\end{equation}

Можно убедиться, что при нулевой начальной скорости решение этого уравнения имеет вид
\begin{equation}
	v=\frac{mg }{k}\left(1-e^{-kt/m}\right).
	\eqmark{3.3.5}
\end{equation}

Установившееся значение скорости равно
\begin{equation}
	v_\text{уст}=\frac{mg }{k}=\frac{\frac 43 \pi\rho r^3g }{6\pi\eta r}=\frac29\frac{\rho}{\eta}g r^2,
	\eqmark{3.3.6}
\end{equation}
где $\rho$~--- плотность масла. Заметим, что \eqref{3.3.6} может быть немедленно получено из \eqref{3.3.4}, если положить $dv/dt=0$.

Как следует из \eqref{3.3.5}, установление скорости происходит с постоянной времени
\begin{equation}
	\tau=\frac{m}{k}=\frac 29 \frac{\rho}{\eta}r^2.
	\eqmark{3.3.7}
\end{equation}
Время установления скорости, таким образом, быстро падает с уменьшением радиуса капли $r$. Для мелких капель оно столь мало, что движение капли всегда можно считать равномерным. Выражение \eqref{3.3.6} в этом случае позволяет определить радиус капли, зная скорость её падения. Обозначая через $h$ путь, пройденный каплей за время $t_0$, найдём
\begin{equation}
	r=\sqrt{\frac{9\eta h}{2\rho g t_0}}.
	\eqmark{3.3.8}
\end{equation}

Рассмотрим теперь движение капли при наличии электрического поля плоского конденсатора, пластины которого расположены горизонтально. Напряжённость поля $E$ в конденсаторе равна
\begin{equation}
	E=\frac{V}{l},
	\eqmark{3.3.9}
\end{equation}
где $l$~--- расстояние между пластинами, а $V$~--- разность потенциалов между ними.

Нас будет интересовать случай, когда под действием электрического поля капля поднимается. Уравнение движения при этом примет вид
\begin{equation}
	m\frac{dv}{dt}=\frac{qV}{l}-mg -kv,
	\eqmark{3.3.10}
\end{equation}
где $q$~--- заряд капли. Появление в правой части постоянного слагаемого не изменяет постоянной времени $\tau$. Для
определения установившейся скорости мы можем снова положить левую часть \eqref{3.3.10} равной нулю.

Измерим время $t$ подъёма капли на начальную высоту. Используя равенства \eqref{3.3.4}, \eqref{3.3.8} и \eqref{3.3.10}, найдём, что заряд капли равен
\begin{equation}
	q=9\pi\sqrt{\frac{2\eta^3 h^3}{g \rho}}\cdot\frac{l(t_0+t)}{Vt_0^{3/2} t}.
	\eqmark{3.3.11}
\end{equation}

Вывод формулы \eqref{3.3.11} предоставляем читателю.

\experiment Схема установки представлена на рис.~\figref{fig3.3.1}. Масло разбрызгивается пульверизатором. Капли масла попадают в конденсатор $C$ через небольшое отверстие в верхней пластине. При этом часть из них вследствие трения о воздух приобретает случайный по абсолютной величине и знаку электрический заряд.

Напряжение на пластины подаётся с регулируемого выпрямителя и измеряется вольтметром $V$. Ключ $K$ позволяет менять направление поля в конденсаторе, чтобы можно было работать  как с отрицательно, так и с положительно заряженными каплями. При размыкании ключа $K$ конденсатор разряжается через дополнительное сопротивление $R\approx 10$~МОм.
\begin{figure}[h!]
	\pic{0.9\textwidth}{3_3_1}
	\caption{Схема экспериментальной установки для измерения заряда электрона}
	\figmark{fig3.3.1}
\end{figure}
Время отсчитывается по электронному секундомеру.

Естественно, что слабые электрические силы, действующие на каплю, несущую всего один или несколько электронных зарядов, способны существенно изменить её движение только в том случае, если сама она очень мала. Опыт производится поэтому с мелкими каплями, наблюдение за которыми возможно только с помощью микроскопа.

В фокальной плоскости окуляра измерительного микроскопа $M$ виден ряд горизонтальных линий, расстояние между которыми было предварительно определено с помощью объектного микрометра. Для облегчения процесса измерений микроскоп может снабжаться камерой с выводом изображения на дисплей ПК. Наблюдая за перемещением капли между линиями, нетрудно определить путь, пройденный каплей. Время $t_0$ свободного падения капли от одной выбранной линии до другой и время $t$ её обратного подъёма, происходящего под действием сил электрического поля, измеряется электронным секундомером.

Из постановки опыта очевидно, что дискретность заряда может быть обнаружена лишь в том случае, если ошибка $\delta q$ в измерении заряда капли существенно меньше абсолютной величины заряда электрона $e$. Допустимая относительная ошибка опыта $\delta q/q$ должна быть поэтому много меньше $e/q=1/n$, где $n$~--- заряд капли, выраженный в числе зарядов электрона. Этому условию тем легче удовлетворить, чем меньше число $n$. В нашем случае трудно определить $q$ с точностью лучше 5\%. Заряд капли должен поэтому быть существенно меньше 20 зарядов электрона~--- лучше всего, если он не превосходит пяти электронных зарядов.

Из всех величин, входящих в формулу \eqref{3.3.11}, на опыте измеряются только $t_0$, $t$, и $V$. От точности определения этих величин зависит в основном ошибка измерения $q$. Из формулы \eqref{3.3.11} можно найти
\begin{equation}
	\frac{\sigma_q}{q}=\sqrt{\frac{\sigma^2_V}{V^2}+\frac{\sigma^2_t t_0^2}{t^2(t_0+t)^2}+
	\frac{\sigma^2_{t_0}}{4t^2_0}\left(\frac{3t+t_0} {t+t_0}\right)^2}.
	\eqmark{3.3.12}
\end{equation}

При $t\approx t_0$ эта формула приобретает вид
\begin{equation}
	\frac{\sigma_q}{q}=\sqrt{\frac{\sigma^2_V}{V^2}+ \frac{\sigma^2_t}{4t^2_0}+ \frac{\sigma^2_{t_0}}{t^2_0}}.
	\eqmark{3.3.13}
\end{equation}

В условиях нашей работы наибольшее влияние на точность эксперимента оказывают два последних стоящих под корнем члена. Ошибка измерения времени $t_0$ и $t$ при визуальном наблюдении капель не может быть сделана меньше 0,1 -- 0,2 секунды. Погрешность в измерении $q$ будет поэтому тем меньше, чем большие значения принимают $t_0$ и $t$. Для увеличения $t_0$ и $t$ можно было бы увеличить расстояние, проходимое каплями, но это сильно усложнило бы экспериментальную установку. Удобнее идти в другом направлении~--- работать с медленно движущимися каплями, т.е. с каплями малого веса. Время падения $t_0$ таких капель достаточно велико. Чтобы время подъёма $t$ было также достаточно большим, нужно использовать не очень большие разности потенциалов $V$.

Заметим, что выбор слишком маленьких капель приводит к снижению точности измерений. Броуновское движение малых капель оказывает существенное влияние на их движение и способно заметно исказить картину их падения и подъёма. Маленькие капли могут испаряться, так что их размеры во время наблюдения могут уменьшаться. При малых скоростях движения делаются особенно опасными конвекционные потоки воздуха, которые возникают при неоднородном нагреве установки (происходящем, например, от осветителя камеры). Практически в наших условиях удобно выбирать $t_0\approx t\approx 10$ -- 30 секунд.

Для капель очень малого размера формула Стокса не вполне применима. Использование формулы Стокса без поправок, впрочем, в наших условиях приводит к искажению значений $q$ и $e$ не более чем на 10\% и почти не мешает обнаружению дискретности электрического заряда. Мы рекомендуем поэтому не вводить в формулу никаких поправок.

\begin{lab:task}

В работе предлагается по измерениям времени свободного падения заряженных капель и времени их подъёма в электрическом поле определить заряд электрона.

\begin{enumerate}

\item{Перед началом работы оцените с помощью формулы \eqref{3.3.11} величину напряжения $V$, которое нужно для подъёма капель, несущих от 1 до 5 зарядов электрона на высоту $h=1$~мм, задав $t_0\approx t=20$~с. Если для подъёма капель потребуются меньшие напряжения, то соответствующие капли слишком сильно заряжены и для эксперимента непригодны.

При вычислениях потребуются значения некоторых величин: расстояние между пластинами $l=0,725$~см; плотность масла
$\rho=0,898$~г/см$^3$; коэффициент внутреннего трения воздуха $\eta=1,83\cdot 10^{-4}$~Пуаз~(СГС) или $1,83\cdot 10^{-5}$~Па$\cdot$с (СИ).}

\item{Включите осветитель. При этом падающий в камеру свет направлен под углом к оси микроскопа и в объектив не попадает. Поле зрения микроскопа остаётся поэтому тёмным. Капли масла рассеивают свет и кажутся светящимися точками на темном фоне.

Не включая электрическое поле \important{слегка} надавите на грушу пульверизатора  и наблюдайте за движением облачка масляных капель в поле зрения микроскопа (изображение перевёрнуто).}

\item{Настройте окуляр микроскопа на резкое изображение делений окулярной шкалы. Затем сфокусируйте объектив на появившиеся в рабочем пространстве капли.}

\item{Наблюдая за движением капель, следует выбирать капли, время падения которых на $h=1$~мм лежит в пределах
10 -- 30 секунд, и научиться отличать их от более крупных, непригодных для работы. Цена деления окулярной шкалы указана на установке.

В случае отсутствия в составе установки ПК с камерой регулировкой и коммутацией напряжения занята одна рука наблюдателя. Вторая рука управляет секундомером. Запись результатов измерений ($t_0$, $t$ и $V$) ведёт второй экспериментатор. Наблюдатель быстро устаёт, поэтому рекомендуется периодически меняться местами. В случае наличия ПК указанные трудности отсутствуют, и работа может быть выполнена одним экспериментатором.

Для уменьшения ошибок в определении $t_0$ и $t$ нужно для пуска и остановки секундомера использовать один и тот же признак~--- всегда нажимать головку секундомера либо в тот момент, когда капля скрывается за линией шкалы, либо, наоборот, когда она появляется из-за линии. Рекомендуется следить за каплей, не отрываясь от окуляра микроскопа, так как в противном случае легко её потерять из виду и весь эксперимент придётся повторить.}

\item{В начале опыта следует позволить капелькам свободно падать \mbox{5 -- 10}~секунд при выключенном электрическом поле для того, чтобы наиболее крупные капли успели упасть на нижнюю пластину.

Из оставшихся в поле зрения капель выберите одну и произведите с ней серию измерений, наблюдая её падение под действием силы тяжести и подъём под действием электрического поля. Серия должна состоять из 5 -- 10 измерений $t_0$ и такого же числа измерений $t$ для одной капли.}

\item{Необходимо проделать не менее 15 таких серий измерений (для 15~различных капель), каждый раз регистрируя величину $V$. При этом нужно иметь в виду, что заряд капли может измениться во время наблюдений; в последнем случае для одной капли получится несколько значений $q$.

Изменение заряда капли может произойти при её подъёме в электрическом поле. Вычисленное с помощью \eqref{3.3.11} значение заряда будет в этом случае соответствовать некоторому среднему из величины заряда капли в начале и в конце опыта. Соответствующий результат непригоден для обработки и только запутывает опыт. Нужно поэтому стараться вовремя отбросить все случаи, когда перезарядка капли произошла во время её подъёма. Это можно сделать, внимательно наблюдая за движением капли и отбрасывая опыты, в которых капля изменила скорость подъёма во время измерения.}

\item{Для оценки точности измерений <<подвесьте>> одну из капель в электрическом поле. Определите соответствующее напряжение, отключите его  и измерьте время падения капли на расстояние 2 -- 3-х делений шкалы. Поменяв полярность напряжения, верните каплю на прежнее место и снова подвесьте её. Снова запишите напряжение. Повторите  процедуру  для одной капли несколько раз  и на месте оцените из этого опыта заряд капли по формуле \eqref{3.3.11}, полагая время подъёма $t=\infty$. По разбросу результатов ($\Delta V$ и  $\Delta t$) оцените точность измерения заряда этой капли.}

\end{enumerate}

\tasksection{Обработка результатов}

\begin{enumerate}

\item{Для всех исследованных капель рассчитайте значения $q$, отложите их на горизонтальной числовой оси и найдите для них общий наибольший делитель. Этот наибольший делитель, вообще говоря, может оказаться равным $e$, $2e$, $3e$ и т.~д. Однако, чем больше значений $q$ было измерено на опыте, тем меньше вероятность получить в качестве делителя число, отличное от $e$. Найденное значение $e$ приведите в системе единиц СИ и в системе СГС.}

\item{Оцените время релаксации $\tau=v_\text{уст}/g$ и расстояние $s$, которое прошла бы капля за это время с установившейся скоростью:
\begin{equation*}
	s=v_\text{уст}\tau=\frac{1}{g}\left(\frac{h}{t_0}\right)^2.
\end{equation*}}

\end{enumerate}

\end{lab:task}

\begin{lab:questions}

\item{Почему не следует выбирать капли слишком большого и слишком маленького размера?}

\item{Какие напряжения соответствуют оптимальным условиям опыта? Приведите расчёты.}

\item{Нарисуйте график зависимости скорости капли в поле силы тяжести от времени и укажите на нём время и путь релаксации.}

\item{Зная параметры установки, оцените ёмкость конденсатора~$C$ и время его разрядки через сопротивление~$R$ (площадь пластин ${\approx} 20$~см$^2$).}

\item{$^*$ Какие ещё способы измерения заряда электрона вам известны?}

\end{lab:questions}

\begin{lab:literature}

\item{ \emph{Сивухин Д.В.} Общий курс физики. Т.~III. Электричество. --- М.:Физматлит, 2015. Гл.~V, \S~90.}

\item{ \emph{Калашников С.Г.} Электричество.~--- М.: Физматлит, 2003. Гл.~XVII, \S~178.}

\end{lab:literature}