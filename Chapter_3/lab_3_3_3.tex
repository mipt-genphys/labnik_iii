\lab{Опыт Милликена}

\aim{измерение элементарного заряда методом масляных капель.}

\equip{плоский конденсатор в защитном кожухе, осветитель, измерительный
микроскоп, электростатический вольтметр,
секундомер, переключатель напряжения, пульверизатор с маслом.}

Если элементарный заряд действительно существует, то заряд~$q$ любого 
тела может принимать только дискретную последовательность значений:
\begin{equation}
q=0,\,\pm e,\,\pm 2e,\,\pm 3e,\,\ldots\pm ne,\,\ldots,
\eqmark{3.3.1}
\end{equation}
где~$e$~--- элементарный заряд. В предлагаемом опыте измеряется заряд небольших
капелек масла, несущих всего несколько элементарных зарядов. Сравнивая между
собой заряды капель, можно убедиться, что все они по модулю кратны одной
и той же величине $e$.

Измерение заряда капель осуществляется по их движению в воздухе 
под действием силы тяжести и вертикального электрического поля. 
В~плоский конденсатор с горизонтальными пластинами через
отверстие в верхней пластине впрыскиваются мелкие капельки масла, 
получаемые с помощью специального распылителя. 
На пластины конденсатора подаётся постоянное напряжение 
(порядка нескольких киловольт), которое можно изменять в ходе опыта.

При распылении капельки масла вследствие трения о воздух приобретают
случайный по величине и знаку электрический заряд. Попадая в конденсатор,
капельки масла движутся в воздухе, опускаясь под действием силы тяжести или
поднимаясь под действием электрического поля. Наблюдая капли с помощью
микроскопа, можно измерить времена опускания и подъёма капли на фиксированное
расстояние, что позволяет определить заряд капли.

%Разумеется, дискретность заряда капель может быть обнаружена только в
%том случае, если абсолютная величина погрешности в измерении заряда капли 
%будет существенно меньше самого элементарного заряда. 
%В~опытах Милликена необходимая точность вполне может быть обеспечена 
%в условиях учебного лабораторного практикума.

%Движение заряженной капли в электрическом поле зависит как от электрических сил,
%так и от массы капли. Масса капли может быть определена по скорости её падения в
%отсутствие поля.

\labsubsection{Уравнение движения капли}
Рассмотрим свободное падение капли. Второй закон Ньютона для неё имеет вид
\begin{equation}
	m\frac{dv}{dt}=mg-F_\text{тр},
	\eqmark{3.3.2}
\end{equation}
где~$m$~--- масса капли, $v$~--- её скорость, $g$~--- ускорение свободного
падения, а~$F_\text{тр}$~--- сила вязкого трения капли в воздухе.
При малых скоростях сила вязкого трения для сферической капли определяется 
формулой Стокса:
\begin{equation}
	F_\text{тр}=6\pi\eta rv=kv,
	\eqmark{3.3.3}
\end{equation}
где~$r$~--- радиус капли, $\eta$~--- коэффициент вязкости воздуха, 
$k=6\pi\eta r$. 

Подставляя \eqref{3.3.3} в \eqref{3.3.2}, и интегрируя уравнение
при нулевой начальной скорости, найдём зависимость скорости падения от времени:
\begin{equation}
	v=v_{\infty}\left(1-e^{-kt/m}\right).
	\eqmark{3.3.5}
\end{equation}
Здесь $v_{\infty}$ --- установившаяся скорость падения, равная
\begin{equation}
	v_{\infty}=\frac{mg}{k}=
    \frac{\frac 43 \pi\rho r^3g }{6\pi\eta r}= \frac29\frac{\rho}{\eta}g r^2.
	\eqmark{3.3.6}
\end{equation}
где $\rho$~--- плотность масла.
%Заметим, что \eqref{3.3.6} может быть немедленно
%получено из \eqref{3.3.4}, если положить $dv/dt=0$.

Как следует из \eqref{3.3.5}, установление скорости происходит 
за характерное время
\begin{equation}
	\tau=\frac{m}{k}= \frac{v_{\infty}}{g} = \frac 29 \frac{\rho}{\eta}r^2.
	\eqmark{3.3.7}
\end{equation}
Время установления скорости, таким образом, быстро падает с уменьшением радиуса
капли~$r$. Для мелких капель оно столь мал\'{о}, что движение капли всегда можно
считать равномерным со скоростью $v_{\infty}$. 

Выражение \eqref{3.3.6} в этом случае позволяет определить
радиус капли, зная время её падения. Обозначая через~$h\approx v_{\infty}t$ 
путь, пройденный каплей за время~$t$, найдём
\begin{equation}
	r=\sqrt{\frac{9\eta h}{2\rho g t}}.
	\eqmark{3.3.8}
\end{equation}

Рассмотрим теперь движение капли при наличии электрического поля плоского
конденсатора. Напряжённость поля в конденсаторе равна
$E=U/l$,
где~$l$~--- расстояние между пластинами, $U$~--- разность потенциалов между
ними. Если капля под действием электрического поля поднимается,
то уравнение движения примет вид
\begin{equation}
	m\frac{dv}{dt}=\frac{qU}{l}-mg -kv,
	\eqmark{3.3.10}
\end{equation}
где~$q$~--- заряд капли. Дополнительная константа в правой части не
изменяет постоянной времени~$\tau=k/m$. 
Новую установившуюся скорость можно найти, положив левую часть 
\eqref{3.3.10} равной нулю:
\begin{equation}
\eqmark{vinfE}
v_{\infty}' = \frac{qU}{kl} - v_{\infty}.
\end{equation}

Пусть $t'=h/v_{\infty}'$~--- время подъёма капли на начальную высоту. 
Используя равенства
\eqref{3.3.3}, \eqref{3.3.8} и \eqref{vinfE}, получим 
окончательную расчётную формулу для заряда капли:
\begin{equation}
	q=9\pi\sqrt{\frac{2\eta^3 h^3}{g \rho}}\cdot\frac{l(t+t')}{Ut^{3/2} t'}.
	\eqmark{3.3.11}
\end{equation}
Вывод формулы \eqref{3.3.11} рекомендуем читателю провести самостоятельно.

\experiment Схема установки представлена на рис.~\figref{Scheme}. Масло
разбрызгивается пульверизатором. Капли масла попадают в конденсатор~$C$ через
небольшое отверстие в верхней пластине. При этом часть из них вследствие трения
о воздух приобретает случайный по абсолютной величине и знаку электрический
заряд.

Напряжение на пластины подаётся от регулируемого выпрямителя и измеряется
вольтметром~$V$. Ключ~К позволяет менять направление поля в конденсаторе,
чтобы можно было работать  как с отрицательно, так и с положительно заряженными
каплями. При размыкании ключа~К конденсатор разряжается через дополнительное
сопротивление $R\approx 10$~МОм.
\begin{figure}[h!]
    \centering
	\pic{8.5cm}{Chapter_3/3_3_1}
	\caption{Схема экспериментальной установки}
	\figmark{Scheme}
\end{figure}

Естественно, что слабые электрические силы, действующие на каплю, несущую всего
один или несколько электронных зарядов, способны существенно изменить её
движение только в том случае, если сама она очень мала. Поэтому опыт производится
с мелкими каплями, наблюдение за которыми возможно только с помощью
микроскопа.

В фокальной плоскости окуляра измерительного микроскопа виден ряд
горизонтальных линий, расстояние между которыми было предварительно определено с
помощью объектного микрометра. Для облегчения процесса измерений микроскоп может
снабжаться камерой с выводом изображения на монитор ПК. Наблюдая за перемещением
капли между линиями, можно определить пройденный каплей путь. Время~$t$
свободного падения капли от одной выбранной линии до другой и время~$t'$ её
обратного подъёма, происходящего под действием сил электрического поля,
измеряется секундомером.

Легко понять, что дискретность заряда может быть обнаружена лишь
в том случае, если погрешность~$\delta q$ в измерении заряда капли 
существенно меньше абсолютной величины заряда электрона~$e$. 
Допустимая относительная погрешность опыта~$\delta q/q$ должна быть 
поэтому много меньше, чем $e/q=1/n$, где~$n$~--- кратность заряда
капли элементарному заряду. Этому условию тем легче
удовлетворить, чем меньше число~$n$. 
В~условиях нашего опыта трудно определить~$q$ с точностью лучше 5\%.
Поэтому необходимо, чтобы заряд капли был меньше 20$e$
(оптимально --- 5$e$).

Проанализируем погрешность формулы \eqref{3.3.11}.
Из всех величин, входящих в формулу \eqref{3.3.11}, 
на опыте измеряются только~$t$, $t'$, и~$U$. 
Напряжение на пластинах может быть измерено достаточно
точно, поэтому погрешность измерения~$q$ определяется в основном 
погрешностью времени~$\delta t$. При визуальных наблюдениях
фактором, определяющим величину погрешности $\delta t$, 
выступает время реакции человека, которое практически 
не бывает меньше $\delta t \sim 0,2$~с.

Рассмотрим погрешность функции
$f=\frac{t+t'}{t^{3/2}t'} = \frac{1}{t'\sqrt{t}} + \frac{1}{t^{3/2}}$,
входящей в соотношение \eqref{3.3.11}.
Будем считать, что времена~$t$ и~$t'$ 
измерены независимо и с одинаковой погрешностью $\delta {t'}\approx \delta t$.
Пользуясь общей формулой для оценки погрешности косвенных 
измерений, после преобразований получим
\begin{equation}
\frac{\delta f}{f} \approx 
%\sigma_t \sqrt{\left(\frac{\partial f}{\partial t}\right)^2+\left(\frac{}{}\right)^2} =
\frac{\delta t}{t+t'} 
\sqrt{\left(\frac{t}{t'} \right)^2 + 
    \left(\frac12 + \frac32 \frac{t'}{t}\right)^2}.
\eqmark{3.3.12}
\end{equation}
Из соотношения \eqref{3.3.12} следует, что погрешность будет минимальна, 
если времена~$t$ и~$t'$ различаются незначительно: $t'\sim t$.
В этом случае для погрешности определения заряда имеем
\begin{equation}
	\frac{\delta q}{q} \sim 
    \sqrt{\left(\frac{\delta U}{U}\right)^2 + 5 \left(\frac{\delta t}{t+t'}\right)^2}.
	\eqmark{3.3.13}
\end{equation}

Абсолютная погрешность в измерении~$q$ будет тем меньше, 
чем больше суммарное время измерения $t+t'$ (при условии, что $t\sim t'$). 
Для увеличения~$t$ и~$t'$ можно было бы увеличить
расстояние, проходимое каплями, однако оно ограничено размерами установки. 
Поэтому необходимо наблюдать \emph{медленно движущиеся капли}, т.\,е. 
каплями \emph{малого веса}. 
Время падения~$t$ лёгких капель относительно велико. 
Чтобы сделать время подъёма~$t'$ также достаточно большим, необходимо
использовать относительно небольшие разности потенциалов~$U$.

С другой стороны, выбор слишком маленьких капель приводит к снижению точности
измерений. \emph{Броуновское движение} малых капель существенное влияет на
их движение и способно заметно исказить картину их падения и подъёма. 
Кроме того, размер маленьких капель может существенно меняться
в ходе опыта из-за их \emph{испарения}. 
Также при малых скоростях становятся особенно 
опасными \emph{конвекционные потоки} воздуха, возникающие при неоднородном нагреве
установки (например, от осветителя камеры). 

И, наконец, если размер капли приближается к длине свободного пробега
молекул воздуха ($\lambda \sim 10^{-5}$~см), \emph{формула Стокса} \eqref{3.3.3}
для них становится неприменимой. 
Милликен в своих опытах использовал уточнённый вариант формулы Стокса 
для капель малого радиуса. 
Однако в наших условиях эти поправки приводят к искажению значений~$q$ не более 
чем на 10\% и практически не мешают обнаружению дискретности электрического заряда. 
Поэтому мы рекомендуем использовать формулу Стокса без поправок.

\begin{lab:task}

\taskpreamble{В работе предлагается определить величину элементарного заряда 
    по измерениям времени свободного падения заряженных капель
и времени их подъёма в электрическом поле.}
    
\item Перед началом работы оцените с помощью формулы \eqref{3.3.11} 
минимальное напряжение~$U_{\rm min}$, которое нужно для подъёма капель, 
несущих 5 зарядов электрона на высоту $h=1$~мм, задав $t\approx t' \sim 20$~с. 
Если в дальнейшем для подъёма капель потребуются меньшие напряжения, 
то соответствующие капли слишком сильно заряжены и для эксперимента непригодны.

Расстояние между пластинами $\ell$ и плотность масла $\rho$ указаны на установке.
Коэффициент вязкости воздуха $\eta=1,85\cdot 10^{-5}\;\text{Па}\cdot \text{с}$
(при температуре 300~К).

\item Включите осветитель. При этом падающий в камеру свет направлен под углом к
оси микроскопа и в объектив не попадает. Поле зрения микроскопа поэтому 
остаётся тёмным. Капли масла рассеивают свет и 
выглядят светящимися точками на темном фоне.

Не включая электрическое поле \important{слегка} надавите на грушу
пульверизатора  и наблюдайте за движением облачка масляных капель в поле зрения
микроскопа (изображение перевёрнуто).

\item Настройте окуляр микроскопа на резкое изображение делений окулярной шкалы.
Затем сфокусируйте объектив на появившиеся в рабочем пространстве капли.

\item Наблюдая за движением капель, следует выбирать капли, время падения
которых на $h=1$~мм лежит в пределах 10--30~с, и научиться отличать 
их от более крупных, непригодных для работы. Цена деления окулярной шкалы 
указана на установке.

В случае отсутствия в составе установки ПК с камерой, регулировкой и коммутацией
напряжения занята одна рука наблюдателя. Вторая рука управляет секундомером.
Запись результатов измерений ($t$, $t'$ и~$U$) ведёт второй экспериментатор.
Наблюдатель быстро устаёт, поэтому рекомендуется периодически меняться местами.
При наличии ПК работа может быть выполнена одним экспериментатором.

Для уменьшения ошибок в определении~$t$ и~$t'$ нужно для пуска и остановки
секундомера использовать один и тот же признак~--- всегда нажимать 
кнопку секундомера либо в тот момент, когда капля скрывается за линией шкалы, либо,
наоборот, когда она появляется из-за линии. Рекомендуется следить за каплей, не
отрываясь от окуляра микроскопа, так как в противном случае её легко потерять из
поля зрения и весь эксперимент придётся повторить.

\item В начале опыта следует позволить капелькам свободно падать 5--10~с 
при выключенном электрическом поле для того, чтобы наиболее крупные
капли успели упасть на нижнюю пластину.

Из оставшихся в поле зрения капель выберите одну и произведите с ней серию
измерений, наблюдая её падение под действием силы тяжести и подъём под действием
электрического поля. Серия должна состоять из 5--10 измерений~$t$ и такого
же числа измерений~$t'$ для одной капли.

\item Необходимо проделать не менее 20 таких серий измерений (для 20~различных
капель), каждый раз регистрируя величину~$U$. 

При этом нужно иметь в виду, что заряд капли может измениться во время наблюдений.
Если перезарядка произошла при падении вниз, тогда для одной
капли получится несколько значений~$q$.

Если же перезарядка произойдет во время подъёма в электрическом поле,
то такой результат необходимо отбросить как непригодный для обработки. 
Это можно сделать, внимательно наблюдая за движением капли и отбрасывая опыты, 
в которых капля изменила скорость подъёма во время измерения.

\item Для оценки погрешности измерений <<подвесьте>> одну из капель в электрическом
поле. Определите соответствующее напряжение, отключите его  и измерьте время
падения капли на расстояние 2--3 делений шкалы. Поменяв полярность
напряжения, верните каплю на прежнее место и снова подвесьте её. Снова запишите
напряжение. Повторите  процедуру  для одной капли несколько раз и 
оцените из этого опыта заряд капли по формуле \eqref{3.3.11}, полагая время
подъёма $t'=\infty$. 
По разбросу результатов ($\delta U$ и~$\delta t$) оцените
погрешность измерения заряда этой капли.

\tasksection{Обработка результатов}

\item Для всех результатов измерений рассчитайте значения заряда капли~$q$
(учтите, что в серии измерений с одной каплей её заряд может поменяться).

\item Отложите результаты на горизонтальной числовой оси. Убедитесь, что
результаты могут быть разбиты на группы, средние значения заряда в которых
кратны некоторому наибольшему общему делителю
(следует учитывать, что при недостаточном числе измерений 
наименьший общий делитель может отличаться от $e$ и быть равным $2e$, $3e$ и т.\,д.).
%Этот наибольший делитель, вообще говоря, может оказаться равным~$e$, $2e$, $3e$ и
%т.~д. Однако, чем больше значений~$q$ было измерено на опыте, тем меньше
%вероятность получить в качестве делителя число, отличное от~$e$. 

\item Найдите значение элементарного заряда~$e$. 
Ответ приведите в системе единиц СИ и в системе СГС

\item Оцените характерное время релаксации $\tau$ и расстояние~$s(\tau)$, 
на которое смещается капля за это время. Убедитесь в том, что эти величины
малы в условиях опыта.

\item Оцените погрешности результатов. Сравните значение элементарного заряда
с табличным.

\end{lab:task}


\begin{lab:questions}
\item Почему не следует выбирать капли слишком большого и слишком маленького
размера?

\item Рассчитайте значение напряжение, соответствующее оптимальным условиям опыта.

\item Нарисуйте график зависимости скорости капли в поле силы тяжести от времени
и укажите на нём время и путь релаксации.

\item Зная параметры установки, оцените ёмкость конденсатора~$C$ и время его
разрядки через сопротивление~$R$ (площадь пластин $\approx 20$~см$^2$).

\item Оцените среднеквадратичную флуктуацию скорости капли.

\item Оцените величину смещения капли из-за броуновского движения за время опыта.

\item Проверьте применимость формулы Стокса, оценив число Рейнольдса для капли.
Критическое число Рейнольдса при обтекании сферы $\mathrm{Re}_{кр} \sim 25$.

\item $^*$Какие ещё способы измерения заряда электрона вам известны?
\end{lab:questions}


\begin{lab:literature}
    \item \Kirichenko~--- \S~7.3.
\item \SivuhinIII~--- Гл.~V, \S~90.
\item \Kalashnikov~--- Гл.~XVII, \S~178.
\end{lab:literature}
