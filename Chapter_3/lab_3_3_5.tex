\lab{Эффект Холла в металлах}

\aim{измерение подвижности и концентрации носителей заряда в металлах.}

\equip{электромагнит с источником питания, источник постоянного тока,
микровольтметр, амперметры, милливеберметр или цифровой магнитометр, образцы из
меди, серебра и цинка.}

Перед выполнением работы необходимо ознакомиться с основами
элементарной теории движения носителей заряда в~металлах и полупроводниках
(п. \ref{sec:halleffect} Введения к разделу).

В работе изучаются особенности проводимости металлов
в геометрии \emph{мостика Холла}.
Ток пропускается по плоской металлической пластинке, 
помещённой в перпендикулярное пластинке магнитное поле,
и измеряется разность потенциалов между краями пластинки в поперечном
к току направлении. По измерениям определяется \emph{константа Холла},
тип проводимости (\emph{электронный} или \emph{дырочный}) и на основе
соотношения \chaptereqref{HallConstant} вычисляется концентрация основных
носителей заряда.

\experiment 

Электрическая схема установки для измерения ЭДС Холла представлена
на рис.~\figref{Scheme}.
В зазоре электромагнита (рис.~\figref{Scheme}а) создаётся постоянное магнитное
поле, величину которого можно менять с помощью источника питания электромагнита.
Разъём~К$_1$ позволяет менять направление тока в обмотках электромагнита. Ток
питания электромагнита измеряется амперметром~А$_1$.

\begin{figure}[h!]
    \centering
    \pic{0.9\textwidth}{Chapter_3/3_5_1}
    \caption{Схема установки для исследования эффекта Холла в~металлах}
    \figmark{Scheme}
\end{figure}

Градуировка электромагнита (связь тока с индукцией поля) проводится 
при помощи милливеберметра (его описание и правила работы 
с ним приведены на с.~\pageref{MWB}) или миллитесламетра на основе
датчика Холла.

Металлические образцы в форме тонких пластинок, смонтированные в специальных
держателях, подключаются к блоку питания через разъём (рис.~\figref{Scheme}б).
Ток через образец регулируется реостатом~$R_2$ и измеряется амперметром~А$_2$.

Для измерений ЭДС Холла используется микровольтметр, в котором
высокая чувствительность по напряжению сочетается с малой величиной тока,
потребляемого измерительной схемой
(например, для микровольтметра Ф116/1 минимальный предел измерения 
напряжения составляет 1~мкВ, а потребляемый ток~--- всего $10^{-8}$~А).

В образце с током, помещённом в зазор электромагнита, между контактами~2 и~4
возникает холловская разность потенциалов $U_{\perp}$, которая измеряется с помощью
микровольтметра, если переключатель~К$_3$ подключён к точке~2 образца. При
подключении~К$_3$ к точке~3 микровольтметр измеряет омическое падение напряжения
$U_{34}$, вызванное током через образец. При нейтральном положении
ключа входная цепь микровольтметра разомкнута.

Ключ~К$_2$ позволяет менять полярность напряжения, поступающего на вход
микровольтметра.

Контакты~2 и~4 вследствие неточности подпайки могут лежать не на одной
эквипотенциали. Тогда напряжение между ними связано не только с эффектом
Холла, но и с омическим падением напряжения вдоль пластинки. 
Исключить этот эффект можно, изменяя направление магнитного поля, 
пронизывающего образец. 
При обращении поля ЭДС Холла меняет знак, а омическое падение напряжения
остаётся неизменным. Поэтому ЭДС Холла~$U_{\perp}$ может быть определена
как половина алгебраической разности показаний вольтметра, полученных для двух
противоположных направлений магнитного поля в~зазоре:
$U_{\perp} = \frac12 (U_{34}^{(+)}-U_{34}^{(-)})$.

Альтернативно, можно исключить влияние омического падения напряжения, 
если при каждом токе через образец измерять напряжение
между точками~3 и~4 в отсутствие магнитного поля. При фиксированном токе через
образец это дополнительное к ЭДС Холла напряжение~$U_0$ остаётся неизменным. 
От него следует (с учётом знака) отсчитывать величину ЭДС Холла:
\begin{equation}
U_{\perp} = U_{24} - U_0.
\eqmark{3.5.1}
\end{equation}
При таком способе измерения нет необходимости проводить повторные измерения
с противоположным направлением магнитного поля.

По знаку~$U_{\perp}$ можно определить характер проводимости~--- электронный или
дырочный. Для этого необходимо знать направление тока в образце и направление
магнитного поля.

Измерив ток~$I$ в образце и напряжение~$U_{34}$ между контактами~3 и~4 в
отсутствие магнитного поля, можно, зная параметры образца, рассчитать удельное
сопотивление $\rho_0$ и проводимость $\lambda_0$ материала образца по формуле:
\begin{equation}
	\rho_0=\frac{U_{34}ah}{Il},
	\eqmark{3.5.2}
\end{equation}
где $l$~--- расстояние между контактами~3 и~4, $a$~--- ширина образца, $h$~---
его толщина.

\begin{lab:task}

\taskpreamble{В работе предлагается исследовать зависимость ЭДС~Холла от величины магнитного
поля при различных токах через образец для определения константы Холла;
определить знак носителей заряда и проводимость различных металлических
образцов.}


\item Подготовьте приборы к работе согласно описанию на установке.

\item Проверьте работу цепи питания образца. Для этого подключите к разъёму
блока управления один из образцов~--- медный или серебряный. Убедитесь, что ток
через образец можно изменять в указанных в описании пределах.

\item Проверьте работу цепи магнита. Установите разъём~$K_1$ в положение~I и,
плавно изменяя ток от минимального до максимального значения, 
определите диапазон изменения тока через электромагнит.

\item Измерьте калибровочную кривую электромагнита~---
зависимость между индукцией~$B$ магнитного поля в его зазоре и 
током~$I_{М}$ через обмотки магнита.
Магнитное поле измеряется милливеберметром или миллитесламетром
(датчиком Холла). Калибровочная кривая должна содержать не менее
15 точек во всём диапазоне изменения токов.

В~случае использования милливеберметра измерьте зависимость 
магнитного потока~$\Phi$, пронизывающего пробную катушку, 
находящуюся в зазоре, от тока~$I_{М}$ ($\Phi=BSN$). 
Значение~$SN$ (произведение площади сечения контура катушки на
число витков в ней) указано на держателе катушки.

\item \label{p1} Вставьте образец в зазор \emph{выключенного} электромагнита 
и определите напряжение $U_0$ между холловскими
контактами~2 и~4 при минимальном токе через образец. 
Это напряжение~$U_0$ вызвано несовершенством контактов~3, 4 и при 
фиксированном токе через образец остаётся неизменным. Значение~$U_0$ с учётом 
знака следует принять за начало отсчёта напряжения.

\item \label{p2} Проведите измерение ЭДС Холла: снимите зависимость 
напряжения~$U_{24}$ от тока электромагнита~$I_{М}$ (5--7 точек) 
при фиксированном токе~$I$ через образец.

Измерения следует проводить при \emph{медленном} увеличении магнитного поля. 
Резкие изменения магнитного поля наводят ЭДС индукции в подводящих проводах 
и вызывают большие отклонения стрелки микровольтметра.

\item Повторите измерения пп. \ref{p1}, \ref{p2} при 5--7 токах $I$ через образец
(рекомендованные токи указаны в описании установки).  
Учтите, что при каждом новом токе $I$ величина~$U_0$ будет иметь 
своё значение.

\item При максимальном токе через образец проведите измерения $U_{24}(I_{M})$ 
при обратном направлении магнитного поля.

\item Для образца из цинка снимите зависимость $U_{24}(I_{M})$ при одном 
(максимальном) значении тока через образец.

\item Определите знак носителей заряда в образце. Для этого необходимо знать
направление тока через образец, направление магнитного поля и знак ЭДС Холла.

Направление тока в образце показано знаками~<<$+$>> и~<<$-$>> на
рис.~\figref{Scheme}. Направление тока в обмотках электромагнита при 
установке разъёма~К$_1$ в положение~1 показано стрелкой на торце магнита.

Зарисуйте в тетради образец. Укажите на рисунке направления тока, магнитного
поля и отклонение носителей. Определите характер проводимости образцов
(дырочный или электронный)

\item Определите удельное сопротивление образцов. Для этого удалите держатель с
образцом из зазора. При необходимости переключите микровольтметр 
в режим измерения милливольтовых напряжений. Ключ~К$_3$ поставьте 
в положение $U_{34}$. При токе через образец порядка максимального 
значения в предыдущих измерениях измерьте падение напряжения между 
контактами~3 и~4 для каждого из двух образцов.

\item Запишите характеристики приборов и параметры образца, указанные на держателе:
длину~$l$ (расстояние между точками~3 и~4), ширину~$a$, толщину~$h$.

\tasksection{Обработка результатов}

\item Постройте калибровочный график зависимости $B(I_{M})$. 
Используйте эту зависимость для дальнейшего пересчёта (интерполяции)
значений поля по измеренным токам $I_{М}$.

\item Рассчитайте ЭДС Холла по формуле \eqref{3.5.1} и постройте на одном листе
семейство характеристик $U_{\perp}(B)$ при разных значениях тока~$I$ через
образец (для меди или серебра). 
Убедитесь в линейности зависимостей и определите угловые 
коэффициенты $k=dU_{\perp}/dB$ полученных прямых.

\item Постройте график $k(I)$. Рассчитайте угловой коэффициент прямой 
и по формуле \chaptereqref{3.19} определите величину постоянной Холла~$R_{\rm H}$.

\item Для цинка изобразите на графике зависимость $U_{\perp}(B)$ и по наклону 
прямой рассчитайте постоянную Холла.

\item Для обоих образцов рассчитайте концентрацию~$n$ носителей тока по формуле
\chaptereqref{HallConstant}.

\item Рассчитайте удельное сопротивление $\rho_0$ и удельную 
проводимость $\lambda_0$ материалов по формуле \eqref{3.5.2}.

\item Используя найденные значения концентрации~$n$ и удельной проводимости
$\lambda_0=1/\rho_0$, с помощью формулы \chaptereqref{3.26} вычислите 
подвижность~$\mu$ носителей тока. Ответ представьте в общепринятых для этой величины 
внесистемных единицах $[\mu]=$~см$^2$/(В$\cdot$с)
(размерность напряжённости $[E]=$~B/см, скорости $[v]=$~см/с,
поэтому $[\mu]=[v/E]=$~см$^2$/(В$\cdot$с)).

\item Дайте качественное объяснение тому, что константа Холла для
цинка отрицательна.

\item Оцените погрешности и сравните результаты с табличными.

\end{lab:task}



\begin{lab:questions}

\item Какие вещества называют диэлектриками, проводниками, полупроводниками?
Чем объясняется различие их электрических свойств? Как зависит от температуры
проводимость металлов и полупроводников?

\item Дайте определение константы Холла. Как зависит константа Холла от
температуры у металлов и полупроводников?

\item Зависит ли результат измерения константы Холла от геометрии образца?

\item Зависит ли сопротивление образца от магнитного поля 
в условиях опыта?

\item Как устроен милливеберметр? Зависят ли его показания от сопротивления
измерительной катушки? Каким должно быть это сопротивление по сравнению с
сопротивлением рамки прибора: большим или маленьким?

\item По результатам измерений оцените частоту столкновений,
длину пробега и коэффициент диффузии носителей тока в исследуемом металле.

\item Получите выражение константы Холла для материалов с~двумя типами
носителей. \emph{Указание}: воспользуйтесь условием равенства нулю поперечного тока.

\end{lab:questions}

\begin{lab:literature}

\item \SivuhinIII~--- \S\S~98, 100.

\item \emph{Парселл Э.} Электричество и магнетизм.~--- М.: Наука, 1983. Гл.~4,
\S\S~4--6; Гл.~6, \S~9

\end{lab:literature}
