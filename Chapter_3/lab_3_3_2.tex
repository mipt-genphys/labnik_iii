{\large \bf 3.3.2 Исследование вольт-амперной характеристики вакуумного диода}
{\bf Цель работы:}{определение удельного заряда электрона на основе закона <<трёх вторых>>.}

{\bf В работе используются:}{радиолампа с цилиндрическим анодом; амперметр; многопредельные микроамперметр и вольтметр постоянного тока; стабилизированные источники постоянного тока и постоянного напряжения.}

В работе исследуется зависимость прямого тока, проходящего через вакуумный диод, от напряжения на нём (положительная ветвь вольт-амперной характеристики). Наибольший физический интерес представляет та область положительного напряжения на диоде, в которой пространственный заряд (электронное облако) существенно влияет на распределение электрического поля между катодом и анодом. В этой области ток диода меньше тока эмиссии катода из-за того, что электрическое поле пространственного заряда препятствует движению электронов, испущенных катодом, и часть их возвращается на катод. Как будет показано ниже, в этом случае величина тока пропорциональна напряжению на диоде в степени 3/2:
\begin{equation}
I\propto V^{3/2}
\end{equation}
(<<закон трёх вторых>>). Коэффициент пропорциональности в этой формуле зависит от удельного заряда электрона (отношения заряда электрона к его массе). Цель работы состоит в измерении удельного заряда электрона из вольт-амперной характеристики диода в~области, описываемой <<законом трёх вторых>>.

Рассмотрим прохождение электрического тока через вакуумный диод. Будем считать, что его катод имеет форму нити с
радиусом~$r_K$, а анод~--- форму полого цилиндра с радиусом $r_A$ (рис. \ref{fig3.2.1}). Между катодом и анодом приложена разность потенциалов $V_A$.

\begin{figure}
\caption{Схема расположения электродов в диоде}
\label{fig3.2.1}
\end{figure}

Для простоты примем, что потенциал катода равен нулю, а потенциал анода равен $V_A$. Ток в лампе переносится
электронами, испускаемыми раскалённым катодом. Будем считать, что длина диода намного превосходит его радиальные
размеры, так что электрическое поле можно считать чисто радиальным.

Движение электронов в лампе происходит под действием электрического поля, распределение которого в свою очередь зависит от плотности электронного облака. Нас будет интересовать задача о~стационарном (не меняющемся с течением времени) распределении потенциала и зарядов. Будем также считать, что вследствие симметрии задачи потенциал электрического поля не зависит ни от координаты $z$, ни от угла $\phi$ и является функцией одного радиуса $r$.

Распределение потенциала внутри диода определяется уравнением Пуассона, которое в цилиндрической системе координат имеет вид (как уже сказано, полагаем $\partial V/\partial z=\partial V/\partial\phi=0$):
\begin{equation}
\Delta V=\frac{d^2 V}{dr^2}+\frac{1}{r}\frac{dV}{dr}=-\frac{\rho(r)}{\epsilon_0},
\label{eq3.2.1}
\end{equation}
где $\rho(r)$~--- плотность электрического заряда. Двумя сечениями, перпендикулярными оси $z$, вырежем в диоде слой
толщиной $l$. Плотность заряда $\rho(r)$ связана с током $I$, протекающим через этот слой, очевидной формулой
\begin{equation}
I=-2\pi r\rho(r)v(r)l,
\label{eq3.2.2}
\end{equation}
где $v(r)$~--- скорость электронов на радиусе $r$. В стационарном случае $I$ не зависит от $r$. Таким образом,
\begin{equation}
I=const.
\label{eq3.2.3}
\end{equation}

Скорость электронов определяется разностью потенциалов, которую они прошли, и скоростью их вылета из катода. Этой
последней скоростью мы будем пренебрегать. Ошибка, связанная с указанным предположением, тем меньше, чем выше $V_A$ (при малых напряжениях она может оказаться существенной). Тогда
\begin{equation}
\frac{mv^2(r)}{2}=eV(r).
\label{eq3.2.4}
\end{equation}
В формуле (\ref{eq3.2.4}) $m$~--- масса, $e$~--- абсолютная величина заряда электрона.

Исключая $v$ и $\rho$ из уравнений (\ref{eq3.2.1}), (\ref{eq3.2.2}) и (\ref{eq3.2.4}), найдём
\begin{equation}
r\frac{d^2V}{dr^2}+\frac{dV}{dr}=\frac{I}{2\pi\epsilon_0l}\sqrt{\frac{m}{2eV}}.
\label{eq3.2.5}
\end{equation}

Мы пришли, таким образом, к дифференциальному уравнению второго порядка, из которого следует определить $V$. Это
уравнение может быть решено, если заданы граничные условия, т.~е. значения потенциала на катоде и на аноде.
Дополнительная трудность состоит в том, что неизвестен ток $I$, зависящий от $V$ и входящий в правую часть (\ref{eq3.2.5}), и, таким образом, не полностью определено само уравнение.

Вместо того чтобы задавать величину тока $I$, можно наложить на потенциал ещё одно условие, например, задавать не только потенциал анода, но и производную $dV/dr$ на катоде. Обычно полагают
\begin{equation}
\left.\frac{dV}{dr}\right|_{r=r_K}=0.
\label{eq3.2.6}
\end{equation}
Производная $dV/dr=-E_r$, где $E_r$~--- радиальная компонента напряжённости электрического поля. Наше предположение
означает, таким образом, что вблизи катода пространственный заряд электронов полностью экранирует электрическое поле,создаваемое анодной разностью потенциалов. В электронных лампах при нормальных рабочих режимах электрическое поле обращается в нуль не на самом катоде, а на расстоянии 0,01--0,1~мм от него. В нашем случае этим расстоянием можно пренебречь. Условие (\ref{eq3.2.6}) и пренебрежение начальной скоростью вылетающих с катода электронов не вполне точны и вносятся для упрощения задачи.

Уравнение (\ref{eq3.2.5}) является нелинейным дифференциальным уравнением. Его решение не может быть найдено простыми методами. Пусть, однако, мы нашли решение этого уравнения при некотором $V_A=V_{A0}$ и пусть при этом ток оказался равным $I=I_0$. Покажем, что в этом случае можно найти решение (\ref{eq3.2.5}) и при любом другом значении потенциала $V_A$. Если $I_0$ и $V_{A0}(r)$ являются решением задачи при напряжении $V_{A0}$, то выражения
\begin{equation}
I=I_0\left(\frac{V_A}{V_{a0}}\right)^{3/2},\qquad V(r)=V_{a0}(r)\frac{V_A}{V_{A0}}
\label{eq3.2.7}
\end{equation}
являются искомыми решениями уравнения (\ref{eq3.2.5}) при потенциале~$V_A$. В~самом деле, подставляя (\ref{eq3.2.7}) в (\ref{eq3.2.5}), найдём
$$
r\frac{V_A}{V_{A0}} \frac{d^2V_{A0}(r)}{dr^2}+\frac{V_A}{V_{A0}} \frac{dV_{A0}(r)}{dr}=
\frac{I_0}{2\pi\epsilon_0l} \left( \frac{V_A}{V_{A0}} \right)^{3/2}
\sqrt{ \frac{m}{2eV_{A0}(r )V_A/V_{A0}}}.
$$

Сокращая это уравнение на $V_A/V_{A0}$, придём к уравнению
$$
r\frac{d^2V_{A0}(r)}{dr^2}+ \frac{dV_{A0}(r)}{dr}=\frac{I_0}{2\pi\epsilon_0l}\sqrt{\frac{m}{2eV_{A0}(r)}},
$$
которое, конечно, выполняется, так как по предположению $I_0$ и $V_{A0}(r)$ являются решениями (\ref{eq3.2.5}).

\begin{figure}
\caption{Схема экспериментальной установки}
\label{fig3.2.2}
\end{figure}

Формула (\ref{eq3.2.7}) представляет собой содержание {\em<<закона трёх вторых>>}, утверждающего, что ток в вакуумном диоде пропорционален напряжению на нём в степени 3/2. Этот закон справедлив при любой~--- а не только при цилиндрической~---
геометрии электродов, если ток не слишком велик (т.~е. пока условие (\ref{eq3.2.6}) нарушается не слишком сильно).

В общем случае решение уравнения~(\ref{eq3.2.5}), удовлетворяющее условию (\ref{eq3.2.6}), записывается обычно в виде
\begin{equation}
I=\frac{8\sqrt{2}\pi\epsilon_0l}{9}\sqrt{\frac{e}{m}}\frac{1}{r_A\beta^2}V^{3/2},
\label{eq3.2.8}
\end{equation}

где $\beta^2$~--- функция от $r_A/r_K$, которая может быть задана бесконечным рядом или графиком. То обстоятельство, что$I$ пропорционально $V^{3/2}$, уже обсуждалось. Линейный характер связи между $I$ и $\sqrt{e/m}$ очевиден из
рассмотрения правой части (\ref{eq3.2.5}). Численный коэффициент при $V^{3/2}$ выбран так, чтобы $r_A/r_K\rightarrow\infty$ при$\beta^2\rightarrow 1$. Последний результат (для $r_K\rightarrow 0$) легко получить, подставив в (\ref{eq3.2.5}) $ V(r)=Ur^{2/3} $, где $U=const$.

{\bf Экспериментальная установка.} Исследования проводятся на диоде с косвенным накалом. Радиус его катода $r_K$, анода $r_A$,  коэффициент $\beta^2$, а также длина его активной области  $l$  указаны в описании, имеющемся на рабочем месте. Отметим, что длина активной области диода (участка катода, покрытого оксидным слоем, обеспечивающим термоэмиссию электронов) обычно существенно (примерно в два раза) меньше полной высоты анода. Благодаря этому рабочая часть катода достаточно удалена от его торцов, а следовательно  электрическое поле в активной части диода с~хорошей точностью можно считать радиальным.

Схема экспериментальной установки изображена на Рис. \ref{fig3.2.2}. Для питпния цепи накала и анода используются два регулируемых источника напряжения ИП1 и ИП2 соответственно. В цепь накала включены амперметр $A$ и предохранительное сопротивление $R$. Анодное напряжение измеряется вольтметром $V$, а анодный ток --- миллиамперметром $mA$. В работе предлагается производить измерения анодных тока и напряжения в широком диапазоне (перекрывающим примерно три порядка величины), в связи с чем указанные вольтметр и миллиамперметр включает в себя устройство ручного или автоматического переключения диапазонов измерения.  

{\Large \bf ЗАДАНИЕ}

В работе предлагается исследовать вольт-амперные характеристики диода при различных токах накала и по результатам
измерений определить удельный заряд электрона.

\begin{enumerate}

\item{ Ознакомьтесь с экспериментальной установкой, изображённой на рис. \ref{fig3.2.2}.}

\item{Настройте вольтметр и миллиамперметр, согласно прилагаемой к установке инструкции, для измерения минимальных значений анодного напряжния и тока.}

\item{ Установите минимальный  ток накала диода $I_{н}$, указанный в инструкции.}

\item{ Регулятором выпрямителя анодной цепи установите минимальное значение анодного напряжения $V_{A}$, указанное в инструкции.}

\item{ Следуя указаниям инструкции, имеющейся на рабочем месте, снимите вольт-амперные характеристики диода $I_{A}(V_{A})$ в указанном диапазоне (обычно это от 0,5 до 50~В, при этом для малых напряжений  --- до 6~В измерения рекомендуется производить с шагом 0,5~В, для средних, до 10~В --- с шагом 1~В и для более высоких --- с шагом 5~В ).}

\item{ Повторите измерения еще при 2-3 значениях токах накала, указанных в инструкции.}
\end{enumerate}

{\rm Обработка результатов}
\begin{enumerate}
\item{ По результатам эксперимента постройте графики зависимости $I_A=f(V_{A}^{3/2})$. Определите интервалы значений $V_{A}$, на которых графики имеют вид прямых линий. Найдите  наклон прямолинейных участков характеристик и используйте его для вычисления $e/m$ электрона.}
\item{ В тех же координатах на другом рисунке постройте участок вольт-амперной характеристики в диапазоне анодных напряжений от 0 до 10~В. Почему вольт-амперная характеристика на этом участке нелинейна?}
\end{enumerate}


{\small
{\bf \Large Контрольные вопросы}
\begin{enumerate}

\item{ Нарисуйте качественные графики распределения потенциала $V(r)$ между катодом и анодом: а)~при нулевой разности потенциалов между катодом и анодом; б)~при большой разности потенциалов (режим насыщения тока диода). Объясните эти распределения.}

 \item{Качественно изобразите зависимость тока диода от напряжения на аноде в области от отрицательных напряжений $V_{A}$ до больших положительных. Покажите на этом графике участок напряжений, при которых выполняется <<закон трёх вторых>>. Чем объясняются отклонения от этого закона при малых и больших напряжениях на аноде?}

\item{ Как влияет ток накала катода на ток диода при неизменном напряжении на аноде? Приводит ли это к погрешности
измерения $e/m$?}
\end{enumerate}

{\bf \Large Список литераратуры}

\begin{enumerate}
\item{Сивухин Д.В. Общий курс физики. Т. III. Электричество.~--- М.: Физматлит, 2015}.
\item{Калашников С.Г. Электричество.М.: Физматлит, 2003}
\end{enumerate}

