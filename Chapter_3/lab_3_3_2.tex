
\lab{Исследование вольт-амперной характеристики вакуумного диода}
\aim{определение удельного заряда электрона на основе закона <<трёх вторых>>.}
\equip{радиолампа с цилиндрическим анодом; амперметр; многопредельные микроамперметр и вольтметр постоянного тока; стабилизированные источники постоянного тока и постоянного напряжения.}

В работе исследуется зависимость прямого тока, проходящего через вакуумный диод, от напряжения на нём (положительная ветвь вольт-амперной характеристики). Наибольший физический интерес представляет та область положительного напряжения на диоде, в которой пространственный заряд (электронное облако) существенно влияет на распределение электрического поля между катодом и анодом. В этой области ток диода меньше тока эмиссии катода из-за того, что электрическое поле пространственного заряда препятствует движению электронов, испущенных катодом, и часть их возвращается на катод. Как будет показано ниже, в этом случае величина тока пропорциональна напряжению на диоде в степени 3/2:
\begin{equation}
	I\propto V^{3/2}
\end{equation}
(<<закон трёх вторых>>). Коэффициент пропорциональности в этой формуле зависит от удельного заряда электрона (отношения заряда электрона к его массе). Цель работы состоит в измерении удельного заряда электрона из вольт-амперной характеристики диода в области, описываемой <<законом трёх вторых>>.

Рассмотрим прохождение электрического тока через вакуумный диод. Будем считать, что его катод имеет форму нити с
радиусом $r_k$, а анод~--- форму полого цилиндра с радиусом $r_a$ (рис.~\figref{Scheme of electrodes}). Между катодом и анодом приложена разность потенциалов $V_A$.
\begin{figure}[h!]
	\pic{0.9\textwidth}{3_2_1}
	\caption{Схема расположения электродов в диоде}
	\figmark{Scheme of electrodes}
\end{figure}

Для простоты примем, что потенциал катода равен нулю, а потенциал анода равен $V_A$. Ток в лампе переносится
электронами, испускаемыми раскалённым катодом. Будем считать, что длина диода намного превосходит его радиальные
размеры, так что электрическое поле можно считать чисто радиальным.

Движение электронов в лампе происходит под действием электрического поля, распределение которого в свою очередь зависит от плотности электронного облака. Нас будет интересовать задача о стационарном (не меняющемся с течением времени) распределении потенциала и зарядов. Будем также считать, что вследствие симметрии задачи потенциал электрического поля не зависит ни от координаты $z$, ни от угла $\varphi$ и является функцией одного радиуса $r$.

Распределение потенциала внутри диода определяется уравнением Пуассона, которое в цилиндрической системе координат имеет вид (как уже сказано, полагаем $\partial V/\partial z=\partial V/\partial\varphi=0$):
\begin{equation}
	\Delta V=\frac{d^2 V}{dr^2}+\frac{1}{r}\frac{dV}{dr}=-\frac{\rho(r)}{\varepsilon_0},
	\eqmark{3.2.1}
\end{equation}
где $\rho(r)$~--- плотность электрического заряда. Двумя сечениями, перпендикулярными оси $z$, вырежем в диоде слой
толщиной $l$. Плотность заряда $\rho(r)$ связана с током $I$, протекающим через этот слой, очевидной формулой
\begin{equation}
	I=-2\pi r\rho(r)v(r)l,
	\eqmark{3.2.2}
\end{equation}
где $v(r)$~--- скорость электронов на радиусе $r$. В стационарном случае $I$ не зависит от $r$. Таким образом,
\begin{equation}
	I=const.
	\eqmark{3.2.3}
\end{equation}

Скорость электронов определяется разностью потенциалов, которую они прошли, и скоростью их вылета из катода. Этой
последней скоростью мы будем пренебрегать. Ошибка, связанная с указанным предположением, тем меньше, чем выше $V_A$ (при малых напряжениях она может оказаться существенной). Тогда
\begin{equation}
	\frac{mv^2(r)}{2}=eV(r).
	\eqmark{3.2.4}
\end{equation}

В формуле \eqref{3.2.4} $m$~--- масса, $e$~--- абсолютная величина заряда электрона.

Исключая $v$ и $\rho$ из уравнений \eqref{3.2.1}, \eqref{3.2.2} и \eqref{3.2.4}, найдём
\begin{equation}
	r\frac{d^2V}{dr^2}+\frac{dV}{dr}=\frac{I}{2\pi\varepsilon_0l}\sqrt{\frac{m}{2eV}}.
	\eqmark{3.2.5}
\end{equation}

Мы пришли, таким образом, к дифференциальному уравнению второго порядка, из которого следует определить $V$. Это
уравнение может быть решено, если заданы граничные условия, т.~е. значения потенциала на катоде и на аноде.
Дополнительная трудность состоит в том, что неизвестен ток $I$, зависящий от $V$ и входящий в правую часть \eqref{3.2.5}, и, таким образом, не полностью определено само уравнение.

Вместо того, чтобы задавать величину тока $I$, можно наложить на потенциал ещё одно условие, например, задавать не только потенциал анода, но и производную $dV/dr$ на катоде. Обычно полагают
\begin{equation}
	\left.\frac{dV}{dr}\right|_{r=r_k}=0.
	\eqmark{3.2.6}
\end{equation}

Производная $dV/dr=-E_r$, где $E_r$~--- радиальная компонента напряжённости электрического поля. Наше предположение
означает, таким образом, что вблизи катода пространственный заряд электронов полностью экранирует электрическое поле, создаваемое анодной разностью потенциалов. В электронных лампах при нормальных рабочих режимах электрическое поле обращается в нуль не на самом катоде, а на расстоянии 0,01 -- 0,1~мм от него. В нашем случае этим расстоянием можно пренебречь. Условие \eqref{3.2.6} и пренебрежение начальной скоростью вылетающих с катода электронов не вполне точны и вносятся для упрощения задачи.

Уравнение \eqref{3.2.5} является нелинейным дифференциальным уравнением. Его решение не может быть найдено простыми методами. Пусть, однако, мы нашли решение этого уравнения при некотором $V_A=V_{A_0}$ и пусть при этом ток оказался равным $I=I_0$. Покажем, что в этом случае можно найти решение \eqref{3.2.5} и при любом другом значении потенциала $V_A$. Если $I_0$ и $V_{A_0}(r)$ являются решением задачи при напряжении $V_{A_0}$, то выражения
\begin{equation}
	I=I_0\left(\frac{V_A}{V_{A_0}}\right)^{3/2},\qquad V(r)=V_{A_0}(r)\frac{V_A}{V_{A_0}}
	\eqmark{3.2.7}
\end{equation}
являются искомыми решениями уравнения \eqref{3.2.5} при потенциале~$V_A$. В~самом деле, подставляя \eqref{3.2.7} в \eqref{3.2.5}, найдём
\begin{equation*}
	r\frac{V_A}{V_{A_0}} \frac{d^2V_{A_0}(r)}{dr^2}+\frac{V_A}{V_{A_0}} \frac{dV_{A_0}(r)}{dr}=
\frac{I_0}{2\pi\varepsilon_0l} \left( \frac{V_A}{V_{A_0}} \right)^{3/2} \sqrt{ \frac{m}{2eV_{A_0}(r )\frac{V_A}{V_{A_0}}}}.
\end{equation*}

Сокращая это уравнение на $V_A/V_{A_0}$, придём к уравнению
\begin{equation*}
	r\frac{d^2V_{A_0}(r)}{dr^2}+ \frac{dV_{A_0}(r)}{dr}=\frac{I_0}{2\pi\varepsilon_0l}\sqrt{\frac{m}{2eV_{A_0}(r)}},
\end{equation*}
которое, конечно, выполняется, так как по предположению $I_0$ и $V_{A_0}(r)$ являются решениями \eqref{3.2.5}.

Формула \eqref{3.2.7} представляет собой содержание \important{<<закона трёх вторых>>}, утверждающего, что ток в вакуумном диоде пропорционален напряжению на нём в степени 3/2. Этот закон справедлив при любой, а не только при цилиндрической, геометрии электродов, если ток не слишком велик (т.~е. пока условие \eqref{3.2.6} нарушается не слишком сильно).

В общем случае решение уравнения \eqref{3.2.5}, удовлетворяющее условию \eqref{3.2.6}, записывается обычно в виде
\begin{equation}
	I=\frac{8\sqrt{2}\pi\varepsilon_0l}{9}\sqrt{\frac{e}{m}}\frac{1}{r_a\beta^2}V^{3/2},
	\eqmark{3.2.8}
\end{equation}
где $\beta^2$~--- функция от $r_a/r_k$, которая может быть задана бесконечным рядом или графиком. То обстоятельство, что $I$ пропорционально $V^{3/2}$, уже обсуждалось. Линейный характер связи между $I$ и $\sqrt{e/m}$ очевиден из
рассмотрения правой части \eqref{3.2.5}. Численный коэффициент при $V^{3/2}$ выбран так, чтобы $r_a/r_k\rightarrow\infty$ при $\beta^2\rightarrow 1$. Последний результат (для $r_k\rightarrow 0$) легко получить, подставив в \eqref{3.2.5} $V(r)=Ur^{2/3} $, где $U=const$.

\experiment Исследования проводятся на диоде с косвенным накалом. Радиус его катода $r_k$, анода $r_a$,  коэффициент $\beta^2$, а также длина его активной области $l$  указаны в описании, имеющемся на рабочем месте. Отметим, что длина активной области диода (участка катода, покрытого оксидным слоем, обеспечивающим термоэмиссию электронов) обычно существенно (примерно в два раза) меньше полной высоты анода. Благодаря этому рабочая часть катода достаточно удалена от его торцов, а, следовательно,  электрическое поле в активной части диода с хорошей точностью можно считать радиальным.
\begin{figure}[h!]
	\pic{0.9\textwidth}{3_2_2}
	\caption{Схема экспериментальной установки}
	\figmark{Scheme}
\end{figure}

Схема экспериментальной установки изображена на рис.~\figref{Scheme}. Для питания цепи накала и анода используются два регулируемых источника напряжения ИП1 и ИП2 соответственно. В цепь накала включены амперметр $A$ и предохранительное сопротивление $R$. Анодное напряжение измеряется вольтметром $V$, а анодный ток --- миллиамперметром $mA$. В работе предлагается производить измерения анодных тока и напряжения в широком диапазоне (перекрывающем примерно три порядка величины), в связи с чем указанные вольтметр и миллиамперметр включают в себя устройства ручного или автоматического переключения диапазонов измерения.  

\begin{lab:task}

В работе предлагается исследовать вольт-амперные характеристики диода при различных токах накала и по результатам
измерений определить удельный заряд электрона.

\begin{enumerate}

\item{Ознакомьтесь с устройством экспериментальной установки, изображённой на рис.~\figref{Scheme}.}

\item{Настройте вольтметр и миллиамперметр, согласно прилагаемой к установке инструкции, для измерения минимальных значений анодного напряжения и тока.}

\item{ Установите минимальный  ток накала диода $I_\text{н}$, указанный в инструкции.}

\item{ Регулятором выпрямителя анодной цепи установите минимальное значение анодного напряжения $V_{A}$, указанное в инструкции.}

\item{ Следуя указаниям инструкции, имеющейся на рабочем месте, снимите вольт-амперные характеристики диода $I_{A}(V_{A})$ в указанном диапазоне (обычно это от 0,5 до 50~В, при этом для малых напряжений (до 6~В) измерения рекомендуется производить с шагом 0,5~В, для средних (до 10~В)~--- с шагом 1~В и для более высоких~--- с шагом 5~В).}

\item{ Повторите измерения еще при 2 -- 3 значениях токах накала, указанных в инструкции.}
\end{enumerate}

\tasksection{Обработка результатов}

\begin{enumerate}
\item{ По результатам проведённого эксперимента постройте графики зависимости $I_A = f(V_{A}^{3/2})$. Определите интервалы значений $V_{A}$, на которых графики имеют вид прямых линий. Найдите  наклон прямолинейных участков характеристик и используйте его для вычисления $e/m$ электрона.}
\item{ В тех же координатах на другом рисунке постройте участок вольт-амперной характеристики в диапазоне анодных напряжений от 0 до 10~В. Почему вольт-амперная характеристика на этом участке нелинейна?}
\end{enumerate}

\end{lab:task}

\begin{lab:questions}
	\item{Нарисуйте качественные графики распределения потенциала $V(r)$ ме-жду катодом и анодом: а)~при нулевой разности потенциалов между катодом и анодом; б)~при большой разности потенциалов (режим насыщения тока диода). Объясните эти распределения.}
	
	\item{Качественно изобразите зависимость тока диода от напряжения на аноде в области от отрицательных напряжений $V_{A}$ до больших положительных. Покажите на этом графике участок напряжений, при которых выполняется <<закон трёх вторых>>. Чем объясняются отклонения от этого закона при малых и больших напряжениях на аноде?}
	
	\item{ Как влияет ток накала катода на ток диода при неизменном напряжении на аноде? Приводит ли это к погрешности измерения $e/m$?}
\end{lab:questions}

\begin{lab:literature}
	\item{\emph{Сивухин Д.В.} Общий курс физики. Т. III. Электричество.~--- М.: Физматлит, 2015}.
	\item{\emph{Калашников С.Г.} Электричество.М.: Физматлит, 2003}
\end{lab:literature}

