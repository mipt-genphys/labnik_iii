\lab{Изучение плазмы газового разряда в неоне}

\aim{изучение вольт-амперной характеристики тлеющего разряда; 
    изучение свойств плазмы методом зондовых характеристик.}

\equip{стеклянная газоразрядная трубка, наполненная неоном;
высоковольтный источник питания; источник питания
постоянного тока; делитель напряжения; потенциометр; амперметры;
вольтметры; переключатели.}

Перед выполнением работы необходимо ознакомиться с 
теоретическим Введением к разделу.
Подробное описание свойств тлеющего разряда можно найти в Приложении к разделу
(см. стр.~\pageref{sec:discharge}).

Схема установки для исследования плазмы газового разряда в неоне представлена на
рис.~\figref{Neon gas discharge}. Стеклянная газоразрядная
трубка имеет холодный (негреваемый) полый катод, три анода и
\important{геттерный узел}~--- стеклянный баллон, на
внутреннюю поверхность которого напылена газопоглощающая плёнка
(\important{геттер}). Трубка наполнена изотопом неона
$^{22}$Ne при давлении 2~мм~рт. ст. Катод и один из анодов (I или II) с~помощью
переключателя $\text{П}_1$ подключаются через
балластный резистор~$R_\text{б}$ ($\sim500$~кОм) к регулируемому высоковольтному
источнику питания (ВИП) с~выходным
напряжением до нескольких киловольт.

\begin{figure}[h!]
    \centering
    \footnotesize
	\pic{\textwidth}{Chapter_5/c5_1_1}
	\caption{Схема установки для исследования газового разряда}
	\figmark{Neon gas discharge}
\end{figure}

При подключении к ВИП анода-I между ним и катодом возникает газовый разряд. Ток
разряда измеряется миллиамперметром~$A_1$, а падение напряжения на разрядной трубке~--- 
вольтметром~$V_{1}$, подключённым к трубке через
высокоомный (несколько десятков МОм) делитель напряжения с коэффициентом
$(R_1+R_2)/R_2$.

При подключении к ВИП анода-II разряд возникает в~проcтранстве между катодом и
анодом-II, где находится двойной зонд,
используемый для диагностики плазмы положительного столба. Зонды изготовлены из
молибденовой проволоки диаметром
$d$ и имеют длину~$l$. Они подключены к источнику питания через
потенциометр~$R$. Переключатель
$\text{П}_2$ позволяет изменять полярность напряжения на зондах. Величина
напряжения на зондах изменяется с~помощью дискретного
переключателя <<$V$>> выходного напряжения источника питания и потенциометра
$R$, а измеряется вольтметром~$V_2$. Для
измерения зондового тока используется микроамперметр~$A_2$.
Анод-III в нашей работе не используется.

\begin{lab:task}

\taskpreamble{В~работе предлагается измерить вольт-амперную характеристику тлеющего разряда и
зондовые характеристики при разных токах
разряда. По результатам измерений рассчитать концентрацию и температуру
электронов в плазме, степень ионизации, плазменную частоту и 
дебаевский радиус экранирования.}
    
\tasksection{Вольт-амперная характеристика разряда}

\item Установите переключатель~$\text{П}_1$ в положение <<Анод-I>> и подготовьте
приборы к работе согласно техническому описанию установки. 
Плавно увеличивая выходное напряжение ВИП, определите напряжение
зажигания разряда $U_{заж}$ (показания вольтметра~$V_{1}$ непосредственно 
\emph{перед} зажиганием).

\item С помощью вольтметра~$V_{1}$ и амперметра~$A_{1}$ измерьте вольт-амперную
характеристику разряда $I_{р}(U_{р})$. Следите, чтобы ток разряда~$I_\text{р}$ 
изменялся в диапазоне, указанном в техническом описании работы 
(при больш\'{и}х токах может сгореть сопротивление).

Измерения проведите как при нарастании, так и при убывания тока.
Обратите внимание, что при уменьшении $U_{р}$ разряд гаснет при
напряжениях, меньших напряжения зажигания $U_{р} < U_{заж}$.

\tasksection{Зондовые характеристики}

\item Уменьшите напряжение ВИП до нуля, переведите переключатель~$\text{П}_1$
в~положение <<Анод-II>> и подготовьте приборы~$\text{П}_{2}$, $V_{2}$ и~$A_{2}$
к работе согласно техническому описанию.

\item Плавно увеличивайте напряжение ВИП до возникновения разряда. Установите
максимальное значение разрядного тока~$I_\text{р}$ согласно техническому описанию
($\sim$\,5~мА).
Подготовьте к работе источник питания. После этого при помощи
потенциометра~$R$ установите на зонде максимальное напряжение $U_{з}^{\rm max}$,
указанное в техническом описании ($\sim$ 25~В).

\item Измерьте вольт-амперную характеристику двойного зонда $I_{з}(U_{з})$
(в диапазоне от~$-U_{з}^{\rm max}$ до~$U_{з}^{\rm max}$) 
при фиксированном токе разряда~$I_\text{p}$ (согласно техническому описанию работы).
Вблизи $U_{з}\approx 0$ точки должны лежать чаще.

В процессе измерений необходимо следить за тем, чтобы ток разряда~$I_{р}$ 
не изменялся. При нулевом токе через зонд $I_{з}=0$ необходимо
переключить полярность подключения зонда (переключатель П$_2$).

%\item Повторите измерения при другой полярности (переключатель~$\text{П}_2$).
%Менять полярность подключения зондов можно только при \important{нулевом токе},
%поддерживая при этом величину тока разряда~$I_\text{p}$ в трубке.

При измерениях рекомендуется одновременно строить приближенный график
$I(U)$ в тетради.
%Отцентрируйте кривую: проведите ось абсцисс на 
%уровне $I=\sum \Delta I/2$, восстановите ось ординат из точки 
%пересения кривой с новой осью абсцисс. 
Убедитесь, что можно
провести асимптоты к участкам кривой при больших напряжениях. Если точек мало,
необходимо провести дополнительные измерения.

\item Получите зондовые характеристики еще при 3-4 токах разряда
$I_{р}$ (согласно
техническому описанию).

\item Переведите ручки регулировки источников питания
в минимальное положение и отключите их. Запишите параметры установки. 

\tasksection{Обработка результатов}

\item Постройте вольт-амперную характеристику разряда 
в координатах~$I_\text{p}(U_{р})$.
По наклону кривой определите максимальное дифференциальное сопротивление разряда
$R_{диф} = \frac{dU}{dI}$. Сравните график с рис.~\ref{fig:5.2.Neon discharge VAC}
Приложения к разделу. Какому участку ВАХ газового разряда соответствует полученный в
работе график?

\item Отцентрируйте зондовые характеристики $I(U)$ и постройте их семейство 
 на одном графике.

\item По каждой зондовой характеристике определите 
а) ионный ток насыщения $I_{i\text{н}}$ по пересечению 
асимптот к графику, проведенных при $U_{з}\to \pm U_{з}^{\rm max}$,
с осью ординат (см. рис.~\chapterfigref{Double probe VAC});
б) наклон характеристики $\left.dI/dU\right|_{U=0}$ в начале координат.

\item По результатам предыдущего пункта рассчитайте а) температуру 
электронов $T_e$, ответ выразите в энергетических
единицах (в электрон-вольтах); б) концентрацию электронов и ионов
в плазме. Ионы считайте однозарядными ($Z=1$).
Площадь поверхности зонда (в пренебрежении
дебаевским слоем): $S\approx \pi d l$, где 
$d$~--- диаметр, $l$~--- длина зонда (указаны на установке).

\item Рассчитайте плазменную частоту колебаний электронов $\omega_p$,
электронную поляризационную длину~$r_{De}$ и дебаевский радиус экранирования~$r_D$
(с учётом того, что температура ионов мала по сравнению с электронной: $T_i\ll T_e$). 
Можно ли считать плазму квазинейтральной?

\item Оцените среднее число ионов в дебаевской сфере $N_D$. 
Является ли плазма разряда идеальной?
%по формуле \chaptereqref{plasma-freq}.
% \begin{equation}
% 	\omega_{p}=\sqrt{\frac{n_{e} e^2}{\varepsilon_{0} m_{e}} }.
% \end{equation}

\item Оцените степень ионизации плазмы (долю ионизованных атомов $\alpha$),
если давление в трубке $P\sim 2$~торр.

\item Постройте графики зависимостей электронной температуры и
концентрации электронов от тока разряда $T_e(I_р)$, $n_e(I_р)$.
Проанализируйте и попытайтесь объяснить полученные зависимости.

%\item Рассчитайте дебаевский радиус экранирования~$r_{D}$ по формуле
%\chaptereqref{debye-general},
%которая в случае $T_{e}\gg T_{i}$ принимает вид
%\begin{equation*}
%	r_{D}=\sqrt{\frac{\kB T_{i}}{4\pi ne^{2}}}.
%\end{equation*}
%Сравните ответ с электронным дебаевским радиусом $r_{De}$
%из формулы \chaptereqref{debye-rad}. Какие процессы характеризует каждая
%из этих величин?

%\item  по формуле:
%\begin{equation*}
%	N_{D}=\frac{4\pi}{3} n_{i}r_{D}^{3}.
%\end{equation*}




\item Оцените погрешности эксперимента.

\end{lab:task}


\begin{lab:questions}
    \item Дайте определение понятия <<плазма>>.
    Назовите различные виды плазм в лаборатории, приложениях и природе.
    
    \item Что такое дебаевская длина экранирования? Выведите формулу
    для дебаевской длины в одномерном случае для равновесной плазмы.
    
    \item Что такое плазменная частота? Выведите формулу для плазменной частоты.
    Какие процессы в плазме характеризуются плазменной частотой?
    
    \item Что такое равновесная и неравновесная плазма? 
    Является ли исследуемая в работе плазма равновесной?
    
   \item Чем определяется потенциал зонда, погружённого в плазму?
   Чем определяется зондовый ток насыщения для одиночного зонда? Для двойного
   зонда?
    
    
    \item Опишите качественно основной механизм зажигания тлеющего разряда.
\end{lab:questions}


