\introsection{Введение}

Как известно, вещество может находиться в трёх агрегатных состояниях~--- твёрдом,
жидком и газообразном, причём эти
состояния последовательно сменяются по мере возрастания температуры. Если~и
дальше нагревать газ, то сначала молекулы диссоциируют на атомы, а~затем и атомы
распадаются на электроны и ионы, так что газ становится \emph{ионизованным},
представляя собой смесь из свободных электронов и ионов, а~также нейтральных
частиц. Если \emph{степень ионизации} газа
(отношение числа ионизованных атомов к их полному числу) оказывается достаточно велика, то
такой газ может обладать качественно новыми свойствами.
Поведение заряженных частиц приобретает \emph{коллективный} характер, 
так что описание свойств среды не может быть 
сведено к обычному газу, содержащему некоторое количество отдельных заряженных частиц.
Такое состояние ионизованного газа называется \term{плазмой}.
Плазму называют также четвёртым состоянием вещества.
Более точное количественное определение этого понятия будет дано ниже.

Из характерных свойств плазмы можно выделить высокую электропроводность и
\term{квазинейтральность}. Ввиду наличия большого числа подвижных
заряженных частиц плазма, в противоположность нейтральному газу, сильно
взаимодействует с электрическим и магнитным полями.
При этом частицы в плазме стремятся распределиться в пространстве таким образом,
чтобы средняя плотность заряда была равна нулю. Равенство концентраций
положительных и отрицательных частиц нарушается, как правило,
лишь в микроскопических масштабах из-за тепловых флуктуаций.

Первое описание газовой плазмы дал И.~Ленгмюр (1923~г.), исследуя электрический
разряд в газе низкого давления (\emph{тлеющий разряд}). Он назвал плазмой <<ярко
светящийся газ, состоящий из электронов, ионов разных сортов и нейтральных
атомов и молекул>>. Он же ввёл сам термин~--- плазма 
(от греческого глагола, обозначающего <<разрыхляться>>, <<расползаться>>).

Свечение плазмы, являющееся при относительно невысоких температурах 
следствием непрерывно идущей рекомбинации электронов и ионов в нейтральные атомы, 
сопровождается выделением энергии и уменьшением концентрации электронов и ионов. 
Стационарное состояние плазмы может существовать лишь при наличии непрерывно 
действующего источника энергии. Им может быть электрический разряд в газе 
(газоразрядная плазма),
происходящий в постоянном электрическом поле (обычный газовый разряд,
дуга и т.~д.) или в высокочастотном поле (индукционные катушки,
запитанные током высокой частоты электроды и т.~д.).
Плазма может образовываться и при \emph{термической} ионизации газа, 
если газовая среда поддерживается при достаточно высокой температуре 
(пламя газовой горелки). Плазма образуется в фокальной области мощных лазерных 
установок и при многих других условиях. 
Звёздная плазма существует за счёт выделения энергии в реакциях 
\emph{ядерного синтеза}.

Практическое применение плазмы чрезвычайно многообразно.
Например, низкотемпературная плазмы применяется 
для проведения химических реакций, которые в горячей
сильно ионизованной газовой среде происходят очень быстро и эффективно.
Методы плазменного травления применяются при создании интегральных
микросхем. Плазма исследуется также в связи с проблемой 
создания магнитогидродинамических генераторов~--- преобразователей
механической энергии движущегося в магнитном поле проводящего газа в
электрическую энергию.
Большой интерес представляет плазма, существующая в атмосфере Земли и планет, а
также в космосе. Атмосферная плазма создаётся ультрафиолетовым излучением
Солнца. Электроны плазмы захватываются магнитным полем Земли (движутся вокруг и
вдоль силовых линий магнитного поля) и образуют радиационные пояса на
расстояниях тысяч километров от поверхности Земли. Широко известны также
плазменные проводящие слои Хевисайда, обеспечивающие дальнюю радиосвязь
на коротких волнах. 
Наконец, физика плазмы является неотъемлемой частью проблемы создания 
\emph{управляемого термоядерного синтеза} (УТС).


В \term{низкотемпературной} плазме ($T\lesssim 10^4$~К) степень ионизации 
атомов обычно невелика. Например, в тлеющем газовом разряде 
люминесцентной лампы коцентрация электронов составляет 
$n_e\sim 10^9\;\text{см}^{-3}$,
а концентрация нейтральных молекул $n_0\sim 10^{14}\;\text{см}^{-3}$.
Лишь внутри звёзд и в установках, используемых для исследования проблем 
УТС, где температуры достигают значений $T \sim 10^{6}\;К$ и более
(\term{высокотемпературная} плазма),
доля ионизованных атомов приближается к единице.

\begin{figure}[ht]
    \centering
    {\footnotesize
    \pic{\textwidth}{Chapter_5/v5_0}}
    \caption{Различные типы плазмы в лаборатории и природе. Температура
    указана в энергетических единицах ($1\;эВ\approx 11\,600\;К$)}
    \figmark{Types of plasma}
\end{figure}

Стационарное состояние плазмы зачастую является \term{неравновесным}.
При этом компоненты плазмы (электроны и ионы), как
правило, имеют разную температуру: $T_e\ne T_i$. 
Такое оказывается возможным, поскольку 
электроны и ионы имеют существенно разные массы ($m_e \ll m_i$).
Поэтому при электрон-ионных столкновениях обмен энергией идёт гораздо медленнее,
чем при столкновениях частиц одного сорта (электрон-электронных и ион-ионных).
Например, в тлеющем газовом разряде обычно имеются 
<<горячие>> электроны и <<холодные>> ионы: $T_e \gg T_i$.
Это связано с тем, что в сильно разреженном газе электроны ускоряются 
внешним электрическим полем, почти не теряя энергии при соударениях 
с ионами и атомами газа или стенками сосуда, --- в результате электроны
нагреваются до высоких температур $T_e\sim 10^4\;К$.
Ионы же, напротив, быстро отдают полученную от поля и от электронов энергию 
нейтральным атомам газа и атомам стенок, поскольку массы их близки, 
--- поэтому их температура оказывается порядка комнатной, $T_i \sim 300\;К$.


Свойствами, характерными для газовой плазмы, обладают и некоторые другие среды,
называемые по этой причине \emph{плазмоподобными} средами, или просто <<плазмами>>.
В~качестве примеров можно назвать плазму металлов,
плазму электролитов, электронно-дырочную плазму полупроводников, 
нуклонную плазму атомного ядра и др.
На рис.~\figref{Types of plasma} на плоскости параметров
$T$\,--\,$n$ (температура плазмы~--- плотность числа частиц) 
представлены различные типы плазм, встречающихся как в лабораторных условиях,
так и в природе.

% Под температурой плазмы в каждом конкретном случае понимают температуру тех
% заряженных частиц, которые определяют плазменные свойства рассматриваемой среды:
% в большинстве случаев это электроны.

Основные свойства газовых разрядов подробно рассмотрены
в Приложении к разделу (см. стр.~\pageref{sec:discharge}).

\introsection{Основные характеристики плазмы}
\label{sec:plasma}

Определяющими свойствами плазмы являются \emph{коллективный} характер её движения
и \emph{квазинейтральность} (равенство нулю средней плотности заряда).
Рассмотрим простейший вид коллективных плазменных колебаний. 

\emph{Здесь и далее
в этом разделе будем использовать систему СГС, как это принято в физике плазмы.}

\introsubsection{Плазменная частота}

\begin{wrapfigure}{o}{0.45\textwidth}
    \centering
    \pic{0.9\linewidth}{Chapter_5/v5_1}
    \caption{Плазменные колебания}
    \figmark{1}
\end{wrapfigure}

Выделим в нейтральной плазме некоторый объём в виде параллелепипеда
(см. рис.~\figref{1}).
Обозначим концентрацию электронов как $n_e$; ионы для простоты будем считать
однозарядными ($Z=1$), тогда их концентрация такая же, как у электронов: $n_i=n_e$.
Предположим, что все электроны сместились на расстояние $x$ относительно ионов.
Ионы как существенно более тяжёлые частицы можно считать неподвижными.
% (ионы занимают объём, изображённый сплошными, а электроны~--- пунктирными линиями).
В результате на боковых гранях параллелепипеда возникнут нескомпенсированные
поверхностные заряды с плотностью
\begin{equation*}
%     \eqmark{5.13}
    \sigma = \pm n_e e \Delta x.
\end{equation*}
Эти заряды --- как две пластины конденсатора --- создадут электрическое поле
\begin{equation*}
%     \eqmark{5.14}
    E=4\pi n_e e \Delta x.
\end{equation*}
В свою очередь это поле будет действовать на электроны,
придавая им ускорение, равное
\begin{equation*}
%     \eqmark{5.15}
    \frac{d^2\Delta x}{dt^2}=-\frac{eE}{m}=-\frac{4\pi n_e e^2}{m} \Delta x.
\end{equation*}
Видно, что полученное уравнение описывает гармонические колебания с частотой
\begin{equation}
    \eqmark{plasma-freq}
    \omega_p=\sqrt{\frac{4\pi n_e e^2}{m_e}}.
\end{equation}

Таким образом, мы получили частоту коллективных колебаний
электронов относительно квазинейтрального состояния. Такие колебания
называют \term{ленгмюровскими}, а частоту $\omega_p$ ---
\term{плазменной} или \term{ленгмюровской}. Эта частота ---
один из важнейших параметров плазмы.
Она определяет характерный \emph{временной} масштаб для плазмы --- время
отклика на флуктуацию плотности заряда в ней. Частота $\omega_p$
определяет многие физические процессы, включая распространение 
электромагнитных волн в плазме.

% Для рассчётов можно использовать практическую формулу
% \begin{equation}
%     \eqmark{omegap-practical}
%  \omega_p =5,65\cdot10^4\sqrt{n[\text{см}^{-3}]}\;рад/с.
% \end{equation}

\introsubsection{Дебаевский радиус}\label{sec:debye_rad}

Плазменные колебания могут быть возбуждены как за счёт внешнего воздействия
(например, при прохождении электромагнитной волны), так и за счёт
тепловой энергии, содержащейся непосредственно в плазме.
Оценим амплитуду колебаний в последнем случае.

Средняя скорость теплового движения электронов 
по порядку величины равна $\bar{v}_e\sim \sqrt{\kB T_e/m_e}$, 
где $T_e$~--- температура электронов. Амплитуду $r$ колебаний 
электронов относительно ионов оценим как смещение с тепловой 
скоростью $\bar{v}_e$ за характерное время плазменных колебаний $1/\omega_p$:
$r = \bar{v}_e / \omega_p$.
Полученную величину обозначают как
\begin{equation}
\eqmark{debye-rad}
r_D=\sqrt{\frac{\kB T_e}{4\pi n_e e^2}}\sim \frac{\bar{v}_e}{\omega_p},
\end{equation}
и называют \term{дебаевским радиусом} 
(или \term{дебаевской длиной}). Это ещё один важный плазменный параметр, 
задающий характерный \emph{пространственный} масштаб многих плазменных явлений.

%Оценим амплитуду колебаний электронов относительно ионов,
%возникающих за счёт тепловых флуктуаций.
%Пусть облако электронов, рассмотренное в предыдущем пункте, 
%колеблется с некоторой амплитудой $x_0$ и частотой $\omega_p$.
%
%Как известно из теории колебаний, амплитуда скорости равна $v_0 = \omega_p x_0$,
%а полная энергия колебаний равна максимальному значению кинетической энергии.
%Тогда в расчёте на единицу объёма среды имеем энергию 
%$w= \frac12 n_e m_e (\omega_p x_0)^2$.
%С другой стороны, из термодинамики известно, что эта энергия должна
%быть равна тепловой энергии, приходящейся на одну степень свободы
%$W_{Т} = \frac12 \kB T_e$. Отсюда находим $x_0^2 = \frac{\kB T_e/m_e}{\omega_p^2}$.


Из рассмотренного примера видно, что дебаевская длина есть амплитуда ленгмюровских колебаний,
возбуждаемых тепловыми флуктуациями. Она задаёт масштаб, на котором возможно
спонтанное нарушение квазинейтральности плазмы.
%Заметим также, что формулу \eqref{debye-rad} можно переписать как
%где $v_{Te}=\sqrt{\kB T_e/m_e}$~--- скорость, по порядку величины равная
%средней тепловой скорости движения электронов.

Таким образом, плазменная частота $\omega_p$ и дебаевская длина $r_D$
есть две важнейшие характеристики плазмы, определяющие временной 
и пространственный масштабы коллективного движения электронов
относительно ионов.

% Как следует из \eqref{plasma-freq}, плазменная частота определяется только плотностью
% электронов (и универсальными постоянными).
% Можно строго доказать, что она не зависит от формы рассматриваемого возмущения и
% является, таким образом,
% локальной характеристикой плазмы. Плазменная частота является не
% единственной~--- но важнейшей~--- характерной частотой
% плазмы. Она определяет коллективное движение электронов относительно ионов.

\introsubsection{Плазменное экранирование}

Рассмотрим ещё одну задачу, в которой дебаевская длина играет роль
ключевого параметра.

\begin{wrapfigure}{o}{0.3\textwidth}
    \centering
    \pic{\linewidth}{Chapter_5/v5-screen1}
    \caption{Упрощенная геометрия задачи об экранировании заряда}
\end{wrapfigure}

Поместим в равновесную плазму с температурой $T$ ($T_e=T_i$) некоторую пробную частицу, 
имеющую фиксированный
положительный заряд~$+q$ и найдём, как распределятся плазменные частицы вокруг неё.
Будем считать, что частица достаточно массивна, так что её можно
считать неподвижной (в качестве такого пробного заряда можно рассмотреть
один из ионов плазмы, поскольку $m_i \gg m_e$).

Заряд будет притягивать к себе плазменные электроны, в результате чего
вокруг него образуется отрицательно заряженное <<облако>>,
\emph{экранирующее} поле заряда на большом расстоянии от него ---
электрическое поле вокруг~$q$ будет убывать с расстоянием~$r$
не по закону $q/r^2$, а существенно быстрее.
Если бы электроны не имели кинетической энергии, то они так <<облепили>>
бы пробный заряд, так что его собственное поле было бы полностью скомпенсировано.
Тепловое движение мешает такой компенсации.

Чтобы наглядно выявить характерные особенности решения данной задачи,
предположим, что радиус~$r_0$ пробной частицы велик (по сравнению с $r_D$)
и будем рассматривать распределение поля вблизи её поверхности ---
так мы сведём задачу к одномерной, сильно упростив выкладки, но не потеряв
качественные особенности решения.

Пространственное распределение электронов в равновесии подчиняется
\important{закону Больцмана}:
\begin{equation}
    \eqmark{5.5}
    n_e=n_{e0} \cdot \exp\left(\frac{e\varphi}{\kB T}\right)
\end{equation}
Здесь $\varphi$~--- потенциал электростатического поля,
$n_{e0}$ --- концентрация электронов вдали от заряда, где $\varphi\to 0$.
Аналогичное соотношение можно записать и для ионов с заменой $-e\to e$
(по-прежнему считаем, что $Z=1$).
% \begin{equation}
%     \eqmark{5.5}
%     n_i(r) \approx n_0 \left(1-\frac{e\varphi}{\kB T}\right).
% \end{equation}
Будем считать, что температура электронов в плазме достаточно велика, так что
можно положить
\begin{equation*}
\frac{e\varphi}{\kB T}\ll 1.
\end{equation*}
Раскладывая больцмановскую экспоненту в ряд по этому малому параметру,
$e^{\pm e\varphi/\kB T}\approx 1 \pm e\varphi/\kB T$,
найдём объёмную плотность заряда:
\begin{equation}
\eqmark{rho_ei}
\rho = -en_e + en_i \approx -en \cdot \frac{e\varphi}{\kB T},
\end{equation}
где $n=n_{e0}+n_{i0}=2n_{e0}$~--- полная концентрация заряженных частиц в плазме
вдали от $q$.

С другой стороны, распределение потенциала $\varphi(x)$ в области 
$x>0$ однозначно связано с распределением плотности 
электрического заряда~$\rho(x)$.
Применяя \important{теорему Гаусса} в дифференциальной форме
\begin{equation*}
\frac{dE}{dx}= 4\pi \rho,
\end{equation*}
и пользуясь определением потенциала электростатического поля
$E = - \frac{d\varphi}{dx}$, получим
\begin{equation}
    \eqmark{poisson-1d}
    \frac{d^2\varphi}{dx^2} = - 4\pi \rho.
\end{equation}
Уравнение \eqref{poisson-1d} представляет собой частный (одномерный)
случай \important{уравнения Пуассона}.

Объединяя \eqref{poisson-1d} и \eqref{rho_ei}, получим окончательно
дифференциальное уравнение для потенциала поля $\varphi(x)$ вблизи пробной частицы:
\begin{equation}
    \frac{d^2\varphi}{dx^2} = \frac{\varphi}{r_D^2},
\end{equation}
где $r_D$ определяется соотношением, аналогичным \eqref{debye-rad}:
\begin{equation}
\eqmark{deb_rad}
r_D = \sqrt{\frac{\kB T}{4\pi n e^2}}.
\end{equation}
%в котором вместо
%$n_e$ стоит полная концентрация частиц в плазме $n=n_{e0}+n_{i0}$.
Решение, удовлетворяющее граничным условиям
$\varphi(\infty)=0$ и~$\varphi(0)=\frac{q}{r_0}$, есть%
\begin{equation}
\varphi(x) = \frac{q}{r_0} e^{-\tfrac{x}{r_D}}.
\end{equation}
\begin{lab:note}
Решая уравнение Пуассона в сферических координатах,
можно показать, что распределение потенциала 
вокруг неподвижного пробного точечного заряда 
(что применимо также к отдельному иону) будет следующим:
    \begin{equation*}
    \varphi(r) = \frac{q}{r} e^{-\tfrac{r}{r_D}}.
    \end{equation*}
\end{lab:note}


Таким образом, потенциал поля и его напряжённость,
а также концентрация плазменных частиц изменяются при удалении от
пробного заряда по экспоненциальному закону с характерной длиной порядка
дебаевского радиуса~$r_D$. На расстояниях, превышающих~$r_D$ в несколько раз,
плазму можно считать квазинейтральной, а поле заряда~$q$ практически
полностью экранированным. В связи с этим дебаевскую длину~\eqref{deb_rad} также называют
\term{радиусом экранирования}.

\begin{figure}[ht]
    \centering
    \pic{0.5\textwidth}{Chapter_5/v5-screen2}
    \caption{Схематичное распределение потенциала (сплошная)
        и плазменных зарядов (пунктиры) и вблизи стороннего
        положительного заряда}
\end{figure}

\paragraph{Экранирование в неравновесной плазме}
В заключение отметим одно обстоятельство, касающееся экранирования
зарядов в неравновесной плазме.
Когда температуры электронов~$T_e$ и ионов~$T_i$ различны, можно определить 
две характерные длины --- \emph{электронную} и 
\emph{ионную}:
\begin{equation}
\eqmark{rde-rdi}
r_{De} = \sqrt{\frac{\kB T_e}{4\pi n_e e^2}},
\qquad r_{Di} = \sqrt{\frac{\kB T_i}{4\pi n_i e^2}}.
\end{equation}
Повторив изложенный выше вывод с учётом различия температур
компонентов,
нетрудно получить \emph{радиус экранирования поля стороннего заряда} 
в общем случае:
\begin{equation}
\eqmark{debye-general}
r_{D} = \left(\frac{1}{r_{De}^2} + \frac{1}{r_{Di}^2}\right)^{-1/2} = 
\sqrt{\frac{\kB}{4\pi n_e e^2}\frac{T_eT_i}{T_e+T_i}}.
\end{equation}

Например, в плазме тлеющего газового разряда $T_e\gg T_i$.
Тогда $r_{De}\gg r_{Di}$ и $r_D\approx r_{Di}$, то есть радиус 
экранирования потенциала электрода,
помещённого в плазму, определяется температурой холодных ионов~$T_i$.
При этом стоит подчеркнуть, что рассуждения п. \ref{sec:debye_rad}, 
остаются в силе:
масштаб, на котором \emph{нарушается квазинейтральность} плазмы из-за тепловых
флуктуаций электронов относительно ионов, определяется именно 
\emph{электронной} дебаевской длиной~$r_{De}$, то есть зависит от температуры 
горячих электронов~$T_e$.
 Чтобы разделить эти два не всегда сопадающие понятия,
электронный дебаевский радиус (левая формула~\eqref{rde-rdi}) иногда
 называют электронной \term{поляризационной длиной}.

\introsubsection{Идеальная и неидеальная плазма}

Теперь можно дать \important{количественное} определение понятию плазма.
%(это определение также принадлежит Ленгмюру).

\term{Плазмой} называется ионизованный газ, дебаевский радиус которого
    $r_D$ существенно меньше характерного размера области $a$, занимаемой этим газом:
\begin{equation*}
	\sqrt{\frac{\kB T}{4\pi ne^2}}\ll a.
\end{equation*}
Именно при $r_D \ll a$ поведение среды носит существенно \emph{коллективный} характер.
В противном случае среда может рассматриваться просто как газ
с примесью индивидуальных заряженных частиц.

Оценим \emph{энергию кулоновского взаимодействия} частиц в плазме.
Распределение потенциала вокруг иона с зарядом~$q$
определяется законом экранирования
\begin{equation*}
\varphi = \frac{q}{r} e^{-r/r_D}.
\end{equation*}
Вычитая потенциал самого иона $\varphi_0=\frac{q}{r}$ (для простоты
считаем ионы однозарядными), найдём
потенциал <<экранирующего облака>>. При $r\lesssim r_D$ имеем
\begin{equation*}
\varphi-\varphi_0 = \frac{q}{r}\left( e^{-r/r_D} - 1\right)
\approx - \frac{q}{r_D}.
\end{equation*}
Воспользуемся известной формулой для электростатической энергии системы
зарядов $\varepsilon=\frac12 \sum_i \varphi_i q_i$, где $\varphi_i$ --- потенциал
в точке нахождения заряда $q_i$. Суммируя по всем ионам,
запишем плотность энергии кулоновского взаимодействия зарядов в плазме:
\begin{equation}
w_{кул} \approx -\frac12 n_i \frac{q^2}{r_D}.
\end{equation}

Сравним полученную энергию с тепловой $w_{тепл} \sim n_i \kB T$:
\begin{equation}
\frac{w_{тепл}}{w_{кул}} \sim
\frac{\kB T r_D}{q^2} = 4\pi n_i r_D^3.
\end{equation}
Видно, что отношение тепловой и кулоновской энергии в плазме по порядку величины
есть число заряженных частиц в сфере радиуса $r_D$:
\begin{equation}
N_D = \frac43 \pi n_i r_D^3.
\end{equation}


Плазму называют \term{идеальной}, если энергия кулоновского взаимодействия
мала по сравнению с тепловой. Видно, что это выполняется, если число частиц
в <<дебаевской сфере>> велико, $N_D\gg 1$. Идеальная плазма во многом подобна
по своим свойствам идеальному газу. В неидеальной плазме ($N_D\lesssim 1$)
взаимодействие между частицами велико, так что она становится в некотором
смысле подобна жидкости, а её описание значительно усложняется.


\begin{lab:example}  
    Характерные параметры тлеющего разряда: $T_e\sim 10^4$~К ($\approx 1$~эВ), 
    $T_i=300\;К$, $n_i=n_e\sim 10^{10} ~\text{см}^{-3}$ 
    (давление газа $10^{-2}$~торр, степень ионизации~$10^{-4}$). Для такой плазмы
    имеем электронную дебаевскую (поляризационную) длину $r_{De}\approx 7\cdot10^{-3}$~см,
    и радиус экранирования $r_{D}\approx 10^{-3}$~см.
    Число частиц в дебаевской сфере
    $N_D = \frac{4}{3}\pi n r_D^3 \sim 40$, то есть плазму
    с удовлетворительной точностью можно считать идеальной.
    % Таким образом, плазму можно считать почти нейтральной (квазинейтральной)
    % в областях, размеры которых существенно превосходят дебаевскую длину.
\end{lab:example}

% Данный раздел имеет мало общего с реальностью. Длина свободного пробега
% электронов в плазме -- отдельный большой вопрос.
% Кроме того, данные формулы нигде не используются // ППВ

% \introsection{Электропроводность плазмы}
%
% Приложим к плазме электрическое поле с напряжённостью $\vec{E}$. Под его
% действием приходят в движение как электроны, так и
% ионы. Действующие на них силы мало отличаются друг от друга, а массы различаются
% очень сильно. Основными носителями тока
% являются поэтому электроны. Свободно двигаясь на пути свободного пробега,
% электроны приобретают направленную (дрейфовую)
% скорость. После очередного соударения скорость электрона может иметь самые
% разные направления, так что среднее значение
% этой скорости в начале пробега близко к нулю. В конце пробега оно равно
% \begin{equation*}
% 	\vec{v}_\text{кон}=-\frac{e\lambda}{m_e\average{v_e}}\vec{E},
% \end{equation*}
% где $\lambda$~--- длина свободного пробега, а $\average{v_e}$~--- тепловая
% скорость электрона, по сравнению с
% которой дрейфовая скорость обычно мала. Среднее значение дрейфовой скорости
% равно поэтому половине $v_\text{кон}$:
% \begin{equation}
% 	\eqmark{5.19}
% 	v_\text{др}=\frac{e\lambda E}{2m_e\average{v_e}}.
% \end{equation}
%
% Средняя тепловая скорость $\average{v_e}$ определяется из обычной формулы:
% \begin{equation}
% 	\eqmark{5.20}
% 	\average{v_e}=\sqrt{\frac{8\kB T_e}{\pi m_e}}.
% \end{equation}
%
% Объединяя эти формулы, найдём
% \begin{equation}
% 	\eqmark{5.21}
% 	\vec{v}_\text{др}=-b\vec{E},
% \end{equation}
% где подвижность электронов $b$ равна
% \begin{equation}
%  	\eqmark{5.22}
% 	b=\frac{e\lambda}{2\sqrt{\frac{8m_e}{\pi} \kB T_e}}.
% \end{equation}
%
% Электропроводность плазмы $\sigma$ определяется совместным дрейфовым движением
% всех электронов, так что
% \begin{equation}
% 	\eqmark{5.23}
% 	\sigma=\frac{j}{E}=\frac{n_eev_{др}}{E}=neb=\frac{e^2\lambda
% n_e}{2\sqrt{\frac{8m_e}{\pi} \kB T_ei}}.
% \end{equation}
%
% Полученная формула показывает, что электропроводность плазмы пропорциональна
% концентрации электронов и уменьшается с
% ростом температуры плазмы. Длина свободного пробега $\lambda$ в
% слабоионизированной плазме определяется не столько
% плотностью электронов $n_e$, сколько плотностью газа.

\introsubsection{Диэлектрическая проницаемость плазмы}
\label{sec:epsilon}
Рассмотрим плазму, помещённую в переменное электрическое поле, меняющееся
по гармоническому закону: $\vec{E}(t)=\vec{E}_0 \cos (\omega t)$. 
Это поле вызывает разделение зарядов в плазме и, соответственно, её \emph{поляризацию}.
Поскольку ионы значительно тяжелее электронов, можно считать, что
под действием поля смещаются только электроны. Электропроводность
плазмы, как правило, очень велика, поэтому столкновениями электронов
с ионами можно пренебречь. Тогда уравнение движения электронов можно записать как
\begin{equation}
\eqmark{free_E}
m_e \frac{d^2 \vec{r}}{dt^2} = - e \vec{E}_0 \cos(\omega t).
\end{equation}
Интегрируя это уравнение, находим, что электроны будут колебаться по 
гармоническому закону
$\vec{r}(t) = \vec{r}_0 \cos(\omega t)$, где амплитуда колебаний равна
\[
\vec{r}_0 = \frac{e}{m_e\omega^2} \vec{E}_0.
\]

Смещение электронов вызывает поляризацию среды. Амплитуда колебаний 
вектора поляризации $\vec{P}$ (объёмной плотности дипольного момента среды) будет равна
$\vec{P}_0 = -n_e e \vec{r}_0$, что 
с учётом выражения для плазменной частоты \eqref{plasma-freq},
можно записать как
\[
\vec{P}_0 = -\frac{\omega_p^2}{\omega^2} \frac{\vec{E}_0}{4\pi}.
\]
Диэлектрическая проницаемость есть коэффициент пропорциональности
между векторами напряжённости $\vec{E}$ и индукции $\vec{D}=\vec{E}+4\pi \vec{P}$.
Таким образом, для плазмы она равна
\begin{equation}
\eqmark{epsilon-plasma}
\varepsilon = 1 - \frac{\omega_p^2}{\omega^2}.
\end{equation}


\begin{lab:example}
Как известно из оптики, диэлектрическая проницаемость есть
квадрат показателя преломления электромагнитной волны: $\varepsilon=n^2$.
Из \eqref{epsilon-plasma} видно, что при $\omega<\omega_p$ проницаемость
ставится отрицательной, а значит показатель преломления будет формально мнимым.
 На деле это означает, что электромагнитные волны с частотами меньше плазменной
 не могут распространяться в плазме. Этим объясняется, в частности, 
 отражение волн с частотой $\lesssim$\,10 МГц от ионосферы Земли, 
 что используется при осуществлении дальней радиосвязи.
\end{lab:example}


\introsection{Исследование плазмы с помощью зондов}
\label{sec:zonds}

\introsubsection{Плавающий потенциал}
\label{sec:single}

Одним из самых простых методов исследования свойств плазмы является измерение
электрических потенциалов с помощью <<зондов>>~---
небольших проводников, вводимых в плазму.
% Как уже говорилось выше, метод зондов был разработан Ленгмюром в начале
% двадцатых годов XX века.

При внесении проводника в плазму, он подвергается <<бомбарировке>>
со стороны её заряженых частиц. Как известно из молекулярной физики,
число частиц, ударяющихся в идеальном газе в секунду о единичную поверхность,
равно
\begin{equation}
    \eqmark{nv4}
j = \frac14 n\bar{v},
\end{equation}
где $n$ --- концентрация частиц, $\bar{v} = \sqrt{\frac{8\kB T}{\pi m}}$ --- 
их средня тепловая скорость. Поскольку $m_e \ll m_i$, тепловые скорости электронов
обычно существенно превосходят скорости ионов: $\bar{v}_e\gg \bar{v}_i$. Поэтому проводник, внесенный
в плазму, в равновесии \emph{зарядится отрицательно}.
Отрицательный потенциал $-U_f$ (относительно плазмы),
до которого заряжается помещённый в неё зонд,
называют \term{плавающим потенциалом}.
При этом вокруг отрицательно заряженного зонда
образуется область положительного пространственного заряда,
экранирующего плазму от зонда (рис.~\figref{Potential distribution}).
Протяженность этой области --- порядка дебаевского радиуса экранирования.

\begin{figure}[h!]
    \centering
    \pic{}{Chapter_5/v5_8}
    \caption{Распределение потенциала в~окрестности зонда}
    \figmark{Potential distribution}
\end{figure}


Найдём связь $U_f$ с параметрами плазмы.
Если бы потенциал зонда был равен потенциалу плазмы ($U_f=0$), то согласно \eqref{nv4} 
электронный и ионный токи были бы равны соответственно
\begin{equation}
    \eqmark{5.24}
    I_{e0}=\frac{n\bar{v}_e}{4}eS,\qquad
    I_{i0}=\frac{n\bar{v}_i}{4}eS,
\end{equation}
где $S$ --- площадь зонда, $n=n_e=n_i$. Эти токи также можно назвать
\emph{тепловыми}. 
Далее учтём наличие разности потенциалов $-U_f$ между плазмой и электродом.
На ионный ток она практически не влияет: $I_i \approx I_{i0}$.
Электронный же ток уменьшится, поскольку лишь часть электронов, летящих к зонду,
способна преодолеть потенциальный барьер. Согласно распределению Больцмана
\begin{equation}
    \eqmark{5.26}
    I_e=I_{e0}\exp\left(-\frac{eU_f}{\kB T_e}\right).
\end{equation}

В равновесии суммарный ток в цепи равен нулю и количество попадающих на зонд ионов и электронов
уравнивается: $I_i=I_e$.
%: до него могут долетать лишь наиболее быстрые электроны и практически все ионы.
%Если суммарный ток в цепи зонда равен нулю,
%то в равновесии токи компенсируются: . 
Тогда из \eqref{5.24} и \eqref{5.26} находим
\begin{equation}
\eqmark{5.27}
U_f=-\frac{\kB T_e}{e}\ln\frac{\bar{v}_e}{\bar{v}_i}=
-\frac12 \frac{\kB T_e}{e}\ln\frac{T_e m_i}{T_i m_e}.
\end{equation}

\begin{lab:example}
В тлеющем газовом разряде $\kB T_e\sim 1\;эВ$, $T_e/T_i\sim 50$, $m_i/m_e\sim 10^4$, откуда
$U_f \approx \frac{1\; В}{2} \ln (50 \cdot 10^4) \approx 6,5\;В$.
% \begin{equation*}
%     \eqmark{5.28}
%     U_f%=\frac12\cdot 1~эВ\ln(40\.10^4)=
%     \approx 6,5~\text{В}.
% \end{equation*}
\end{lab:example}

\begin{lab:note}
Формула \eqref{5.27} даёт правильную оценку по порядку величины, но с количественной
точки зрения её нельзя считать надёжной.
Во-первых существование <<дебаевского слоя>> вокруг зонда вносит некоторую
неопределённость в величину $S$. Если размер зонда значительно превышает
дебаевский радиус, это обстоятельство несущественно; однако если дебаевский
радиус велик, поправка может быть значимой.
Во-вторых, при выводе предполагалось,
что движение ионов у зонда близко к тепловому.
Это справедливо \emph{вдали} от дебаевского слоя зонда,
но не вблизи него (и тем более в нём), где ионы подвергаются
ускоряющему действию довольно сильного электрического поля.

Уточним выражение для ионного тока в состоянии равновесия.
Будем считать, что скорости ионов вблизи зонда определяются не столько 
температурой плазмы, сколько разностью потенциалов между плазмой и зондом:
\begin{equation*}
\eqmark{5.30}
v_i\approx\sqrt{\frac{2eU_f}{m_i}}.
\end{equation*}
При проведении оценки по порядку величины
воспользуемся формулой \eqref{5.27}, 
в которой отбросим численные коэффициенты и несущественный 
логарифмический множитель. Тогда получим
\begin{equation}
\eqmark{I0i}
I_{i\text{0}} = n_i e S v_i \sim n_i eS\sqrt{\frac{\kB T_e}{m_i}}.
\end{equation}
Видно, что поправка будет существенна, 
если плазма неравновесна и~$T_e \gg T_i$.
%На практике требуется вносить поправку в формулу ионного тока~$I_{i0}$
%(см. далее ф-лу \eqref{5.31}).
% Тем не менее для грубых оценок \eqref{5.27} может быть использована.
\end{lab:note}

\introsubsection{Измерения методом одиночного зонда}

Рассмотрим схему исследования параметров плазмы с помощью метода
\emph{одиночного зонда}.
Схема измерений изображена на рис.~\figref{Plasma study with single probe}. 
Два электрода погружены в плазму.
Один из них имеет существенно б\'{о}льшую 
площадь поверхности, контактирующую с плазмой,
--- он выполняет роль \emph{опорного электрода}.
Второй электрод с малой поверхностью и есть наш <<зонд>>.
Расстояние между электродами значительно превышает радиус экранирования,
поэтому можно пренебречь их взаимным влиянием. 

С помощью источника ЭДС $\mathcal{E}$ на электродах 
можно создавать регулируемую разность потенциалов $U$
(регулировка осуществляется потенциометром $R$).
Измеряя ток~$I$ через электроды, можно получить 
вольт-амперную характеристику (ВАХ) зонда $U(I)$,
вид которой зависит от характеристик исследуемой плазмы ---
температуры и концентрации частиц.

\begin{figure}[h]
	\centering
	\pic{}{Chapter_5/v5_9}
	\caption{Исследование плазмы методом одиночного зонда. Пунктиром
        отмечен дебаевский слой вблизи электродов}
	\figmark{Plasma study with single probe}
\end{figure}


Пусть положение потенциометра~$R$ подобрано так, что
ток в цепи и в плазме отсутствует: $I=0$.
Тогда потенциал каждого электрода будет равен потенциалу плазмы 
в точке его размещения за вычетом соответствующего плавающего потенциала $U_f$.
Причём, если температура в плазме постоянна, то и как видно из~\eqref{5.27},
постоянен будет и плавающий потенциал. В таком случае потенциал зонда $U$ 
(относительно опорного электрода) будет равен просто разности потенциалов 
между соответствующими точками плазмы. Перемещая зонд
и поддерживая нулевой ток в цепи, можно измерить 
пространственно распределение электрического поля в плазме.

При изменении потенциала зонда~$U$ по измерительной цепи и через плазму
начнёт протекать ток, так как баланс между электронным и ионным потоками 
на зонд нарушится. 
При этом плотность тока через опорный электрод будет мала,
поскольку его площадь велика, --- поэтому его потенциал относительно плазмы
останется практически всегда равным $-U_f$. При небольшом размере зонда наибольшая
плотность тока возникает около него, так что практически всё падение 
напряжения~$U$ будет приходится на дебаевский слой, окружающий зонд.
 
Зависимость тока через зонд~$I$ 
от потенциала зонда $U$ (для эквипотенциальной плазмы) имеет вид, 
показанный на рис.~\figref{Single probe VAC}. 
Эту кривую называют \term{зондовой характеристикой}.

\begin{figure}[h]
    \centering
    \pic{}{Chapter_5/v5_10}
    \caption{Вольт-амперная характеристика одиночного~зонда}
    \figmark{Single probe VAC}
\end{figure}

Ток зонда равен сумме ионной и электронной составляющих
$I=I_e + I_i$. 
На \emph{левой ветви} характеристики ($U<0$) весь ионный ток,
приходящий на границу дебаевского слоя, достигает зонда.
Ионный ток $I_i$ равен, следовательно, своему максимальному значению
--- \important{ионному току насыщения}~$I_{iн}$.
Электронный ток резко убывает при смещении потенциала 
в сторону отрицательных значений и в пределе $U\to -\infty$ 
прекращается, $I_e\to 0$.

На \emph{правой ветви} характеристики ($U>0$) потенциал зонда превышает
потенциал опорного электрода, но вначале (вплоть до точки~$A$)
остаётся ниже потенциала плазмы ($U<U_f$). При этом ионный ток на зонд
практически не меняется ($I_i\approx I_{iн}$),
а электронный ток возрастает. В точке~$A$~--- при~$U=U_f$~---
слой пространственного заряда (дебаевский слой) исчезает и оба тока~---
электронный и ионный~--- подходят к зонду беспрепятственно.
При этом электронный ток существенно превосходит ионный
поскольку плотности электронов и ионов близки друг к другу, 
а тепловые скорости существенно различаются
($n_i=n_e$, $\overline{v}_e\gg \overline{v}_i$).

%Отметим, что, хотя на первый взгляд, величина ионного тока
%насыщения не должна зависеть от потенциала зонда ($I_{iн}=\const$),
%на самом деле это не так. Дело в том, что при изменении потенциала
%% во-первых, изменяется площадь поверхности дебаевского слоя и, во-вторых,
%изменяются скорости ионов, которые быстро увеличиваются при подлёте иона 
%к электроду~--- от тепловых значений до значений, определяемых 
%величиной потенциала (см.~ниже формулу \eqref{5.31}).
%Поэтому при дальнейшем сдвиге потенциала зонда в сторону отрицательных значений
%ток зонда возрастает, хотя и не очень сильно.



%При дальнейшем увеличении~$U$ ионный ток подавляется, а ток электронов
%достигает насыщения $I\approx I_{eн}$.
%(при этом на самом деле ток насыщения медленно возрастает по тем же причинам,
%по которым изменяется ионный ток насыщения при $U_{з} < 0$).

Участок зондовой характеристики, расположенный слева от точки~$A$, носит 
название \term{ионной} ветви (ионный ток достигает насыщения), 
а участок справа от точки $A$ называется \term{электронной}
ветвью (электронный ток достигает насыщения).

Электронный ток насыщения можно оценить по формуле \eqref{5.24}:
\begin{equation}
\eqmark{Ien}
I_{eн} \approx I_{e0} \approx \frac14 n_e S \sqrt{\frac{8\kB T_e}{\pi m_e}}. 
\end{equation}
Однако для ионного тока аналогичная оценка может оказаться слишком груба,
поскольку скорости ионов вблизи зонда определяются не столько 
температурой плазмы, сколько разностью потенциалов между плазмой и зондом.
Правильнее было бы воспользоваться уточнённой оценкой \eqref{I0i}.
Для количественных расчётов можно использовать 
полуэмпирическую формулу, предложенную Бомом:
%\begin{equation*}
%	\eqmark{5.30}
%	v_i\approx\sqrt{\frac{2eU_f}{m_i}}.
%\end{equation*}
%Для нахождения $U_f$ воспользуемся формулой \eqref{5.27}.
%Отбрасывая численные коэффициенты и логарифмический множитель, 
%можно предложить следующую оценку для ионного тока насыщения:
%\begin{equation}
%\eqmark{5.31x}
%I_{i\text{н}} \sim n_i eS\sqrt{\frac{\kB T_e}{m_i}}.
%\end{equation}
%Для количественных расчётов можно использовать следующую 
%полуэмпирическую оценку, предложенную Бомом:
\begin{equation}
	\eqmark{5.31}
	I_{i\text{н}}\approx 0,4 n_i eS\sqrt{\frac{2\kB T_e}{m_i}}.
\end{equation}

Кроме того, эффект ускорения частиц под действием разности 
потенциалов между плазмой и зондом приводит к тому, что  
при $|U|\gg U_f$ ток насыщения не является постоянным и 
медленно возрастает при $U \to \pm \infty$.
%Структуру этой формулы нетрудно понять, замечая, что, согласно формуле
%\eqref{5.27}, $U_f$ пропорционально $T_e$ (логарифмической
%зависимостью $U_f$ при оценках следует пренебрегать).
% Численный коэффициент в формуле \eqref{5.31} требует более подробных расчётов.


%Отметим также, что величина как элекионного тока
%насыщения не должна зависеть от потенциала зонда ($I_{iн}=\const$),
%на самом деле это не так. Дело в том, что при изменении потенциала
%% во-первых, изменяется площадь поверхности дебаевского слоя и, во-вторых,
%изменяются скорости ионов, которые быстро увеличиваются при подлёте иона 
%к электроду~--- от тепловых значений до значений, определяемых 
%величиной потенциала (см.~ниже формулу \eqref{5.31}).
%Поэтому при дальнейшем сдвиге потенциала зонда в сторону отрицательных значений
%ток зонда возрастает, хотя и не очень сильно.

\begin{lab:note}
Вид выражения \eqref{5.31}, в которое входят температура электронов и масса
ионов, характерен для многих явлений в плазме.
Внешние поля вызывают быстрое перемещение электронов и существенно более
медленное движение ионов. Однако разделение электронов и ионов невозможно,
так как оно нарушило бы квазинейтральность плазмы.
Поэтому движение плазмы как целого определяется массой ионов. 
В~то же время перемещение электронов существенно зависит как от
приложенных полей, так и от электронной температуры. Процессы, которые
определяются параметрами, одни из которых
характерны для электронов (здесь~$T_e$), а другие~--- для ионов (в
рассматриваемой формуле~--- $m_i$), называются \term{амбиполярными}.
\end{lab:note}

% При измерениях с помощью одиночного зонда в качестве опорного электрода часто
% используется анод газоразрядной трубки. Мы
% уже отмечали, что падение напряжения в положительном столбе разряда невелико,
% поэтому разности потенциалов, возникающие
% между анодом и зондом, также оказываются небольшими и легко доступны измерениям.
% Одиночные зонды используются для
% исследования распределения потенциала в плазме, для измерения электронной
% температуры и плотности электронов. Ещё лучше
% делать это с помощью двойных зондов.

\introsubsection{Измерения с помощью двойного зонда}
\label{sec:double}

Двойным зондом называется система, состоящая из двух одинаковых зондов,
расположенных на небольшом расстоянии друг от
друга. Между зондами создаётся разность потенциалов~$U$, которая по величине много
меньше плавающего потенциала $U_f$, $|U|\ll U_f$. При
этом оба зонда имеют относительно плазмы близкий к плавающему отрицательный
потенциал, т.~е. находятся на \important{ионной} ветви
вольт-амперной характеристики (см. выше).

При отсутствии разности потенциалов ток между зондами равен нулю. Рассчитаем
величину тока, проходящего через двойной
зонд вблизи точки $I=0$. При небольших разностях потенциалов ионные токи на оба
зонда равны ионному току насыщения и
компенсируют друг друга. Величина результирующего тока целиком связана с
различием в электронных токах. Пусть потенциал
на первом зонде равен
\begin{equation*}
	\eqmark{5.32}
	U_1=-U_f+\Delta U_1,
\end{equation*}
а на втором
\begin{equation*}
	\eqmark{5.33}
	U_2=-U_f+\Delta U_2.
\end{equation*}
Предполагается, что $\Delta U_1, \Delta U_2 \ll U_f$.
Напряжение $U$ между зондами равно
\begin{equation*}
	\eqmark{5.34}
	U=U_2-U_1=\Delta U_2-\Delta U_1.
\end{equation*}

Найдём ток, приходящий на первый электрод
(см. также \eqref{5.26}):
\begin{equation*}
	\begin{gathered}
        I_1=I_{i\text{н}}-I_{e0}
\exp\left(\frac{eU_1}{\kB T_e}\right)=  \\
	 	=
        I_{i\text{н}}-\left[I_{e0}\exp\left(-\frac{eU_f}{\kB T_e}\right)\right]
\exp\left(\frac{e\Delta U_1} {\kB T_e} \right).
	\end{gathered}
\end{equation*}
Заметим, что при $\Delta U_1=0$ (при $U_1=U_f$) электронный и ионный ток
компенсируют друг друга. Это означает, что
заключённый в квадратные скобки множитель равен $I_{i\text{н}}$. Имеем поэтому
\begin{equation}
	\eqmark{5.35}
	I_1=I_{i\text{н}}\left[1-\exp\left(\frac{e\Delta U_1}{\kB T_e}\right)\right].
\end{equation}
Аналогично для второго электрода
\begin{equation}
	\eqmark{5.36}
	I_2=I_{i\text{н}}\left[1-\exp\left(\frac{e\Delta U_2}{\kB T_e}\right)\right].
\end{equation}

Заметим, что зонды 1 и 2 соединены \important{последовательно}~--- через плазму~---
поэтому $I_1 = - I_2 = I$.
Выразим $\Delta U_1$ и $\Delta U_2$ из \eqref{5.35} и \eqref{5.36}:
\begin{equation*}
	\eqmark{5.38}
	\Delta U_1=\frac{\kB T_e}{e}\ln\left(1-\frac{I}{I_{i\text{н}}}\right),
\end{equation*}
\begin{equation*}
	\eqmark{5.39}
	\Delta U_2=\frac{\kB T_e}{e}\ln\left(1+\frac{I}{I_{i\text{н}}}\right).
\end{equation*}
Наконец, вычитая второе равенство из первого, найдём
\begin{equation*}
 	\eqmark{5.40}
	U=\Delta U_1-\Delta
U_2=\frac{\kB T_e}{e}\ln\frac{I_{i\text{н}}-I}{I_{i\text{н}}+I},
\end{equation*}
и разрешая это равенство относительно $I$, получим
\begin{equation}
	\eqmark{5.41}
	I=I_{i\text{н}}\th\frac{eU}{2\kB T_e}.
\end{equation}
Эту формулу можно использовать для определения температуры электронов по форме
вольт-амперной характеристики двойного зонда.

Наблюдаемая на опыте зависимость тока от напряжения изображена на
рис.~\figref{Double probe VAC}. Заметим, что
эта кривая отличается от \eqref{5.41} существованием наклона у асимптот
в области больших $|U|$, что связано с ускорением частиц
плазмы приложенным полем, которое не учтено при выводе \eqref{5.41}
(см. также замечание в конце п.~\ref{sec:single}).
% Наклон асимптот в первом приближении является линейным.
% Поэтому вместо \eqref{5.41} лучше писать
% \begin{equation}
% 	\eqmark{5.42}
% 	I=I_{i\text{н}}\th\frac{eU}{2\kB T_e}+aU,
% \end{equation}
% где $a$~--- некоторая константа, величина которой может быть найдена из опыта.

\begin{figure}[h!]
    \centering
	\pic{0.7\textwidth}{Chapter_5/v5_11}
	\caption{Вольт-амперная характеристика двойного зонда}
	\figmark{Double probe VAC}
\end{figure}

Графики типа рис.~\figref{Double probe VAC} проще всего обрабатывать следующим
образом. Сначала находится ток насыщения $I_{i\text{н}}$ из пересечения асимптот
с осью $U=0$.
Затем находится наклон графика в начале координат,
% Затем, по наклону асимптот, находится величина $a$.
из которого можно определить температуру электронов~$T_e$.
Дифференцируя \eqref{5.41} по~$U$ в точке~$U=0$ и принимая во внимание, 
что при малых аргументах $\th x\approx x$,
% а при малых наклонах кривой насыщения $a\to 0$,
найдём
\begin{equation}
	\eqmark{5.43}
	\kB T_e=\frac12\frac{eI_{i\text{н}}}{\left.\frac{dI}{dU}\right|_{U=0}},
\end{equation}
где $\left.\frac{dI}{dU}\right|_{U=0}$ --- наклон характеристики зонда вблизи
начала координат. По известным~$T_e$ и~$I_{iн}$
можно из формулы \eqref{5.31} найти концентрацию заряженных
частиц $n_i=n_e$.

Таким образом, двойные зонды удобно применять для измерения электронной 
температуры и концентрации частиц в плазме.


\begin{lab:literature}
    \item \SivuhinIII~--- Гл.~IX.
    
    \item \Kirichenko~--- Гл.~16.
    
    \item \textit{Арцимович Л.А., Сагдеев Р.З.} Физика плазмы для физиков.~---
    М.: Атомиздат, 1979.
    % \item \textit{Кингсеп А.С.} Элементы физики плазмы: Учебно-методическое пособие.
    % ~--- М:МФТИ 1985.
    
    \item \textit{Чен Ф.} Введение в физику плазмы. Перевод с английского ~--- М.:
    Мир, 1987.
\end{lab:literature}
