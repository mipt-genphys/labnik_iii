\lab{3.5.2}{Индукционный газовый разряд}

\cel{Изучение свойств плазмы методом зондовых характеристик.}

\prin{газоразрядная трубка с высокочастотным (ВЧ)-генератором, источник постоянного тока, генератор звуковой частоты
(ЗГ), осциллограф, форвакуумный насос, вакуумметр, натекатель, вакуумный кран.}

Газоразрядную плазму можно получить, используя электрические разряды в переменных высокочастотных (ВЧ) полях. Существуют
различные способы введения ВЧ-поля в разрядный объём. Один из них основан на использовании электромагнитной индукции:
через катушку-соленоид, в которую вставлена диэлектрическая газоразрядная камера, пропускается ток высокой частоты, и
внутри катушки индуцируется вихревое электрическое поле. Силовые линии этого поля, а вместе с~ними линии разрядного
тока, представляют собой замкнутые окружности. Такой разряд называется кольцевым, индукционным или разрядом $H$-типа,
что указывает на определяющую роль магнитного поля. Именно такой способ возбуждения газового разряда используется в нашей
установке.

\eo Схема установки представлена на \p{1}. Заполненная газом диэлектрическая камера представляет собой цилиндрическую
стеклянную трубку диаметром 15~мм и длиной 100~мм, на одном из торцов которой впаяны две молибденовые проволочки (зонды)
диаметром~$d=0,5$~мм и длиной~$l=10$~мм, расположенные на расстоянии~5~мм друг от друга. Другой конец трубки не запаян.
Он служит для откачки и для заполнения камеры газом. Трубка вставлена в катушку индуктивности колебательного контура
ВЧ-генератора, работающего на частоте~10~МГц. Камера откачивается форвакуумным насосом и с помощью натекателя
заполняется воздухом до давления $2\cdot10^{-1}$--$2\cdot10^{-2}$~мм~рт. ст. Давление контролируется вакуумметром
(термопарным манометром).

\fcpic[1]{5_3_1}{Схема установки для исследования газового разряда}{1}

На зонды поступает синусоидальное напряжение от звукового генератора~ЗГ. Это же напряжение через делитель (1:10)
подаётся на вход~$X$ осциллографа. Напряжение, пропорциональное току, текущему через плазму, подаётся на вход~$Y$
с~сопротивления~$200$~кОм. Две катушки, подключённые к~зондам, не пропускают на осциллограф высокочастотный сигнал. На
экране осциллографа наблюдается кривая, представляющая собой вольт-амперную характеристику двойного зонда (см.
рис.~\oref{v5_r6}). Следует отметить, что на некоторых частотах в~измерительной цепи могут возникать фазовые сдвиги, и
характеристика зондов приобретает вид петли. Такие частоты для измерений непригодны.

Для получения абсолютных значений тока и напряжения необходимо прокалибровать оси~$X$ и~$Y$ осциллографа по известному
напряжению.

\zad

В работе предлагается при различных давлениях газа в~трубке получить зондовые вольт-амперные характеристики на экране
осциллографа и рассчитать с~их помощью температуру и~концентрацию электронов в~плазме, степень ионизации, плазменную
частоту и дебаевский радиус экранирования.

\zn Подготовка приборов к работе

\n Соберите схему для измерений согласно \p{1}.

\n Включите форвакуумный насос и вакуумметр. Откачайте трубку до давления~$\sim 2\cdot10^{-1}$~мм~рт. ст. Давление
регулируется с~помощью натекателя (микровентиля) при постоянной откачке.

\n Включите источник питания ВЧ-генератора и проследите за разрядом в трубке: после зажигания разряд должен устойчиво
гореть по всей трубке, включая область расположения зондов.

\n Включите осциллограф и звуковой генератор. Подайте на зонды переменное напряжение от звукового генератора (рабочее
значение частоты около~20~Гц, напряжение~$\sim20$~В).

На экране осциллографа должна появиться кривая, похожая на теоретическую зависимость, изображённую на рис.~\oref{v5_r6}.

Если на кривой не наблюдаются области насыщения, следует увеличить выходное напряжение~ЗГ. Если вместо кривой на экране
возникает петля, следует изменить частоту~ЗГ.

\n[p5] Посмотрите, как ведёт себя разряд, насколько он устойчив при изменении давления в 
диапазоне~$2\cdot10^{-1}$--$2\cdot10^{-2}$. Отметьте, в какой области давлений наблюдаемая кривая соответствует
теоретической.

\zn Измерения

\n[p6] Получите на экране осциллографа вольт-амперную характеристику зондов, соответствующую рис.~\oref{v5_r6}.

Убедитесь, что ручка плавной регулировки усиления по оси~$Y$ выведена вправо до щелчка (при таком положении ручки
чувствительность канала К$_Y$ указана возле дискретного переключателя усиления). Регулируя напряжение звукового
генератора, добейтесь того, чтобы кривая занимала почти весь экран. Зарисуйте кривую с~экрана на кальку. Укажите на
кальке показания вакуумметра и чувствительность осциллографа по оси~$Y$.

\n[p7] Для калибровки оси $X$ в В/см (не трогать ручки усиления по $X$!) уберите дискретным переключателем усиление по
оси~$Y$ и измерьте в см сигнал, поступающий на $X$ с~клеммы~6.

Затем подайте тот же сигнал на прокалиброванный вход~$Y$ (переключите вход~$Y$ осциллографа с~клеммы 5 на 6), подберите
положение дискретного переключателя по оси~$Y$ и измерьте отклик на сигнал по оси $Y$ в~см.

\n Повторите измерения п.~\r{p6} для 3--4-х давлений внутри интервала, выбранного Вами в п.~\r{p5}.

При изменении чувствительности по оси~$X$ повторите калибровку оси~$X$ (п.~\r{p7}).

\znr Обработка результатов

\n Для каждой кривой пересчитайте масштаб по оси~$Y$ из В/см в А/см, зная сопротивление, с~которого сигнал,
пропорциональный зондовому току, подавался на ось~$Y$ осциллографа.

\n Рассчитайте масштаб по оси~$X$ в~В/см с учётом делителя в~блоке~$X$ (1:10) и калибровки (п.~\r{p7}).

\n По зондовым характеристикам определите температуру~$T_e$ электронов по формуле (\oref{v5_040}): ток~$I_{iн}$ найдите
из пересечения асимптоты к~току насыщения с~осью $U=0$ (см. рис.~\oref{v5_r11}); $(dI/dU)|_{u=0}$~--- наклон
характеристики $I=f(U)$ в~точке $U=0$, $I=0$; взяв $\Delta U$ в~вольтах и приняв заряд электрона $e=1$, определите
энергию (<<температуру>>) электронов~$kT_e$ в~электрон-вольтах.

\n Концентрацию электронов $n_e$ определите из формулы (\oref{v5_028}), в которую вместо $n$ следует подставить $n_e$:
\[
I_{iн}=0,4n_e eS\sqrt{\frac{2kT_e}{m_i}}.
\]
Здесь $S=\pi\cdot d\cdot l$~--- площадь поверхности зонда; значения $d$ и $l$ приведены в описании экспериментальной
установки; $m_i=14\cdot 1,66\cdot 10^{-24}$~г~--- масса иона азота.


\n Рассчитайте плазменную частоту колебаний электронов:
\[
\omega_p=\sqrt{\frac{n_e e^2}{\e_0 m_e}}.
\]

\n Рассчитайте дебаевский радиус~$\rd$ экранирования по формуле (\oref{v5_5l}), приняв температуру ионов равной
комнатной: $T_i\approx 300$~K.

Оцените среднее число ионов в дебаевской сфере по формуле (\oref{v5_41l}) % : \[n_{\Deb}=n_0\frac43\pi\rd^3.\]

\n Оцените степень ионизации в плазме (долю ионизованных атомов~$\alpha$).

