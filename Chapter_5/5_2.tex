\lab{Изучение плазмы индукционного газового разряда}

\aim{изучение свойств плазмы высокочастотного газового разряда
    в воздухе методом зондовых характеристик.}

\equip{газоразрядная трубка с высокочастотным генератором; источник
    постоянного тока; генератор звуковой частоты; электронный осциллограф;
    форвакуумный насос; вакуумметр; натекатель; вакуумный кран; делитель;
    повторители--фазовращатели.}

Перед выполнением работы необходимо ознакомиться с
теоретическим Введением к разделу.
% Подробное описание свойств тлеющего разряда можно найти в Приложении к разделу
% (см. стр.~\pageref{sec:discharge}).

Ионизацию в плазме можно получить с помощью высокочастотных электромагнитных
переменных полей. Существуют различные способы введения ВЧ-поля
в разрядный объём. Один из них основан на использовании электромагнитной
индукции: через катушку-соленоид, в которую вставлена диэлектрическая
(например, стеклянная) газоразрядная камера, пропускается ток высокой частоты.
Внутри катушки индуцируется вихревое электрическое поле.
Силовые линии этого поля, а вместе с~ними и линии разрядного тока,
представляют собой окружности. Такой разряд  называется \term{кольцевым},
\term{индукционным} или разрядом $H$-типа, что указывает
на определяющую роль магнитного поля. 
Именно такой способ возбуждения газового разряда используется в нашей установке.

Другой способ возбуждения заключается в том, что на два электрода,
пространство между которыми заполнено разреженным газом,
подаётся высокочастотное переменное напряжение. 
Такие разряды называются \term{ёмкостными} или
разрядами $E$-типа.

Высокочастотные разряды широко используются в технике. Индукционные разряды
применяются в безэлектродных генераторах плотной низкотемпературной плазмы
(в плазмотронах), используемых, например, для плазмохимического производства
чистых веществ. Разряды ёмкостного типа применяются в мощных 
газоразрядных лазерах.

\paragraph{Механизм возникновения пробоя} 
В высокочастотных разрядах --- как $H$- так и $E$-типа --- электроны
ускоряются под действием высокочастотного переменного электрического поля.

Предположим, что давление газа мало, так что столкновения электронов
с молекулами газа происходят редко. 
Ограничимся рассмотрением случая однородного поля, 
меняющегося по гармоническому закону: $\vec{E}=\vec{E}_0 \cos \omega t$, 
где $\vec{E}_0=\const$ --- амплитуда электрического поля. 
Эта задача была рассмотрена также в п.~\ref{sec:epsilon} (см. формулу \chaptereqref{free_E}).
Направив ось~$x$ вдоль~$\vec{E}_0$, получим
\[
    m_e\frac{d^2x}{dt^2}=-eE_0\cos\omega t.
\]
Интегрируя это уравнение, найдём скорость электрона:
\begin{equation*}
    v(t)=v_0+\frac{eE_0}{\omega m_e}\sin\omega t,
\end{equation*}
где $v_0$~--- начальная скорость.

Видно, что скорость электрона периодически меняется, 
но в среднем энергия электрона остаётся \emph{неизменной}. 
Так обстоит дело, пока газ достаточно разрежен. С увеличением
давления соударения электронов с молекулами газа происходят всё чаще.
Однако если амплитуда напряжённости поля невелика,
легкие электроны не смогут ионизировать молекулы 
и будут соударятся с ними упруго, передавая им при этом
лишь малую долю своей энергии (поскольку масса электрона
много меньше массы молекулы).

Чтобы понять, как возникает зажигание разряда (пробой), рассмотрим
подробнее, что происходит при упругом соударении электрона с молекулами газа. 
Хотя энергия электрона при ударе об ион почти не меняется, 
направление его скорости претерпевает существенные изменения.
Если частота поля достаточно велика, новое направление
скорости электрона может совпасть с изменившимся за время соударения направлением 
электрического поля --- в таком случае электрон при
дальнейшем движении не возвращает полю энергию, а вновь её получает! 
Часть
электронов может поэтому ускоряться высокочастотным полем вплоть
до энергии ионизации, даже если амплитуда поля невелика. 
После ионизации процесс повторяется: при упругих столкновениях с молекулами газа 
часть электронов в электрическом поле ускоряется, вследствие чего 
происходят новые акты ионизации. 

Таким образом в газе накапливаются заряженные частицы --- электроны и ионы. По мере
увеличения их концентрации возрастает роль процессов \emph{рекомбинации}
(то есть объединения электрона и иона обратно в нейтральный атом). 
Отметим, что наблюдаемое свечение разряда есть в основном следствие именно рекомбинации
электронов и ионов.
В результате
действия двух факторов~--- ионизации и рекомбинации~---
устанавливается стационарное состояние плазмы.
Её концентрация и температура зависят от сорта газа, его давления,
а также от частоты и амплитуды высокочастотного поля.


\experiment

Схема установки представлена на рис.~\figref{Induction gas discharge}.
Заполненная газом диэлектрическая камера представляет собой цилиндрическую
стеклянную трубку (ее диаметр и длина указаны в описании), на одном из торцов
которой впаяны две молибденовые проволочки (зонды)
диаметром~$d$ и длиной~$l$, расположенные на некотором расстоянии друг от друга.
Трубка вставлена в катушку индуктивности колебательного контура высокочастотного 
(ВЧ) генератора, 
работающего на частоте~$\sim10$~МГц. 
Камера закрыта экраном (металлической сеткой), защищающим измерительную
цепь зондов от высокочастотного поля.

\begin{figure}[h]
    \centering
    %    \footnotesize
    \pic{}{Chapter_5/c5_2_1}
    \caption{Схема установки для исследования газового разряда}
    \figmark{Induction gas discharge}
\end{figure}

Другой конец трубки не запаян и служит для откачки и для заполнения камеры газом. 
Камера откачивается форвакуумным насосом~(ФН), а давление в ней может
регулироваться с помощью натекателя, соединенного с атмосферой.
Значение давления контролируется термопарным вакуумметром.

\paragraph{Регулировка давления в системе}
Отметим важную особенность процедуры установки рабочего давления в
разрядной камере. Она осуществляется путем изменения
проходного сечения натекателя, соединенного с атмосферой, при 
\emph{непрерывной} работе откачивающего насоса. 
При этом скорость откачки сохраняется постоянной, а приток
воздуха в разрядную камеру определяется изменением аэродинамического
сопротивления натекателя. Таким образом, устанавливается новое значение
давления, при котором скорость откачки насоса уравновешивается расходом воздуха
через натекатель. Изменение сечения натекателя выполняется микрометрическим 
винтом, который сжимает расположенную под ним пружину и изменяет 
ее давление на подвижную
мембрану клапана. Такая система обладает очень большой \emph{инерционностью} и
реагирует на вращение винта со значительным запозданием.

\paragraph{Измерение зондовой характеристики}
Измерительная цепь зондов изображена на рис.~\figref{zond-measure}.
Две катушки~$L_{1}$ и~$L_{2}$, подключённые последовательно с зондами, 
не пропускают на осциллограф высокочастотный сигнал.
На зонды подаётся синусоидальное напряжение небольшой частоты 
($\sim$\,50~Гц) от звукового генератора~ЗГ.
Это же напряжение через делитель~Д и регулируемый 
повторитель-фазовращатель~ПФ2 подаётся на вход~$X$ электронного осциллографа
(ЭО). Последовательно с зондом подключён резистор~$R$ 
($R\sim 100\;КОм$, конкретное значение указано на установке),
напряжение на котором пропорционально току через зонды (т.\,е. через плазму).
Это напряжение подаётся на вход~$Y$ осциллографа 
через нерегулируемый повторитель-фазовращатель ПФ1. 

\begin{figure}[h!]
    \centering
    %    \footnotesize
    \pic{\textwidth}{Chapter_5/c5_2_2}
    \caption{Схема измерения зондовых характеристик в разряде}
    \figmark{zond-measure}
\end{figure}

На экране осциллографа наблюдается кривая, представляющая собой вольт-амперную
характеристику (ВАХ) двойного зонда (см. рис.~\chapterfigref{Double probe VAC}).
Регулируя амплитуду на звуковом генераторе, можно изменять диапазон значений
напряжения  на ВАХ. Из-за фазовых сдвигов, возникающих в измерительной цепи, 
картина на экране~ЭО в общем случае имеет вид петли.
Для компенсации разности фаз и устранения этого эффекта 
используется регулятор фазовращателя ПФ2. Если регулировка ПФ2 не помогает,
следует изменить частоту ЗГ.

По пересечению асимптот зондовой характеристики с осью ориднат
можно определить ионный ток насыщения зондов $I_{iн}$, а по наклону 
характеристики вблизи начала координат~--- электронную температуру $T_e$
(см. п.~\ref{sec:double} Введения).




\begin{lab:task}

\taskpreamble{В работе предлагается при различных давлениях газа в~трубке получить зондовые
вольт-амперные характеристики на экране
осциллографа и рассчитать с~их помощью температуру и~концентрацию электронов
в~плазме, степень ионизации, плазменную
частоту и дебаевский радиус экранирования.}

\tasksection{Подготовка к работе}

\item До включения приборов ознакомьтесь с установкой по техническому описанию 
в лаборатории.

\item Подготовьте к работе термопарный вакуумметр (см. техническое описание). 
Включите форвакуумный насос и вакуумметр. 
Откачайте трубку до давления $P\sim 10^{-1}\div 10^{-2}$~торр (рекомендованные
значения см. в техническом описании). Давление
регулируется с~помощью натекателя (микровентиля) при непрерывной работе
насоса.

\begin{lab:warning}
    Не выключайте насос до конца работы!
\end{lab:warning}

\item Включите источник питания ВЧ-генератора. Убедитесь, что
в трубке возник разряд: после зажигания он должен устойчиво гореть 
по всей трубке, включая область расположения зондов.

\item  Включите осциллограф и звуковой генератор. 
%При этом на зонды подается
%переменное напряжение от генератора с частотой, указанной в описании в лаборатории.
Подберите напряжение на генераторе, при котором на экране осциллографа 
появится кривая, похожая на теоретическую зависимость, изображённую на
рис.~\chapterfigref{Double probe VAC}.
Выходное напряжение генератора должно быть таким, чтобы на кривой 
наблюдались начальные участки области насыщения (асимптоты ВАХ).

Если вместо кривой на экране возникает петля, воспользуйтесь ручкой 
регулировкой фазовращателя ПФ2. Если регулировка не помогает, следует 
изменить частоту генератора.

При чрезмерно большом напряжении генератора
возникает искажение зондовой характеристики, в ее оконечных областях появляются
изломы и <<выпучины>>. Это происходит вследствие влияния электрического поля зондов
на характер плазменного разряда. Такого допускать не следует, уменьшая
при необходимости напряжение питания зондов до исчезновения искажений.


\item \label{p521} Посмотрите, как ведёт себя разряд, насколько он устойчив при изменении
давления в рабочем диапазоне. Проверьте, что наблюдаемая 
на экране кривая в целом соответствует теоретической.

\tasksection{Измерения}

\item Получите на экране осциллографа вольт-амперную характеристику зондов
(см. рис.~\chapterfigref{Double probe VAC}) для максимального давления из
выбранного диапазона.

\item Регулируя напряжение звукового генератора и коэффициент усиления 
осциллографа по осям~$X$ и $Y$ (вольт/деление), 
добейтесь того, чтобы кривая занимала большую часть экрана. 

Если у осциллографа есть ручка плавной регулировки масштаба
(например, осциллограф GOS-620), убедитесь, что она находится в 
<<калиброванном>> положении, при котором чувствительность каналов 
соответствует положению дискретного переключателя усиления.

\item Зафиксируйте изображение кривой с экрана осциллографа 
(сфотографируйте или зарисуйте с~экрана на кальку). 
Запишите давление в системе $P$ и чувствительность осциллографа по 
осям~$X$ и~$Y$.

\item Повторите измерения для 5--6 значений давления внутри диапазона, 
в котором зондовая характеристика соответствует теоретической
(см. п. \ref{p521}).

\item Закончив работу, выключите сначала насос и \textbf{сразу же откройте натекатель}
(для предотвращения выдавливания масла из насоса).
Затем отключите вакуумметр и остальные приборы.

\item Перепишите параметры установки: сопротивление датчика тока~$R$, коэффициент
делителя~Д, геометрические характеристики зондов и разрядной трубки.

\tasksection{Обработка результатов}

\item По значению сопротивление $R$, с которого подавался
сигнал на канал~$Y$ осциллографа  (см. рис.~\figref{zond-measure}), 
для каждой кривой пересчитайте масштабы по оси~$Y$ из единиц
напряжения в единицы тока.
 
Рассчитайте масштаб по оси~$X$ с учётом наличия делителя~Д в канале 
зондового напряжения.

\item По зондовым характеристикам определите температуру~$T_e$ электронов
(см. п. \ref{sec:double} Введения). Ответ
выразите в энергетических единицах (электрон-вольтах).

\item По зондовым характеристикам определите значения ионных токов насыщения
зондов $I_{iн}$. 
Определите концентрацию $n$ электронов и ионов в плазме, 
считая ионы однозарядными ($Z=1$).
Площадь поверхности зонда (в пренебрежении
дебаевским слоем): $S\approx \pi d l$, где 
$d$~--- диаметр, $l$~--- длина зонда.

\item Рассчитайте плазменную частоту колебаний электронов $\omega_p$,
электронную поляризационную длину~$r_{De}$ и дебаевский радиус экранирования~$r_D$
(с учётом того, что температура ионов мала по сравнению с электронной: $T_i\ll T_e$). 
Можно ли считать плазму квазинейтральной?

\item Оцените среднее число ионов в дебаевской сфере $N_D$. 
Является ли плазма разряда идеальной?

\item Оцените степень ионизации плазмы (долю ионизованных атомов $\alpha$)
для каждого давления трубке $P$.

\item Оцените погрешности эксперимента.

\end{lab:task}

\begin{lab:questions}
    \item Дайте определение понятия <<плазма>>. Назовите различные виды плазм в лаборатории, 
    технике и природе.
    
    \item Что такое дебаевская длина экранирования? Выведите формулу
    для дебаевской длины в одномерном случае для равновесной плазмы.
    
    \item Что такое плазменная частота? Выведите формулу для плазменной частоты.
    Какие процессы в плазме характеризуются плазменной частотой?

    \item Что такое равновесная и неравновесная плазма? 
    Является ли исследуемая в работе плазма равновесной?
        
    \item Чем определяется потенциал зонда, погружённого в плазму? 
    Чем определяется зондовый ток насыщения для одиночного зонда? Для двойного
    зонда?
    
    \item Каков механизм зажигания разряда в высокочастотном поле?
    
    \item Для чего нужны защитный экран вокруг разрядной трубки и катушки $L_1$, $L_2$ в измерительной
    цепи зонда?
\end{lab:questions}


