\begin{booksupplement}[Основные сведения о комплексных числах]
\label{sec:cplx}

\renewcommand{\Re}{\mathop{\mathrm{Re}}}
\renewcommand{\Im}{\mathop{\mathrm{Im}}}

% TODO: сделать вступительный абзац?
% Компл\'{е}ксные числа исторически возникли из задачи о поиске
% корней многочленов с произвольными коэффициентами. Уравнение
% $x^2=-1$ очевидно не имеет решений в действительных числах.
% При этом введение \term{мнимой единицы} $i=\sqrt{-1}$
% как корня такого уравнения, позволяет...
%
% Использование комплексных чисел позволяет в ряде случаев существенно
% упростить расчеты и по этой причине находит самое широкое
% применение в различных областях математики, физики и техники:
% в теории колебаний, электротехнике, квантовой механике, гидродинамике и др.

\labsection{Различные представления комплексных чисел}
    \emph{Алгебраическая форма} комплексного числа:
    \begin{equation*}
        z = x + iy,
    \end{equation*}
где $x$ и~$y$~--- действительные числа, $i = \sqrt{- 1}$~--- \term{мнимая единица},
то есть число, являющееся корнем уравнения $i^2 = -1$.
Число $x$ называется \term{действительной частью} комплексного числа $z$,
число $y$~--- его \term{мнимой частью}. Используются обозначения:
    \begin{equation*}
        x = \Re z,\quad y = \Im z.
    \end{equation*}
Если $\Im z = 0$, то $z = x$ --- действительное число, если
    $\Re z = 0$, то $z = iy$~--- <<чисто мнимое>> число.

    \begin{wrapfigure}{o}{0.35\textwidth}
     \pic{\linewidth}{complex}
    \end{wrapfigure}

    \emph{Геометрически} комплексное число $z=x+iy$ может быть изображено точкой
    на плоскости с декартовой системой координат с абсциссой~$x$ и ординатой~$y$
    (\term{комплексная плоскость}).
    Действительные числа лежат на оси абсцисс (\term{действительная ось}),
    чисто мнимые --- на оси ординат (\term{мнимая ось}).
    Так как каждая точка плоскости однозначно определяется радиусом-вектором этой точки,
    то каждому комплексному числу $z$ соответствует \emph{вектор},
    лежащий в плоскости и идущий из полюса (0, 0) в точку ($\Re{z},\Im{z}$).

    \emph{Тригонометрическая форма} комплексного числа определяется через
    \emph{полярную систему координат} на комплексной плоскости:
    \[
        z = r\left(\cos\varphi + i\sin\varphi\right),
    \]
где $r = \left| z \right| = \sqrt{x^{2} + y^{2}}$~--- длина
    радиус-вектора, изображающего число $z$ на комплексной плоскости,
    называется \term{модулем}. Угол $\varphi$ (в радианах)~---
    \term{аргумент} (или \term{фаза}) комплексного числа $z$, определяемый равенством
    \[
        \varphi = \arg z + 2 \pi k,\quad k \in \mathbb{Z},
    \]
    где функция $\arg z$ определена на промежутке $(-\pi,\pi]$
    и называется \term{главной ветвью} аргумента.
    При этом положительное направление отсчёта угла $\varphi$
    соответствует вращению \emph{против часовой стрелки}, так что
    $\Re z = r\cos\varphi$ и $\Im z = r\sin\varphi$.

    \emph{Показательная форма} комплексного числа основана на \emph{формуле
    Эйлера}:
    \[
\cos \varphi + i \sin \varphi = e^{i\varphi}.
    \]
    Таким образом, комплексное число можно представить в виде
    \[
    z = re^{i\varphi},
    \]
    где, как и в тригонометрической форме, $r$ --- модуль, $\varphi$ ---
    аргумент (<<фаза>>) комплексного числа.

    \begin{lab:example}
    Число $z=\sqrt{3} + i$ ($\Re z = \sqrt{3}$, $\Im z = 1$) может быть 
    записано в виде
    \[
        \sqrt{3} + i = 2\left(\cos\tfrac{\pi}{6} + i\sin\tfrac{\pi}{6}\right) =
        2e^{i\pi/6},
    \]
    причём, если не ограничиваться главным значением,
    к аргументу $\pi/6$ можно добавить $2\pi k$, где $k$ --- целое.
    \end{lab:example}

    \begin{lab:exercise}
     Убедитесь в справедливости соотношений
     \[
      e^{i\pi/2} = i,\quad e^{i\pi} = -1,\quad e^{3i\pi/2} = e^{-i\pi/2} =-i,\quad
      e^{2\pi i} = 1.
     \]
    \end{lab:exercise}

\labsection{Операциии с комплексными числами}
    Два комплексных числа считаются
    \emph{равными}, если равны отдельно действительные и мнимые части:
    $z_1 = z_2$ при $\Re z_1=\Re z_2$ и $\Im z_1=\Im z_2$. Геометрически
    это соответствует равенству изображающих числа векторов.
    Операция сравнения для комплексных чисел не определена.




    Основные арифметические операции (сложение, вычитание, умножение, деление)
    над комплексными числами имеют те же свойства, что и соответствующие операции с вещественными числами.
    Вычисления производятся так же, как и над обыкновенными двучленами,
    но с учётом равенства $(\pm i)^{2} = - 1$.
    Сложение и вычитание производится по-компонентно:
    \[
     z_1 \pm z_2 = (x_1 \pm x_2) + i (y_1 \pm y_2).
    \]
    В геометрическом представлении операции суммы/разности комплексных чисел соответствуют
    сложению (вычитанию) представляющих их векторов.

    Умножение в алгебраической форме:
    \[
     z_1 z_2 = (x_1 +iy_1)(x_2 + iy_2) = x_1 x_2 - y_1 y_2 + i(x_1y_2 + x_2 y_1).
    \]
    Умножение некоторого числа $z$ на действительное число изменяет модуль
    $z$ (в геометрическом представлении --- длину вектора), а умножение
    на чисто мнимое --- изменяет аргумент $z$ (в геометрическом представлении
    происходит поворот исходного вектора на аргумент множителя). В показательной форме:
    \[
     z_1 z_2 = r_1 e^{i\varphi_1} \cdot r_2 e^{i\varphi_2} = r_1r_1 e^{i(\varphi_1+\varphi_2)}.
    \]


    Деление эквивалентно умножению на обратное: в показательной
    форме
    \[
      \frac{1}{z} = \frac{1}{r} e^{-i\varphi}.
    \]
 В~алгебраической форме числитель и знаменатель дроби можно умножить на число,
 сопряженное (см. ниже) знаменателю:
    \[
     \frac{1}{x+iy} = \frac{x-iy}{(x+iy)(x-iy)}=\frac{x-iy}{x^2+y^2}.
    \]

    \begin{wrapfigure}[9]{o}{0.35\textwidth}
     \pic{\linewidth}{complex2}
    \end{wrapfigure}

    Два комплексных числа $z$ и $z^{\star}$ называются
    \term{сопряжёнными}, если
    $\Re z = \Re z^{\star}$ и $\Im z = - \Im z^{\star}$. В геометрическом
    представлении точки, изображающие сопряжённые числа, расположены симметрично
    относительно действительной оси. Модули сопряжённых чисел
    равны, аргументы отличаются знаком:
    \begin{equation*}
    \begin{aligned}
        z &= x + iy = r(\cos\varphi + i\sin\varphi) = re^{i\varphi}, \\
        z^{\star} &= x - iy = r(\cos\varphi - i\sin\varphi) = re^{- i\varphi}.
    \end{aligned}
    \end{equation*}
    Мнимые и действительные части числа могут быть выражены с помощью
    операции сопряжения:
    \[
     \Re z = \frac{z+z^{\star}}{2},\qquad \Im z =\frac{z-z^{\star}}{2i}.
    \]
    Произведение числа на его сопряжённое даёт квадрат модуля:
    \[
     z z^{\star} = |z|^2.
    \]

%   \textbf{Формула Эйлера} для комплексных чисел:
%     \begin{equation*}
%         e^{\pm \text{iz}} = \cos z \pm i\sin z.
%     \end{equation*}
    \emph{Возведение в целую степень} осуществляется по \emph{формуле Муавра}:
      \begin{equation*}
        \left[r(\cos\varphi + i\sin\varphi)\right]^{n} =
        r^{n}(\cos n\varphi + i\sin n\varphi)\qquad (\forall n\in \mathbb{Z}).
    \end{equation*}
    Или в показательной форме:
    \[
     z^n = (re^{i\varphi})^n = r^n e^{in\varphi}.
    \]
    \begin{lab:exercise}
     Покажите, что $|z^2| = |z|^2$.
    \end{lab:exercise}

    \emph{Извлечение корня} (возведение в \emph{дробную} степень) не является
    однозначной операцией. Поскольку аргумент любого комплексного числа определён
    с точностью до $2\pi k$, где $k$ --- целое, для корня $n$-й степени
    можно записать
    \begin{equation*}
        \sqrt[n]{z} = r^{1/n}\left( \cos\frac{\varphi + 2k\pi}{n} +
        i\sin\frac{\varphi + 2k\pi}{n} \right) = r^{1/n} e^{i\varphi/n}
        e^{i2\pi k/n},\quad k\in \mathbb{Z}.
    \end{equation*}
    Здесь $r^{1/n}$ --- обычный (арифметический) корень степени $n$ из модуля~$z$. 
    Таким образом, если $n$ --- целое, то для любого комплексного $c$
    существует $n$ различных комплексных корней уравнения $z^n=c$. 

    В общем случае для комплексных чисел справедлива
    \emph{основная теорема алгебры}: любой полином степени $n\ge 1$ имеет~$n$
    корней.

    \begin{lab:exercise}
     Покажите, что $\sqrt{i}=\pm \frac{1+i}{\sqrt{2}}$.
    \end{lab:exercise}
    \begin{lab:exercise}
     Найдите $\sqrt[n]{1}$.
    \end{lab:exercise}

\end{booksupplement}
