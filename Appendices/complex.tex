\section*{Основные сведения о комплексных числах}

\begin{enumerate}
\item
    \textbf{Алгебраическая форма} комплексного числа:
    $\hat{z} = x + \text{iy}$, где \textbf{мнимая единица}
    $i = \sqrt{- 1}$ -- число, для которого $( \pm i)^{2} = - 1$,
    $x$ и $y$ -- любые действительные числа. Число $x$ называется
    \textbf{действительной частью} комплексного числа $\hat{z}$,
    число $\text{iy}$ -- его \textbf{мнимой частью}, число $y$ --
    \textbf{коэффициентом при мнимой части}. Используются обозначения:
    
    \begin{equation*}        
        x = \operatorname{Re}(\hat{z}),~y = \operatorname{Im}(\hat{z}),
    \end{equation*}

    при этом символ «\^{}» над комплексным числом и скобки обычно
    опускаются. Если $y = 0$, то $z = x$ -- действительное число, если
    $x = 0$, то $z = \text{iy}$ -- «чисто мнимое» число.

\item
    \textbf{Геометрически} комплексные числа могут быть изображены точками
    плоскости: число $z = x + \text{iy}$ изображается точкой с абсциссой
    $x$ и ординатой $y$ \textbf{декартовой системы координат}.
    Действительные числа изображаются точками оси абсцисс
    (\textbf{действительная ось}), чисто мнимые -- точками оси ординат
    (\textbf{мнимая ось}). Так как каждая точка плоскости вполне
    определяется радиусом-вектором этой точки, то каждому комплексному
    числу соответствует определённый \textbf{вектор}, лежащий в плоскости
    и идущий из полюса (0, 0) в точку, соответствующую комплексному числу.

\item
    \textbf{Тригонометрическая форма} комплексного числа связана с
    \textbf{полярной системой координат}:
    $z = r(\cos\varphi + i\sin\varphi)$, где
    $r = \left| z \right| = \sqrt{x^{2} + y^{2}}$ -- длина
    радиус-вектора, изображающего число $z$ на комплексной плоскости,
    называется \textbf{модулем}, а угол $\varphi$ (в радианах) --
    \textbf{аргументом} комплексного числа $z$, который определяется
    равенством $\varphi = \arg z + 2\text{kπ}$,
    где функция $\arg z$, называемая \textbf{главной ветвью (главным
    значением) аргумента}, определена и однозначна в промежутке
    $( - \pi,\pi\rbrack$\textbf{,} а $k$ -- произвольное целое число.
    При этом {положительное направление отсчёта угла} $\varphi$
    {соответствует вращению против часовой стрелки}, так что
    $x = r\cos\varphi$, а $y = r\sin\varphi$.

\item
    \textbf{Показательная форма} комплексного числа:
    $z = re^{\text{iφ}}$.
    Таким образом, например, число $\sqrt{3} + i$ может записано в виде:
    $\sqrt{3} + i = 2(\cos\frac{\pi}{6} + i\sin\frac{\pi}{6}) = 2e^{\frac{\pi}{6}i}$,
    причём к аргументу $\frac{\pi}{6}$ надо добавить $2\text{πk}$, если не ограничиваться его главным значением.

\item
    Равенство комплексных чисел: два комплексных числа считаются
    \textbf{равными}, если равны отдельно действительные части и
    коэффициенты при мнимых частях. Геометрически комплексные числа равны,
    если равны изображающие их векторы. В противном случае числа не равны;
    понятий «больше» и «меньше» для комплексных чисел не существует.

\item 
    Два комплексных числа $z$ и $\bar{z}$ называются
    \textbf{сопряжёнными}, если
    $\bar{\operatorname{Re}z} = \operatorname{Re}z$, а
    $\bar{\operatorname{Im}z} = - \operatorname{Im}z$. В геометрическом
    представлении точки, изображающие сопряжённые числа, расположены
    симметрично относительно действительной оси. Модули сопряжённых чисел
    равны, аргументы отличаются знаком:
    \begin{equation*}
        z = x + iy = r(\cos\varphi + i\sin\varphi) = re^{\text{iφ}},
        \bar{z} = x - iy = r(\cos\varphi - i\sin\varphi) = re^{- \text{iφ}}.
    \end{equation*}

\item
  \textbf{Арифметические действия} над комплексными числами производятся
  так же, как и над обыкновенными двучленами, но с учётом равенства
  $( \pm i)^{2} = - 1$. При делении одного комплексного числа на
  другое умножают числитель и знаменатель соответствующей дроби на
  число, сопряжённое знаменателю, и, пользуясь равенством
  $(x + \text{iy})(x - \text{iy}) = x^{2} + y^{2}$, устраняют мнимость
  в знаменателе. В \textbf{геометрическом} представлении для получения
  вектора, изображающего сумму или разность двух чисел, следует сложить
  или вычесть соответствующие векторы по правилу действий над векторами.
\item
    \textbf{Формула Эйлера} для комплексных чисел:
    \begin{equation*}
        e^{\pm \text{iz}} = \cos z \pm i\sin z.
    \end{equation*}

\item
  \textbf{Возведение в} $\mathbf{n}$ \textbf{-ю степень} комплексного
    числа производится по \textbf{формуле Муавра}
    \begin{equation*}
        \lbrack r(\cos\varphi + i\sin\varphi)\rbrack^{n} = r^{n}(\cos n\varphi + i\sin n\varphi)
    \end{equation*}
    при любом значении $n$: целом, дробном, положительном, отрицательном.
\item
    \textbf{Извлечение корня} $\mathbf{n}$ \textbf{-ой степени}
    производится по формуле Муавра для дробного показателя:
    \begin{equation*}
        \sqrt[n]{z} = \sqrt[n]{z}\left( \cos\frac{\varphi + 2k\pi}{n} + i\sin\frac{\varphi + 2k\pi}{n} \right),
    \end{equation*}
    причём это действие даёт всегда $n$ различных значений.
\end{enumerate}
