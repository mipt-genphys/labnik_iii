\section*{Перевод единиц}

\begin{longtable}{|L{4cm}|c|c|c|}
	\caption{Перевод числовых значений физических величин из~системы СИ в~систему СГС}
	\hline
	Наименование                       &        Обозн.         &         СИ         &                       СГС                        \\ \hline
	Длина                              &          $l$          &     1 м (метр)     &                    $10^2$~см                     \\
	Масса                              &          $m$          &  1 кг (килограмм)  &                     $10^3$ г                     \\
	Время                              &          $t$          &   1 с (секунда)    &                       1 с                        \\
	Работа, энергия                    &       $A$, $W$        &   1 Дж (джоуль)    &                    $10^7$ эрг                    \\
	Мощность                           &          $N$          &    1 Вт (ватт)     &          $10^7~\frac{эрг}{с\mathstrut}$          \\
	Давление                           &          $P$          &   1 Па (паскаль)   &         $10~\frac{дин}{см^2\mathstrut}$          \\
	Сила электрического тока		   &          $I$          &    1 А (ампер)     &                   $3\cdot10^9$                   \\
	Электр. заряд                      &          $q$          &    1 Кл (кулон)    &                   $3\cdot10^9$                   \\
	Поляризация                        &       $\vec{P}$       & $1~\frac{Кл}{м^2}$ &                   $3\cdot10^5$                   \\
	Электрическая индукция	           &       $\vec{D}$       & $1~\frac{Кл}{м^2}$ &                 $12\pi\cdot10^5$                 \\
	Электр. ёмкость                    &          $C$          &    1 Ф (фарад)     &                $9\cdot10^{11}$ см                \\
	Электрическое сопротивление   	   &          $R$          &     1 Ом (ом)      & $\frac{1}{9\cdot10^{11}}~\frac{с}{см\mathstrut}$ \\
	Удельное сопротивление        &        $\rho$         &     1 Ом$\cdot$м   &       $\frac{1}{9\cdot10^{9}\mathstrut}~с$       \\
	Электрическая проводимость    & $\Lambda=\frac{1}{R}$ &   1 См (сименс)    &      $9\cdot10^{11}~\frac{см}{с\mathstrut}$      \\
	Удельная проводимость         &       $\sigma$        &  $1~\frac{См}{м}$  &               $9\cdot10^9~с^{-1}$                \\
	Магнитный поток                    &        $\Phi$         &    1 Вб (вебер)    &                    $10^8$ Мкс                    \\
	Магнитная индукция            &       $\vec{B}$       &    1 Тл (тесла)    &                    $10^4$ Гс                     \\
	Напряжённость магнитного поля &       $\vec{H}$       &  $1~\frac{А}{м}$   &               $4\pi\cdot10^{-3}$ Э               \\
	Намагниченность                    &       $\vec{M}$       &  $1~\frac{А}{м}$   &          $\frac{1}{4\pi}\cdot 10^4$ Гс           \\
	Индуктивность                      &          $L$          &    1 Гн (генри)    &                    $10^9$ см                     \\
	Электрический потенциал       &        $\phi$         &   $1~В$ (вольт)    &                 $\frac{1}{300}$                  \\
	Напряжённость электр. поля    &       $\vec{E}$       &  $1~\frac{В}{м}$   &            $\frac{1}{3}\cdot 10^{-4}$            \\
	\hline
\end{longtable}

\clearpage

\section*{Основные формулы в СИ и СГС}

\begin{longtable}{|L{4cm}|c|c|}
	\caption{Основные формулы в СИ и СГС}
	\hline
	Наименование                                 &                                    СИ                                     &                                              СГС                                               \\ \hline
	Уравнения Максвелла                          &                            $\Div\vec{D}=\rho$                             &                                     $\Div\vec{D}=4\pi\rho$                                     \\ 
	в дифференциальной                           &                              $\Div\vec{B}=0$                              &                                        $\Div\vec{B}=0$                                         \\
	форме                                        &                   $\Rot\vec{E}=-\frac{d\vec{B}}{d t}$                   &                        $\Rot\vec{E}=-\frac{1}{c}\frac{d\vec{B}}{d t}$                        \\
	                                             &               $\Rot\vec{H}=\vec{j}+\frac{d\vec{D}}{d t}$                &             $\Rot\vec{H}=\frac{4\pi}{c}\vec{j}+\frac{1}{c}\frac{d\vec{D}}{d t}$              \\ \hline
	Электрическая индукция                       &                    $\vec{D}=\varepsilon_0\vec{E}+\vec{P}$                    &                                 $\vec{D}=\vec{E}+4\pi\vec{P}$                                  \\ \hline
	Напряжённость магнитного поля                &                 $\vec{H}=\frac{1}{\mu_0}\vec{B}-\vec{M}$                  &                                 $\vec{H}=\vec{B}-4\pi\vec{M}$                                  \\ \hline
	Материальные                                 &                     $\vec{P}=\alpha\varepsilon_0\vec{E}$                     &                                    $\vec{P}=\alpha\vec{E}$                                     \\
	уравнения                                    &                    $\vec{D}=\varepsilon\varepsilon_0\vec{E}$                    &                                   $\vec{D}=\varepsilon\vec{E}$                                    \\
	%Связь между $\vec{D$ и $\vec{E}$ в~вакууме} &                        $\vec{D}=\varepsilon_0\vec{E}$                        &                                       $\vec{D}=\vec{E}$                                        \\
												 &                          $\vec{M}=\kappa\vec{H}$                          &                                    $\vec{M}=\kappa\vec{H}$                                     \\
												 &                         $\vec{B}=\mu\mu_0\vec{H}$                         &                                      $\vec{B}=\mu\vec{H}$                                      \\
	                                             &                          $\vec{j}=\sigma\vec{E}$                          &                                    $\vec{j}=\sigma\vec{E}$                                     \\ \hline
	%Связь между $\vec{B$ и $\vec{H}$ в~вакууме} &                          $\vec{B}=\mu_0\vec{H}$                           &                                       $\vec{B}=\vec{H}$                                        \\ \hline
	Уравнения Максвелла                          &             $\oint\limits_{S}\vec{D}\,d\vec{S}=\int_{V}\rho\,dV$              &                      $\oint\limits_{S}\vec{D}\,d\vec{S}=4\pi\int_{V}\rho\,dV$                      \\
	в интегральной форме                         &                     $\oint\limits_{S}\vec{B}\,d\vec{S}=0$                     &                               $\oint\limits_{S}\vec{B}\,d\vec{S}=0$                                \\
	                                             & $\oint\limits_{L}\vec{E}\,d\vec{l}=-\int_{S}\frac{d\vec{B}}{d t}\,d\vec{S}$ &      $\oint\limits_{L}\vec{E}\,d\vec{l}=-\frac{1}{c}\int_{S}\frac{d\vec{B}}{d t}\,d\vec{S}$      \\
	                                             &   $\oint\limits_{L}\vec{H}\,d\vec{l}=\int_{S}\vec{j}\,d\vec{S}+\int_{S}\frac{d\vec{D}}{d t}\,d\vec{S}$   & $\oint\limits_{L}\vec{H}\,d\vec{l}=\frac{4\pi}{c}\int_{S}\vec{j}\,d\vec{S}+\frac{1}{c}\int_{S}\frac{d\vec{D}}{d t}\,d\vec{S}$ \\ \hline
	Сила Лоренца                                 &                  $\vec{F}=q\vec{E}+\vp{\vec{v}}{\vec{B}}$                   &            $ds\vec{F}=q\vec{E}+\frac{q\mathstrut}{c\mathstrut}\vp{\vec{v}}{\vec{B}}$            \\ \hline
	Закон Кулона                                 &   $ds \vec{F}=\frac{1}{4\pi\varepsilon_0}\frac{q_1q_2}{\varepsilon r^3}\vec{r}$    &                          $ds\vec{F}=\frac{q_1q_2}{\varepsilon r^3}\vec{r}$                           \\[2ex] \hline
	Закон Био--Савара                            &      $ds d\vec{H}=\frac{I}{4\pi}\frac{\vp{d\vec{l}}{\vec{r}}}{r^3}$      &                  $ds d\vec{H}=\frac{I}{c}\frac{\vp{d\vec{l}}{\vec{r}}}{r^3}$                  \\[2ex] \hline
	Закон Ампера                                 &                    $d\vec{F}=I\vp{d\vec{l}}{\vec{B}}$                     &                          $d\vec{F}=\frac{I}{c}\vp{d\vec{l}}{\vec{B}}$                          \\ \hline
	Плотность энергии электромагнитного поля     &           $w=\frac12\bigl(\vec{E}\vec{D}+\vec{B}\vec{H}\bigr)$            &                  $w=\frac{1}{8\pi}\bigl(\vec{E}\vec{D}+\vec{B}\vec{H}\bigr)$                   \\ \hline
	Вектор Пойнтинга                             &                     $\vec{\Pi}=\vp{\vec{E}}{\vec{H}}$                     &                       $ds\vec{\Pi}=\frac{c}{4\pi}\vp{\vec{E}}{\vec{H}}$                       \\[1ex] \hline
	Энергия магнитного поля тока                 &                          $ds W=\frac{LI^2}{2}$                           &                              $ds W=\frac{1}{c^2}\frac{LI^2}{2}$                               \\[1ex] \hline
	Плотность импульса электромагнитного поля    &              $ds\vec{g}=\frac{1}{c^2}\vp{\vec{E}}{\vec{H}}$              &                       $ds\vec{g}=\frac{1}{4\pi c}\vp{\vec{E}}{\vec{H}}$                       \\  \hline
	Индуктивность (определение)				  	 &		$\Phi=LI$															  &			$ds\Phi=\frac{1}{c}LI$																   \\ \hline
	Индуктивность длинного соленоида			 &		$ds L=\frac{\mu\mu_0N^2S}{l}$										  &			$ds L=\frac{4\pi\mu N^2S}{l}$														   \\ \hline
	Магнитный момент витка с током				 &  	$\vec{\Mgot}=I\vec{S}$												  &			$ds\vec{\Mgot}=\frac{1}{c}I\vec{S}$												   \\ \hline
	Момент сил, действующий на виток с~током	 &		$\vec{M}=\vp{\vec{\Mgot}}{\vec{B}}$									  &			$\vec{M}=\vp{\vec{\Mgot}}{\vec{B}}$													   \\ \hline
	Поле точечного магнитного диполя			 &	$\vec{B}=\frac{\mu_0}{4\pi}\left(\frac{3(\vec{\Mgot}\vec{r})}{r^5}\vec{r}-\frac{\vec{\Mgot}}{r^3}\right)$	&$\vec{B}=\frac{3(\vec{\Mgot}\vec{r})}{r^5}\vec{r}-\frac{\vec{\Mgot}}{r^3}$\\ \hline
	Сила, действующая на магнитный диполь в неоднородном поле   &$\vec{F}=(\vec{\Mgot}\vec{\nabla})\vec{B}$					  &	$\vec{F}=(\vec{\Mgot}\vec{\nabla})\vec{B}$													   \\ \hline
	Поле точечного электрического диполя         &	$\vec{E}=\frac{1}{4\pi\varepsilon_0}\left(\frac{3(\vec{p}\vec{r})}{r^5}\vec{r}-\frac{\vec{p}}{r^3}\right)$&	$\vec{E}=\frac{3(\vec{p}\vec{r})}{r^5}\vec{r}-\frac{\vec{p}}{r^3}$	   \\ \hline
	Ёмкость плоского конденсатора				 &		$ds C=\frac{\varepsilon\varepsilon_0S}{d}$								  &			$ds C=\frac{\e S}{4\pi d}$															   \\ \hline
	Энергия заряженного конденсатора			 &		$W=\frac{q^2}{2C}=\frac{qU}{2}=\frac{CU^2}{2}$						  &			$W=\frac{q^2}{2C}=\frac{qU}{2}=\frac{CU^2}{2}$										   \\
	\hline
\end{longtable}