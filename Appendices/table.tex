%\let\np=\newpage
%\def\newpage{\np\mbox{}\vskip-\bls\vfil}

\newcommand{\etab}[2]{
      \begin{table*}%
            \caption{#1}%
            #2%
      \end{table*}
}

\newcommand{\bt}[1]{\begin{tabular}{#1}}
\newcommand{\et}{\end{tabular}}

\def\tabline#1#2#3{%
\parbox{0.4\textwidth}{\raggedright\utabstrut #1\dtabstrut}&%
\parbox{0.1\textwidth}{\utabstrut\centering #2\dtabstrut}&%
\parbox{0.36\textwidth}{\raggedright\utabstrut #3\dtabstrut}\\ \hline
}

\etab{Основные физические постоянные}{\small
\bt{|l|c|l|}\hline
\tabline{\centering Величина}{Обозна\-чение}{\centering Числовое значение}
\tabline{Скорость света в вакууме}{$c$}{$299\,792\,458$~м/с (точно)}
\tabline{Постоянная Планка}{$h$\\ $\hbar$}{$6,626 068 96(33)\.10^{-34}$~Дж\.с\\ $1,054 571 628(53)\.10^{-34}$~Дж\.с}
\tabline{Постоянная Больцмана}{$k$}{$1,3806504(24)\.10^{-23}$~Дж/К}
\tabline{Постоянная Авогадро}{$N_A$}{$6,022 141 79(30)\.10^{23}~моль^{-1}$}
\tabline{Атомная единица массы}{1 a.e.м}{$1,660 538 782(83)\.10^{-27}$~кг}
%\tabline{Газовая постоянная}{$R=kN_A$}{8,314472(15)~Дж/(моль\.К)}
%\tabline{Объём моля идеального газа при нормальных условиях\\
%($T_0=273,15$~К, $P_0=101325$~Па)}{$\ds V_0=\frac{RT_0}{P_0}$}{$22,413996(39)\.10^{-3}~\ds\frac{м^3}{моль}$}
%\tabline{Число Лошмидта}{$N_л=N_A/V_0$}{$2,6867774(47)\.10^{19}~см^{-3}$}
\tabline{Гравитационная постоянная}{$G$}{$6,67428(67)\.10^{-11}~Н\.м^2/кг^2$}
\tabline{Постоянная Фарадея}{$F$}{$96\,485,3399(24)$~Кл/моль}
%\tabline{Постоянная Стефана--Больцмана}{$\sigma$}{$5,670 400(40)\.10^{-8}~Вт/(м^2\.К^4)$}
%\tabline{Постоянная Ридберга}{$R_{\infty}$}{$1,097 373 156 8527(73)\.10^7\,м^{-1}$}
%\tabline{Постоянная тонкой~структуры}{$\alpha$\\$\alpha^{-1}$}{$7,297 352 5376(50)\.10^{-3}$\\$137,035 999 679(94)$}
\tabline{Магнитная постоянная}{$\mu_0$}{$1,256 637 061 4...\.10^{-6}~Гн/м$}
\tabline{Электрическая постоянная}{$\e_0$}{$8,854 187 817\.10^{-12}~Ф/м$}
\tabline{Радиус первой боровской орбиты для атома водорода}{$a_0$}{$0,529 177 208 59(36)\.10^{-10}$~м}
\tabline{Радиус электрона классический}{$r_e$}{$2,817 940 2894(58)\.10^{-15}$~м}
\tabline{Элементарный заряд}{$e$}{$1,602 176 487(40)\.10^{-19}$~Кл\\ $4,803 204 2\.10^{-10}$~ед. СГСЭ}
\tabline{Удельный заряд электрона}{$e/m_e$}{$1,758 820 150(44)\.10^{11}$~Кл/кг}
\tabline{Масса электрона}{$m_e$}{$0,910 938 215(45)\.10^{-30}$~кг}
\tabline{Масса протона}{$m_p$}{$1,672 621 637(83)\.10^{-27}$~кг}
\tabline{Масса нейтрона}{$m_n$}{$1,674 927 211(84)\.10^{-27}$~кг}
\tabline{Магнетон Бора}{$\mb$}{$9,274 009 15(23)\.10^{-24}~А\.м^2$}
%\tabline{Ядерный магнетон}{$\mu_я$}{$5,050 783 24(13)\.10^{-27}~А\.м^2$}
\tabline{Магнитный момент протона}{$\mu_p$}{$1,410 606 662(37)\.10^{-26}~А\.м^2$}
\tabline{Магнитный момент электрона}{$\mu_e$}{$9,284 76377(23)\.10^{-24}~А\.м^2$}
\et
}

%\smallskip

%{\small В скобках указана погрешность последних знаков.}

%-------------------------------------------

\newpage

\etab{Важнейшие единицы физических величин Международной~системы~СИ}{\small
\bt{|@{\tabstrut~}l|c|c|c|l|}
\hline
%\multicolumn{2}{|c|}{Величины}&
%\multicolumn{2}{|c|}{Единицы}&\\
%\multicolumn{1}{|c|}{Соотношение}\\ \cline{1-4}
%Наименование&Обозна-&Наименование&Обозна-&\mbox{\ \ \ }единиц системы СИ и\\
%&чение&&чение&\mbox{\ \ \ }единиц других систем\\ \hline
%\multicolumn{5}{|c|}{}\\
\multicolumn{5}{|c|}{Основные единицы}\\
\hline
длина&$l$&метр&м&1~\AA~(Ангстрем)~$=10^{-10}$~м\\
масса&$m$&килограмм&кг&1~а.е.м.~$=1,66\cdot 10^{-27}$~кг\\
время&$t$&секунда&с&1~мин~$=60$~с\\
сила тока&$I$&Ампер&А&1~ед.~СГСМ~$=$\\
 &     &           &       &$=3{\cdot}10^{10}$~ед.~СГСЭ~$=10$~А\\ \hline
\multicolumn{5}{|c|}{}\\
\multicolumn{5}{|c|}{Производные единицы}\\
\hline
сила, вес&$F$&Ньютон&Н&1~дина~$=10^{-5}$~Н \\
давление       &$P$& Паскаль  & Па & 1~атм=$760$~мм~Hg~$\approx10^5$~Па\\
работа,        &$A$&          &       & 1~эрг~$=10^{-7}$~Дж\\
энергия        &$W$& Джоуль   &  Дж   & 1~эВ~$=1,6{\cdot}10^{-19}$~Дж\\
мощность       &$P$& Ватт     &  Вт   & 1~эрг/с~$=10^{-7}$~Вт\\
эл. заряд      &$q$& Кулон    &  Кл    & 1~ед.СГС~$=1/(3{\cdot}10^9)$~Кл\\
зл. напряж.    &$U$&   Вольт&  В     & 1~ед.СГС~$=300$~В \\
эл. сопрот.    &$R$& Ом     &  Ом    &1~ед.СГС(с/см)$=9{\cdot}10^{11}$~Ом\\
эл. проводим   &$G$& Сименс & См   &1~ед.СГС~$=1/(9{\cdot}10^{11})$~См\\
уд. сопрот.    &$\rho$&Ом$\cdot$метр &Ом$\cdot$м&1~ед.СГС($с^{-1})=
                               9\.10^9\mbox{Ом}\.\mbox{м}$\\
уд. проводим.&$\sigma$&$\frac{Сименс}{метр}$&См/м&1~ед.~СГС~$=1/(9{\cdot}10^9)$~См/м\\
напряжённость  &      &           &    &                        \\
\qquad эл. поля  &$E$&$\frac{Вольт}{метр}$&  В/м   &1~ед.~СГС~$=3{\cdot}10^4$~В/м\\
эл. индукция   &$D$&$\frac{Кулон}{метр^2}$&Кл/м\^2&$12\pi\.10^{5}~ед.~СГС=1~Кл/м^2$\\
эл. ёмкость    &$C$&Фарада & Ф &$9\.10^{11}~ед.~СГС~(см)=1~Ф$\\
напряжённость  & & & & \\
\qquad магн. поля&$H$&$\frac{Ампер}{метр}$& А/м & 1~Э (эрстед)~$=79,6$~А/м \\
магн. поток      &$\Phi$& Вебер  & Вб&1~Мкс~(максвелл)$=10^{-8}$~Вб\\
магн. индукция   &$B$& Тесла     & Тл     & 1~Гс (гаусс)~$=10^{-4}$~Тл\\
индуктивность    &$L$& Генри     & Гн     & 1~ед.~СГС (см)~$=10^{-9}$~Гн\\
\hline
\et
}

%%%%%%%%%%%%%%%%%%%%%%%%%%%%%%%%%%%%%%%%%%%

\newpage

%t3
\def\tabline#1#2#3#4#5#6#7#8{#1&#2&#3&#4&#5&#6&#7&#8\\}

\etab[$\rho$~--- плотность (при 20\C); $t_{пл}$ и $t_{кип}$~--- температуры плавления и кипения;
$\alpha$~--- температурный коэффициент линейного расширения изотропных элементов при 0\C]
{Некоторые постоянные элементов {\rm при давлении 760~мм~рт.~ст.}}
{\def\pb#1#2{\parbox{#1em}{\utabstrut #2\dtabstrut}}\footnotesize
\bt{l|c|c|c|c|c|c|c}\hline
\pb{8}{\hfil Элемент}&
\kern-0.5em\pb{3}{\centering Сим\-вол}\kern-0.6em&
$Z$&$A$&
%\kern-0.5em\pb{4.5}{\centering Атомная масса$^{1)}$}\kern-0.6em&
$\rho$, $\frac{г}{см^3}$&
$t_{пл}$,\unsp\C&$t_{кип}$,\unsp\C&
\kern-0.5em\pb{3}{\centering $\alpha$,\\10\^{-6}\,К\^{-1}}\\
\hline
\tabline{Алюминий}{Al}{13}{26,98}{2,70}{660}{2447}{22,58}
\tabline{Барий}{Ba}{56}{137,34}{3,78}{710}{1637}{19,45}
\tabline{Бериллий}{Be}{4}{9,01}{1,84}{1283}{2477}{10,5}
%\tabline{Бор (крист.)}{B}{5}{10,81}{3,33}{2030}{3900}{8}
\tabline{Бром}{Br}{35}{79,90}{3,12}{$-7,3$}{58,2}{8,3}
\tabline{Ванадий}{V}{23}{50,94}{5,96}{1730}{3380}{---}
\tabline{Висмут}{Вi}{83}{209,98}{9,75}{271,3}{1559}{$16,6^{2)}$}
\tabline{Вольфрам}{W}{74}{183,85}{18,6--19,1}{3380}{5530}{4,3}
\tabline{Германий}{Ge}{32}{72,59}{5,46}{937,2}{2830}{5,8}
\tabline{Железо}{Fe}{26}{55,85}{7,87}{1535}{---}{12,1}
\tabline{Золото}{Au}{79}{196,97}{\tle19,3\pd}{1063}{2700}{14,0$^{2)}$}
\tabline{Индий}{In}{49}{114,82}{7,28}{156,01}{2075}{30,5$^{2)}$}
%\tabline{Йод}{I}{53}{126,90}{4,94}{113,6}{182,8}{93,0}
\tabline{Иридий}{Ir}{77}{192,2}{\tle22,42}{2443}{4350}{6,5}
\tabline{Кадмий}{Cd}{48}{112,40}{8,65}{321,03}{765}{29,0}
\tabline{Калий}{K}{19}{39,1}{0,87}{63,4}{753}{84}
\tabline{Кальций}{Ca}{20}{40,08}{1,55}{850}{1487}{22(0)}
\tabline{Кобальт}{Co}{27}{58,93}{8,71}{1492}{2255}{12,0}
\tabline{Кремний(крист.)}{Si}{14}{28,09}{2,42}{1423}{2355}{2,3}
\tabline{Литий}{Li}{3}{6,94}{0,53\tre4}{180,5}{1317}{---}
\tabline{Магний}{Mg}{12}{24,3}{1,74}{649}{1120}{---}
\tabline{Марганец}{Mn}{25}{54,94}{7,42}{1244}{2095}{22,6}
\tabline{Медь}{Cu}{29}{63,55}{8,93}{1083}{2595}{16,6$^{2)}$}
\tabline{Молибден}{Mo}{42}{95,94}{9,01}{2625}{4800}{5,19}
\tabline{Натрий}{Na}{11}{22,99}{0,971}{97,82}{890}{72}
\tabline{Неодим}{Nd}{60}{144,24}{6,96}{1019}{3110}{8,6}
\tabline{Никель}{Ni}{28}{58,71}{8,6--8,9}{1453}{2800}{14,0}
\tabline{Олово (серое)}{Sn}{50}{118,69}{5,8}{231,9}{2687}{---}
\tabline{Палладий}{Pd}{46}{106,4}{12,16}{1552}{3560}{12,4$^{2)}$}
\tabline{Платина}{Pl}{78}{195,09}{21,37}{1769}{4310}{9}
\tabline{Родий}{Rh}{45}{102,91}{12,44}{1960}{3960}{8,7}
\tabline{Ртуть (жидк.)}{Hg}{80}{200,59}{13,546}{$-38,86$}{356,73}{---}
\tabline{Рубидий}{Rb}{37}{85,47}{1,53}{38,7}{701}{90}
\tabline{Свинец}{Pb}{82}{207,19}{11,34}{327,3}{1751}{28,3}
\tabline{Селен (крист.)}{Se}{34}{78,96}{4,5}{217,4}{657}{20,3}
%\tabline{Сера (ромбич.)}{S}{16}{32,06}{2,1}{115,18}{444,6}{74}
\tabline{Серебро}{Ag}{47}{107,87}{10,4--10,6}{960,8}{2212}{19,0$^{2)}$}
%\tabline{Стронций}{Sr}{38}{87,62}{2,54}{770}{1367}{20,6}
\tabline{Сурьма}{Sb}{51}{121,75}{6,62}{630,5}{1637}{9,2}
%\tabline{Тантал}{Ta}{73}{180,95}{16,6}{2996}{5400}{6,2}
%\tabline{Теллур (крист.)}{Te}{52}{127,6}{6,25}{449,5}{989,8}{17,0}
\tabline{Титан}{Ti}{22}{47,9}{4,5}{1668}{3280}{7,7}
%\tabline{Торий}{Th}{90}{232,04}{11,1--11,3}{1695}{4200}{9,8}
\tabline{Углерод (графит)}{C}{6}{12,01}{2,25}{3500}{3900}{---}
%\tabline{Фосфор (белый)}{P}{15}{30,97}{1,83}{44,2}{---}{125}
\tabline{Хром}{Cr}{24}{52,00}{7,1}{1903}{2642}{7,78}
\tabline{Цезий}{Cs}{55}{132,90}{1,87}{28,64}{685}{97}
\tabline{Цинк}{Zn}{30}{65,37}{6,97}{419,5}{907}{32}
%\tabline{Цирконий}{Zr}{40}{91,22}{6,44}{1855}{4380}{5,1}
\hline
\et
}

{\small $^{1)}$ Атомная масса дана по отношению  к углероду: $m({}^{12}\mbox{C})=12$ а.е.м.

$^{2)}$~При 20\C.}

%---------------------------------------------------
\newpage

\etab{ЭДС термопар при различных температурах}{%
\bt{c|c|c|c|c}\hline
\tabstrut&\multicolumn{4}{c}{ЭДС, мВ}\\
\cline{2-5}
\utabstrut $t$,&Платина --- плати-&Хромель ---&Железо ---&Медь ---\\
\dtabstrut \C  &на + 10\% родия &алюмель  &константан&константан\\
\hline
100 &0,64 &\pd4,1 &\pd5 &\pd4 \\
200 &1,44 &\pd8,1 &11 &\pd9 \\
300 &2,31 &12,2 &16 &15 \\
400 &3,25 &16,4 &22 &21 \\
500 &4,22 &20,6 &27 & \\
600 &5,23 &24,9 &33 & \\
700 &6,26 &29,1 &39 & \\
800 &7,34 &33,3 &45 & \\
900 &8,45 &37,4 &52 & \\
\tle1000 &9,59 &41,3 &58 & \\
\tle1200 &\tle11,95 &48,9 & \\
\tle1400 &\tle14,37 &55,9 & \\
\tle1600 &\tle16,77 & & & \\ \hline
\et
}
%-------------------------------------------------

\bv\bv

\etab{Удельное сопротивление и температурный коэффициент сопротивления металлических проволок {\rm(при~18\C)}}
{%
\bt{l|c|c}\hline
\tabstrut\hfil Вещество &
\kern-0.5em\parbox{6em}{\centering\utabstrut$\rho$,\,10\^{-8} Ом\.м\dtabstrut}\kern-0.5em&
\kern-0.5em\parbox{4em}{\centering\utabstrut$\alpha$,\,10\^{-4}\,К\^{-1}\dtabstrut}\kern-0.5em \\ \hline
Алюминий &\pd3,2\tre1 &38 \\
Вольфрам &\pd5,5 &51 \\
Железо (0,1\% C)&12,0 &62 \\
Золото &\pd2,4\tre2 &40 \\
Латунь &6--9 &10 \\
Манганин (3\% Ni, 12\% Mn, 85\% Cu)&44,5 &0,02--0,5 \\
Медь &\pd1,7\tre8 &42\tre{,8} \\
Никель &11,8 &27 \\
Константан ({\footnotesize 40\% Ni, 1,2\% Mn, 58,8\% Cu}) &49,0 &$-0,4\div0,1$ \\
Нихром ({\footnotesize 67,5\% Ni, 1,5\% Mn, 16\% Fe, 15\% Cr})&\tle110\phantom{,0} &\pd1\tre{,7} \\
Олово &11,3 &45 \\
Платина &11,0 &38 \\
Свинец &20,8 &43 \\
Серебро &\pd1,6\tre6 &40 \\
Цинк &\pd6,1 &37 \\ \hline
\et
}

%---------------------------------------------------------------------

\newpage
\def\tabline#1#2#3#4{#1&#2&#3&#4\\}
\etab{Электрические свойства металлов {\rm (при 20\C)}}
{%
\bt{l|c|c|c}\hline

\tabstrut     &{Электро-}&{Постоянная}&{Подвижность }\\
         Металл&провододность&Холла    &носителей тока\\
               &$\sigma,\;10^7$(Ом\.м)\^{-1}
&$R,\,10^{-10}$~м\^3/Kл&$b$,~см\^2/(В\.с)\\
\hline
\tabline{Алюминий }{3,1}{$-0,33$}{12,3}
\tabline{Вольфрам}{1,8}{$+1,1\pd$}{20}
\tabline{Золото}{4,1}{$-0,7\pd$}{32}
\tabline{Медь}{5,6}{$-0,53$}{32}
\tabline{Молибден}{1,7}{$+1,8\pd$}{30}
\tabline{Олово}{0,9}{$-0,02\tre2$}{0,17}
\tabline{Платина}{0,9}{$-1,27$}{12}
\tabline{Серебро}{6,0}{$-0,9\pd$}{56}
\tabline{Цинк}{1,6}{$+1,04$}{17,5}\hline
\et
}

%---------------------------------------------------------------------

\bv

\def\tabline#1#2#3#4#5{#1&#2&#3&#4&#5\\}
\def\pb#1#2{\parbox{#1ex}{\utabstrut #2\dtabstrut}}
\def\mc#1{\multicolumn{2}{c}{#1}}

\etab{Электрические свойства полупроводников}
{%
\bt{l|c|c|c|c}\hline
\tabstrut Вещество&
\pb{13.4}{Собственное удельное сопротивление при 20\C, $\rho,~Ом\cdot м$}&
\pb{10}{\centering Диэлектр. прониц. $\varepsilon$}&
\mc{\pb{20}{Подвижности носителей тока в области собственной проводимости при 20\C, см\^2/(В\.с)}}\\\cline{4-5}
&            &           &электроны&дырки\\
\hline
\tabline{Алмаз}{10\^6--10\^{10}}{5,5--16,5}{1800}{1200}
\tabline{Германий}{0,43}{16}{3800}{1800}
\tabline{Кремний}{2,6\.10\^3}{11,7}{1300}{500}
\tabline{Селен (крист.)}{10\^3--10\^{10}}{6}{---}{---}
\tabline{Теллур}{10\^{-3}}{25}{1700}{1200}
\tabline{Сульфид~свинца}{2\.10\^{-3}}{---}{600}{200}
\tabline{Антимонид~индия}{7\.10\^{-5}}{17}{78000}{750}
\tabline{Арсенид~~галия}{1,5}{12,7}{85000}{420}\hline
\et
}

\bv

\etab{Работа выхода электронов из металлов}
{%
\bt{@{ \tabstrut}l|l||l|l||l|l}\hline
Металл&$W$, эВ&Металл&$W$, эВ&Металл&$W$, эВ\\ \hline

Алюминий&4,25&Медь&4,40&Ртуть&4,52\\
Барий&2,49&Никель&4,50&Серебро&4,3\\
Вольфрам&4,54&Олово&4,38&Цезий&1,81\\
Железо&4,31&Платина&5,32&Цинк&4,24\\ \hline
\et
}

%------------------------------------------
\newpage

\def\tabline#1#2#3{#1&#2&#3\\}
\def\mc#1{\hline\multicolumn{3}{c}{\tabstrut\it#1}\\}

\etab{Удельное сопротивление и диэлектрическая проницаемость диэлектриков\\
(при 20\C для не очень высоких частот)}
{%
\bt{l|c|c}\hline
\tabstrut{Вещество}&{~~$\rho$, Ом\.м~~}&$\varepsilon$\\
\mc{Твёрдые тела}
%\tabline{Бакелит}{10\^{11}--10\^{12}}{4,5}
\tabline{Битум}{10\^{13}--10\^{14}}{2,5--3}
\tabline{Бумага сухая}{10\^{11}--10\^{12}}{2--2,5}
\tabline{Гетинакс}{10\^8--10\^9}{5--6}
\tabline{Каучук}{10\^{14}}{2,4}
\tabline{Кварц}{10\^{12}--10\^{13}}{3,5--4,5}
\tabline{Керамика конденсаторная~~~~~~}{10\^9}{10--200}
\tabline{Метатитанат бария}{3\.10\^{16}}{2000}
\tabline{Парафин}{---}{2--2,3}
\tabline{Плексиглас}{10\^{11}}{3,5}
\tabline{Полистирол}{10\^{15}--10\^{17}}{2,4--2,6}
\tabline{Полихлорвинил}{10\^{14}}{3}
\tabline{Полиэтилен}{10\^{14}}{2,3--2,4}
\tabline{Сегнетова соль}{---}{500}
\tabline{Слюда}{10\^{14}}{5,7--7}
\tabline{Стекло}{10\^6--10\^{15}}{4--16}
\tabline{Текстолит}{10\^7--10\^8}{--}
\tabline{Фарфор}{10\^{13}}{4,5--7}
%\tabline{Шеллак}{10\^{13}--10\^{14}}{3,5}
\tabline{Эбонит}{10\^{13}--10\^{14}}{2,5--3}
\tabline{Янтарь}{10\^{15}--10\^{18}}{2,8}
\mc{Жидкости}
\tabline{Бензин}{10\^{10}}{2}
\tabline{Вода дистиллированная}{10\^3--10\^4}{81}
\tabline{Масло вазелиновое}{10\^{14}}{2}
\tabline{Масло касторовое}{10\^9}{4,6--4,8}
\tabline{Масло трансформаторное}{10\^{10}--10\^{13}}{2,2}
\tabline{Скипидар}{10\^{11}}{2,2}
\tabline{Спирт этиловый}{10\^{4}--10\^{5}}{27}
\mc{Газы  {\rm(760 мм рт. ст.)}}
\tabline{Азот}{---}{1,00054}
\tabline{Воздух сухой}{10\^{14}--10\^{15}}{1,00025}
\tabline{Гелий}{---}{1,00007}
\tabline{Кислород}{---}{1,00055}
\tabline{Углекислый газ}{---}{1,0009\pd}\hline
\et
}

%-------------------------------------------------

\newpage

\def\tabline#1#2#3#4{#1&#2&#3&#4\\}
\def\tablinez#1#2#3{#1&#2&\multicolumn{2}{c}{#3}\\}

\etab{Магнитная восприимчивость элементов и~соединений при~20\C ($B=\mu_0(1+\chi)H$)}
{%
\bt{@{\tabstrut~}l|c||l|c}\hline
Вещество&{$\chi,\;10^{-6}$}&Вещество&{$\chi,\;10^{-6}$}\\
\hline
\tabline{Алюминий}{23}{Серебро}{$-26,25$}
\tabline{Висмут}{$-176$}{Стекло}{$-12,6$}
\tabline{Вода}{$-9$}{Цинк}{$-12,3$}
\tabline{Вольфрам}{176}{Эбонит}{14,0} \cline{3-4}
\tablinez{Золото}{$-36,7$}{}
\tablinez{Калий}{5,6}{\it Газы}
\tabline{Каменная соль}{$-12,6$}{Азот}{0,013}
\tabline{Кварц}{$-15,1$}{Водород}{$-0,063$}
\tabline{Кислород жидкий}{3400}{Воздух}{0,38}
\tabline{Медь}{$-10,3$}{Гелий}{$-1,1$}
\tabline{Платина}{360}{Кислород}{1,9}\hline
\et
}

%---------------------------------------------------------

\bv\bv

\etab{Точки Кюри некоторых веществ}
{%
\bt{@{\tabstrut~}l|c}\hline
Вещество&Точка Кюри, \C\\
\hline
{\quad\it Сегнетоэлектрики}&{}\\
Метатитанат бария&100\\
Сегнетова соль&Верхняя $+22,5$\\
&нижняя\ $-15$~~~\\
{\quad\it Ферромагнетики}&\\
Железо&770\\
Железо кремнистое (Fe + 4,3\% Si)&690\\
Кобальт&1130\\
Никель&358\\
Пермаллой (22\% Fe + 78\% Ni)&550\\
Гадолиний&16\\
Магнетит (Fe$_3$O$_4$)&572\\
Ферриты&100--600\\
\hline
\et
}

%--------------------------------------------------
\newpage
\def\tabline#1#2#3#4#5#6{\tabstrut #1&#2&#3&#4&#5&#6\\}
\def\pb#1#2{\parbox{#1ex}{\centering\utabstrut #2\dtabstrut}}

\etab{Свойства ферромагнитных материалов\\ {\bf Магнитомягкие материалы}}
{\small%
\bt{l|c|c|c|c|c}\hline
\tabstrut Вещество&
\pb{11}{Состав~(\%), остальное железо и примеси}&
\pb{9}{Началь\-ная проница\-емость}&
\pb{9}{Макси\-мальная проница\-емость}&
\pb{9}{Коэрци\-тивная сила,\\ \hfil $H_C$, А/м}&
\pb{9}{Индукция насыщения\\ \hfil $B$, Тл}\\ \hline
\tabline{ Железо}{}{}{}{}{}
\tabline{\quad чистое}{0,05(прим.)}{10 000}{200 000}{4}{2,15}
\tabline{\quad техническое}{0,2(прим.)}{150}{5 000}{80}{2,15}
\tabline{\quad кремнистое}{3~Si}{1 500}{40 000}{8}{2,0}
\tabline{Сталь мягкая}{0,2 C}{120}{2 000}{140}{2,12}
\tabline{Пермаллой }{78,5 Ni}{8 000}{100 000}{4}{1,08}
\tabline{Пермендюр}{50 Co}{800}{5 000}{160}{2,45}
\tabline{Кобальт}{99 Co}{70}{250}{800}{1,79}
\tabline{Никель}{99 Ni}{110}{600}{400}{0,61}
\tabline{Ферриты}{}{1000}{(3--10)\.10\^3}{8--600}{0,2--0,4}\hline
\et
}

\bv\bv

\def\tabline#1#2#3#4{\tabstrut #1&#2&#3&#4\\}
\etab{Свойства ферромагнитных материалов\\ {\bf Магнитожёсткие материалы}}{\small%
\bt{l|c|c|c}\hline
Вещество&
\pb{25}{Состав ($\%$),\\ остальное~--- железо}&
\pb{10}{Коэрци\-тивная сила\\ $H_C$,~А/м }&
\pb{12}{Остаточная индукция $B$,~Тл}\\ \hline
\tabline{ Сталь}{}{}{}
\tabline{\quad углеродистая}{0,9 C, 1 Mn}{4 000}{1,0}
\tabline{\quad вольфрамовая}{0,4 C, 6 W}{5 200}{1,05}
\tabline{\quad кобальтовая}{1,0 C, 3 Co, 4 Cr, 0,4 Mn}{6 400}{1,0}
\tabline{\quad Альнико}{19 Ni, 10 Al, 18 Co, 3 Cu}{52 000}{0,9}
\tabline{\quad Магнико}{13,5 Ni, 9 Al, 24 Co, 3 Cu}{40 000}{1,23}
\tabline{Платина-железо}{78 Pt}{120 000}{0,6}
\tabline{Платина-кобальт}{77 Pt, 23 Co}{320 000}{0,5}
\tabline{Ферриты}{}{(120--300)10\^3}{0,2--0,4}\hline
\et
}
