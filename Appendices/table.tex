\pagestyle{empty}
\newgeometry{top=1cm,left=1.5cm,right=1.7cm,bottom=1.5cm}
% \vspace*{-5\baselineskip}
\begin{labsupplement}[Таблицы физических величин]
\begin{longtable}{p{46mm}>{\centering}p{14mm}p{45mm}}
\caption{Основные физические постоянные [\ref{bib:codata}]}\\
% \def\tabline#1#2#3{#1 & #2 & #3 \\ \hline}
% \centering\renewcommand{\arraystretch}{1.15}
% \begin{tabular}{p{46mm}>{\centering}p{14mm}p{45mm}}
\toprule[1pt]
\textbf{Величина}     & \textbf{Обозн.} & \textbf{Значение (погрешность)} \\
\midrule[1pt]
Скорость света в вакууме
                      & $c$
                      & 299\,792\,458~м/с (точно)                         \\ \hline
Постоянная Планка     & \hfil$h$\newline \newline
                        $\hbar=\frac{h}{2\pi}$
                      & $6,626\,070\,040(81)\times \!\!
                            \begin{array}{l}
                                10^{-34}\;Дж\cdot с \\[-2pt]
                                10^{-27}\;эрг\cdot с
                            \end{array} $ \newline
                        $1,054\,571\,800(13)\times \!\!
                            \begin{array}{l}
                                10^{-34}\;Дж\cdot с \\[-2pt]
                                10^{-27}\;эрг\cdot с
                            \end{array}$                                 \\ \hline
Постоянная Больцмана  & $k_{Б}$
                      & $1,380\,648\,52(79)\times \!\!
                        \begin{array}{l}
                            10^{-23}\;Дж/К \\[-2pt]
                            10^{-16}\;эрг/К
                        \end{array}$                                    \\ \hline
Постоянная Авогадро   & $N_{А}$
                      & $6,022\,140\,857(74)\cdot 10^{23}~моль^{-1}$        \\ \hline
%\tabline{Газовая постоянная}{$R=kN_A$}{8,314472(15)~Дж/(моль\cdot К)}
%\tabline{Объём моля идеального газа при нормальных условиях\\>{\small}m{45mm} >{\small}m{14mm} >{\small}m{45mm}}
%($T_0=273,15$~К, $P_0=101325$~Па)}{$\ds V_0=\frac{RT_0}{P_0}$}{$22,413996(39)\cdot 10^{-3}~\ds\frac{м^3}{моль}$}
%\tabline{Число Лошмидта}{$N_л=N_A/V_0$}{$2,6867774(47)\cdot 10^{19}~см^{-3}$}
Гравитационная постоянная
                      & $G$
                      & $6,674\,08(31)\cdot 10^{-11}\;\frac{Н\cdot м^2}{кг^2}$  \\ \hline
%\tabline{Постоянная Фарадея}{$F$}{$96\,485,3399(24)$~Кл/моль}
%\tabline{Постоянная тонкой~структуры}{$\alpha$\\$\alpha^{-1}$}{$7,297 352 5376(50)\cdot 10^{-3}$\\$137,035 999 679(94)$}
Магнитная постоянная  & $\mu_0$
                      & $4\pi\cdot 10^{-7} \approx
                12,566\,370...\cdot 10^{-7}\;\frac{\text{Гн}}{\text{м}}$ \\ \hline
Электрическая постоянная
                      & $\varepsilon_0=\frac{1}{\mu_0c^2}$
                      & $8,854\,187...\cdot 10^{-12}~\frac{Ф}{м}$  \bigstrut\\ \hline
Элементарный заряд    & $e$
                      & $1,602\,176\,6208(98)\cdot 10^{-19}$~Кл\newline
                        $4,803\,204\,6730(33)\cdot 10^{-10}$~ед.\,СГС   \\ \hline
Электрон-вольт        & 1 эВ
                      & $1,602\,176\,6208(98)\times \!\!
                            \begin{array}{l}
                                10^{-19}~Дж \\[-2pt]
                                10^{-12}~эрг
                            \end{array}$                                \\ \hline
Атомная единица массы & 1 a.e.м.
                      & $1,660\,539\,040(20)\cdot 10^{-27}$~кг\newline
                        ($=\frac{1}{N_A}\cdot 10^{-3} \frac{кг}{моль}$) \\ \hline
Масса электрона  & $m_e$ \newline
                   $m_ec^2$  & $0,910 938 356(11)\cdot 10^{-30}$~кг \newline
%                              $8,187\,105\,64(10)\cdot 10^{-24}~Дж$\newline
                               $0,510\,998\,9461(31)~МэВ$               \\ \hline
%\tabline{Удельный заряд электрона}{$e/m_e$}{$1,758 820 150(44)\cdot 10^{11}$~Кл/кг}
Масса протона    & $m_p$ \newline
                   $m_pc^2$ \newline
                   $m_p/m_e$
                 & $1,672 621 637(83)\cdot 10^{-27}$~кг\newline
                   $938,272\,0813(58)$~MэВ \newline
                   $1836,152\,673\,89(17)$                              \\ \hline
Масса нейтрона   & $m_n$ \newline
                   $m_nc^2$
                 & $1,674\,927\,471(21)\cdot 10^{-27}$~кг\newline
                   $939,565\,4133(58)$~MэВ                              \\ \hline
Магнетон Бора    & $\mathfrak{m}_{Б}=\frac{e\hbar}{2m_e}$
                 & $9,274\, 009\,994(57)\cdot 10^{-24}~Дж/Тл$       \bigstrut\\ \hline
Ядерный магнетон & $\mathfrak{m}_{я}=\frac{e\hbar}{2m_p}$
                 & $5,050\,783\,699(31)\cdot 10^{-27}~Дж/Тл$        \bigstrut\\ \hline
Магнитный момент электрона\newline
(в единицах $\mathfrak{m}_{Б}$)
                 & $\mathfrak{m}_e/\mathfrak{m}_{Б}$
                 & $1,001\,159\,652\,180\,91(26)$                       \\ \hline
Магнитный момент протона\newline
(в единицах $\mathfrak{m}_{я}$)
%(собственный, в единицах $\mathfrak{m}_я$)
                 & $\mathfrak{m}_p/\mathfrak{m}_{я}$
                 & $2,792\,847\,3508(85)$                               \\ \hline
Магнитный момент нейтрона
                 & $\mathfrak{m}_n/\mathfrak{m}_{я}$
                 & $-1,913\,042\,73(45)$                                \\ \hline
Радиус первой орбиты\newline
атома водорода (радиус Бора)
                 & $a_0$
                 & $0,529\,177\,210\,67(12)\cdot 10^{-10}$~м                \\ \hline
Постоянная Ридберга
                 & $R_{\infty}$ \newline
                   $\mathrm{Ry} = hcR_{\infty}$
                 & $10\,973\,731,568\,508(65)$~м \newline
                   $2,179\,872\,325(27)\cdot 10^{-18}\;Дж$ \newline
                   $13,605\,693\,009(84)$~эВ                            \\ \hline
%Классический радиус электрона
%                 & $r_e = \frac{e^2}{4\pi \varepsilon_0 m_e c^2}$
%                 & $2,817\,940\,3227(19)\cdot 10^{-15}$~м \\ \hline
\mbox{Постоянная Стефана--Больцмана}
                 & $\sigma_{СБ}$
                 & $5,670\,367(13)\cdot 10^{-8}\;\frac{Вт}{м^2\cdot К^4}$       \\
\bottomrule[1pt]
\end{longtable}
% \end{tabular}\par
% \vspace*{-\baselineskip}\smallskip
% \flushleft
% \noindent\footnotesize{}* Данные согласно \textit{CODATA Recommended Values of the Fundamental
% Physical Constants}, 2014.
% Запись $x = 1,234\,56(78)$ эквивалентна $x=1,234\,56\pm 0,000\,78$.
% \end{table}

\begin{table}
    \begingroup
    % \def\tabstrut{\vrule width 0pt height 1.3em depth 0.5em\relax}
    % % \def\vs{\vphantom{$\ds\frac{A}{B}$}}%
    % \let\vs=\tabstrut
    % \def\bbx{\raggedright\baselineskip=9pt}%
    % \def\pbl#1#2{\parbox{#1}{\bbx #2}}%
    % \def\vr{\vbox to 3pt{}}%
    % \def\pb#1{$\vcenter{\vr\hbox{\pbl{30mm}{#1\hfill}}\vr}$}%
    % \def\pb#1{\parbox{30mm}{#1}}%
    % \def\tabline#1#2#3{%
    % \parbox{0.4\textwidth}{\raggedright\utabstrut #1\dtabstrut}&%
    % \parbox{0.1\textwidth}{\utabstrut\centering #2\dtabstrut}&%
    % \parbox{0.36\textwidth}{\raggedright\utabstrut #3\dtabstrut}\\ \hline
    % }
    \caption{Связь основных единиц измерения в системах СИ и СГС}
    \centering
    \small
    \renewcommand{\arraystretch}{1.15}
%    \def\tabstrut{\vrule width 0pt height 1em depth 0.5em\relax}
    \begin{tabular}{m{29mm}m{11mm}m{26mm}m{35mm}}
\toprule[1pt]
        \bf Наименование & \small\bf Обозн. & \small\bf СИ & \small\bf СГС $^{1)}$ \\
\midrule[1pt]
        Длина    & $l$ & 1~м (\emph{метр})&$10^2$~см\\ \hline
        Масса    & $m$ & 1~кг (\emph{килограмм})&$10^3$ г\\ \hline
        Время    & $t$ & 1~с (\emph{секунда})&1 с\\ \hline
        Сила     & $F$ & 1~Н (\emph{ньютон}) & $10^5$~дин \\ \hline
        Работа,
        энергия  &$A$, $W$&1 Дж (\emph{джоуль})
                 &$10^7$ эрг                                           \bigstrut\\ \hline
        Мощность & $N$ & 1 Вт (\emph{ватт})
                 & $10^7~\frac{эрг}{с\mathstrut}$                      \bigstrut\\ \hline
        Давление & $P$ & 1~Па (\emph{паскаль})
                 & $10~\frac{дин}{см^2\mathstrut}$                     \bigstrut\\ \hline
        Электрический\newline
        заряд    & $q$& 1~Кл (\emph{кулон})
                 & $\frac{c}{10}\approx 3\cdot 10^9$~Фр\newline
                   (\emph{франклин})$^{2)}$                                     \\ \hline
        Сила тока & $I$ & 1~А (\emph{ампер})
                  & $\frac{c}{10}\approx 3\cdot 10^9\;\frac{Фр}{с}$ $^{3)}$     \bigstrut\\ \hline
        Электрический\newline
        потенциал     & $\varphi$& $1~В$ (\emph{вольт})
                      & $\frac{10^{8}}{c}\approx \frac{1}{300}\;\text{ед.~СГС}^{3)}$ \\ \hline
        Напряжённость\newline
        электр. поля  & ${E}$ & $1~\frac{В}{м}$
                      & $\frac{10^{6}}{c}\approx 3,34\cdot 10^{-5}\;\frac{Фр}{см^2}$ \\ \hline
        Поляризация & ${P}$ & $1~\frac{Кл}{м^2}$
                    & $\frac{c}{10}\approx 3\cdot 10^9\;\frac{Фр}{см^2}$ \bigstrut\\ \hline
        Электрическая\newline
        индукция & ${D}$ & $1~\frac{Кл}{м^2}$
                 & $4\pi c\cdot 10^{-5}\approx 3,77\cdot 10^6\;\frac{Фр}{см^2}$ \\ \hline
        Электрическая\newline
        ёмкость  & $C$ & 1~Ф (\emph{фарад})
                 & $c^2\cdot 10^{-9} \approx 9 \cdot 10^{11}$ см                \\ \hline
        Электрическое\newline
        сопротивление & $R$ & 1~Ом (\emph{ом})
                      & $\frac{10^9}{c^2}\approx \frac19 \cdot 10^{-11}~\frac{с}{см\mathstrut}$       \\ \hline
        Удельное\newline
        сопротивление & $\rho$ & $1\;Ом\cdot м$
                      & $\frac{10^{11}}{c^2}\approx \frac19 \cdot 10^{-9}\;\text{с}$ \\ \hline
        Электрическая\newline
        проводимость  & $\Lambda=\frac{1}{R}$ & 1~См (\emph{сименс})
                      & $\approx 9\cdot 10^{11}~\frac{см}{с\mathstrut}$         \\ \hline
        Удельная\newline
        проводимость & $\lambda=\frac{1}{\rho}$ & $1~\frac{См}{м}$
                     & $\approx 9\cdot 10^9~с^{-1}$                             \\ \hline
        Магн. индукция     & ${B}$ & 1~Тл (тесла)
                     & $10^4$~Гс (\emph{гаусс})                        \bigstrut\\ \hline
        Напряжённость\newline
        магнитного поля & ${H}$ & $1~\frac{А}{м}$
                        & $4\pi\cdot10^{-3}$~Э (\emph{эрстэд})                  \\ \hline
        Магн. поток  & $\Phi$ & 1 Вб (\emph{вебер})
                     & $10^8$ Мкс (\emph{максвелл})                    \bigstrut\\ \hline
        Магнитный момент    & $\mathfrak{m}$ & $1\;А\cdot м^2 = 1\;\frac{Дж}{Тл}$
                        & $10^3\;\frac{эрг}{Гс}$                       \bigstrut\\ \hline
        Индуктивность & $L$ & 1~Гн (\emph{генри})
                      & $10^9$~см                                      \bigstrut\\
\bottomrule[1pt]
    \end{tabular}
    \endgroup
    \vspace*{-0.7\baselineskip}
    \flushleft
    \noindent\footnotesize{}$^{1)}$ Здесь $c = 2,99792458\cdot 10^{10}$~см/с --- скорость света в ед. СГС\\
    $^{2)}$ Употребляется также название \emph{статкулон} (статКл),
    либо <<ед. СГС>> заряда. \\
    $^{3)}$ Для единиц тока и напряжения возможно использование
    названий \emph{статампер} (статА) и \emph{статвольт} (статВ) соответственно,
    либо <<ед. СГС>> тока/напряжения.
\end{table}

\newpage


\begingroup
\setlength{\bigstrutjot}{5pt}
\newcommand{\divv}{\mathop{\mathrm{div}}}
\newcommand{\rot}{\mathop{\mathrm{rot}}}
\small
\begin{longtable}{p{40mm}p{30mm}p{30mm}}
\caption{Основные формулы электродинамики в системах СИ и СГС} \\
\toprule[1pt]
\textbf{Наименование} & \textbf{СИ} & \textbf{СГС} \\
\midrule[1pt]
\bigstrut[t]
\multirow[t]{4}{40mm}{Уравнения Максвелла в дифференциальной форме}
      & $\divv\vec{D}=\rho$ & $\divv\vec{D}=4\pi\rho$ \\
      & $\divv\vec{B}=0$    & $\divv\vec{B}=0$ \\
      & $\rot\vec{E}=-\frac{\partial\vec{B}}{\partial t}$ & $\rot\vec{E}= -\frac{1}{c}\frac{\partial\vec{B}}{\partial t}$ \\
      & $\rot\vec{H}=\vec{j}+\frac{\partial\vec{D}}{\partial t}$
          & $\rot\vec{H}=\frac{4\pi}{c}\vec{j}+
              \frac{1}{c}\frac{\partial\vec{D}}{\partial t}$  \bigstrut[b] \\  \hline
\parbox{40mm}{Электрическая\\[-2pt] индукция}
    & $\vec{D}=\varepsilon_0\vec{E}+\vec{P}$
            & $\vec{D}=\vec{E}+4\pi\vec{P}$ \bigstrut \\ \hline
\parbox{40mm}{Напряжённость\\[-2pt] магнитного поля}
    & $\vec{H}=\frac{1}{\mu_0}\vec{B}-\vec{M}$
                & $\vec{H}=\vec{B}-4\pi\vec{M}$ \bigstrut \\ \hline
\bigstrut[t]
\multirow[t]{5}{40mm}{Материальные уравнения}
    & $\vec{P}=\alpha\varepsilon_0\vec{E}$      &  $\vec{P}=\alpha\vec{E}$      \\
    & $\vec{D}=\varepsilon\varepsilon_0\vec{E}$ &  $\vec{D}=\varepsilon\vec{E}$ \\
    & $\vec{M}=\chi\vec{H}$                     &  $\vec{M}=\chi\vec{H}$        \\
    & $\vec{B}=\mu\mu_0\vec{H}$                 &  $\vec{B}=\mu\vec{H}$         \\
    & $\vec{j}=\lambda\vec{E}$&$\vec{j}=\lambda\vec{E}$       \bigstrut[b]   \\ \hline
\bigstrut[t]
\multirow[t]{5}{40mm}{Уравнения Максвелла в интегральной форме}
    & $\oint\limits_{S}\vec{D}\,d\vec{S}=\int\limits_{V}\rho\,dV$
    & $\oint\limits_{S}\vec{D}\,d\vec{S}=4\pi\int\limits_{V}\rho\,dV$ \\
    & $\oint\limits_{S}\vec{B}\,d\vec{S}=0$
    & $\oint\limits_{S}\vec{B}\,d\vec{S}=0$\\
    & $\oint\limits_{L}\vec{E}\,d\vec{l}=-\int\limits_{S}\frac{\partial\vec{B}}{\partial t}\,d\vec{S}$
    & $\oint\limits_{L}\vec{E}\,d\vec{l}=-\frac{1}{c}\int\limits_{S}\frac{\partial\vec{B}}{\partial t}\,d\vec{S}$\\
    & $\oint\limits_{L}\vec{H}\,d\vec{l}=$
    & $\oint\limits_{L}\vec{H}\,d\vec{l}=$\\
    & $=\int\limits_{S}\vec{j}\,d\vec{S}+\int\limits_{S}\frac{\partial\vec{D}}{\partial t}\,d\vec{S}$
    & $=\frac{4\pi}{c}\int\limits_{S}\vec{j}\,d\vec{S}+\frac{1}{c}\int\limits_{S}\frac{\partial\vec{D}}{\partial t}\,d\vec{S}$
    \bigstrut[b] \\ \hline
\bigstrut
Сила Лоренца
                & $\vec{F}=q\vec{E}+\vp{\vec{v}}{\vec{B}}$
                & $\vec{F}=q\vec{E}+\frac{q}{c}\vp{\vec{v}}{\vec{B}}$ \\ \hline
\bigstrut
Закон Кулона
                & $\vec{F}=\frac{1}{4\pi\varepsilon_0}\frac{q_1q_2}{\varepsilon r^3}\vec{r}$
                & $\vec{F}=\frac{q_1q_2}{\varepsilon r^3}\vec{r}$ \\ \hline
Закон Био--Савара
                & $d\vec{H}=\frac{I}{4\pi}\frac{\vp{d\vec{l}}{\vec{r}}}{r^3}$
                & $d\vec{H}=\frac{I}{c}\frac{\vp{d\vec{l}}{\vec{r}}}{r^3}$ \bigstrut \\ \hline
Закон Ампера
                & $d\vec{F}=I\vp{d\vec{l}}{\vec{B}}$
                & $d\vec{F}=\frac{I}{c}\vp{d\vec{l}}{\vec{B}}$ \bigstrut\\ \hline
\parbox{40mm}{Плотность энергии э/м\\[-2.5pt] поля
    ($\varepsilon=\const$, $\mu=\const$)}
& $w=\frac{\varepsilon\varepsilon_0E^2}{2}
    +\frac{B^2}{2\mu\mu_0}$
& $w=\frac{\varepsilon E^2}{8\pi} + \frac{B^2}{8\pi \mu}$
                \bigstrut \\ \hline
Вектор Пойнтинга
                & $\vec{\Pi}=\vp{\vec{E}}{\vec{H}}$
                & $\vec{\Pi}=\frac{c}{4\pi}\vp{\vec{E}}{\vec{H}}$ \bigstrut\\ \hline
\parbox{40mm}{Энергия магнитного\\[-2.5pt] поля тока}
                & $W=\frac{LI^2}{2}$
                & $W=\frac{1}{c^2}\frac{LI^2}{2}$ \bigstrut\\ \hline
%\parbox{40mm}{Плотность импульса\\[-2.5pt] э/м поля}
%        & $\vec{g}=\frac{1}{c^2}\vp{\vec{E}}{\vec{H}}$
%        & $\vec{g}=\frac{1}{4\pi c}\vp{\vec{E}}{\vec{H}}$ \bigstrut\\ \hline
Магнитный поток
        & $\Phi=LI$	&    $\Phi=\frac{1}{c}LI$	\bigstrut\\ \hline
\parbox{40mm}{Индуктивность длинного\\[-2.5pt] соленоида}
        & $L=\mu\mu_0\frac{N^2S}{l}$    &   $L=4\pi\mu \frac{N^2S}{l}$	\bigstrut\\ \hline
\newpage
\caption[]{Основные формулы электродинамики в системах СИ и СГС (продолжение)}  \\
\toprule[1pt]
\textbf{Наименование} & \textbf{СИ} & \textbf{СГС} \\
\midrule[1pt]
\parbox{40mm}{Магнитный момент\\[-2.5pt] витка с током}
        & $\vec{\mathfrak{m}  }=I\vec{S}$		&   $\vec{\mathfrak{m}  }=\frac{1}{c}I\vec{S}$	\bigstrut\\ \hline
\parbox{40mm}{Поле точечного\\[-2.5pt] магнитного диполя}
        & $\vec{B}=\frac{\mu_0}{4\pi}\!\left(\!\frac{3(\vec{\mathfrak{m}}\vec{r})\vec{r}}{r^5}-\frac{\vec{\mathfrak{m}}}{r^3}\!\right)$
        & $\vec{B}=\frac{3(\vec{\mathfrak{m}}\vec{r})\vec{r}}{r^5}-\frac{\vec{\mathfrak{m}}}{r^3}$ \bigstrut\\ \hline
\parbox{40mm}{Поле точечного\\[-2.5pt] электрического диполя}
        & $\vec{E}=\frac{1}{4\pi\varepsilon_0}\!\left(\!\frac{3(\vec{p}\vec{r})\vec{r}}{r^5}-\frac{\vec{p}}{r^3}\!\right)$
        & $\vec{E}=\frac{3(\vec{p}\vec{r})\vec{r}}{r^5}-\frac{\vec{p}}{r^3}$ \bigstrut\\ \hline
\parbox{40mm}{Момент сил, действующий\\[-2.5pt] на виток с~током}
        & \multicolumn{2}{c}{$\vec{M}=\vp{\vec{\mathfrak{m}  }}{\vec{B}}$}\bigstrut\\ \hline
\parbox{40mm}{Сила, действующая на\\[-2.5pt] магнитный диполь}
        & \multicolumn{2}{c}{$\vec{F}=(\vec{\mathfrak{m}  }\vec{\nabla})\vec{B}$} \bigstrut\\ \hline
\parbox{40mm}{Магнитное поле\\[-2.pt] прямого провода}
    & $H = \frac{I}{2\pi r}$ & $H=\frac{2I}{cr}$ \bigstrut \\ \hline
\parbox{40mm}{Ёмкость плоского\\[-2.5pt] конденсатора}
        & $C=\frac{q}{U}=\frac{\varepsilon\varepsilon_0S}{d}$
        & $C=\frac{q}{U}=\frac{\varepsilon S}{4\pi d}$ \bigstrut \\ \hline
Энергия конденсатора
        & \multicolumn{2}{c}{$W=\frac{CU^2}{2}$} \bigstrut\\
\bottomrule[1pt]
\end{longtable}
\endgroup

%%\smallskip
%
%%{\small В скобках указана погрешность последних знаков.}
%
%%-------------------------------------------
%
%\newpage
%
%\etab{Важнейшие единицы физических величин Международной~системы~СИ}{\small
%\bt{|@{\tabstrut~}l|c|c|c|l|}
%\hline
%%\multicolumn{2}{|c|}{Величины}&
%%\multicolumn{2}{|c|}{Единицы}&\\
%%\multicolumn{1}{|c|}{Соотношение}\\ \cline{1-4}
%%Наименование&Обозна-&Наименование&Обозна-&\mbox{\ \ \ }единиц системы СИ и\\
%%&чение&&чение&\mbox{\ \ \ }единиц других систем\\ \hline
%%\multicolumn{5}{|c|}{}\\
%\multicolumn{5}{|c|}{Основные единицы}\\
%\hline
%длина&$l$&метр&м&1~\AA~(Ангстрем)~$=10^{-10}$~м\\
%масса&$m$&килограмм&кг&1~а.е.м.~$=1,66\cdot 10^{-27}$~кг\\
%время&$t$&секунда&с&1~мин~$=60$~с\\
%сила тока&$I$&Ампер&А&1~ед.~СГСМ~$=$\\
% &     &           &       &$=3{\cdot}10^{10}$~ед.~СГСЭ~$=10$~А\\ \hline
%\multicolumn{5}{|c|}{}\\
%\multicolumn{5}{|c|}{Производные единицы}\\
%\hline
%сила, вес&$F$&Ньютон&Н&1~дина~$=10^{-5}$~Н \\
%давление       &$P$& Паскаль  & Па & 1~атм=$760$~мм~Hg~$\approx10^5$~Па\\
%работа,        &$A$&          &       & 1~эрг~$=10^{-7}$~Дж\\
%энергия        &$W$& Джоуль   &  Дж   & 1~эВ~$=1,6{\cdot}10^{-19}$~Дж\\
%мощность       &$P$& Ватт     &  Вт   & 1~эрг/с~$=10^{-7}$~Вт\\
%эл. заряд      &$q$& Кулон    &  Кл    & 1~ед.СГС~$=1/(3{\cdot}10^9)$~Кл\\
%зл. напряж.    &$U$&   Вольт&  В     & 1~ед.СГС~$=300$~В \\
%эл. сопрот.    &$R$& Ом     &  Ом    &1~ед.СГС(с/см)$=9{\cdot}10^{11}$~Ом\\
%эл. проводим   &$G$& Сименс & См   &1~ед.СГС~$=1/(9{\cdot}10^{11})$~См\\
%уд. сопрот.    &$\rho$&Ом$\cdot$метр &Ом$\cdot$м&1~ед.СГС($с^{-1})=
%                               9\cdot 10^9\mbox{Ом}\cdot \mbox{м}$\\
%уд. проводим.&$\sigma$&$\frac{Сименс}{метр}$&См/м&1~ед.~СГС~$=1/(9{\cdot}10^9)$~См/м\\
%напряжённость  &      &           &    &                        \\
%\qquad эл. поля  &$E$&$\frac{Вольт}{метр}$&  В/м   &1~ед.~СГС~$=3{\cdot}10^4$~В/м\\
%эл. индукция   &$D$&$\frac{Кулон}{метр^2}$&Кл/м\^2&$12\pi\cdot 10^{5}~ед.~СГС=1~Кл/м^2$\\
%эл. ёмкость    &$C$&Фарада & Ф &$9\cdot 10^{11}~ед.~СГС~(см)=1~Ф$\\
%напряжённость  & & & & \\
%\qquad магн. поля&$H$&$\frac{Ампер}{метр}$& А/м & 1~Э (эрстед)~$=79,6$~А/м \\
%магн. поток      &$\Phi$& Вебер  & Вб&1~Мкс~(максвелл)$=10^{-8}$~Вб\\
%магн. индукция   &$B$& Тесла     & Тл     & 1~Гс (гаусс)~$=10^{-4}$~Тл\\
%индуктивность    &$L$& Генри     & Гн     & 1~ед.~СГС (см)~$=10^{-9}$~Гн\\
%\hline
%\et
%}
%
%%%%%%%%%%%%%%%%%%%%%%%%%%%%%%%%%%%%%%%%%%%%
%
%\newpage
%
%%t3
\def\tabline#1#2#3#4#5#6#7#8{#1 & #2 & #3 & #4 & #5 & #6 & #7 & #8\\}

\begin{table}
\caption{Некоторые постоянные элементов при атмосферном давлении ($101,325$~кПа).
$\rho$~--- плотность (при 20\oC  ); $t_{пл}$ и $t_{кип}$~---
температуры плавления и кипения;
$\alpha=\frac{1}{l}\left(\frac{\partial l}{\partial T}\right)_P$~---
температурный коэффициент линейного расширения (для изотропных элементов при 300~К)
[\ref{bib:grig}, \ref{bib:handbook}]}
\footnotesize
\begin{tabular}{l|c|c|c|c|c|c|c}\toprule[1pt]
Элемент & \kern-2mm\parbox{5mm}{\centering Сим-\\[-1pt]вол}
        & $Z$ & $A, а.е.м.^{1)}$
        & $\rho,\;\frac{г}{см^3}$
        & $t_{пл},\oC$
        & $t_{кип},\oC$
        & \kern-2mm\parbox{8mm}{\centering $\alpha$, $10^{-6}\;\text{К}^{-1}$}
        \rule[-10pt]{0pt}{22pt} \\ \midrule[1pt]
\tabline{Алюминий}{Al}{13}{26,9815}{2,6889}{660,3}{2519}{23,3}
\tabline{Барий}{Ba}{56}{137,33}{3,59}{727}{1897}{16,4}
\tabline{Бериллий}{Be}{4}{9,0122}{1,848}{1283}{2477}{}
%\tabline{Бор (крист.)}{B}{5}{10,81}{3,33}{2030}{3900}{8}
\tabline{Бром}{Br}{35}{79,904}{3,119}{$-7,2$}{58,8}{}
\tabline{Ванадий}{V}{23}{50,942}{5,96}{1910}{3407}{}
\tabline{Висмут}{Вi}{83}{208,98}{9,78}{271,4}{1564}{}
\tabline{Вольфрам}{W}{74}{183,84}{19,35}{3414}{5555}{4,6}
\tabline{Германий}{Ge}{32}{72,64}{5,32}{938,3}{2833}{5,8}
\tabline{Железо}{Fe}{26}{55,845}{7,874}{1538}{2861}{12,0}
\tabline{Золото}{Au}{79}{196,967}{19,32}{1064,2}{2856}{14,0}
\tabline{Индий}{In}{49}{114,818}{7,31}{156,01}{2075}{}
%\tabline{Йод}{I}{53}{126,90}{4,94}{113,6}{182,8}{93,0}
\tabline{Иридий}{Ir}{77}{192,217}{22,42}{2443}{4350}{}
\tabline{Кадмий}{Cd}{48}{112,411}{8,65}{321,07}{767}{}
\tabline{Калий}{K}{19}{39,0983}{0,862}{63,5}{759}{79,6}
\tabline{Кальций}{Ca}{20}{40,078}{1,55}{842}{1484}{22,4}
\tabline{Кобальт}{Co}{27}{58,9332}{8,90}{1495}{2927}{12,2}
\tabline{Кремний (крист.)}{Si}{14}{28,0855}{2,33}{1414}{3265}{2,54}
\tabline{Литий}{Li}{3}{6,941}{0,534}{180,5}{1342}{47,1}
\tabline{Магний}{Mg}{12}{24,305}{1,738}{650}{1090}{}
\tabline{Марганец}{Mn}{25}{54,9381}{7,21--7,44}{1246}{2061}{}
\tabline{Медь}{Cu}{29}{63,546}{8,96}{1084,6}{2562}{16,7}
\tabline{Молибден}{Mo}{42}{95,94}{10,22}{2622}{4639}{5,27}
\tabline{Натрий}{Na}{11}{22,9898}{0,971}{97,79}{882,9}{71,5}
\tabline{Неодим (гекс.)}{Nd}{60}{144,24}{7,01}{1021}{3074}{7,0}
\tabline{Никель}{Ni}{28}{58,6934}{8,6--8,9}{1455}{2913}{13,0}
\tabline{Олово (сер./бел.)}{Sn}{50}{118,710}{5,85/7,29}{13,2/232}{2602}{}
%\tabline{Палладий}{Pd}{46}{106,42}{12,16}{1552}{3560}{12,4$^{2)}$}
\tabline{Платина}{Pl}{78}{195,078}{21,45}{1768,2}{3825}{8,99}
\tabline{Родий}{Rh}{45}{102,906}{12,41}{1963}{3695}{8,50}
\tabline{Ртуть (жидк.)}{Hg}{80}{200,59}{13,5461}{$-38,829$}{356,62}{}
%\tabline{Рубидий}{Rb}{37}{85,47}{1,53}{38,7}{701}{90}
\tabline{Свинец}{Pb}{82}{207,2}{11,336}{327,46}{1749}{28,5}
\tabline{Селен (крист.)}{Se}{34}{78,96}{4,46}{220,5}{685}{}
%\tabline{Сера (ромбич.)}{S}{16}{32,065}{2,1}{115,18}{444,6}{74}
\tabline{Серебро}{Ag}{47}{107,868}{10,50}{961,78}{2162}{18,9}
%\tabline{Стронций}{Sr}{38}{87,62}{2,54}{770}{1367}{20,6}
\tabline{Сурьма}{Sb}{51}{121,760}{6,691}{630,63}{1587}{}
%\tabline{Тантал}{Ta}{73}{180,95}{16,6}{2996}{5400}{6,2}
%\tabline{Теллур (крист.)}{Te}{52}{127,6}{6,25}{449,5}{989,8}{17,0}
\tabline{Титан}{Ti}{22}{47,867}{4,505}{1668}{3280}{8,3}
%\tabline{Торий}{Th}{90}{232,04}{11,1--11,3}{1695}{4200}{9,8}
\tabline{Углерод (графит)}{C}{6}{12,0107}{1,9--2,3}{---}{3825}{}
%\tabline{Фосфор (белый)}{P}{15}{30,97}{1,83}{44,2}{---}{125}
\tabline{Хром}{Cr}{24}{51,996}{7,18--7,20}{1907}{2671}{5,00}
\tabline{Цезий}{Cs}{55}{132,906}{1,873}{28,5}{671}{97,0}
\tabline{Цинк}{Zn}{30}{65,41}{6,77}{419,5}{907}{}
\tabline{Цирконий}{Zr}{40}{91,224}{6,45}{1854}{4409}{}
\bottomrule[1pt]
\end{tabular}
\par\smallskip
$^{1)}$~Стандартный атомный вес (по естественному изотопному составу)\\
\end{table}


\begin{table}
\caption{Удельное сопротивление $\rho_0$ и температурный коэффициент
    сопротивления  $\alpha_0 = \frac{1}{\rho_0}\frac{d\rho}{dT}$ при~$0\oC$
    для чистых металлов$^{1)}$ и сплавов$^{2)}$ [\ref{bib:grig}, \ref{bib:handbook}]}
\small
\begingroup\centering
\begin{tabular}{lcc}
\toprule[1pt]
Вещество &
$\rho_0,\,10^{-8}\;Ом\cdot м$ &
$\alpha_0,\,10^{-3}\;К^{-1}$ \\
\midrule[1pt]
Алюминий & 2,417 & 4,60 \\
Вольфрам & 4,82 & 5,10 \\
Железо & 8,57 & 6,51 \\
Золото & 2,051 & 4,02 \\
Константан (54\% Cu, 45\% Ni, 1\% Mn) & 50 & $-0,03$ \\
Латунь (62\% Cu, 38\% Zn) & 7,1 & 1,7 \\
Манганин (86\% Cu, 12\% Mn, 2\% Ni) & 43 & 0,02 \\
Медь & 1,543 & 4,33 \\
Молибден & 4,85 & 4,73 \\
Неодим & 71 & 2,00 \\
Никель & 6,16 & 6,92 \\
Нихром (63\% Ni, 20\% Fe, 15\% Cr, 2\% Mn) & 112 & 0,15 \\
Олово & 11,5 & 4,65 \\
Платина & 9,60 & 3,96 \\
Ртуть & 96,1 & 0,99 \\
Свинец & 19,2 & 4,28 \\
Серебро & 1,467 & 4,30 \\
%Сталь (разные марки) & 5,7 -- 10,2 & --- \\
Хром & 11,8 & 3,01 \\
Цинк & 5,46 & 4,17 \\
\bottomrule[1pt]
\end{tabular}
\par\endgroup
\smallskip
$^{1)}$ Сопротивление реального образца определяется формулой
$\rho = \rho_{ост} + \rho_{ид}(T)$, где $\rho_{ост}$~---
не зависящее от $T$ остаточное сопротивление, определяемое чистотой
изготовления образца (\emph{правило Матиссена}).\\
$^{2)}$~Данные для сплавов даны при 20\oC.
\end{table}

\newpage

\begin{table}
\newcommand*{\z}{\phantom{0}}
\newcommand*{\zz}{\phantom{00}}
\caption{ЭДС термопар при различных температурах [\ref{bib:grig}]}
\small\centering
\begin{tabular}{c|c|c|c|c}
\toprule[1pt]
&\multicolumn{4}{c}{ЭДС, мВ}\\
\cline{2-5}
$\Delta t$,& \footnotesize Платина --- плати-& \footnotesize Хромель --- &
\footnotesize Железо --- & \footnotesize Медь ---\\
\oC    &\footnotesize на + 10\% родия &
\footnotesize алюмель  & \footnotesize константан & \footnotesize константан\\
\midrule[1pt]
0   &\z0,00 &\z0,00 &\z0,00 &\z0,000 \\
10  &\z0,06 &\z0,40 &\z0,51 &\z0,391 \\
20  &\z0,11 &\z0,80 &\z1,02 &\z0,789 \\
30  &\z0,17 &\z1,20 &\z1,54 &\z1,196 \\
40  &\z0,23 &\z1,61 &\z2,06 &\z1,611 \\
50  &\z0,30 &\z2,02 &\z2,59 &\z2,035 \\
60  &\z0,36 &\z2,44 &\z3,12 &\z2,467 \\
70  &\z0,43 &\z2,85 &\z3,65 &\z2,908 \\
80  &\z0,50 &\z3,27 &\z4,19 &\z3,357 \\
90  &\z0,57 &\z3,68 &\z4,73 &\z3,813 \\
100 &\z0,64 &\z4,10 &\z5,27 &\z4,277 \\
150 &\z1,03 &\z6,14 &\z8,01 &\z6,702 \\
200 &\z1,44 &\z8,14 &10,78  &\z9,286 \\
250 &\z1,87 &10,15  &13,53  &12,01\z \\
300 &\z2,31 &12,21  &16,33  &14,86\z \\
400 &\z3,25 &16,40  &21,8\z &20,87\z \\
500 &\z4,22 &20,6\z &27,4\z & \\
600 &\z5,23 &24,9\z &33,1\z & \\
700 &\z6,26 &29,1\z &39,1\z & \\
800 &\z7,34 &33,3\z &45,5\z & \\
900 &\z8,45 &37,3\z &51,9\z & \\
1000 &\z9,59 &41,3\z &57,9\z & \\
1200 &11,95 &48,8\z &69,5\z & \\
1400 &14,37 & & & \\
1600 &16,77 & & & \\
\bottomrule[1pt]
\end{tabular}
\end{table}


\newpage

\begin{table}
\caption{Электрические свойства металлов при 20\oC $^{1)}$ [\ref{bib:grig}, \ref{bib:handbook}]}
\small
\begingroup\centering
\begin{tabular}{lccc}\toprule[1pt]
 &  & Постоянная & Подвижность \\
Металл & Проводимость & Холла    & носителей \\
       & $\lambda$, $10^{7}$~См/м & $R_H,\,10^{-10}$~м$^3$/Kл & $\mu=\lambda |R_H|$,~см$^2$/(В$\cdot$с)\\
\midrule[1pt]
Алюминий & 3,79 & $-0,34$  & 13 \\
Вольфрам & 1,88  & $+1,1$ & 21 \\
Золото & 4,51 & $-0,71$ & 32 \\
Медь & 5,96 & $-0,55$ & 33 \\
Молибден & 1,88 & $+1,26$ & 24 \\
Олово & 0,80 & $-0,02$ & 0,16 \\
Платина & 0,97 & $-0,23$ & 2,2 \\
Серебро & 6,28 & $-0,9$ & 57 \\
Цинк & 1,69 & $+0,55$ & 9,3 \\
\bottomrule[1pt]
\end{tabular}
\par\endgroup
\smallskip
{\footnotesize
$^{1)}$ Электропроводящие свойства реальных образцов металлов существенно зависят
от их чистоты (наличия примесей и дефектов).}
\end{table}



\def\tabline#1#2#3#4#5#6{#1&#2&#3&#4&#5&#6\\}
%\def\pb#1#2{\parbox{#1ex}{\utabstrut #2\dtabstrut}}
%\def\mc#1{\multicolumn{2}{c}{#1}}
%
\begin{table}
\caption{Электрические свойства полупроводников (при 300~К)
[\ref{bib:grig}, \ref{bib:semi}]}
\small
\begin{tabular}{l|l|c|c|c|c}
\toprule[1pt]
Вещество & \parbox{0.5cm}{Сим-\\вол} &
\parbox{13.4ex}{Собств. про\-водимость $\lambda_i,~(Ом\cdot см)^{-1}$}&
\parbox{10ex}{\centering Диэлектр. проницаемость $\varepsilon$, отн. ед.}&
\multicolumn{2}{c}{\parbox{18ex}{Подвижность носителей тока, $10^3$~см$^2$/(В$\cdot$с)}}\\ \cline{5-6}
& & & & \parbox{7ex}{\hfil$\mu_e$} & \parbox{7ex}{\hfil$\mu_h$} \\
\midrule[1pt]
\tabline{Алмаз}{C}{$<10^{-13}$}{5,7}{$2,0$}{$2,1$}
\tabline{Германий}{Ge}{$2,1\cdot 10^{-2}$}{16}{$3,8$}{$1,8$}
\tabline{Кремний}{Si}{$3,1\cdot 10^{-6}$}{11,8}{$1,5$}{0,50}
\tabline{Олово (серое, 0\oC)}{$\alpha$-Sn}{$2,0\cdot 10^{3}$}{24}{$1,6$}{$10$}
% \tabline{Сера (крист.)}{$\alpha$-S}{$>10^{15}$}{3,6}{$6,2\cdot 10^{-4}$}{$6,5\cdot10^{-4}$}
\tabline{Антимонид~индия}{InSb}{$2,2\cdot 10^2$}{17}{$77$}{0,85}
\tabline{Арсенид~галлия}{GaAs}{$2,5\cdot 10^{-9}$}{12,7}{$9$}{0,4}
\bottomrule[1pt]
\end{tabular}
\end{table}



\begin{table}
\caption{Работа выхода электронов с поверхности [\ref{bib:handbook}]}
\small\centering
\begin{tabular}{ll||ll||ll}
\toprule[1pt]
Элемент &$W$, эВ& Элемент &$W$, эВ& Элемент &$W$, эВ\\
\midrule[1pt]
Алюминий& 4,06 -- 4,26* & Золото & 5,31 -- 5,47* & Платина & 5,64\\
Барий & 2,52 & Литий & 2,93  & Ртуть & 4,48\\
Вольфрам & 4,55 & Медь & 4,53 -- 5,10* & Серебро & 4,52 -- 4,74*\\
Германий & 5,0 & Никель & 5,04 -- 5,35* & Цезий & 1,95\\
Железо* & 4,67 -- 4,81* & Олово & 4,42 &  Цинк & 3,63\\
\bottomrule[1pt]
\end{tabular}\par
\smallskip
{\footnotesize
* Для монокристаллических образцов работа выхода зависит от ориентации грани.}
\end{table}


\begingroup
\def\tabline#1#2#3#4{#1&#2&#3&#4\\}
\begin{table}
\caption{Удельное сопротивление, диэлектрическая проницаемость
и напряженность пробоя диэлектриков
(при 20\oC~для низких частот) [\ref{bib:grig}, \ref{bib:kikoin}]}
\centering\small
\begin{tabular}{lccc}
\toprule[1pt]
Вещество & $\rho$, Ом\,$\cdot$\,м & $\varepsilon$ & $E_{пр}$, кВ/мм \\
\midrule[1pt]
\multicolumn{4}{c}{Твёрдые тела} \bigstrut \\
%\tabline{Бакелит}{10\^{11}--10\^{12}}{4,5}
% \tabline{Битум}{$10^{13}$--$10^{14}$}{2,5--3}
\tabline{Бумага сухая}{$10^{11}$--$10^{12}$}{2--2,5}{}
\tabline{Гетинакс}{$10^{10}$--$10^{11}$}{7--8}{20--35}
\tabline{Каучук}{$10^{16}$}{2,4}{}
% \tabline{Кварц}{$10^{12}$--$10^{13}$}{3,5--4,5}
% \tabline{Керамика конденсаторная}{$10^9$}{10--200}
\tabline{Парафин}{$10^{15}$--$10^{17}$}{2,1}{20--30}
\tabline{Плексиглас}{$10^{10}$--$10^{11}$}{3,6}{15--25}
\tabline{Полистирол}{$10^{14}$--$10^{15}$}{2,5}{20--25}
\tabline{Поливинилхлорид}{$10^{14}$--$10^{16}$}{3--5}{14--20}
\tabline{Полиэтилен}{$10^{15}$}{2,3}{25--60}
% \tabline{Сегнетова соль}{---}{500}
% \tabline{Слюда}{$10^{14}$}{5,7--7}
\tabline{Стекло}{$10^6$--$10^{15}$}{3,7--16}{}
\tabline{Текстолит}{$10^8$--$10^9$}{8}{4,5--12}
\tabline{Титанат бария}{}{2000}{}
\tabline{Фарфор (электротехнический)}{$10^{12}$--$10^{13}$}{6--7}{20--28}
%\tabline{Шеллак}{10\^{13}--10\^{14}}{3,5}
\tabline{Эбонит}{$10^{12}$--$10^{13}$}{2,8--3,5}{20--35}
\tabline{Янтарь}{$10^{17}$}{2,8}{} \hline
\multicolumn{4}{c}{Жидкости} \bigstrut \\
\tabline{Ацетон}{$10^7$}{20,7}{}
% \tabline{Бензин}{$10^{10}$}{2}
\tabline{Вода дистиллированная}{1--4\,$\cdot 10^4$}{78,3(25\oC)}{}
\tabline{Масло вазелиновое}{$10^{12}$--$10^{13}$}{3,9}{20--22}
\tabline{Масло касторовое}{$10^8$--$10^{11}$}{4,0--4,5}{14--16}
\tabline{Масло конденсаторное}{$10^{12}$--$10^{13}$}{2,2}{20--25}
\tabline{Масло трансформаторное}{$10^{11}$--$10^{12}$}{2,2}{12--26}
% \tabline{Скипидар}{$10^{11}$}{2,2}
\tabline{Спирт этиловый}{$6\cdot 10^{6}$}{27}{}
\hline
\multicolumn{4}{c}{Газы  при $10^5$~Па$^{1)}$} \bigstrut \\
\tabline{Азот}{}{1,00058}{3,2}
\tabline{Аргон}{}{1,000554}{0,8}
\tabline{Воздух сухой}{}{1,00058}{3,2}
\tabline{Гелий}{}{1,000072}{0,6}
\tabline{Кислород}{}{1,00055}{2,9}
\tabline{Неон}{}{1,000127}{0,5}
\tabline{Углекислый газ}{}{1,00096}{2,9}
\bottomrule[1pt]
\end{tabular}
\par\smallskip
\footnotesize\noindent\raggedright
$^{1)}$ Напряжённость пробоя для газов дано при длине промежутка $d=1$~см.
\end{table}
\endgroup

%%-------------------------------------------------
%
%\newpage
%
%\def\tabline#1#2#3#4{#1&#2&#3&#4\\}
%\def\tablinez#1#2#3{#1&#2&\multicolumn{2}{c}{#3}\\}
%
%\etab{Магнитная восприимчивость элементов и~соединений при~20\oC   ($B=\mu_0(1+\chi)H$)}
%{%
%\bt{@{\tabstrut~}l|c||l|c}\hline
%Вещество&{$\chi,\;10^{-6}$}&Вещество&{$\chi,\;10^{-6}$}\\
%\hline
%\tabline{Алюминий}{23}{Серебро}{$-26,25$}
%\tabline{Висмут}{$-176$}{Стекло}{$-12,6$}
%\tabline{Вода}{$-9$}{Цинк}{$-12,3$}
%\tabline{Вольфрам}{176}{Эбонит}{14,0} \cline{3-4}
%\tablinez{Золото}{$-36,7$}{}
%\tablinez{Калий}{5,6}{\it Газы}
%\tabline{Каменная соль}{$-12,6$}{Азот}{0,013}
%\tabline{Кварц}{$-15,1$}{Водород}{$-0,063$}
%\tabline{Кислород жидкий}{3400}{Воздух}{0,38}
%\tabline{Медь}{$-10,3$}{Гелий}{$-1,1$}
%\tabline{Платина}{360}{Кислород}{1,9}\hline
%\et
%}
%
%%---------------------------------------------------------
%
%\bv\bv
%
%\etab{Точки Кюри некоторых веществ}
%{%
%\bt{@{\tabstrut~}l|c}\hline
%Вещество&Точка Кюри, \oC  \\
%\hline
%{\quad\it Сегнетоэлектрики}&{}\\
%Метатитанат бария&100\\
%Сегнетова соль&Верхняя $+22,5$\\
%&нижняя\ $-15$~~~\\
%{\quad\it Ферромагнетики}&\\
%Железо&770\\
%Железо кремнистое (Fe + 4,3\% Si)&690\\
%Кобальт&1130\\
%Никель&358\\
%Пермаллой (22\% Fe + 78\% Ni)&550\\
%Гадолиний&16\\
%Магнетит (Fe$_3$O$_4$)&572\\
%Ферриты&100--600\\
%\hline
%\et
%}
%
%%--------------------------------------------------
%\newpage
%\def\tabline#1#2#3#4#5#6{\tabstrut #1&#2&#3&#4&#5&#6\\}
%\def\pb#1#2{\parbox{#1ex}{\centering\utabstrut #2\dtabstrut}}
%
%\etab{Свойства ферромагнитных материалов\\ {\bf Магнитомягкие материалы}}
%{\small%
%\bt{l|c|c|c|c|c}\hline
%\tabstrut Вещество&
%\pb{11}{Состав~(\%), остальное железо и примеси}&
%\pb{9}{Началь\-ная проница\-емость}&
%\pb{9}{Макси\-мальная проница\-емость}&
%\pb{9}{Коэрци\-тивная сила,\\ \hfil $H_C$, А/м}&
%\pb{9}{Индукция насыщения\\ \hfil $B$, Тл}\\ \hline
%\tabline{ Железо}{}{}{}{}{}
%\tabline{\quad чистое}{0,05(прим.)}{10 000}{200 000}{4}{2,15}
%\tabline{\quad техническое}{0,2(прим.)}{150}{5 000}{80}{2,15}
%\tabline{\quad кремнистое}{3~Si}{1 500}{40 000}{8}{2,0}
%\tabline{Сталь мягкая}{0,2 C}{120}{2 000}{140}{2,12}
%\tabline{Пермаллой }{78,5 Ni}{8 000}{100 000}{4}{1,08}
%\tabline{Пермендюр}{50 Co}{800}{5 000}{160}{2,45}
%\tabline{Кобальт}{99 Co}{70}{250}{800}{1,79}
%\tabline{Никель}{99 Ni}{110}{600}{400}{0,61}
%\tabline{Ферриты}{}{1000}{(3--10)\cdot 10\^3}{8--600}{0,2--0,4}\hline
%\et
%}
%
%\bv\bv
%
%\def\tabline#1#2#3#4{\tabstrut #1&#2&#3&#4\\}
%\etab{Свойства ферромагнитных материалов\\ {\bf Магнитожёсткие материалы}}{\small%
%\bt{l|c|c|c}\hline
%Вещество&
%\pb{25}{Состав ($\%$),\\ остальное~--- железо}&
%\pb{10}{Коэрци\-тивная сила\\ $H_C$,~А/м }&
%\pb{12}{Остаточная индукция $B$,~Тл}\\ \hline
%\tabline{ Сталь}{}{}{}
%\tabline{\quad углеродистая}{0,9 C, 1 Mn}{4 000}{1,0}
%\tabline{\quad вольфрамовая}{0,4 C, 6 W}{5 200}{1,05}
%\tabline{\quad кобальтовая}{1,0 C, 3 Co, 4 Cr, 0,4 Mn}{6 400}{1,0}
%\tabline{\quad Альнико}{19 Ni, 10 Al, 18 Co, 3 Cu}{52 000}{0,9}
%\tabline{\quad Магнико}{13,5 Ni, 9 Al, 24 Co, 3 Cu}{40 000}{1,23}
%\tabline{Платина-железо}{78 Pt}{120 000}{0,6}
%\tabline{Платина-кобальт}{77 Pt, 23 Co}{320 000}{0,5}
%\tabline{Ферриты}{}{(120--300)10\^3}{0,2--0,4}\hline
%\et
%}

\clearpage

\begin{lab:literature}
 \item \label{bib:codata} CODATA Recommended Values of the Fundamental Physical Constants, 2014.
 \item \label{bib:grig} Физические величины: Справочник. / под ред. И.С. Григорьева, Е.З. Мейлихова.~---
 М.\,: Энергоатомиздат, 1991.
 \item \label{bib:kikoin} ???
 \item \label{bib:handbook} CRC Handbook of Chemistry and Physics / ed. by David R. Lide.~---
 CRC Press, Boca Raton, FL, 2005.
 \item \label{bib:semi} \textit{O. Madelung} Semiconductors: Data Handbook.~---
 Springer-Verlag, Berlin, 2004.

\end{lab:literature}

\end{labsupplement}
