\clearpage
{\noindent\large\bfseries ПРЕДИСЛОВИЕ}

\vspace{3pc}

Предлагаемое учебное пособие представляет собой руководство по выполнению 
лабораторных работ по электричеству и магнетизму,
выполняемых в рамках курса общей физики в 3-м семестре студентами 
\mbox{МФТИ}, обучающимися по направлению <<Прикладные математика и физика>>. 
Предыдущие варианты сборника работ выходили в 1964, 1973, 1983 и 2007~годах.
Работа по совершенствованию практикума и руководств к лабораторным занятиям
считается традиционно одной из важнейших задач кафедры общей физики МФТИ.
Эта работа ведётся непрерывно всем коллективом кафедры и данное издание 
отражает актуальное состояние курса. С момента последнего издания были 
поставлены новые работы, обновлён парк приборов, усовершенствованы 
методики выполнения работ и обновлены теоретические описания.

В связи с тем, что используемое оборудование модернизируется быстрее,
чем выходят издания книги, из настоящего издания исключены упоминания
о конкретных моделях приборов, задействованных в работах. Актуальные 
технические описания и инструкции к приборам постоянно обновляются
и доступны студентам на сайте кафедры.

В курсе общей физики МФТИ лабораторному практикуму отводится ключевая роль.
В~течение пяти семестров студент выполняет не менее 40 работ, на выполнение
которых отводится в общей сложности 300 аудиторных часов~--- половина 
аудиторных часов, отводимых весь на курс общей физики. Ещё столько же часов учебным планом отводится 
на самостоятельную работу студента для обработки результатов, оформления
отчёта и подготовки к его защите. В~последние годы особое внимание 
уделяется использованию современных методов регистрации, таких как цифровые 
осциллографы, видеокамеры, оптические датчики и т.\,д. Важное место занимает 
обучение студента корректной обработке экспериментальных данных, 
оценке погрешностей, а также представлению результатов своей работы 
согласно требованиям, приближающимся к современным стандартам научных 
публикаций.

Непреложным правилом в практикуме является постановка работ только 
в <<железе>>, без использования виртуальных работ, в которых реальные 
процессы заменены вычислительными моделями. 
Реальность всегда существенно богаче любой попытки её симуляции. 
При этом цифровые технологии активно используются как средства измерения 
и обработки данных.

Основателем физического практикума МФТИ является \textit{К.\,А.~Рогозинский}.
Профессор \textit{Л.\,Л.~Гольдин} в течение многих лет был бессменным 
редактором <<лабника>>~--- неоднократно переиздаваемой 
книги <<Лабораторные занятия по физике>>. С~2003 года
практикум издаётся в новом формате, в котором объединенным
по разделам описаниям работ предшествует подробное теоретическое введение.
Редактором сборника в обновленном формате был профессор \textit{А.\,Д. Гладун}.

В теоретических обзорах к разделам описаны фундаментальные принципы, лежащие 
в основе рассматриваемых физических явлений, основные формулы и их 
качественный анализ. Тексты введений не дублируют соответствующие разделы учебников,
но позволяют студенту получить ясное представление об изучаемом явлении 
в том случае, когда выполнение работы опережает лекционный и семинарский курс.

Высокий научный и методический уровень лабораторных работ является результатом 
большой работы всего коллектива кафедры общей физики МФТИ. Несмотря на то, что 
во многих случаях конкретные работы имеют определенных авторов, предложивших или
поставивших их впервые, они являются фактически плодом многолетнего труда всей 
кафедры. Авторы книги взяли на себя лишь скромный труд по систематизации и 
обобщению уникального опыта кафедры.
Приведём далеко не полный список преподавателей и инженеров кафедры, 
в разные годы внесших вклад в развитие практикума по электричеству и магнетизму: 
\textit{Д.\,А.~Александров}, \textit{Т.\,Ф.~Буклинова}, \textit{В.\,А.~Данилин}, 
\textit{Ф.\,Ф.~Игошин}, \textit{C.\,М.~Козел}, \textit{Л.\,В.~Ногинова}, 
\textit{В.\,В.~Можаев}, \textit{С.\,К.~Моршнев}, \textit{А.\,Г.~Пряжников}, 
\textit{Ю.\,А.~Самарский}, \textit{Т.\,М.~Сидоренко}, \textit{А.\,А.~Теврюков}, 
\textit{А.\,В.~Францессон}, \textit{Ю.\,М.~Ципенюк}, и многие другие.

\bigskip

{\raggedleft \textit{Авторы}\par}
